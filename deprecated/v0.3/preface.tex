% A preface generally covers the story of how the book came into being, or how the idea for the book was developed; this is often followed by thanks and acknowledgments to people who were helpful to the author during the time of writing.
\chapter*{Preface}
\addcontentsline{toc}{chapter}{Preface}

\section*{About the book}
Consider the following scenario: You have a computational problem to solve that is difficult either because of the inherent complexities of the problem (number of variables, shape of response surface) or because of the constraints imposed on the problem solving process (lack of resources, time, or expertise). 

This is a classical situation where a good solution (satisfies the problem) is required quickly (within broader constraints). A powerful collection of techniques intended for exactly this class of problem are \textit{Inspired Algorithms}. Such problem solving approaches are systematically engineered like any other technique, although the spark that ignited the general thrust came from analogy or metaphor with a natural process. Back to the scenario. An approximation technique is suggested from the field inspired algorithms. Each technique has its general area of application, although which set of techniques are appropriate? For a given technique, what is the canonical algorithm and which papers or books should be consulted? There are many such techniques, and each approach has tens, hundreds and in some cases thousands of research papers written about it, many of them insightful. This describes a situation of information overload for the problem solving practitioner. This book addresses this need by first presenting a diverse collection of such techniques, and by secondly presenting each technique in terms of its inspiration, problem solving strategy, and a practical programming tutorial that results in a compact and working canonical realization of the approach. 

The practitioner may read the inspiration and abstract their own problem solving strategy. The practitioner may read the canonical strategy and consider their own specialization of the approach to their problem solving needs. Finally, the practitioner may rapidly implement one or evaluate a handful of the canonical techniques, selecting the most appropriate for the present problem solving needs. Together, the inspiration, the strategy, and the canonical algorithm and its application heuristics provide three points of entry for the experienced system designer or research scientist, the systems engineer, and novice programmer or interest amateur to the application and study of inspired algorithms for practical computational problem solving.

\section*{About the author}
Dr Jason Brownlee's passion for programming and machine learning manifest early in the development of open source computer game modifications and tutorials. He started his technical carrier as a Java Enterprise consultant working closely and onsite with large corporate clients in the utility, retail, and directory industries. After a number of years of corporate enterprise experience, Jason followed his passion for basic research back to university and completed is doctorate in biologically inspired artificial intelligence. Since completing his PhD, Jason has pursued his entrepreneurial aspirations, founding a number of web businesses.

\section*{Acknowledgments}
This book would not exist today without the support, suggestions, corrections contributed by my small army of editors: \textit{your name here...}. 

Finally,  this book would not have been possible without the continued support of my love \textit{Ying Liu}. 
