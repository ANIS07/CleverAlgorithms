% The Clever Algorithms Project: http://www.CleverAlgorithms.com
% (c) Copyright 2010 Jason Brownlee. Some Rights Reserved. 
% This work is licensed under a Creative Commons Attribution-Noncommercial-Share Alike 2.5 Australia License.

% This is a chapter

\renewcommand{\bibsection}{\subsection{\bibname}}
\begin{bibunit}

\chapter{Probabilistic Algorithms}
\label{ch:probabilistic}
\index{Probabilistic Algorithms}
\index{Estimation of Distribution Algorithms}
\index{Population Model-Building Genetic Algorithms}

\section{Overview}
This chapter describes Probabilistic Algorithms


% probabilistic
\subsection{Probabilistic Models}
Probabilistic Algorithms are those algorithms that model a problem or search a problem space using an probabilistic model of candidate solutions. Many Metaheuristics and Computational Intelligence algorithms may be considered probabilistic, although the difference with algorithms is the explicit (rather than implicit) use of the tools of probability in problem solving. The majority of the algorithms described in this Chapter are referred to as Estimation of Distribution Algorithms.

% EDA
\subsection{Estimation of Distribution Algorithms}
Estimation of Distribution Algorithms (EDA) also called Probabilistic Model-Building Genetic Algorithms (PMBGA) are an extension of the field of Evolutionary Computation that model a population of candidate solutions as a probabilistic model. They generally involve iterations that alternate between creating candidate solutions in the problem space from a probabilistic model, and reducing a collection of generated candidate solutions into a probabilistic model. 

The model at the heart of an EDA typically provides the probabilistic expectation of a component or component configuration comprising part of an optimal solution. This estimation is typically based on the observed frequency of use of the component in better than average candidate solutions. The probabilistic model is used to generate candidate solutions in the problem space, typically in a component-wise or step-wise manner using a domain specific construction method to ensure validity.

% papers
Pelikan et al. provide a comprehensive summary of the field of probabilistic optimization algorithms, summarizing the core approaches and their differences \cite{Pelikan2002b}.
% books
The edited volume by Pelikan, Sastry, and Cantu-Paz provides a collection of studies on the popular Estimation of Distribution algorithms as well as methodology for designing algorithms and application demonstration studies \cite{Pelikan2006}.
An edited volume on studies of EDAs by Larranaga and Lozano \cite{Larranaga2002} and the follow-up volume by Lozano et al. \cite{Lozano2006} provide an applied foundation for the field.

% 
% Extensions
% 
\subsection{Extensions}
There are many other algorithms and classes of algorithm that were not described from the field of Estimation of Distribution Algorithm, not limited to:

\begin{itemize}
	\item \textbf{Extensions to UMDA}: Extensions to the Univariate Marginal Distribution Algorithm such as the Bivariate Marginal Distribution Algorithm (BMDA) \cite{Pelikan1998, Pelikan1999} and the Factorized Distribution Algorithm (FDA) \cite{Muhlenbein1999}.
	\item \textbf{Extensions to cGA}: Extensions to the Compact Genetic Algorithm such as the Extended Compact Genetic Algorithm (ECGA) \cite{Harik1999a, Harik2006}.
	\item \textbf{Extensions to BOA}: Extensions to the Bayesian Optimization Algorithm such as the Hierarchal Bayesian Optimization Algorithm (hBOA) \cite{Pelikan2000, Pelikan2001b} and the Incremental Bayesian Optimization Algorithm (iBOA) \cite{Pelikan2008}.
	\item \textbf{Bayesian Network Algorithms}: Other Bayesian network algorithms such as The Estimation of Bayesian Network Algorithm \cite{Etxeberria1999}, and the Learning Factorized Distribution Algorithm (LFDA) \cite{Muehlenbein1999}.
	\item \textbf{PIPE}: The Probabilistic Incremental Program Evolution that uses EDA methods for constructing programs \cite{Salustowicz1997}. 
	\item \textbf{SHCLVND}: The Stochastic Hill-Climbing with Learning by Vectors of Normal Distributions algorithm \cite{Rudlof1996}.
\end{itemize}

\putbib
\end{bibunit}

\newpage\begin{bibunit}% The Clever Algorithms Project: http://www.CleverAlgorithms.com
% (c) Copyright 2010 Jason Brownlee. Some Rights Reserved. 
% This work is licensed under a Creative Commons Attribution-Noncommercial-Share Alike 2.5 Australia License.

% This is an algorithm description, see:
% Jason Brownlee. A Template for Standardized Algorithm Descriptions. Technical Report CA-TR-20100107-1, The Clever Algorithms Project http://www.CleverAlgorithms.com, January 2010.

% Name
% The algorithm name defines the canonical name used to refer to the technique, in addition to common aliases, abbreviations, and acronyms. The name is used in terms of the heading and sub-headings of an algorithm description.
\section{Population-Based Incremental Learning} 
\label{sec:pbil}
\index{Population-Based Incremental Learning}
\index{PBIL}

% other names
% What is the canonical name and common aliases for a technique?
% What are the common abbreviations and acronyms for a technique?
\emph{Population-Based Incremental Learning, PBIL.}

% Taxonomy: Lineage and locality
% The algorithm taxonomy defines where a techniques fits into the field, both the specific subfields of Computational Intelligence and Biologically Inspired Computation as well as the broader field of Artificial Intelligence. The taxonomy also provides a context for determining the relation- ships between algorithms. The taxonomy may be described in terms of a series of relationship statements or pictorially as a venn diagram or a graph with hierarchical structure.
\subsection{Taxonomy}
% To what fields of study does a technique belong?
Population-Based Incremental Learning is an Estimation of Distribution Algorithm (EDA), also referred to as Population Model-Building Genetic Algorithms (PMBGA) an extension to the field of Evolutionary Computation.
% What are the closely related approaches to a technique?
PBIL is related to other EDAs such as the Compact Genetic Algorithm (Section~\ref{sec:compact_genetic_algorithm}), the Probabilistic Incremental Programing Evolution Algorithm, and the Bayesian Optimization Algorithm (Section~\ref{sec:boa}). The fact the the algorithm maintains a single prototype vector that is updated competitively shows some relationship to the Learning Vector Quantization algorithm (Section~\ref{sec:lvq}).

% Inspiration: Motivating system
% The inspiration describes the specific system or process that provoked the inception of the algorithm. The inspiring system may non-exclusively be natural, biological, physical, or social. The description of the inspiring system may include relevant domain specific theory, observation, nomenclature, and most important must include those salient attributes of the system that are somehow abstractly or conceptually manifest in the technique. The inspiration is described textually with citations and may include diagrams to highlight features and relationships within the inspiring system.
% Optional
\subsection{Inspiration}
% What is the system or process that motivated the development of a technique?
Population-Based Incremental Learning is a population-based technique without an inspiration. It is related to the Genetic Algorithm and other Evolutionary Algorithms that are inspired by the biological theory of evolution by means of natural selection.
% Which features of the motivating system are relevant to a technique?

% Metaphor: Explanation via analogy
% The metaphor is a description of the technique in the context of the inspiring system or a different suitable system. The features of the technique are made apparent through an analogous description of the features of the inspiring system. The explanation through analogy is not expected to be literal scientific truth, rather the method is used as an allegorical communication tool. The inspiring system is not explicitly described, this is the role of the ‘inspiration’ element, which represents a loose dependency for this element. The explanation is textual and uses the nomenclature of the metaphorical system.
% Optional
% \subsection{Metaphor}
% What is the explanation of a technique in the context of the inspiring system?
% What are the functionalities inferred for a technique from the analogous inspiring system?
% A textual description of the algorithm by analogy.

% Strategy: Problem solving plan
% The strategy is an abstract description of the computational model. The strategy describes the information processing actions a technique shall take in order to achieve an objective. The strategy provides a logical separation between a computational realization (procedure) and a analogous system (metaphor). A given problem solving strategy may be realized as one of a number specific algorithms or problem solving systems. The strategy description is textual using information processing and algorithmic terminology.
\subsection{Strategy}
% What is the information processing objective of a technique?
The information processing objective of the PBIL algorithm is to reduce the memory required by the genetic algorithm.
% What is a techniques plan of action?
This is done by reducing the population of a candidate solutions to a single prototype vector of attributes from which candidate solutions can be generated and assessed. Updates and mutation operators are also performed to the prototype vector, rather than the generated candidate solutions.

% Procedure: Abstract computation
% The algorithmic procedure summarizes the specifics of realizing a strategy as a systemized and parameterized computation. It outlines how the algorithm is organized in terms of the data structures and representations. The procedure may be described in terms of software engineering and computer science artifacts such as Pseudocode, design diagrams, and relevant mathematical equations.
\subsection{Procedure}
% What are the data structures and representations used in a technique?
The Population-Based Incremental Learning algorithm maintains a real-valued prototype vector that represents the probability of each component being expressed in a candidate solution. 
% What is the computational recipe for a technique?
Algorithm~\ref{alg:pbil} provides a pseudocode listing of the Population-Based Incremental Learning algorithm for maximizing a cost function.

\begin{algorithm}[ht]
	\SetLine  

	% data
	\SetKwData{Best}{$S_{best}$}
	\SetKwData{ProbabilityMutate}{$P_{mutation}$}
	\SetKwData{LearningRate}{$Learn_{rate}$}
	\SetKwData{MutationFactor}{$Mutation_{factor}$}
	\SetKwData{NumSamples}{$Samples_{num}$}
	\SetKwData{NumBits}{$Bits_{num}$}
	
	\SetKwData{Vector}{$V$}
	\SetKwData{CurrentBest}{$S_{current}$}
	\SetKwData{Sample}{$S_i$}
	\SetKwData{Bit}{$S_{bit}^i$}
	\SetKwData{VectorBit}{$V_{bit}^i$}

	% functions
	\SetKwFunction{InitializeVector}{InitializeVector}  
	\SetKwFunction{StopCondition}{StopCondition} 
	\SetKwFunction{GenerateSamples}{GenerateSamples} 
	\SetKwFunction{Cost}{Cost}
	\SetKwFunction{Rand}{Rand}
  
	% I/O
	\KwIn{\NumBits, \NumSamples, \LearningRate, \ProbabilityMutate, \MutationFactor}		
	\KwOut{\Best}

  % Algorithm
	\Vector $\leftarrow$ \InitializeVector{\NumBits}\;
	\Best $\leftarrow$ $\emptyset$\;
	
	\While{$\neg$\StopCondition{}} {
		\CurrentBest $\leftarrow$ $\emptyset$\;		
		\For{$i$ \KwTo \NumSamples} {
			\Sample $\leftarrow$ \GenerateSamples{\Vector}\;
			\If{\Cost{\Sample} $\leq$ \Cost{\CurrentBest}} {
				\CurrentBest $\leftarrow$ \Sample\;
				\If{\Cost{\Sample} $\leq$ \Cost{\Best}} {
					\Best $\leftarrow$ \Sample\;
				}
			}			
		}
		\ForEach{\Bit $\in$ \CurrentBest} {
			\VectorBit $\leftarrow$ \VectorBit $\times$ (1.0$-$\LearningRate) $+$ \Bit $\times$ \LearningRate\;
			\If{\Rand{} $<$ \ProbabilityMutate} {
				\VectorBit $\leftarrow$ \VectorBit $\times$ (1.0$-$\MutationFactor) $+$ \Rand{} $\times$ \MutationFactor\;
			}
		}
	}
	\Return{\Best}\;
	
	% end
	\caption{Pseudocode for PBIL.}
	\label{alg:pbil}
\end{algorithm}

% Heuristics: Usage guidelines
% The heuristics element describe the commonsense, best practice, and demonstrated rules for applying and configuring a parameterized algorithm. The heuristics relate to the technical details of the techniques procedure and data structures for general classes of application (neither specific implementations not specific problem instances). The heuristics are described textually, such as a series of guidelines in a bullet-point structure.
\subsection{Heuristics}
% What are the suggested configurations for a technique?
% What are the guidelines for the application of a technique to a problem instance?
\begin{itemize}
	\item PBIL was designed to optimize the probability of components from low cardinality sets, such as bit's in a binary string.
	\item The algorithm has a very small memory footprint (compared to some population-based evolutionary algorithms) given the compression of information into a single prototype vector.
	\item Extensions to PBIL have been proposed that extend the representation beyond sets to real-valued vectors.
	\item Variants of PBIL that were proposed in the original paper include updating the prototype vector with more than one competitive candidate solution (such as an average of top candidate solutions), and moving the prototype vector away from the least competitive candidate solution each iteration.
	\item Low learning rates are preferred, such as 0.1.
\end{itemize}

% Code Listing
% The code description provides a minimal but functional version of the technique implemented with a programming language. The code description must be able to be typed into an appropriate computer, compiled or interpreted as need be, and provide a working execution of the technique. The technique implementation also includes a minimal problem instance to which it is applied, and both the problem and algorithm implementations are complete enough to demonstrate the techniques procedure. The description is presented as a programming source code listing.
\subsection{Code Listing}
% How is a technique implemented as an executable program?
% How is a technique applied to a concrete problem instance?
Listing~\ref{pbil} provides an example of the Population-Based Incremental Learning algorithm implemented in the Ruby Programming Language. 
% problem
The demonstration problem is a maximizing binary optimization problem called OneMax that seeks a binary string of unity (all `1' bits). The objective function only provides an indication of the number of correct bits in a candidate string, not the positions of the correct bits.
% algorithm
The algorithm is an implementation of the simple PBIL algorithm that updates the prototype vector based on the best candidate solution generated each iteration. 

% the listing
\lstinputlisting[firstline=7,language=ruby,caption=Population-Based Incremental Learning algorithm in the Ruby Programming Language, label=pbil]{../src/algorithms/probabilistic/pbil.rb}

% References: Deeper understanding
% The references element description includes a listing of both primary sources of information about the technique as well as useful introductory sources for novices to gain a deeper understanding of the theory and application of the technique. The description consists of hand-selected reference material including books, peer reviewed conference papers, journal articles, and potentially websites. A bullet-pointed structure is suggested.
\subsection{References}
% What are the primary sources for a technique?
% What are the suggested reference sources for learning more about a technique?

% 
% Primary Sources
% 
\subsubsection{Primary Sources}
% seminal
The Population-Based Incremental Learning algorithm was proposed by Baluja in a technical report that proposed the base algorithm as well as a number of variants inspired by the Learning Vector Quantization algorithm \cite{Baluja1994}.
% early


% 
% Learn More
% 
\subsubsection{Learn More}
% reviews
Baluja and Caruana provide an excellent overview of PBIL and compare it to the standard Genetic Algorithm, released as a technical report \cite{Baluja1995} and later published \cite{Baluja1995a}. Baluja provides a detailed comparison between the Genetic algorithm and PBIL on a range of problems and scales in another technical report \cite{Baluja1995b}.
Greene provided an excellent account on the applicability of PBIL as a practical optimization algorithm \cite{Greene1996}.
H\"ohfeld and Rudolph provide the first theoretical analysis of the technique and provide a convergence proof \cite{Hohfeld1997}.
% books


\putbib\end{bibunit}
\newpage\begin{bibunit}% The Clever Algorithms Project: http://www.CleverAlgorithms.com
% (c) Copyright 2010 Jason Brownlee. Some Rights Reserved. 
% This work is licensed under a Creative Commons Attribution-Noncommercial-Share Alike 2.5 Australia License.

% This is an algorithm description, see:
% Jason Brownlee. A Template for Standardized Algorithm Descriptions. Technical Report CA-TR-20100107-1, The Clever Algorithms Project http://www.CleverAlgorithms.com, January 2010.

% Name
% The algorithm name defines the canonical name used to refer to the technique, in addition to common aliases, abbreviations, and acronyms. The name is used in terms of the heading and sub-headings of an algorithm description.
\section{Univariate Marginal Distribution Algorithm} 
\label{sec:umda}

% other names
% What is the canonical name and common aliases for a technique?
% What are the common abbreviations and acronyms for a technique?
\emph{The heading and alternate headings for the algorithm description.}

% Taxonomy: Lineage and locality
% The algorithm taxonomy defines where a techniques fits into the field, both the specific subfields of Computational Intelligence and Biologically Inspired Computation as well as the broader field of Artificial Intelligence. The taxonomy also provides a context for determining the relation- ships between algorithms. The taxonomy may be described in terms of a series of relationship statements or pictorially as a venn diagram or a graph with hierarchical structure.
\subsection{Taxonomy}
% To what fields of study does a technique belong?
% What are the closely related approaches to a technique?
A small tree diagram showing related fields and algorithms.

% Inspiration: Motivating system
% The inspiration describes the specific system or process that provoked the inception of the algorithm. The inspiring system may non-exclusively be natural, biological, physical, or social. The description of the inspiring system may include relevant domain specific theory, observation, nomenclature, and most important must include those salient attributes of the system that are somehow abstractly or conceptually manifest in the technique. The inspiration is described textually with citations and may include diagrams to highlight features and relationships within the inspiring system.
% Optional
\subsection{Inspiration}
% What is the system or process that motivated the development of a technique?
% Which features of the motivating system are relevant to a technique?
A textual description of the inspiring system.

% Metaphor: Explanation via analogy
% The metaphor is a description of the technique in the context of the inspiring system or a different suitable system. The features of the technique are made apparent through an analogous description of the features of the inspiring system. The explanation through analogy is not expected to be literal scientific truth, rather the method is used as an allegorical communication tool. The inspiring system is not explicitly described, this is the role of the ‘inspiration’ element, which represents a loose dependency for this element. The explanation is textual and uses the nomenclature of the metaphorical system.
% Optional
\subsection{Metaphor}
% What is the explanation of a technique in the context of the inspiring system?
% What are the functionalities inferred for a technique from the analogous inspiring system?
A textual description of the algorithm by analogy.

% Strategy: Problem solving plan
% The strategy is an abstract description of the computational model. The strategy describes the information processing actions a technique shall take in order to achieve an objective. The strategy provides a logical separation between a computational realization (procedure) and a analogous system (metaphor). A given problem solving strategy may be realized as one of a number specific algorithms or problem solving systems. The strategy description is textual using information processing and algorithmic terminology.
\subsection{Strategy}
% What is the information processing objective of a technique?
% What is a techniques plan of action?
A textual description of the information processing strategy.

% Procedure: Abstract computation
% The algorithmic procedure summarizes the specifics of realizing a strategy as a systemized and parameterized computation. It outlines how the algorithm is organized in terms of the data structures and representations. The procedure may be described in terms of software engineering and computer science artifacts such as pseudo code, design diagrams, and relevant mathematical equations.
\subsection{Procedure}
% What is the computational recipe for a technique?
% What are the data structures and representations used in a technique?
A pseudo code description of the algorithms procedure.

% Heuristics: Usage guidelines
% The heuristics element describe the commonsense, best practice, and demonstrated rules for applying and configuring a parameterized algorithm. The heuristics relate to the technical details of the techniques procedure and data structures for general classes of application (neither specific implementations not specific problem instances). The heuristics are described textually, such as a series of guidelines in a bullet-point structure.
\subsection{Heuristics}
% What are the suggested configurations for a technique?
% What are the guidelines for the application of a technique to a problem instance?
A bullet-point listing of best practice usage.

% Code Listing
% The code description provides a minimal but functional version of the technique implemented with a programming language. The code description must be able to be typed into an appropriate computer, compiled or interpreted as need be, and provide a working execution of the technique. The technique implementation also includes a minimal problem instance to which it is applied, and both the problem and algorithm implementations are complete enough to demonstrate the techniques procedure. The description is presented as a programming source code listing.
\subsection{Code Listing}
% How is a technique implemented as an executable program?
% How is a technique applied to a concrete problem instance?
A code listing and a terse description of the listing.

% References: Deeper understanding
% The references element description includes a listing of both primary sources of information about the technique as well as useful introductory sources for novices to gain a deeper understanding of the theory and application of the technique. The description consists of hand-selected reference material including books, peer reviewed conference papers, journal articles, and potentially websites. A bullet-pointed structure is suggested.
\subsection{References}
% What are the primary sources for a technique?
% What are the suggested reference sources for learning more about a technique?
An bullet-point annotated reference list of primary sources and useful resources.


\putbib\end{bibunit}
\newpage\begin{bibunit}% The Clever Algorithms Project: http://www.CleverAlgorithms.com
% (c) Copyright 2010 Jason Brownlee. Some Rights Reserved. 
% This work is licensed under a Creative Commons Attribution-Noncommercial-Share Alike 2.5 Australia License.

% This is an algorithm description, see:
% Jason Brownlee. A Template for Standardized Algorithm Descriptions. Technical Report CA-TR-20100107-1, The Clever Algorithms Project http://www.CleverAlgorithms.com, January 2010.

% Name
% The algorithm name defines the canonical name used to refer to the technique, in addition to common aliases, abbreviations, and acronyms. The name is used in terms of the heading and sub-headings of an algorithm description.
\section{Compact Genetic Algorithm} 
\label{sec:compact_genetic_algorithm}
\index{Compact Genetic Algorithm}
\index{cGA}

% other names
% What is the canonical name and common aliases for a technique?
% What are the common abbreviations and acronyms for a technique?
\emph{Compact Genetic Algorithm, CGA, cGA.}

% Taxonomy: Lineage and locality
% The algorithm taxonomy defines where a techniques fits into the field, both the specific subfields of Computational Intelligence and Biologically Inspired Computation as well as the broader field of Artificial Intelligence. The taxonomy also provides a context for determining the relation- ships between algorithms. The taxonomy may be described in terms of a series of relationship statements or pictorially as a venn diagram or a graph with hierarchical structure.
\subsection{Taxonomy}
% To what fields of study does a technique belong?
The Compact Genetic Algorithm is an Estimation of Distribution Algorithm (EDA), also referred to as Population Model-Building Genetic Algorithms (PMBGA), an extension to the field of Evolutionary Computation.
% What are the closely related approaches to a technique?
The Compact Genetic Algorithm is the basis for extensions such as the Extended Compact Genetic Algorithm (ECGA).
It is related to other EDAs such as the Univariate Marginal Probability Algorithm (Section~\ref{sec:umda}), the Population-Based Incremental Learning algorithm (Section~\ref{sec:pbil}), and the Bayesian Optimization Algorithm (Section~\ref{sec:boa}).

% Inspiration: Motivating system
% The inspiration describes the specific system or process that provoked the inception of the algorithm. The inspiring system may non-exclusively be natural, biological, physical, or social. The description of the inspiring system may include relevant domain specific theory, observation, nomenclature, and most important must include those salient attributes of the system that are somehow abstractly or conceptually manifest in the technique. The inspiration is described textually with citations and may include diagrams to highlight features and relationships within the inspiring system.
% Optional
\subsection{Inspiration}
% What is the system or process that motivated the development of a technique?
The Compact Genetic Algorithm is a probabilistic technique without an inspiration. It is related to the Genetic Algorithm and other Evolutionary Algorithms that are inspired by the biological theory of evolution by means of natural selection.
% Which features of the motivating system are relevant to a technique?

% Metaphor: Explanation via analogy
% The metaphor is a description of the technique in the context of the inspiring system or a different suitable system. The features of the technique are made apparent through an analogous description of the features of the inspiring system. The explanation through analogy is not expected to be literal scientific truth, rather the method is used as an allegorical communication tool. The inspiring system is not explicitly described, this is the role of the ‘inspiration’ element, which represents a loose dependency for this element. The explanation is textual and uses the nomenclature of the metaphorical system.
% Optional
% \subsection{Metaphor}
% What is the explanation of a technique in the context of the inspiring system?
% What are the functionalities inferred for a technique from the analogous inspiring system?
% A textual description of the algorithm by analogy.

% Strategy: Problem solving plan
% The strategy is an abstract description of the computational model. The strategy describes the information processing actions a technique shall take in order to achieve an objective. The strategy provides a logical separation between a computational realization (procedure) and a analogous system (metaphor). A given problem solving strategy may be realized as one of a number specific algorithms or problem solving systems. The strategy description is textual using information processing and algorithmic terminology.
\subsection{Strategy}
% What is the information processing objective of a technique?
The information processing objective of the algorithm is to simulate the behavior of a Genetic Algorithm with a much smaller memory footprint (without requiring a population to be maintained). 
% What is a techniques plan of action?
This is achieved by maintaining a vector that specifies the probability of including each component in a solution in new candidate solutions. Candidate solutions are probabilistically generated from the vector and the components in the better solution are used to make small changes to the probabilities in the vector.

% Procedure: Abstract computation
% The algorithmic procedure summarizes the specifics of realizing a strategy as a systemized and parameterized computation. It outlines how the algorithm is organized in terms of the data structures and representations. The procedure may be described in terms of software engineering and computer science artifacts such as Pseudocode, design diagrams, and relevant mathematical equations.
\subsection{Procedure}
% What are the data structures and representations used in a technique?
The Compact Genetic Algorithm maintains a real-valued prototype vector that represents the probability of each component being expressed in a candidate solution. 
% What is the computational recipe for a technique?
Algorithm~\ref{alg:cga} provides a pseudocode listing of the Compact Genetic Algorithm for maximizing a cost function. The parameter $n$ indicates the amount to update probabilities for conflicting bits in each algorithm iteration.

\begin{algorithm}[htp]
	\SetLine  

	% data
	\SetKwData{Best}{$S_{best}$}
	\SetKwData{NumBits}{$Bits_{num}$}
	\SetKwData{VectorUpdate}{$n$}
	
	\SetKwData{Vector}{$V$}
	\SetKwData{CandidateOne}{$S_{1}$}
	\SetKwData{CandidateTwo}{$S_{2}$}	
	\SetKwData{Winner}{$S_{winner}$}
	\SetKwData{Loser}{$S_{loser}$}
	
	\SetKwData{WinnerBit}{$S_{winner}^i$}
	\SetKwData{LoserBit}{$S_{loser}^i$}
	\SetKwData{VectorBit}{$V_{i}^i$}

	% functions
	\SetKwFunction{InitializeVector}{InitializeVector}  
	\SetKwFunction{StopCondition}{StopCondition} 
	\SetKwFunction{GenerateSamples}{GenerateSamples} 
	\SetKwFunction{Cost}{Cost}
	\SetKwFunction{SelectWinnerAndLoser}{SelectWinnerAndLoser}
  
	% I/O
	\KwIn{\NumBits, \VectorUpdate}		
	\KwOut{\Best}

  % Algorithm
	\Vector $\leftarrow$ \InitializeVector{\NumBits, 0.5}\;
	\Best $\leftarrow$ $0$\;
	
	\While{$\neg$\StopCondition{}} {
		\CandidateOne $\leftarrow$ \GenerateSamples{\Vector}\;
		\CandidateTwo $\leftarrow$ \GenerateSamples{\Vector}\;
		\Winner, \Loser $\leftarrow$ \SelectWinnerAndLoser{\CandidateOne, \CandidateTwo}\;

		\If{\Cost{\Winner} $\leq$ \Cost{\Best}} {
			\Best $\leftarrow$ \Winner\;
		}

		\For{$i$ \KwTo \NumBits} {
			\If{\WinnerBit $\neq$ \LoserBit} {
				\eIf{\WinnerBit $\equiv$ 1} {
					\VectorBit $\leftarrow$ \VectorBit $+$ $\frac{1}{\VectorUpdate}$\;
				}{
					\VectorBit $\leftarrow$ \VectorBit $-$ $\frac{1}{\VectorUpdate}$\;
				}
			}
		}		
	}
	\Return{\Best}\;
	
	% end
	\caption{Pseudocode for the Compact Genetic Algorithm.}
	\label{alg:cga}
\end{algorithm}

% Heuristics: Usage guidelines
% The heuristics element describe the commonsense, best practice, and demonstrated rules for applying and configuring a parameterized algorithm. The heuristics relate to the technical details of the techniques procedure and data structures for general classes of application (neither specific implementations not specific problem instances). The heuristics are described textually, such as a series of guidelines in a bullet-point structure.
\subsection{Heuristics}
% What are the suggested configurations for a technique?
% What are the guidelines for the application of a technique to a problem instance?
\begin{itemize}
	\item The vector update parameter ($n$) influences the amount that the probabilities are updated each algorithm iteration.
	\item The vector update parameter ($n$) may be considered to be comparable to the population size parameter in the Genetic Algorithm.
	\item Early results demonstrate that the cGA may be comparable to a standard Genetic Algorithm on classical binary string optimization problems (such as OneMax).	
	\item The algorithm may be considered to have converged if the vector probabilities are all either $0$ or $1$.
\end{itemize}

% Code Listing
% The code description provides a minimal but functional version of the technique implemented with a programming language. The code description must be able to be typed into an appropriate computer, compiled or interpreted as need be, and provide a working execution of the technique. The technique implementation also includes a minimal problem instance to which it is applied, and both the problem and algorithm implementations are complete enough to demonstrate the techniques procedure. The description is presented as a programming source code listing.
\subsection{Code Listing}
% How is a technique implemented as an executable program?
% How is a technique applied to a concrete problem instance?
Listing~\ref{compact_genetic_algorithm} provides an example of the Compact Genetic Algorithm implemented in the Ruby Programming Language. 
% problem
The demonstration problem is a maximizing binary optimization problem called OneMax that seeks a binary string of unity (all `1' bits). The objective function only provides an indication of the number of correct bits in a candidate string, not the positions of the correct bits.
% algorithm
The algorithm is an implementation of Compact Genetic Algorithm that uses integer values to represent 1 and 0 bits in a binary string representation. 

% the listing
\lstinputlisting[firstline=7,language=ruby,caption=Compact Genetic Algorithm in the Ruby Programming Language, label=compact_genetic_algorithm]{../src/algorithms/probabilistic/compact_genetic_algorithm.rb}

% References: Deeper understanding
% The references element description includes a listing of both primary sources of information about the technique as well as useful introductory sources for novices to gain a deeper understanding of the theory and application of the technique. The description consists of hand-selected reference material including books, peer reviewed conference papers, journal articles, and potentially websites. A bullet-pointed structure is suggested.
\subsection{References}
% What are the primary sources for a technique?
% What are the suggested reference sources for learning more about a technique?

% 
% Primary Sources
% 
\subsubsection{Primary Sources}
% seminal
The Compact Genetic Algorithm was proposed by Harik, Lobo, and Goldberg in 1999 \cite{Harik1999}, based on a random walk model previously introduced by Harik et al.\ \cite{Harik1997}. In the introductory paper, the cGA is demonstrated to be comparable to the Genetic Algorithm on standard binary string optimization problems.
% early

% 
% Learn More
% 
\subsubsection{Learn More}
% reviews
Harik et al.\ extended the Compact Genetic Algorithm (called the Extended Compact Genetic Algorithm) to generate populations of candidate solutions and perform selection (much like the Univariate Marginal Probabilist Algorithm), although it used Marginal Product Models \cite{Harik1999a, Harik2006}. Sastry and Goldberg performed further analysis into the Extended Compact Genetic Algorithm applying the method to a complex optimization problem \cite{Sastry2000}.
% books


\putbib\end{bibunit}
\newpage\begin{bibunit}% The Clever Algorithms Project: http://www.CleverAlgorithms.com
% (c) Copyright 2010 Jason Brownlee. Some Rights Reserved. 
% This work is licensed under a Creative Commons Attribution-Noncommercial-Share Alike 2.5 Australia License.

% This is an algorithm description, see:
% Jason Brownlee. A Template for Standardized Algorithm Descriptions. Technical Report CA-TR-20100107-1, The Clever Algorithms Project http://www.CleverAlgorithms.com, January 2010.

% Name
% The algorithm name defines the canonical name used to refer to the technique, in addition to common aliases, abbreviations, and acronyms. The name is used in terms of the heading and sub-headings of an algorithm description.
\section{Bayesian Optimization Algorithm} 
\label{sec:boa}
\index{Bayesian Optimization Algorithm}
\index{BOA}

% other names
% What is the canonical name and common aliases for a technique?
% What are the common abbreviations and acronyms for a technique?
\emph{Bayesian Optimization Algorithm, BOA.}

% Taxonomy: Lineage and locality
% The algorithm taxonomy defines where a techniques fits into the field, both the specific subfields of Computational Intelligence and Biologically Inspired Computation as well as the broader field of Artificial Intelligence. The taxonomy also provides a context for determining the relation- ships between algorithms. The taxonomy may be described in terms of a series of relationship statements or pictorially as a venn diagram or a graph with hierarchical structure.
\subsection{Taxonomy}
% To what fields of study does a technique belong?
The Bayesian Optimization Algorithm belongs to the field of Estimation of Distribution Algorithms, also referred to as Population Model-Building Genetic Algorithms (PMBGA) an extension to the field of Evolutionary Computation. More broadly, BOA belongs to the field of Computational Intelligence.
% What are the closely related approaches to a technique?
The Bayesian Optimization Algorithm is related to other Estimation of Distribution Algorithms such as the Population Incremental Learning Algorithm, and the Univariate Marginal Distribution Algorithm.
It is also the basis for extensions such as the Hierarchal Bayesian Optimization Algorithm (hBOA) and the Incremental Bayesian Optimization Algorithm (iBOA).

% Inspiration: Motivating system
% The inspiration describes the specific system or process that provoked the inception of the algorithm. The inspiring system may non-exclusively be natural, biological, physical, or social. The description of the inspiring system may include relevant domain specific theory, observation, nomenclature, and most important must include those salient attributes of the system that are somehow abstractly or conceptually manifest in the technique. The inspiration is described textually with citations and may include diagrams to highlight features and relationships within the inspiring system.
% Optional
\subsection{Inspiration}
% What is the system or process that motivated the development of a technique?
% Which features of the motivating system are relevant to a technique?
Bayesian Optimization Algorithm is a technique without an inspiration. It is related to the Genetic Algorithm and other Evolutionary algorithms that are inspired by the biological theory of evolution by means of natural selection.

% Metaphor: Explanation via analogy
% The metaphor is a description of the technique in the context of the inspiring system or a different suitable system. The features of the technique are made apparent through an analogous description of the features of the inspiring system. The explanation through analogy is not expected to be literal scientific truth, rather the method is used as an allegorical communication tool. The inspiring system is not explicitly described, this is the role of the ‘inspiration’ element, which represents a loose dependency for this element. The explanation is textual and uses the nomenclature of the metaphorical system.
% Optional
% \subsection{Metaphor}
% What is the explanation of a technique in the context of the inspiring system?
% What are the functionalities inferred for a technique from the analogous inspiring system?
% A textual description of the algorithm by analogy.

% Strategy: Problem solving plan
% The strategy is an abstract description of the computational model. The strategy describes the information processing actions a technique shall take in order to achieve an objective. The strategy provides a logical separation between a computational realization (procedure) and a analogous system (metaphor). A given problem solving strategy may be realized as one of a number specific algorithms or problem solving systems. The strategy description is textual using information processing and algorithmic terminology.
\subsection{Strategy}
% What is the information processing objective of a technique?
The information processing objective of the technique is to construct a probabilistic model that describes the relationships between the components of fit solutions in the problem space.
% What is a techniques plan of action?
This is achieved by repeating the process of creating and sampling from a Bayesian network that contains the conditional dependancies, independencies, and conditional probabilities between the components of a solution. The network is constructed from the relative frequencies of the components within a population of high fitness candidate solutions. Once the network is constructed, the candidate solutions are discarded and a new population of candidate solutions are generated from the model. The process is repeated until the model converges on a fit prototype solution.

% Procedure: Abstract computation
% The algorithmic procedure summarizes the specifics of realizing a strategy as a systemized and parameterized computation. It outlines how the algorithm is organized in terms of the data structures and representations. The procedure may be described in terms of software engineering and computer science artifacts such as pseudo code, design diagrams, and relevant mathematical equations.
\subsection{Procedure}
% What is the computational recipe for a technique?
% What are the data structures and representations used in a technique?
Algorithm~\ref{alg:boa} provides a pseudo-code listing of the Bayesian Optimization Algorithm for minimizing a cost function.
% network construction
The Bayesian network is constructed each iteration using a greedy algorithm. The network is assessed based on its fit of the information in the population of candidate solutions using either a Bayesian Dirichlet Metric (BD) \cite{Pelikan1999a}, or a Bayesian Information Criterion (BIC) \cite{Pelikan2005} (Chapter~3).

\begin{algorithm}[htp]
	\SetLine  
	% data
	\SetKwData{Best}{$S_{best}$}
	\SetKwData{NumBits}{$Bits_{num}$}
	\SetKwData{PopulationSize}{$Population_{size}$}
	\SetKwData{SelectionSize}{$Selection_{size}$}
	% data
	\SetKwData{Model}{Model}
	\SetKwData{Selected}{Selected}
	\SetKwData{Population}{Population}
	\SetKwData{OffSpring}{OffSpring}
	% functions
	\SetKwFunction{InitializePopulation}{InitializePopulation}  
	\SetKwFunction{EvaluatePopulation}{EvaluatePopulation} 
	\SetKwFunction{GetBestSolution}{GetBestSolution} 
	\SetKwFunction{StopCondition}{StopCondition}
	\SetKwFunction{SelectFitSolutions}{SelectFitSolutions}
	\SetKwFunction{ConstructBayesianNetwork}{ConstructBayesianNetwork}
	\SetKwFunction{ProbabilisticallyConstructSolution}{ProbabilisticallyConstructSolution}
	\SetKwFunction{Combine}{Combine}
  
	% I/O
	\KwIn{\NumBits, \PopulationSize, \SelectionSize}
	\KwOut{\Best}

  % Algorithm
	\Population $\leftarrow$ \InitializePopulation{\NumBits, \PopulationSize}\;
	\EvaluatePopulation{\Population}\;
	\Best $\leftarrow$ \GetBestSolution{\Population}\;
	
	\While{$\neg$\StopCondition{}} {
		\Selected $\leftarrow$ \SelectFitSolutions{\Population}\;
		\Model $\leftarrow$ \ConstructBayesianNetwork{\Selected}\;
		\OffSpring $\leftarrow$ $0$\;
		\For{$i$ \KwTo \PopulationSize} {
			\OffSpring $\leftarrow$ \ProbabilisticallyConstructSolution{\Model}\;
		}
		\EvaluatePopulation{\OffSpring}\;
		\Best $\leftarrow$ \GetBestSolution{\OffSpring}\;
		\Population $\leftarrow$ \Combine{\Population, \OffSpring}\;
	}
	\Return{\Best}\;
	
	% end
	\caption{Pseudo Code for the Bayesian Optimization Algorithm.}
	\label{alg:boa}
\end{algorithm}

% Heuristics: Usage guidelines
% The heuristics element describe the commonsense, best practice, and demonstrated rules for applying and configuring a parameterized algorithm. The heuristics relate to the technical details of the techniques procedure and data structures for general classes of application (neither specific implementations not specific problem instances). The heuristics are described textually, such as a series of guidelines in a bullet-point structure.
\subsection{Heuristics}
% What are the suggested configurations for a technique?
% What are the guidelines for the application of a technique to a problem instance?
\begin{itemize}
	\item The Bayesian Optimization algorithm was designed and investigated on binary string-base problems, most commonly representing binary function optimization problems.
	\item Bayesian networks are typically constructed (grown) from scratch each iteration using an iterative process of adding, removing, and reversing links. Additionally, past networks may be used as the basis for the process that are in turn verified.
	\item A greedy hill-climbing algorithm is used to each algorithm iteration optimize a Bayesian network to represent a population of candidate solutions.
	\item The fitness of constructed bayesian networks may be assessed using the Bayesian Dirichlet Metric (BD) or a Minimum Description length method called the Bayesian Information Criterion (BIC).
	
\end{itemize}

% Code Listing
% The code description provides a minimal but functional version of the technique implemented with a programming language. The code description must be able to be typed into an appropriate computer, compiled or interpreted as need be, and provide a working execution of the technique. The technique implementation also includes a minimal problem instance to which it is applied, and both the problem and algorithm implementations are complete enough to demonstrate the techniques procedure. The description is presented as a programming source code listing.
\subsection{Code Listing}
% How is a technique implemented as an executable program?
% How is a technique applied to a concrete problem instance?
% Listing~\ref{boa} provides an example of the Bayesian Optimization Algorithm implemented in the Ruby Programming Language. 
% % problem
% The demonstration problem is a maximizing binary optimization problem called OneMax that seeks a binary string of unity (all `1' bits). The objective function provides only an indication of the number of correct bits in a candidate string, not the positions of the correct bits.
% 
% % algorithm
% The algorithm is an implementation based on the description of the technique by Pelikan in Chapter~3 of his book \cite{Pelikan2005a}.
% 
% % the listing
% \lstinputlisting[firstline=7,language=ruby,caption=Bayesian Optimization Algorithm in the Ruby Programming Language, label=boa]{../src/algorithms/probabilistic/boa.rb}
A listing of the algorithm is currently not provided.

% References: Deeper understanding
% The references element description includes a listing of both primary sources of information about the technique as well as useful introductory sources for novices to gain a deeper understanding of the theory and application of the technique. The description consists of hand-selected reference material including books, peer reviewed conference papers, journal articles, and potentially websites. A bullet-pointed structure is suggested.
\subsection{References}
% What are the primary sources for a technique?
% What are the suggested reference sources for learning more about a technique?

% 
% Primary Sources
% 
\subsubsection{Primary Sources}
% seminal
The Bayesian Optimization Algorithm was proposed by Pelikan, Goldberg, and Cantu-Paz in the technical report \cite{Pelikan1998a}, that was later published \cite{Pelikan2002}. The technique was proposed as an extension to the state of Estimation of Distribution algorithms (such as the Univariate Marginal Distribution Algorithm and the Bivariate Marginal Distribution Algorithm) that used a Bayesian Network to model the relationships and conditional probabilities for the components expressed in a population of fit candidate solutions.
% early 
Pelikan, Goldberg, and Cantu-Paz also described the approach applied to decelptive binary optimization problems (trap functions) in a paper that was published before the seminal journal article \cite{Pelikan1999a}.

% 
% Learn More
% 
\subsubsection{Learn More}
% hBOA
Pelikan and Goldberg described an extension to the approach called the Hierarchical Bayesian Optimization Algorithm (hBOA) \cite{Pelikan2000, Pelikan2001b}. The differences in the hBOA algorithm are that it replaces the decision tables (used to store the probabilities) with decision graphs and used a niching method called Restricted Tournament Replacement to maintain diversity in the selected set of candidate solutions used to construct the network models.
% thesis
Pelikan's work on BOA culminated in his PhD thesis that provides a detailed treatment of the approach, its configuration and application \cite{Pelikan2002a}.
% iBOA
Pelikan, Sastry, and Goldberg proposed the Incremental Bayesian Optimization Algorithm (iBOA) extension of the approach that removes the population and adds incremental updates to the Bayesian network \cite{Pelikan2008}.

% books
Pelikan published a book that focused on the technique, walking through the development of probabilistic algorithms inspired by evolutionary computation, a detailed look at the Bayesian Optimization Algorithm (Chapter~3), the hierarchic extension to Hierarchical Bayesian Optimization Algorithm and demonstration studies of the approach on test problems \cite{Pelikan2005}.


\putbib\end{bibunit}
\newpage\begin{bibunit}% The Clever Algorithms Project: http://www.CleverAlgorithms.com
% (c) Copyright 2010 Jason Brownlee. Some Rights Reserved. 
% This work is licensed under a Creative Commons Attribution-Noncommercial-Share Alike 2.5 Australia License.

% This is an algorithm description, see:
% Jason Brownlee. A Template for Standardized Algorithm Descriptions. Technical Report CA-TR-20100107-1, The Clever Algorithms Project http://www.CleverAlgorithms.com, January 2010.

% Name
% The algorithm name defines the canonical name used to refer to the technique, in addition to common aliases, abbreviations, and acronyms. The name is used in terms of the heading and sub-headings of an algorithm description.
\section{Cross-Entropy Method} 
\label{sec:cross_entropy}
\index{Cross-Entropy Method}

% other names
% What is the canonical name and common aliases for a technique?
% What are the common abbreviations and acronyms for a technique?
\emph{Cross-Entropy Method, Cross Entropy Method, CEM.}

% Taxonomy: Lineage and locality
% The algorithm taxonomy defines where a techniques fits into the field, both the specific subfields of Computational Intelligence and Biologically Inspired Computation as well as the broader field of Artificial Intelligence. The taxonomy also provides a context for determining the relation- ships between algorithms. The taxonomy may be described in terms of a series of relationship statements or pictorially as a venn diagram or a graph with hierarchical structure.
\subsection{Taxonomy}
% To what fields of study does a technique belong?
The Cross-Entropy Method is a probabilistic optimization belonging to the field of Stochastic Optimization.
% What are the closely related approaches to a technique?
It is similar to other Stochastic Optimization and algorithms such as Simulated Annealing (Section~\ref{sec:simulated_annealing}), and to Estimation of Distribution Algorithms such as the Probabilistic Incremental Learning Algorithm (Section~\ref{sec:pbil}).

% Inspiration: Motivating system
% The inspiration describes the specific system or process that provoked the inception of the algorithm. The inspiring system may non-exclusively be natural, biological, physical, or social. The description of the inspiring system may include relevant domain specific theory, observation, nomenclature, and most important must include those salient attributes of the system that are somehow abstractly or conceptually manifest in the technique. The inspiration is described textually with citations and may include diagrams to highlight features and relationships within the inspiring system.
% Optional
\subsection{Inspiration}
% What is the system or process that motivated the development of a technique?
The Cross-Entropy Method does not have an inspiration. It was developed as an efficient estimation technique for rare-event probabilities in discrete event simulation systems and was adapted for use in optimization.
The name of the technique comes from the Kullback-Leibler cross-entropy method for measuring the amount of information (bits) needed to identify an event from a set of probabilities.
% Which features of the motivating system are relevant to a technique?

% Metaphor: Explanation via analogy
% The metaphor is a description of the technique in the context of the inspiring system or a different suitable system. The features of the technique are made apparent through an analogous description of the features of the inspiring system. The explanation through analogy is not expected to be literal scientific truth, rather the method is used as an allegorical communication tool. The inspiring system is not explicitly described, this is the role of the ‘inspiration’ element, which represents a loose dependency for this element. The explanation is textual and uses the nomenclature of the metaphorical system.
% Optional
% \subsection{Metaphor}
% What is the explanation of a technique in the context of the inspiring system?
% What are the functionalities inferred for a technique from the analogous inspiring system?
% A textual description of the algorithm by analogy.

% Strategy: Problem solving plan
% The strategy is an abstract description of the computational model. The strategy describes the information processing actions a technique shall take in order to achieve an objective. The strategy provides a logical separation between a computational realization (procedure) and a analogous system (metaphor). A given problem solving strategy may be realized as one of a number specific algorithms or problem solving systems. The strategy description is textual using information processing and algorithmic terminology.
\subsection{Strategy}
% What is the information processing objective of a technique?
The information processing strategy of the algorithm is to sample the problem space and approximate the distribution of good solutions.
% What is a techniques plan of action?
This is achieved by assuming a distribution of the problem space (such as Gaussian), sampling the problem domain by generating candidate solutions using the distribution, and updating the distribution based on the better candidate solutions discovered. Samples are constructed step-wise (one component at a time) based on the summarized distribution of good solutions. As the algorithm progresses, the distribution becomes more refined until it focuses on the area or scope of optimal solutions in the domain.

% Procedure: Abstract computation
% The algorithmic procedure summarizes the specifics of realizing a strategy as a systemized and parameterized computation. It outlines how the algorithm is organized in terms of the data structures and representations. The procedure may be described in terms of software engineering and computer science artifacts such as Pseudocode, design diagrams, and relevant mathematical equations.
\subsection{Procedure}
% What is the computational recipe for a technique?
% What are the data structures and representations used in a technique?
Algorithm~\ref{alg:cross_entropy_method} provides a pseudocode listing of the Cross-Entropy Method algorithm for minimizing a cost function.

\begin{algorithm}[ht]
	\SetLine

	% params
	\SetKwData{ProblemSize}{$Problem_{size}$}
	\SetKwData{NumSamples}{$Samples_{num}$}
	\SetKwData{NumUpdateSamples}{$UpdateSamples_{num}$}
	\SetKwData{LearningRate}{$Learn_{rate}$}
	\SetKwData{MinVariance}{$Variance_{min}$}
	% data
	\SetKwData{Means}{Means}
	\SetKwData{Variances}{Variances}
	\SetKwData{Best}{$S_{best}$}
	\SetKwData{Samples}{Samples}
	\SetKwData{SamplesZero}{$Samples_{0}$}
	\SetKwData{SelectedSamples}{$Samples_{selected}$}
	\SetKwData{Meani}{$Means_i$}
	\SetKwData{Variancei}{$Variances_i$}
	% functions
	\SetKwFunction{InitializeMeans}{InitializeMeans} 
	\SetKwFunction{InitializeVariances}{InitializeVariances} 
	\SetKwFunction{Max}{Max} 	
	\SetKwFunction{GenerateSample}{GenerateSample} 
	\SetKwFunction{EvaluateSamples}{EvaluateSamples} 
	\SetKwFunction{SortSamplesByQuality}{SortSamplesByQuality} 
	\SetKwFunction{Cost}{Cost} 
	\SetKwFunction{SelectBestSamples}{SelectBestSamples} 	
	\SetKwFunction{Mean}{Mean} 
	\SetKwFunction{Variance}{Variance} 
	
	% I/O
	\KwIn{\ProblemSize, \NumSamples, \NumUpdateSamples, \LearningRate, \MinVariance}		
	\KwOut{\Best}

  % Algorithm
	\Means $\leftarrow$ \InitializeMeans{}\;
	\Variances $\leftarrow$ \InitializeVariances{}\;
	\Best $\leftarrow$ $\emptyset$\;
	\While{\Max{\Variances} $\leq$ \MinVariance} {
		\Samples $\leftarrow$ $0$\;
		\For{$i=0$ \KwTo \NumSamples} {
			\Samples $\leftarrow$ \GenerateSample{\Means, \Variances}\;
		}
		\EvaluateSamples{\Samples}\;
		\SortSamplesByQuality{\Samples}\;
		\If{\Cost{\SamplesZero} $\leq$ \Cost{\Best}} {
			\Best $\leftarrow$ \SamplesZero\;
		}
		\SelectedSamples $\leftarrow $\SelectBestSamples{\Samples, \NumUpdateSamples}\;
		\For{$i=0$ \KwTo \ProblemSize} {
			\Meani $\leftarrow$ \Meani $+$ \LearningRate $\times$ \Mean{\SelectedSamples, $i$}\;
			\Variancei $\leftarrow$ \Variancei $+$ \LearningRate $\times$ \Variance{\SelectedSamples, $i$}\;
		}
	}
	\Return{\Best}\;
	% end
	\caption{Pseudocode for the Cross-Entropy Method.}
	\label{alg:cross_entropy_method}
\end{algorithm}

% Heuristics: Usage guidelines
% The heuristics element describe the commonsense, best practice, and demonstrated rules for applying and configuring a parameterized algorithm. The heuristics relate to the technical details of the techniques procedure and data structures for general classes of application (neither specific implementations not specific problem instances). The heuristics are described textually, such as a series of guidelines in a bullet-point structure.
\subsection{Heuristics}
% What are the suggested configurations for a technique?
% What are the guidelines for the application of a technique to a problem instance?
\begin{itemize}
	\item The Cross-Entropy Method was adapted for combinatorial optimization problems, although has been applied to continuous function optimization as well as noisy simulation problems.
	\item A alpha ($\alpha$) parameter or learning rate $\in [0.1]$ is typically set high, such as 0.7.
	\item A smoothing function can be used to further control the updates the summaries of the distribution(s) of samples from the problem space. For example, in continuous function optimization a $\beta$ parameter may replace $\alpha$ for updating the standard deviation, calculated at time $t$ as $\beta_{t} = \beta - \beta \times (1-\frac{1}{t})^q$, where $\beta$ is initially set high $\in [0.8, 0.99]$ and $q$ is a small integer $\in [5, 10]$.
\end{itemize}

% Code Listing
% The code description provides a minimal but functional version of the technique implemented with a programming language. The code description must be able to be typed into an appropriate computer, compiled or interpreted as need be, and provide a working execution of the technique. The technique implementation also includes a minimal problem instance to which it is applied, and both the problem and algorithm implementations are complete enough to demonstrate the techniques procedure. The description is presented as a programming source code listing.
\subsection{Code Listing}
% How is a technique implemented as an executable program?
% How is a technique applied to a concrete problem instance?
Listing~\ref{cross_entropy_method} provides an example of the Cross-Entropy Method algorithm implemented in the Ruby Programming Language. 
% problem
The demonstration problem is an instance of a continuous function optimization problem that seeks $\min f(x)$ where $f=\sum_{i=1}^n x_{i}^2$, $-5.0\leq x_i \leq 5.0$ and $n=3$. The optimal solution for this basin function is $(v_0,\ldots,v_{n-1})=0.0$.

% algorithm
The algorithm was implemented based on a description of the Cross-Entropy Method algorithm for continuous function optimization by Rubinstein and Kroese in Chapter 5 and Appendix A of their book on the method \cite{Rubinstein2004}. The algorithm maintains means and standard deviations of the distribution of samples for convenience. The means and standard deviations are initialized based on random positions in the problem space and the bounds of the whole problem space respectively. A smoothing parameter is not used on the standard deviations.

% the listing
\lstinputlisting[firstline=7,language=ruby,caption=Cross-Entropy Method in Ruby, label=cross_entropy_method]{../src/algorithms/probabilistic/cross_entropy_method.rb}

% References: Deeper understanding
% The references element description includes a listing of both primary sources of information about the technique as well as useful introductory sources for novices to gain a deeper understanding of the theory and application of the technique. The description consists of hand-selected reference material including books, peer reviewed conference papers, journal articles, and potentially websites. A bullet-pointed structure is suggested.
\subsection{References}
% What are the primary sources for a technique?
% What are the suggested reference sources for learning more about a technique?

% 
% Primary Sources
% 
\subsubsection{Primary Sources}
% seminal
The Cross-Entropy method was proposed by Rubinstein in 1997 \cite{Rubinstein1997} for use in optimizing discrete event simulation systems. It was later generalized by Rubinstein and proposed as an optimization method for combinatorial function optimization in 1999 \cite{Rubinstein1999}.
% early
This work was further elaborated by Rubinstein providing a detailed treatment on the use of the Cross-Entropy method for combinatorial optimization \cite{Rubinstein2001}.



% 
% Learn More
% 
\subsubsection{Learn More}
% reviews
De~Boer et al.\ provide a detailed presentation of Cross-Entropy method including its application in rare event simulation, its adaptation to combinatorial optimization, and example applications to the max-cut, traveling salesman problem, and a clustering numeric optimization example \cite{DeBoer2005}.
% books
Rubinstein and Kroese provide a thorough presentation of the approach in their book, summarizing the relevant theory and the state of the art \cite{Rubinstein2004}.


\putbib\end{bibunit}
