% The Clever Algorithms Project: http://www.CleverAlgorithms.com
% (c) Copyright 2010 Jason Brownlee. Some Rights Reserved. 
% This work is licensed under a Creative Commons Attribution-Noncommercial-Share Alike 2.5 Australia License.

% This is an algorithm description, see:
% Jason Brownlee. A Template for Standardized Algorithm Descriptions. Technical Report CA-TR-20100107-1, The Clever Algorithms Project http://www.CleverAlgorithms.com, January 2010.

% Name
% The algorithm name defines the canonical name used to refer to the technique, in addition to common aliases, abbreviations, and acronyms. The name is used in terms of the heading and sub-headings of an algorithm description.
\section{Back-propagation} 
\label{sec:backpropagation}
\index{Back-propagation}
\index{Error Back-propagation}

% other names
% What is the canonical name and common aliases for a technique?
% What are the common abbreviations and acronyms for a technique?
\emph{Back-propagation, Backpropagation, Error Back Propagation, Backprop, Delta-rule.}

% Taxonomy: Lineage and locality
% The algorithm taxonomy defines where a techniques fits into the field, both the specific subfields of Computational Intelligence and Biologically Inspired Computation as well as the broader field of Artificial Intelligence. The taxonomy also provides a context for determining the relation- ships between algorithms. The taxonomy may be described in terms of a series of relationship statements or pictorially as a venn diagram or a graph with hierarchical structure.
\subsection{Taxonomy}
% To what fields of study does a technique belong?
The Back-propagation algorithm is a supervised learning method for multi-layer feed-forward networks from the field of Artificial Neural Networks and more broadly Computational Intelligence.
% What are the closely related approaches to a technique?
The name refers to the backward propagation of error during the training of the network. Back-propagation is the basis for many variations and extensions for training multi-layer feed-forward networks not limited to Vogl's Method (Bold Drive), Delta-Bar-Delta, Quickprop, and Rprop.

% Inspiration: Motivating system
% The inspiration describes the specific system or process that provoked the inception of the algorithm. The inspiring system may non-exclusively be natural, biological, physical, or social. The description of the inspiring system may include relevant domain specific theory, observation, nomenclature, and most important must include those salient attributes of the system that are somehow abstractly or conceptually manifest in the technique. The inspiration is described textually with citations and may include diagrams to highlight features and relationships within the inspiring system.
% Optional
\subsection{Inspiration}
% What is the system or process that motivated the development of a technique?
Feed-forward neural networks are inspired by the information processing of one or more neural cells (called a neuron). 
% Which features of the motivating system are relevant to a technique?
A neuron accepts input signals via the dendrites, a chemical process occurs within the cell based on the input signals, and the cell may or may not produce an output signal on its axon. The point where one cell's axon interfaces another cell's dendrite is called the synapse, which may fire if the cell is activated.
% backprop 
The Back-propagation algorithm is a training regime for multi-layer feed forward neural networks and is not directly inspired by the learning processes the biological system.

% Metaphor: Explanation via analogy
% The metaphor is a description of the technique in the context of the inspiring system or a different suitable system. The features of the technique are made apparent through an analogous description of the features of the inspiring system. The explanation through analogy is not expected to be literal scientific truth, rather the method is used as an allegorical communication tool. The inspiring system is not explicitly described, this is the role of the ‘inspiration’ element, which represents a loose dependency for this element. The explanation is textual and uses the nomenclature of the metaphorical system.
% Optional
% \subsection{Metaphor}
% What is the explanation of a technique in the context of the inspiring system?
% What are the functionalities inferred for a technique from the analogous inspiring system?
% A textual description of the algorithm by analogy.

% Strategy: Problem solving plan
% The strategy is an abstract description of the computational model. The strategy describes the information processing actions a technique shall take in order to achieve an objective. The strategy provides a logical separation between a computational realization (procedure) and a analogous system (metaphor). A given problem solving strategy may be realized as one of a number specific algorithms or problem solving systems. The strategy description is textual using information processing and algorithmic terminology.
\subsection{Strategy}
% What is the information processing objective of a technique?
The information processing objective of the technique is to model a given function by modifying internal weightings of input signals to produce an expected output signal.
% What is a techniques plan of action?
The system is trained using a supervised learning method, where the error between the system's output and a known expected output is presented to the system and used to modify its internal state. State is maintained in a set of weightings on the input signals. The weights are used to represent an abstraction of the mapping of input vectors to the output signal for the examples that the system was exposed to during training.
Each layer of the network provides an abstraction of the information processing of the previous layer, allowing the combination of sub-functions and higher order modeling.

% Procedure: Abstract computation
% The algorithmic procedure summarizes the specifics of realizing a strategy as a systemized and parameterized computation. It outlines how the algorithm is organized in terms of the data structures and representations. The procedure may be described in terms of software engineering and computer science artifacts such as Pseudocode, design diagrams, and relevant mathematical equations.
\subsection{Procedure}
% What are the data structures and representations used in a technique?
The Back-propagation algorithm is a method for training the weights in a multi-layer feed-forward neural network. As such, it requires a network structure to be defined of one or more layers where one layer is fully connected to the next layer. A standard network structure is one input layer, one hidden layer, and one output layer. The method is primarily concerned with adapting the weights to the calculated error in the presence of input patterns, and the method is applied backward from the network output layer through to the input layer.

% What is the computational recipe for a technique?
Algorithm~\ref{alg:train} provides a high-level pseudocode for preparing a network using the Back-propagation training method. A weight is initialized for each input plus an additional weight for a fixed bias constant input that is almost always set to 1.0. The activation of a single neuron to a given input pattern is calculated as follows:
\begin{equation}
	activation = \bigg(\sum_{k=1}^{n} w_{k} \times x_{ki}\bigg) + w_{bias} \times 1.0
\end{equation}

where $n$ is the number of weights and inputs, $x_{ki}$ is the $k^{th}$ attribute on the $i^{th}$ input pattern, and $w_{bias}$ is the bias weight. A logistic transfer function (sigmoid) is used to calculate the output for a neuron $\in [0,1]$ and provide nonlinearities between in the input and output signals: $\frac{1}{1+exp(-a)}$, where $a$ represents the neuron activation. 

The weight updates use the delta rule, specifically a changed delta rule where error is backwardly propagated through the network, starting at the output layer and weighted back through the previous layers. The following describes the back-propagation of error and weight updates for a single pattern.

An error signal is calculated for each node and propagated back through the network. For the output nodes this is the sum of the error between the node outputs and the expected outputs: 

\begin{equation}
	es_i = (c_i - o_i) \times td_i
\end{equation}

where $es_i$ is the error signal for the $i^{th}$ node, $c_i$ is the expected output and $o_i$ is the actual output for the $i^{th}$ node. The $td$ term is the derivative of the output of the $i^{th}$ node. If the sigmod transfer function is used, $td_i$ would be $o_i \times (1-o_i)$ For the hidden nodes, the error signal is the sum of the weighted error signals from the next layer.

\begin{equation}
	es_i = \bigg(\sum_{k=1}^n (w_{ik} \times es_k)\bigg) \times td_i
\end{equation}

where $es_i$ is the error signal for the $i^{th}$ node, $w_{ik}$ is the weight between the $i^{th}$ and the $k^{th}$ nodes, and $es_k$ is the error signal of the $k_th$ node.

The error derivatives for each weight are calculated by combining the input to each node and the error signal for the node.

\begin{equation}
	ed_i = \sum_{k=1}^n es_i \times x_k
\end{equation}

where $ed_i$ is the error derivative for the $i^{th}$ node, $es_i$ is the error signal for the $i^{th}$ node and $x_k$ is the input from the $k^{th}$ node in the previous layer. This process include the bias input that has a constant value.

Weights are updated in a direction that reduces the error derivative $ed_i$ (error assigned to the weight), metered by a learning coefficient.

\begin{equation}
	w_i(t+1) = w_i(t) + (ed_k \times learn_{rate})
\end{equation}

where $w_i(t+1)$ is the updated $i^{th}$ weight, $ed_k$ is the error derivative for the $k^{th}$ node and $learn_{rate}$ is an update coefficient parameter.

\begin{algorithm}[ht]
	\SetLine

	% data
	\SetKwData{ProblemSize}{ProblemSize}
	\SetKwData{MaxIterations}{$iterations_{max}$}
	\SetKwData{LearningRate}{$learn_{rate}$}
	\SetKwData{Network}{Network}
	\SetKwData{NetworkWeights}{$Network_{weights}$}
	\SetKwData{InputPatterns}{InputPatterns}
	\SetKwData{Pattern}{$Pattern_i$}
	\SetKwData{Output}{$Output_i$}

	% functions
	\SetKwFunction{ConstructNetworkLayers}{ConstructNetworkLayers}
	\SetKwFunction{SelectInputPattern}{SelectInputPattern}
	\SetKwFunction{InitializeWeights}{InitializeWeights}
	\SetKwFunction{ForwardPropagate}{ForwardPropagate}
	\SetKwFunction{BackwardPropagateError}{BackwardPropagateError}
	\SetKwFunction{UpdateWeights}{UpdateWeights}
	
	% I/O
	\KwIn{\ProblemSize, \InputPatterns, \MaxIterations, \LearningRate}		
	\KwOut{\Network}
  
	% Algorithm
	\Network $\leftarrow$ \ConstructNetworkLayers{}\;
	\NetworkWeights $\leftarrow$ \InitializeWeights{\Network, \ProblemSize}\;
	% loop
	\For{$i=1$ \KwTo \MaxIterations} {
		\Pattern $\leftarrow$ \SelectInputPattern{\InputPatterns}\;
		\Output $\leftarrow$ \ForwardPropagate{\Pattern, \Network}\;
		\BackwardPropagateError{\Pattern, \Output, \Network}\;		
		\UpdateWeights{\Pattern, \Output, \Network, \LearningRate}\;
	}
	\Return{\Network}\;
	% end
	\caption{Pseudocode for Back-propagation.}
	\label{alg:train}
\end{algorithm}

% Heuristics: Usage guidelines
% The heuristics element describe the commonsense, best practice, and demonstrated rules for applying and configuring a parameterized algorithm. The heuristics relate to the technical details of the techniques procedure and data structures for general classes of application (neither specific implementations not specific problem instances). The heuristics are described textually, such as a series of guidelines in a bullet-point structure.
\subsection{Heuristics}
% What are the suggested configurations for a technique?
% What are the guidelines for the application of a technique to a problem instance?
\begin{itemize}
	\item The Back-propagation algorithm can be used to train a multi-layer network to approximate arbitrary non-linear functions and can be used for regression or classification problems.
	\item Input and output values should be normalized such that $x \in [0,1)$.
	\item The weights can be updated in an online manner (after the exposure to each input pattern) or in batch (after a fixed number of patterns have been observed).
	\item Batch updates are expected to be more stable than online updates for some complex problems.
	\item A logistic function (sigmoid) transfer function is commonly used to transfer the activation to a binary output value, although other transfer functions can be used such as the hyperbolic tangent (tanh), Gaussian, and softmax.
	\item It is good practice to expose the system to input patterns in a different random order each enumeration through the input set.
	\item The initial weights are typically small random values, typically $\in [0, 0.5]$.
	\item Typically a small number of layers are used such as 2-4 given that the increase in layers result in an increase in the complexity of the system and the time required to train the weights.
	\item The learning rate can be varied during training, and it is common to introduce a momentum term to limit the rate of change.
	\item The weights of a give network can be initialized with a global optimization method before being refined using the Back-propagation algorithm.
	\item One output node is common for regression problems, where as one output node per class is common for classification problems.
\end{itemize}

% The code description provides a minimal but functional version of the technique implemented with a programming language. The code description must be able to be typed into an appropriate computer, compiled or interpreted as need be, and provide a working execution of the technique. The technique implementation also includes a minimal problem instance to which it is applied, and both the problem and algorithm implementations are complete enough to demonstrate the techniques procedure. The description is presented as a programming source code listing.
\subsection{Code Listing}
% How is a technique implemented as an executable program?
% How is a technique applied to a concrete problem instance?
Listing~\ref{backpropagation} provides an example of the Back-propagation algorithm implemented in the Ruby Programming Language. 
% problem
The problem is the classical XOR boolean problem, where the inputs of the boolean truth table are provided as inputs and the result of the boolean XOR operation is the expected as output. This is a classical problem for Back-Propagation because it was the problem instance referenced by Minsky and Papert in their analysis of the Perceptron highlighting the limitations of their simplified models of neural networks \cite{Minsky1969}.

% algorithm
The algorithm was implemented using an online learning method, meaning the weights are updated after each input pattern is observed. A logistic (sigmoid) transfer function is used to convert the activation into an output signal. Random samples are taken from the domain to train the weights, and similarly, random samples are drawn from the domain to demonstrate what the network has learned. 

A three layer network is demonstrated with 2 nodes in the input layer (two inputs), 2 nodes in the hidden layer and 1 node in the output layer, which is sufficient for the chosen problem. A bias weight is used on each neuron for stability with a constant input of 1.0. The learning process is separated into four steps: forward propagation, backward propagation of error, calculation of error derivatives (assigning blame to the weights) and the weight update. This separation facilities easy extensions such as adding a momentum term and/or weight decay to the update process.

Extensions to this implementation should cosider changes such as batch updates (rather than online learning) and the use of a weight decay and momentium (dampen large changes) coefficients. 

% the listing
\lstinputlisting[firstline=7,language=ruby,caption=Back-propagation in Ruby, label=backpropagation]{../src/algorithms/neural/backpropagation.rb}

% References: Deeper understanding
% The references element description includes a listing of both primary sources of information about the technique as well as useful introductory sources for novices to gain a deeper understanding of the theory and application of the technique. The description consists of hand-selected reference material including books, peer reviewed conference papers, journal articles, and potentially websites. A bullet-pointed structure is suggested.
\subsection{References}
% What are the primary sources for a technique?
% What are the suggested reference sources for learning more about a technique?

% 
% Primary Sources
% 
\subsubsection{Primary Sources}
% seminal
The backward propagation of error method is credited to Bryson and Ho in \cite{Bryson1969}. It was applied to the training of multi-layer networks and called back-propagation by Rumelhart, Hinton and Williams in 1986 \cite{Rumelhart1986b, Rumelhart1986c}. 
% early
This effort and the collection of studies edited by Rumelhart and McClelland that helped to define the field of Artificial Neural Networks in the late 1980's \cite{Rumelhart1986, Rumelhart1986a}.


% 
% Learn More
% 
\subsubsection{Learn More}
% reviews
% books
A seminal book on the approach was ``Backpropagation: theory, architectures, and applications'' by Chauvin and Rumelhart that provided an excellent introduction (chapter 1) but also a collection of studies applying and extending the approach \cite{Chauvin1995}.
Reed and Marks provide an excellent treatment of feed-forward neural networks called ``Neural Smithing'' that includes chapters dedicated to Back-propagation, the configuration of its parameters, error surface and speed improvements \cite{Reed1999}.


