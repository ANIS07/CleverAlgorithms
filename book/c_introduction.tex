% The Clever Algorithms Project: http://www.CleverAlgorithms.com
% (c) Copyright 2010 Jason Brownlee. All Rights Reserved. 
% This work is licensed under a Creative Commons Attribution-Noncommercial-Share Alike 2.5 Australia License.

% This is a chapter


% The argument for the book
% The information the user requires to read and understand the book
\chapter{Introduction}
\label{chap:intro}

hello and welcome statement.

% 
% What is AI
% 
\section{What is AI}
\label{intro:sec:what_is_ai}
What are all the fields and subfields we need to know about?

\subsection{Artificial Intelligence}
Messy and Neat AI

\subsection{Computational Intelligence}
Fuzzy Logic, Artificial Neural Networks, and Evolutionary Computation

\subsection{Natural Computation}
Biologically Inspired Computation, Computation with Biology, Computational Biology 

\subsection{Metaheuristics}
Heuristics for driving heuristics

\subsection{Machine Learning}
Statistical methods for learning

% 
% Algorithms
% 
\section{Algorithms}
\label{intro:sec:algorithms}
What do we need to know about this general class of algorithms.

\subsection{Black Box Methods}
They make little or few assumptions about the problem domain

\subsection{Randomness}
The are stochastic processes.

\subsection{State Space}
The typically require the problem to be phrased as a search space which is traversed and sampled.
We care about the size of moves, the patters of sampling and re-sampling, the number of samples managed.

\subsection{Induction}
The typically learn using indiction (trial and error)

\subsection{Problems}
What types of computational problems are we solving with these algorithms?
Give example classes for each, give canonical instances for each (all covered in this book)

\subsubsection{Function Optimization}
Generate a set of parameters (continuous) or something like a permutation (combinatorial).

\subsubsection{Function Approximation}
Generate a representation that produces outputs in the presence of inputs.


% 
% Book Organization
% 
\section{Book Organization}
\label{intro:sec:organization}
How the book is structured?

\subsection{Background}
Provide you with enough information to understand the presented algorithms.

\subsection{Algorithms}
The presentation of all the algorithms, partitioned by a taxonomy. About the kingdoms, about the standardized algorithm description template.

\subsection{Extensions}
Things to think about once you have mastered a number of a algorithms.


% 
% How to Read this Book
% 
\section{How to Read this Book}
\label{intro:sec:how_to_read}
Who is this book for and how should it be read?

\subsection{Research Scientists}
ways in which scientists may read this material

\subsection{Developers}
ways in which programmers and developers may read this material

\subsection{Students}
ways in which students may read this material

\subsection{Interested Amateurs}
ways in which amateurs may read this material

