% The Clever Algorithms Project: http://www.CleverAlgorithms.com
% (c) Copyright 2010 Jason Brownlee. Some Rights Reserved. 
% This work is licensed under a Creative Commons Attribution-Noncommercial-Share Alike 2.5 Australia License.

% This is an algorithm description, see:
% Jason Brownlee. A Template for Standardized Algorithm Descriptions. Technical Report CA-TR-20100107-1, The Clever Algorithms Project http://www.CleverAlgorithms.com, January 2010.

% Name
% The algorithm name defines the canonical name used to refer to the technique, in addition to common aliases, abbreviations, and acronyms. The name is used in terms of the heading and sub-headings of an algorithm description.
\section{Ant Colony System} 
\label{sec:ant_colony_system}
\index{Ant Colony System}
\index{Ant-Q}

% other names
% What is the canonical name and common aliases for a technique?
% What are the common abbreviations and acronyms for a technique?
\emph{Ant Colony System, ACS, Ant-Q.}

% Taxonomy: Lineage and locality
% The algorithm taxonomy defines where a techniques fits into the field, both the specific subfields of Computational Intelligence and Biologically Inspired Computation as well as the broader field of Artificial Intelligence. The taxonomy also provides a context for determining the relation- ships between algorithms. The taxonomy may be described in terms of a series of relationship statements or pictorially as a venn diagram or a graph with hierarchical structure.
\subsection{Taxonomy}
% To what fields of study does a technique belong?
The Ant Colony System algorithm is an example of an Ant Colony Optimization method from the field of Swarm Intelligence, Metaheuristics and Computational Intelligence.
% What are the closely related approaches to a technique?
Ant Colony System is an extension to the Ant System algorithm and is related to other Ant Colony Optimization methods such as Elite Ant System, Rank-based Ant System, and Ant Colony System.

% Inspiration: Motivating system
% The inspiration describes the specific system or process that provoked the inception of the algorithm. The inspiring system may non-exclusively be natural, biological, physical, or social. The description of the inspiring system may include relevant domain specific theory, observation, nomenclature, and most important must include those salient attributes of the system that are somehow abstractly or conceptually manifest in the technique. The inspiration is described textually with citations and may include diagrams to highlight features and relationships within the inspiring system.
% Optional
\subsection{Inspiration}
% What is the system or process that motivated the development of a technique?
The Ant Colony System algorithm is inspired by the foraging behavior of ants, specifically the pheromone communication between ants regarding a good path between the colony and a food source in an environment. This mechanism is called stigmergy.
% Which features of the motivating system are relevant to a technique?

% Metaphor: Explanation via analogy
% The metaphor is a description of the technique in the context of the inspiring system or a different suitable system. The features of the technique are made apparent through an analogous description of the features of the inspiring system. The explanation through analogy is not expected to be literal scientific truth, rather the method is used as an allegorical communication tool. The inspiring system is not explicitly described, this is the role of the ‘inspiration’ element, which represents a loose dependency for this element. The explanation is textual and uses the nomenclature of the metaphorical system.
% Optional
\subsection{Metaphor}
% What is the explanation of a technique in the context of the inspiring system?
% What are the functionalities inferred for a technique from the analogous inspiring system?
Ants initially wander randomly around their environment. Once food is located an ant will begin laying down pheromone in the environment. Numerous trips between the food and the colony are performed and if the same route is followed that leads to food then additional pheromone is laid down. Pheromone decays in the environment, so that older paths are less likely to be followed. Other ants may discover the same path to the food and in turn may follow it and also lay down pheromone. A positive feedback process routes more and more ants to productive paths that are in turn further refined through use.

% Strategy: Problem solving plan
% The strategy is an abstract description of the computational model. The strategy describes the information processing actions a technique shall take in order to achieve an objective. The strategy provides a logical separation between a computational realization (procedure) and a analogous system (metaphor). A given problem solving strategy may be realized as one of a number specific algorithms or problem solving systems. The strategy description is textual using information processing and algorithmic terminology.
\subsection{Strategy}
% What is the information processing objective of a technique?
The objective of the strategy is to exploit historic and heuristic information to construct candidate solutions and fold the information learned from constructing solutions into the history.
% What is a techniques plan of action?
Solutions are constructed one discrete piece at a time in a probabilistic step-wise manner. The probability of selecting a component is determined by the heuristic contribution of the component to the overall cost of the solution and the quality of solutions from which the component has historically known to have been included. History is updated proportional to the quality of candidate solutions and is uniformly deceased ensuring the most recent and useful information is retained.

% Procedure: Abstract computation
% The algorithmic procedure summarizes the specifics of realizing a strategy as a systemized and parameterized computation. It outlines how the algorithm is organized in terms of the data structures and representations. The procedure may be described in terms of software engineering and computer science artifacts such as pseudo code, design diagrams, and relevant mathematical equations.
\subsection{Procedure}
% What is the computational recipe for a technique?
% What are the data structures and representations used in a technique?
Algorithm~\ref{alg:acs} provides a pseudo-code listing of the main Ant Colony System algorithm for minimizing a cost function. 
The probabilistic step-wise construction of solution makes use of both history (pheromone) and problem-specific heuristic information to incrementally construction a solution piece-by-piece. Each component can only be selected if it has not already been chosen (for most combinatorial problems), and for those components that can  be selected from given the current component $i$, their probability for selection is defined as:

\begin{equation}
P_{i,j} \leftarrow \frac{\tau_{i,j}^{\alpha} \times \eta_{i,j}^{\beta}}{\sum_{k=1}^c \tau_{i,k}^{\alpha} \times \eta_{i,k}^{\beta}}
\end{equation}

where $\tau_{i,j}$ is the maximizing contribution to the overall score of selecting the component (such as $\frac{1.0}{distance_{i,j}}$ for the Traveling Salesman Problem), $\alpha$ is the heuristic coefficient (commonly fixed at 1.0), $\eta_{i,j}$ is the pheromone value for the component, $\beta$ is the history coefficient, and $c$ is the set of usable components. A greediness factor ($q0$) is used to influences when to use the above probabilistic component selection and when to greedily select the best possible component. 

A local pheromone update is performed for each solution that is constructed to dissuade following solutions to use the same components in the same order, as follows:

\begin{equation}
\tau_{i,j} \leftarrow (1-\sigma) \times \tau_{i,j} + \sigma \times \tau_{i,j}^{0}
\end{equation}

where $\tau_{i,j}$ represents the pheromone for the component (graph edge) ($i,j$), $\sigma$ is the local pheromone factor, and $\tau_{i,j}^{0}$ is the initial pheromone value.

At the end of each iteration, the pheromone is updated and decayed using the best candidate solution found thus far (or the best candidate solution found for the iteration), as follows:

\begin{equation}
\tau_{i,j} \leftarrow (1-\rho) \times \tau_{i,j} + \Delta_{i,j}^{best}
\end{equation}

where $\tau_{i,j}$ represents the pheromone for the component (graph edge) ($i,j$), $\rho$ is the decay factor, and $\Delta_{i,j}^{best} $ is the maximizing solution cost for the best solution found so far. 



\begin{algorithm}[ht]
	\SetLine  

	% data
	\SetKwData{Best}{$P_{best}$}
	\SetKwData{BestCost}{$Pbest_{cost}$}
	\SetKwData{ProblemSize}{ProblemSize}
	\SetKwData{DecayFactor}{$\rho$}
	\SetKwData{LocalPheromone}{$\sigma$}
	\SetKwData{HeuristicContribution}{$\beta$}
	\SetKwData{GreedinessFactor}{$q0$}
	\SetKwData{NumAnts}{$m$}
	\SetKwData{PopulationSize}{$Population_{size}$}
	\SetKwData{Pheromone}{Pheromone}
	\SetKwData{Candidates}{Candidates}
	\SetKwData{Solution}{$S_{i}$}
	\SetKwData{SolutionCost}{$Si_{cost}$}
	\SetKwData{HeuristicSolution}{$S_{h}$}
	\SetKwData{HeuristicSolutionCost}{$Sh_{cost}$}
	\SetKwData{InitialPheromone}{$Pheromone_{init}$}
	
	% functions
	\SetKwFunction{Cost}{Cost}
	\SetKwFunction{StopCondition}{StopCondition}
	\SetKwFunction{InitializePheromone}{InitializePheromone}
	\SetKwFunction{ConstructSolution}{ConstructSolution}
	\SetKwFunction{LocalUpdateAndDecayPheromone}{LocalUpdateAndDecayPheromone}
	\SetKwFunction{GlobalUpdateAndDecayPheromone}{GlobalUpdateAndDecayPheromone}
	\SetKwFunction{CreateHeuristicSolution}{CreateHeuristicSolution}
	
	% I/O
	\KwIn{\ProblemSize, \PopulationSize, \NumAnts, \DecayFactor, \HeuristicContribution, \LocalPheromone, \GreedinessFactor}		
	\KwOut{\Best}
  % Algorithm
	\Best $\leftarrow$ \CreateHeuristicSolution{\ProblemSize}\;
	\BestCost $\leftarrow$ \Cost{\HeuristicSolution}\;
	\InitialPheromone $\leftarrow$ $\frac{1.0}{\ProblemSize \times \BestCost}$\;
	\Pheromone $\leftarrow$ \InitializePheromone{\InitialPheromone}\;
	% loop
	\While{$\neg$\StopCondition{}} {		
		\For{$i=1$ $\KwTo$ \NumAnts} {
			\Solution $\leftarrow$ \ConstructSolution{\Pheromone, \ProblemSize, \HeuristicContribution, \GreedinessFactor}\;
			\SolutionCost $\leftarrow$ \Cost{\Solution}\;
			\If{\SolutionCost $\leq$ \BestCost} {
				\BestCost $\leftarrow$ \SolutionCost\;
				\Best $\leftarrow$ \Solution\;
			}
			\LocalUpdateAndDecayPheromone{\Pheromone, \Solution, \SolutionCost, \LocalPheromone}\;
		}
		\GlobalUpdateAndDecayPheromone{\Pheromone, \Best, \BestCost, \DecayFactor}\;
	}
	\Return{\Best}\;
	% end
	\caption{Pseudo Code for the Ant Colony System algorithm.}
	\label{alg:acs}
\end{algorithm}

% Heuristics: Usage guidelines
% The heuristics element describe the commonsense, best practice, and demonstrated rules for applying and configuring a parameterized algorithm. The heuristics relate to the technical details of the techniques procedure and data structures for general classes of application (neither specific implementations not specific problem instances). The heuristics are described textually, such as a series of guidelines in a bullet-point structure.
\subsection{Heuristics}
% What are the suggested configurations for a technique?
% What are the guidelines for the application of a technique to a problem instance?
\begin{itemize}
	\item The Ant Colony System algorithm was designed for use with combinatorial problems such as the TSP, knapsack problem, quadratic assignment problems, graph coloring problems and many others.
	\item The local pheromone (history) coefficient ($\sigma$) controls the amount of contribution history plays in a components probability of selection and is commonly set to 0.1.
	\item The heuristic coefficient ($\beta$) controls the amount of contribution problem-specific heuristic information plays in a components probability of selection and is commonly between 2 and 5, such as 2.5.
	\item The decay factor ($\rho$) controls the rate at which historic information is lost and is commonly set to 0.1.
	\item The greediness factor ($q0$) is commonly set to 0.9.
	\item The total number of ants ($m$) is commonly set low, such as 10.
\end{itemize}

% Code Listing
% The code description provides a minimal but functional version of the technique implemented with a programming language. The code description must be able to be typed into an appropriate computer, compiled or interpreted as need be, and provide a working execution of the technique. The technique implementation also includes a minimal problem instance to which it is applied, and both the problem and algorithm implementations are complete enough to demonstrate the techniques procedure. The description is presented as a programming source code listing.
\subsection{Code Listing}
% How is a technique implemented as an executable program?
% How is a technique applied to a concrete problem instance?
Listing~\ref{ant_colony_system} provides an example of the Ant Colony System algorithm implemented in the Ruby Programming Language. 
% problem
The algorithm is applied to the Berlin52 instance of the Traveling Salesman Problem (TSP), taken from the TSPLIB. The problem seeks a permutation of the order to visit cities (called a tour) that minimized the total distance traveled. The optimal tour distance for Berlin52 instance is 7542 units.
% algorithm
Some extensions to the algorithm implementation for speed improvements may consider pre-calculating a distance matrix for all the cities in the problem, and pre-computing a probability matrix for choices during the probabilistic step-wise construction of tours. 

% the listing
\lstinputlisting[firstline=7,language=ruby,caption=Ant Colony System algorithm in the Ruby Programming Language, label=ant_colony_system]{../src/algorithms/swarm/ant_colony_system.rb}

% References: Deeper understanding
% The references element description includes a listing of both primary sources of information about the technique as well as useful introductory sources for novices to gain a deeper understanding of the theory and application of the technique. The description consists of hand-selected reference material including books, peer reviewed conference papers, journal articles, and potentially websites. A bullet-pointed structure is suggested.
\subsection{References}
% What are the primary sources for a technique?
% What are the suggested reference sources for learning more about a technique?

% 
% Primary Sources
% 
\subsubsection{Primary Sources}
The algorithm was initially investigated by Dorigo and Gambardella under the name Ant-Q \cite{Dorigo1996a, Gambardella1995}.
% seminal
It was renamed Ant Colony System and further investigated first in a technical report by Dorigo and Gambardella \cite{Dorigo1997a}, and later published \cite{Dorigo1997}.

% 
% Learn More
% 
\subsubsection{Learn More}
% reviews
% books
The seminal book on Ant Colony Optimization in general with a detailed treatment of Ant Colony System is ``Ant colony optimization'' by Dorigo and St\"utzle \cite{Dorigo2004}. An earlier book ``Swarm intelligence: from natural to artificial systems'' by Bonabeau, Dorigo, and Theraulaz also provides an introduction to Swarm Intelligence with a detailed treatment of Ant Colony System \cite{Bonabeau1999}.


