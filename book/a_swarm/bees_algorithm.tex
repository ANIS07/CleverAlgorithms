% The Clever Algorithms Project: http://www.CleverAlgorithms.com
% (c) Copyright 2010 Jason Brownlee. Some Rights Reserved. 
% This work is licensed under a Creative Commons Attribution-Noncommercial-Share Alike 2.5 Australia License.

% This is an algorithm description, see:
% Jason Brownlee. A Template for Standardized Algorithm Descriptions. Technical Report CA-TR-20100107-1, The Clever Algorithms Project http://www.CleverAlgorithms.com, January 2010.

% Name
% The algorithm name defines the canonical name used to refer to the technique, in addition to common aliases, abbreviations, and acronyms. The name is used in terms of the heading and sub-headings of an algorithm description.
\section{Bees Algorithm} 
\label{sec:bees_algorithm}
\index{Bees Algorithm}

% other names
% What is the canonical name and common aliases for a technique?
% What are the common abbreviations and acronyms for a technique?
\emph{Bees Algorithm, BA.}

% Taxonomy: Lineage and locality
% The algorithm taxonomy defines where a techniques fits into the field, both the specific subfields of Computational Intelligence and Biologically Inspired Computation as well as the broader field of Artificial Intelligence. The taxonomy also provides a context for determining the relation- ships between algorithms. The taxonomy may be described in terms of a series of relationship statements or pictorially as a venn diagram or a graph with hierarchical structure.
\subsection{Taxonomy}
% To what fields of study does a technique belong?
The Bees Algorithm beings to Bee Inspired Algorithms and the field of Swarm Intelligence, and more broadly the fields of Computational Intelligence and Metaheuristics.
% What are the closely related approaches to a technique?
The Bees Algorithm is related to other Bee Inspired Algorithms, such as Bee Colony Optimization, and other Swarm Intelligence algorithms such as Ant Colony Optimization and Particle Swarm Optimization.

% Inspiration: Motivating system
% The inspiration describes the specific system or process that provoked the inception of the algorithm. The inspiring system may non-exclusively be natural, biological, physical, or social. The description of the inspiring system may include relevant domain specific theory, observation, nomenclature, and most important must include those salient attributes of the system that are somehow abstractly or conceptually manifest in the technique. The inspiration is described textually with citations and may include diagrams to highlight features and relationships within the inspiring system.
% Optional
\subsection{Inspiration}
% What is the system or process that motivated the development of a technique?
The Bees Algorithm is inspired by the foraging behavior of honey bees.
% Which features of the motivating system are relevant to a technique?
Honey bees collect nectar from vast areas around their hive (more than 10 kilometers). The have been observed to send bees to collect nectar from patches relative to the amount of food available at each flower patch.
Finally, bees communicate with each other via a waggle dance that informs other bees in the hive as to the direction, distance, and quality rating of food sources.

% Metaphor: Explanation via analogy
% The metaphor is a description of the technique in the context of the inspiring system or a different suitable system. The features of the technique are made apparent through an analogous description of the features of the inspiring system. The explanation through analogy is not expected to be literal scientific truth, rather the method is used as an allegorical communication tool. The inspiring system is not explicitly described, this is the role of the ‘inspiration’ element, which represents a loose dependency for this element. The explanation is textual and uses the nomenclature of the metaphorical system.
% Optional
\subsection{Metaphor}
% What is the explanation of a technique in the context of the inspiring system?
% What are the functionalities inferred for a technique from the analogous inspiring system?
Honey bees collect nectar from flower patches as a food source for the hive. The hive sends out scout's that locate patches of flowers, then return to the hive and inform other bees as to the fitness and location of a food source via a waggle dance. The scout returns to the flower patch with follower bees. A small number of scouts continue to search for new patches, while bees returning from flower patches continue to communicate the quality of the patch.

% Strategy: Problem solving plan
% The strategy is an abstract description of the computational model. The strategy describes the information processing actions a technique shall take in order to achieve an objective. The strategy provides a logical separation between a computational realization (procedure) and a analogous system (metaphor). A given problem solving strategy may be realized as one of a number specific algorithms or problem solving systems. The strategy description is textual using information processing and algorithmic terminology.
\subsection{Strategy}
% What is the information processing objective of a technique?
The information processing objective of the algorithm is to locate and explore good sites within a problem search space.
% What is a techniques plan of action?
Scouts are sent out to randomly sample the problem space and locate good sites. The good sites are exploited via the application of a local search, where a smaller number of good sites are explored more than the others. Good sites are continually explored, although many scouts are sent out each iteration always in search of additional good sites.

% Procedure: Abstract computation
% The algorithmic procedure summarizes the specifics of realizing a strategy as a systemized and parameterized computation. It outlines how the algorithm is organized in terms of the data structures and representations. The procedure may be described in terms of software engineering and computer science artifacts such as pseudo code, design diagrams, and relevant mathematical equations.
\subsection{Procedure}
% What is the computational recipe for a technique?
% What are the data structures and representations used in a technique?
Algorithm~\ref{alg:bees_algorithm} provides a pseudo-code listing of the Bees Algorithm for minimizing a cost function. 

\begin{algorithm}[ht]
	\SetLine

	% params
	\SetKwData{NumBees}{$Bees_{num}$}
	\SetKwData{NumSites}{$Sites_{num}$}
	\SetKwData{NumEliteSites}{$EliteSites_{num}$}
	\SetKwData{InitialPatchSize}{$PatchSize_{init}$}
	\SetKwData{NumEliteBees}{$EliteBees_{num}$}
	\SetKwData{NumOtherBees}{$OtherBees_{num}$}
	\SetKwData{ProblemSize}{$Problem_{size}$}
	% data
	\SetKwData{Neighborhood}{Neighborhood}
	\SetKwData{Population}{Population}
	\SetKwData{Best}{$Bee_{best}$}
	\SetKwData{NextGeneration}{NextGeneration}
	\SetKwData{BestSites}{$Sites_{best}$}
	\SetKwData{PatchDecreaseFactor}{$PatchDecrease_{factor}$}
	\SetKwData{Site}{$Site_{i}$}
	\SetKwData{NumRecruitedBees}{$RecruitedBees_{num}$}
	\SetKwData{PatchSize}{$Patch_{size}$}
	\SetKwData{RemainingBees}{$RemainingBees_{num}$}
	
	% functions
	\SetKwFunction{InitializePopulation}{InitializePopulation}  
	\SetKwFunction{StopCondition}{StopCondition} 
	\SetKwFunction{GetBestSolution}{GetBestSolution} 
	\SetKwFunction{SelectBestSites}{SelectBestSites}
	\SetKwFunction{CreateNeighbourhoodBee}{CreateNeighbourhoodBee}
	\SetKwFunction{CreateRandomBee}{CreateRandomBee}
	\SetKwFunction{EvaluatePopulation}{EvaluatePopulation}
  
	% I/O
	\KwIn{\ProblemSize, \NumBees, \NumSites, \NumEliteSites, \InitialPatchSize, \NumEliteBees, \NumOtherBees}		
	\KwOut{\Best}
  % Algorithm

	\Population $\leftarrow$ \InitializePopulation{\NumBees, \ProblemSize}\;

	\While{$\neg$\StopCondition{}} {
		\EvaluatePopulation{\Population}\;
		\Best $\leftarrow$ \GetBestSolution{\Population}\;
		\NextGeneration $\leftarrow 0$\;		
		\PatchSize $\leftarrow$ ( \InitialPatchSize $\times$ \PatchDecreaseFactor )\;
		\BestSites $\leftarrow$ \SelectBestSites{\Population, \NumSites}\;
		
		\ForEach{\Site $\in$ \BestSites} {
				\NumRecruitedBees $\leftarrow$ $0$\;
				\eIf{$i <$ \NumEliteSites} {
					\NumRecruitedBees $\leftarrow$ \NumEliteBees\;
				} {
					\NumRecruitedBees $\leftarrow$ \NumOtherBees\;
				}
				\Neighborhood $\leftarrow$ $0$\;
				\For{$j$ \KwTo \NumRecruitedBees} {
					\Neighborhood $\leftarrow$ \CreateNeighbourhoodBee{\Site, \PatchSize}\;
				}
				\NextGeneration $\leftarrow$ \GetBestSolution{\Neighborhood}\;
			}
		
		\RemainingBees $\leftarrow$ (\NumBees - \NumSites)\;
		\For{$j$ \KwTo \RemainingBees} {
			\NextGeneration $\leftarrow$ \CreateRandomBee{}\;
		}
		\Population $\leftarrow$ \NextGeneration\;
	}
	\Return{\Best}\;
	% end
	\caption{Pseudo Code for the Bees Algorithm.}
	\label{alg:bees_algorithm}
\end{algorithm}

% Heuristics: Usage guidelines
% The heuristics element describe the commonsense, best practice, and demonstrated rules for applying and configuring a parameterized algorithm. The heuristics relate to the technical details of the techniques procedure and data structures for general classes of application (neither specific implementations not specific problem instances). The heuristics are described textually, such as a series of guidelines in a bullet-point structure.
\subsection{Heuristics}
% What are the suggested configurations for a technique?
% What are the guidelines for the application of a technique to a problem instance?
\begin{itemize}
	\item The Bees Algorithm was developed to be used with continuous and combinatorial function optimization problems.
	\item The $Patch_{size}$ variable is used as the neighborhood size. For example, in a continuous function optimization problem, each dimension of a site would be sampled as $x_i \pm (rand() \times Patch_{size})$.
	\item The $Patch_{size}$ variable is decrease each iteration, typically by a constant amount (such as 0.95).
	\item The number of elite sites ($EliteSites_{num}$) must be $<$ the number of sites ($Sites_{num}$), and the number of elite bees ($EliteBees_{num}$) is traditionally $<$ the number of other bees ($OtherBees_{num}$).
\end{itemize}

% Code Listing
% The code description provides a minimal but functional version of the technique implemented with a programming language. The code description must be able to be typed into an appropriate computer, compiled or interpreted as need be, and provide a working execution of the technique. The technique implementation also includes a minimal problem instance to which it is applied, and both the problem and algorithm implementations are complete enough to demonstrate the techniques procedure. The description is presented as a programming source code listing.
\subsection{Code Listing}
% How is a technique implemented as an executable program?
% How is a technique applied to a concrete problem instance?
Listing~\ref{bees_algorithm} provides an example of the Bees Algorithm implemented in the Ruby Programming Language. 
% problem
The demonstration problem is an instance of a continuous function optimization that seeks $min f(x)$ where $f=\sum_{i=1}^n x_{i}^2$, $-5.0\leq x_i \leq 5.0$ and $n=3$. The optimal solution for this basin function is $(v_0,\ldots,v_{n-1})=0.0$.
% algorithm
The algorithm is an implementation of the Bees Algorithm as described in the seminal paper \cite{Pham2006}. A fixed patch size decrease factor of 0.95 was applied each iteration.

% the listing
\lstinputlisting[firstline=7,language=ruby,caption=Bees Algorithm in the Ruby Programming Language, label=bees_algorithm]{../src/algorithms/swarm/bees_algorithm.rb}

% References: Deeper understanding
% The references element description includes a listing of both primary sources of information about the technique as well as useful introductory sources for novices to gain a deeper understanding of the theory and application of the technique. The description consists of hand-selected reference material including books, peer reviewed conference papers, journal articles, and potentially websites. A bullet-pointed structure is suggested.
\subsection{References}
% What are the primary sources for a technique?
% What are the suggested reference sources for learning more about a technique?

% 
% Primary Sources
% 
\subsubsection{Primary Sources}
% seminal
The Bees Algorithm was proposed by Pham, et al. in a technical report in 2005 \cite{Pham2005}, and later published \cite{Pham2006}. In this work, the algorithm was applied to standard instances of continuous function optimization problems.
% early

% 
% Learn More
% 
\subsubsection{Learn More}
% reviews
The majority of the work on the algorithm has concerned its application to various problem domains.
The following is a selection of popular application papers: the optimization of linear antenna arrays by Guney and Onay \cite{Guney2007}, the optimization of codebook vectors in the Learning Vector Quantization algorithm for classification by Pham, et al. \cite{Pham2006a}, optimization of neural networks for classification by Pham, et al. \cite{Pham2006b}, and the optimization of clustering methods by Pham, et al. \cite{Pham2007}.
% books


