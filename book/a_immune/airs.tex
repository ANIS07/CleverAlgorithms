% The Clever Algorithms Project: http://www.CleverAlgorithms.com
% (c) Copyright 2010 Jason Brownlee. Some Rights Reserved. 
% This work is licensed under a Creative Commons Attribution-Noncommercial-Share Alike 2.5 Australia License.

% This is an algorithm description, see:
% Jason Brownlee. A Template for Standardized Algorithm Descriptions. Technical Report CA-TR-20100107-1, The Clever Algorithms Project http://www.CleverAlgorithms.com, January 2010.

% Name
% The algorithm name defines the canonical name used to refer to the technique, in addition to common aliases, abbreviations, and acronyms. The name is used in terms of the heading and sub-headings of an algorithm description.
\section{Artificial Immune Recognition System} 
\label{sec:airs}
\index{Artificial Immune Recognition System}
\index{AIRS}

% other names
% What is the canonical name and common aliases for a technique?
% What are the common abbreviations and acronyms for a technique?
\emph{Artificial Immune Recognition System, AIRS.}

% Taxonomy: Lineage and locality
% The algorithm taxonomy defines where a techniques fits into the field, both the specific subfields of Computational Intelligence and Biologically Inspired Computation as well as the broader field of Artificial Intelligence. The taxonomy also provides a context for determining the relation- ships between algorithms. The taxonomy may be described in terms of a series of relationship statements or pictorially as a venn diagram or a graph with hierarchical structure.
\subsection{Taxonomy}
% To what fields of study does a technique belong?
The Artificial Immune Recognition System belongs to the field of Artificial Immune Systems, and more broadly to the field of Computational Intelligence.
% What are the closely related approaches to a technique?
It is was extended early to the canonical version called the  Artificial Immune Recognition System 2 (AIRS2) and provides the basis for extensions such as the Parallel Artificial Immune Recognition System \cite{Watkins2004}.
It is related to other Artificial Immune System algorithms such as the Dendritic Cell Algorithm, the Clonal Selection Algorithm, and the Negative Selection Algorithm.

% Inspiration: Motivating system
% The inspiration describes the specific system or process that provoked the inception of the algorithm. The inspiring system may non-exclusively be natural, biological, physical, or social. The description of the inspiring system may include relevant domain specific theory, observation, nomenclature, and most important must include those salient attributes of the system that are somehow abstractly or conceptually manifest in the technique. The inspiration is described textually with citations and may include diagrams to highlight features and relationships within the inspiring system.
% Optional
\subsection{Inspiration}
% What is the system or process that motivated the development of a technique?
The Artificial Immune Recognition System is inspired by the Clonal Selection theory of acquired immunity.
% Which features of the motivating system are relevant to a technique?
The clonal selection theory credited to Burnet was proposed to account for the behavior and capabilities of antibodies in the acquired immune system \cite{Burnet1957, Burnet1959}. Inspired itself by the principles of Darwinian natural selection theory of evolution, the theory proposes that antigens select-for lymphocytes (both B and T-cells). When a lymphocyte is selected and binds to an antigenic determinant, the cell proliferates making many thousands more copies of itself and differentiates into different cell types (plasma and memory cells). Plasma cells have a short lifespan and produce vast quantities of antibody molecules, whereas memory cells live for an extended period in the host anticipating future recognition of the same determinant. The important feature of the theory is that when a cell is selected and proliferates, it is subjected to small copying errors (changes to the genome called somatic hypermutation) that change the shape of the expressed receptors. It also affects the  subsequent determinant recognition capabilities of both the antibodies bound to the lymphocytes cells surface, and the antibodies that plasma cells produce.

% Metaphor: Explanation via analogy
% The metaphor is a description of the technique in the context of the inspiring system or a different suitable system. The features of the technique are made apparent through an analogous description of the features of the inspiring system. The explanation through analogy is not expected to be literal scientific truth, rather the method is used as an allegorical communication tool. The inspiring system is not explicitly described, this is the role of the ‘inspiration’ element, which represents a loose dependency for this element. The explanation is textual and uses the nomenclature of the metaphorical system.
% Optional
\subsection{Metaphor}
% What is the explanation of a technique in the context of the inspiring system?
% What are the functionalities inferred for a technique from the analogous inspiring system?
The theory suggests that starting with an initial repertoire of general immune cells, the system is able to change itself (the compositions and densities of cells and their receptors) in response to experience with the environment. Through a blind process of selection and accumulated variation on the large scale of many billions of cells, the acquired immune system is capable of acquiring the necessary information to protect the host organism from the specific pathogenic dangers of the environment. It also suggests that the system must anticipate (guess) at the pathogen to which it will be exposed, and requires exposure to pathogen that may harm the host before it can acquire the necessary information to provide a defense.

% Strategy: Problem solving plan
% The strategy is an abstract description of the computational model. The strategy describes the information processing actions a technique shall take in order to achieve an objective. The strategy provides a logical separation between a computational realization (procedure) and a analogous system (metaphor). A given problem solving strategy may be realized as one of a number specific algorithms or problem solving systems. The strategy description is textual using information processing and algorithmic terminology.
\subsection{Strategy}
% What is the information processing objective of a technique?
The information processing objective of the technique is to prepare a set of real-valued vectors to classify patterns. 
% What is a techniques plan of action?
The Artificial Immune Recognition System maintains a pool of memory cells that are prepared by exposing the system to a single iteration of the training data. Candidate memory cells are prepared when the memory cells are insufficiently stimulated for a given input pattern. A process of cloning and mutation of cells occurs for the most stimulated memory cell. The clones compete with each other for entry into the memory pool based on stimulation and on the amount of resources each cell is using. This concept of resources comes from prior work on Artificial Immune Networks, where a single cell (an Artificial Recognition Ball or ARB) represents a set of similar cells. Here, a cell's resources are a function of its stimulation to a given input pattern and the number of clones it may create.

% Procedure: Abstract computation
% The algorithmic procedure summarizes the specifics of realizing a strategy as a systemized and parameterized computation. It outlines how the algorithm is organized in terms of the data structures and representations. The procedure may be described in terms of software engineering and computer science artifacts such as pseudo code, design diagrams, and relevant mathematical equations.
\subsection{Procedure}
% What is the computational recipe for a technique?
Algorithm~\ref{alg:train} provides a high-level pseudo-code for preparing memory cell vectors using the Artificial Immune Recognition System, specifically the canonical AIRS2. 
An affinity (distance) measure between input patterns must be defined. For real-valued vectors, this is commonly the Euclidean distance:

\begin{equation}
	dist(x,c) = \sum_{i=1}^{n} (x_i - c_i)^2
\end{equation}

where $n$ is the number of attributes, $x$ is the input vector and $c$ is a given cell vector. The variation of cells during cloning (somatic hypermutation) occurs inversely proportional to the stimulation of a given cell to an input pattern.

\begin{algorithm}[htp]
	\SetLine

	% paramd
	\SetKwData{InputPatterns}{InputPatterns}
	\SetKwData{CloneRate}{$clone_{rate}$}
	\SetKwData{MutateRate}{$mutate_{rate}$}
	\SetKwData{StimulationThreshold}{$stim_{thresh}$}
	\SetKwData{AffinityThreshold}{$affinity_{thresh}$}
	\SetKwData{MaxResources}{$resources_{max}$}
	% data
	\SetKwData{MemoryPool}{$Cells_{memory}$}
	\SetKwData{Clones}{$Cells_{clones}$}
	\SetKwData{Clone}{$Cell_{i}$}
	\SetKwData{Candidate}{$Cell_{c}$}
	\SetKwData{CandidateStimulation}{$Cell_{c}^{stim}$}
	\SetKwData{InputPattern}{$InputPattern_i$}
	\SetKwData{BestMatch}{$Cell_{best}$}
	\SetKwData{BestMatchClass}{$Cell_{best}^{class}$}
	\SetKwData{InputPatternClass}{$InputPattern_{i}^{class}$}
	\SetKwData{NumClones}{$Clones_{num}$}
	\SetKwData{BestMatchStimulation}{$Cell_{best}^{stim}$}
	% functions
	\SetKwFunction{InitializeMemoryPool}{InitializeMemoryPool}
	\SetKwFunction{Stimulate}{Stimulate}
	\SetKwFunction{GetMostStimulated}{GetMostStimulated}
	\SetKwFunction{CreateNewMemoryCell}{CreateNewMemoryCell}
	\SetKwFunction{CloneAndMutate}{CloneAndMutate}
	\SetKwFunction{ReducePoolToMaximumResources}{ReducePoolToMaximumResources}
	\SetKwFunction{Affinity}{Affinity}
	\SetKwFunction{DeleteCell}{DeleteCell}
	\SetKwFunction{AverageStimulation}{AverageStimulation}
	
	% I/O
	\KwIn{\InputPatterns, \CloneRate, \MutateRate, \StimulationThreshold, \MaxResources, \AffinityThreshold}		
	\KwOut{\MemoryPool}
  
	% Algorithm
	\MemoryPool $\leftarrow$ \InitializeMemoryPool{\InputPatterns}\;
	% loop
	\ForEach{\InputPattern $\in$ \InputPatterns}{
		\Stimulate{\MemoryPool, \InputPatterns}\;
		\BestMatch $\leftarrow$ \GetMostStimulated{\InputPattern, \MemoryPool}\;
		\eIf{\BestMatchClass $\neq$ \InputPatternClass} {
			\MemoryPool $\leftarrow$ \CreateNewMemoryCell{\InputPattern}\;
		}{
			\NumClones $\leftarrow$ \BestMatchStimulation $\times$ \CloneRate $\times$ \MutateRate\;
			\Clones $\leftarrow$ \BestMatch\;
			\For{$i$ \KwTo \NumClones} {
				\Clones $\leftarrow$ \CloneAndMutate{\BestMatch}\;
			}
			\While{\AverageStimulation{\Clones} $\leq$ \StimulationThreshold} {
				\ForEach{\Clone $\in$ \Clones} {
					\Clones $\leftarrow$ \CloneAndMutate{\Clone}\;
				}
				\Stimulate{\Clones, \InputPatterns}\;
				\ReducePoolToMaximumResources{\Clones, \MaxResources}\;
			}
			\Candidate $\leftarrow$ \GetMostStimulated{\InputPattern, \Clones}\;
			\If{\CandidateStimulation $>$ \BestMatchStimulation} {
				\MemoryPool $\leftarrow$ \Candidate\;
				\If{\Affinity{\Candidate, \BestMatch} $\leq$ \AffinityThreshold} {
					\DeleteCell{\BestMatch, \MemoryPool}\;
				}
			}			
		}
	}
	\Return{\MemoryPool}\;
	% end
	\caption{Pseudo Code for training the Artificial Immune Recognition System (AIRS2).}
	\label{alg:train}
\end{algorithm}

% Heuristics: Usage guidelines
% The heuristics element describe the commonsense, best practice, and demonstrated rules for applying and configuring a parameterized algorithm. The heuristics relate to the technical details of the techniques procedure and data structures for general classes of application (neither specific implementations not specific problem instances). The heuristics are described textually, such as a series of guidelines in a bullet-point structure.
\subsection{Heuristics}
% What are the suggested configurations for a technique?
% What are the guidelines for the application of a technique to a problem instance?
\begin{itemize}
	\item The AIRS was designed as a supervised algorithm for classification problem domains.
	\item The AIRS is non-parametric, meaning that it does not rely on assumptions about that structure of the function that is is approximating.
	\item Real-values in input vectors should be normalized such that $x \in [0,1)$. 
	\item Euclidean distance is commonly used to measure the distance between real-valued vectors (affinity calculation), although other distance measures may be used (such as dot product), and data specific distance measures may be required for non-scalar attributes.
	\item Cells may be initialized with small random values or more commonly with values from instances in the training set.
	\item A cell's affinity is typically minimizing, where as a cells stimulation is maximizing and typically $\in [0,1]$.
\end{itemize}

% Code Listing
% The code description provides a minimal but functional version of the technique implemented with a programming language. The code description must be able to be typed into an appropriate computer, compiled or interpreted as need be, and provide a working execution of the technique. The technique implementation also includes a minimal problem instance to which it is applied, and both the problem and algorithm implementations are complete enough to demonstrate the techniques procedure. The description is presented as a programming source code listing.
\subsection{Code Listing}
% How is a technique implemented as an executable program?
% How is a technique applied to a concrete problem instance?
Listing~\ref{airs} provides an example of the Artificial Immune Recognition System implemented in the Ruby Programming Language. 
% problem
The problem is a contrived classification problem in a 2-dimensional domain $x\in[0,1], y\in[0,1]$ with two classes: `A' ($x\in[0,0.4999999], y\in[0,0.4999999]$) and `B' ($x\in[0.5,1], y\in[0.5,1]$).

% algorithm
The algorithm is an implementation of the AIRS2 algorithm \cite{Watkins2002b}. An initial pool of memory cells is created, one cell for each class. Euclidean distance divided by the maximum possible distance in the domain is taken as the affinity and stimulation is taken as $1.0-affinity$. The meta-dynamics for memory cells (competition for input patterns) is not performed and may be added into the implementation as an extension.

% the listing
\lstinputlisting[firstline=7,language=ruby,caption=Artificial Immune Recognition System in the Ruby Programming Language, label=airs]{../src/algorithms/immune/airs.rb}

% References: Deeper understanding
% The references element description includes a listing of both primary sources of information about the technique as well as useful introductory sources for novices to gain a deeper understanding of the theory and application of the technique. The description consists of hand-selected reference material including books, peer reviewed conference papers, journal articles, and potentially websites. A bullet-pointed structure is suggested.
\subsection{References}
% What are the primary sources for a technique?
% What are the suggested reference sources for learning more about a technique?

% 
% Primary Sources
% 
\subsubsection{Primary Sources}
% seminal
The Artificial Immune Recognition System was proposed in the Masters work by Watkins \cite{Watkins2001}, and later published \cite{Watkins2002a}.
% early
Early works included the application the AIRS by Watkins and Boggess to a suite of benchmark classification problems \cite{Watkins2002}, and a similar study by Goodman and Boggess comparing to a conceptually similar approach called Learning Vector Quantization \cite{Goodman2002}.

% 
% Learn More
% 
\subsubsection{Learn More}
% reviews
Marwah and Boggess investigated the algorithm seeking issues that affect the algorithms performance \cite{Marwah2002}. They compared various variations of the algorithm with modified resource allocation schemes, tie-handling within the ARB pool, and ARB pool organization.
Watkins and Timmis proposed a new version of the algorithm called AIRS2 which became the replacement for AIRS1 \cite{Watkins2002b}. The updates reduced the complexity of the approach while maintaining the accuracy of the results. An investigation by Goodman, et~al. into the so called `\emph{source of power}' in AIRS indicated that perhaps the memory cell maintenance procedures played an important role \cite{Goodman2003}.
% books
Watkins, et al. provide a detailed review of the technique and its application \cite{Watkins2004a}.


