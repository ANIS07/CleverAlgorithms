% The Clever Algorithms Project: http://www.CleverAlgorithms.com
% (c) Copyright 2010 Jason Brownlee. Some Rights Reserved. 
% This work is licensed under a Creative Commons Attribution-Noncommercial-Share Alike 2.5 Australia License.

% This is an algorithm description, see:
% Jason Brownlee. A Template for Standardized Algorithm Descriptions. Technical Report CA-TR-20100107-1, The Clever Algorithms Project http://www.CleverAlgorithms.com, January 2010.

% Name
% The algorithm name defines the canonical name used to refer to the technique, in addition to common aliases, abbreviations, and acronyms. The name is used in terms of the heading and sub-headings of an algorithm description.
\section{Genetic Algorithm} 
\label{sec:genetic_algorithm}
\index{Genetic Algorithm}

% other names
% What is the canonical name and common aliases for a technique?
% What are the common abbreviations and acronyms for a technique?
\emph{Genetic Algorithm, GA, Simple Genetic Algorithm, SGA, Canonical Genetic Algorithm, CGA.}

% Taxonomy: Lineage and locality
% The algorithm taxonomy defines where a techniques fits into the field, both the specific subfields of Computational Intelligence and Biologically Inspired Computation as well as the broader field of Artificial Intelligence. The taxonomy also provides a context for determining the relation- ships between algorithms. The taxonomy may be described in terms of a series of relationship statements or pictorially as a venn diagram or a graph with hierarchical structure.
\subsection{Taxonomy}
% To what fields of study does a technique belong?
The Genetic Algorithm is an Adaptive Strategy and a Global Optimization technique. It is an Evolutionary Algorithm and belongs to the broader study of Evolutionary Computation.
% What are the closely related approaches to a technique?
The Genetic Algorithm is a sibling of other Evolutionary Algorithms such as Genetic Programming (Section~\ref{sec:genetic_programming}), Evolution Strategies (Section~\ref{sec:evolution_strategies}), Evolutionary Programming (Section~\ref{sec:evolutionary_programming}), and Learning Classifier Systems (Section~\ref{sec:learning_classifier_system}). The Genetic Algorithm is a parent of a large number of variant techniques and sub-fields too numerous to list.

% Inspiration: Motivating system
% The inspiration describes the specific system or process that provoked the inception of the algorithm. The inspiring system may non-exclusively be natural, biological, physical, or social. The description of the inspiring system may include relevant domain specific theory, observation, nomenclature, and most important must include those salient attributes of the system that are somehow abstractly or conceptually manifest in the technique. The inspiration is described textually with citations and may include diagrams to highlight features and relationships within the inspiring system.
% Optional
\subsection{Inspiration}
% What is the system or process that motivated the development of a technique?
% Which features of the motivating system are relevant to a technique?
The Genetic Algorithm is inspired by population genetics (including heredity and gene frequencies), and evolution at the population level, as well as the Mendelian understanding of the structure (such as chromosomes, genes, alleles) and mechanisms (such as recombination and mutation). This is the so-called new or modern synthesis of evolutionary biology. 

% Metaphor: Explanation via analogy
% The metaphor is a description of the technique in the context of the inspiring system or a different suitable system. The features of the technique are made apparent through an analogous description of the features of the inspiring system. The explanation through analogy is not expected to be literal scientific truth, rather the method is used as an allegorical communication tool. The inspiring system is not explicitly described, this is the role of the ‘inspiration’ element, which represents a loose dependency for this element. The explanation is textual and uses the nomenclature of the metaphorical system.
% Optional
\subsection{Metaphor}
% What is the explanation of a technique in the context of the inspiring system?
% What are the functionalities inferred for a technique from the analogous inspiring system?
Individuals of a population contribute their genetic material (called the genotype) proportional to their suitability of their expressed genome (called their phenotype) to their environment, in the form of offspring. The next generation is created through a process of mating that involves recombination of two individuals genomes in the population with the introduction of random copying errors (called mutation). This iterative process may result in an improved adaptive-fit between the phenotypes of individuals in a population and the environment.

% Strategy: Problem solving plan
% The strategy is an abstract description of the computational model. The strategy describes the information processing actions a technique shall take in order to achieve an objective. The strategy provides a logical separation between a computational realization (procedure) and a analogous system (metaphor). A given problem solving strategy may be realized as one of a number specific algorithms or problem solving systems. The strategy description is textual using information processing and algorithmic terminology.
\subsection{Strategy}
% What is the information processing objective of a technique?
The objective of the Genetic Algorithm is to maximize the payoff of candidate solutions in the population against a cost function from the problem domain. 
% What is a techniques plan of action?
The strategy for the Genetic Algorithm is to repeatedly employ surrogates for the recombination and mutation genetic mechanisms on the population of candidate solutions, where the cost function (also known as objective or fitness function) applied to a decoded representation of a candidate governs the probabilistic contributions a given candidate solution can make to the subsequent generation of candidate solutions.

% Procedure: Abstract computation
% The algorithmic procedure summarizes the specifics of realizing a strategy as a systemized and parameterized computation. It outlines how the algorithm is organized in terms of the data structures and representations. The procedure may be described in terms of software engineering and computer science artifacts such as Pseudocode, design diagrams, and relevant mathematical equations.
\subsection{Procedure}
% What is the computational recipe for a technique?
% What are the data structures and representations used in a technique?
Algorithm~\ref{alg:genetic_algorithm} provides a pseudocode listing of the Genetic Algorithm for minimizing a cost function. 

\begin{algorithm}[htp]
	\SetLine  

	% data
	\SetKwData{Best}{$S_{best}$}
	\SetKwData{ProbabilityMutate}{$P_{mutation}$}
	\SetKwData{ProbabilityCrossover}{$P_{crossover}$}
	\SetKwData{Parents}{Parents}
	\SetKwData{Children}{Children}
	\SetKwData{ProblemSize}{$Problem_{size}$}
	\SetKwData{Population}{Population}
	\SetKwData{PopulationSize}{$Population_{size}$}
	\SetKwData{ParentOne}{$Parent_{1}$}
	\SetKwData{ParentTwo}{$Parent_{2}$}
	\SetKwData{ChildOne}{$Child_{1}$}
	\SetKwData{ChildTwo}{$Child_{2}$}	
	% functions
	\SetKwFunction{InitializePopulation}{InitializePopulation}  
	\SetKwFunction{EvaluatePopulation}{EvaluatePopulation} 
	\SetKwFunction{GetBestSolution}{GetBestSolution} 
	\SetKwFunction{SelectParents}{SelectParents}
	\SetKwFunction{Replace}{Replace}
	\SetKwFunction{StopCondition}{StopCondition}
	\SetKwFunction{Crossover}{Crossover}
	\SetKwFunction{Mutate}{Mutate}
  
	% I/O
	\KwIn{\PopulationSize, \ProblemSize, \ProbabilityCrossover, \ProbabilityMutate}		
	\KwOut{\Best}
  	% Algorithm
	% initialize	
	\Population $\leftarrow$ \InitializePopulation{\PopulationSize, \ProblemSize}\;
	% evaluate
	\EvaluatePopulation{\Population}\;
	% best
	\Best $\leftarrow$ \GetBestSolution{\Population}\;
	% loop
	\While{$\neg$\StopCondition{}} {
		% select
		\Parents $\leftarrow$ \SelectParents{\Population, \PopulationSize}\;
		% recombine
		\Children $\leftarrow \emptyset$\;
		\ForEach{\ParentOne, \ParentTwo $\in$ \Parents}{
			\ChildOne, \ChildTwo $\leftarrow$ \Crossover{\ParentOne, \ParentTwo, \ProbabilityCrossover}\;
			\Children $\leftarrow$ \Mutate{\ChildOne, \ProbabilityMutate}\;
			\Children $\leftarrow$ \Mutate{\ChildTwo, \ProbabilityMutate}\;
		}
		% evaluate
		\EvaluatePopulation{\Children}\;
		% best
		\Best $\leftarrow$ \GetBestSolution{\Children}\;
		% replace
		\Population $\leftarrow$ \Replace{\Population, \Children}\;
	}
	\Return{\Best}\;
	% end
	\caption{Pseudocode for the Genetic Algorithm.}
	\label{alg:genetic_algorithm}
\end{algorithm}

% Heuristics: Usage guidelines
% The heuristics element describe the commonsense, best practice, and demonstrated rules for applying and configuring a parameterized algorithm. The heuristics relate to the technical details of the techniques procedure and data structures for general classes of application (neither specific implementations not specific problem instances). The heuristics are described textually, such as a series of guidelines in a bullet-point structure.
\subsection{Heuristics}
% What are the suggested configurations for a technique?
% What are the guidelines for the application of a technique to a problem instance?
\begin{itemize}
	\item Binary strings (referred to as `bitstrings') are the classical representation as they can be decoded to almost any desired representation. Real-valued and integer variables can be decoded using the binary coded decimal method, one's or two's complement methods, or the gray code method, the latter of which is generally preferred.
	\item Problem specific representations and customized genetic operators should be adopted, incorporating as much prior information about the problem domain as possible.
	\item The algorithm is highly-modular and a sub-field exists to study each sub-process, specifically: selection, recombination, mutation, and representation. 
	\item The Genetic Algorithm is most commonly used as an optimization technique, although it should also be considered a general adaptive strategy \cite{Jong1992}.
	\item The schema theorem is a classical explanation for the power of the Genetic Algorithm proposed by Holland  \cite{Holland1975}, and investigated by Goldberg under the name of the building block hypothesis \cite{Goldberg1989}.
	\item The size of the population must be large enough to provide sufficient coverage of the domain and mixing of the useful sub-components of the solution  \cite{Goldberg1992}.
	\item The Genetic Algorithm is classically configured with a high probability of recombination (such as 95\%-99\% of the selected population) and a low probability of mutation (such as $\frac{1}{L}$ where $L$ is the number of components in a solution) \cite{Muhlenbein1992, Back1993}.
	\item The fitness-proportionate selection of candidate solutions to contribute to the next generation should be neither too greedy (to avoid the takeover of fitter candidate solutions) nor too random.
\end{itemize}

% Code Listing
% The code description provides a minimal but functional version of the technique implemented with a programming language. The code description must be able to be typed into an appropriate computer, compiled or interpreted as need be, and provide a working execution of the technique. The technique implementation also includes a minimal problem instance to which it is applied, and both the problem and algorithm implementations are complete enough to demonstrate the techniques procedure. The description is presented as a programming source code listing.
\subsection{Code Listing}
% How is a technique implemented as an executable program?
% How is a technique applied to a concrete problem instance?
Listing~\ref{genetic_algorithm} provides an example of the Genetic Algorithm implemented in the Ruby Programming Language. 
% problem
The demonstration problem is a maximizing binary optimization problem called OneMax that seeks a binary string of unity (all `1' bits). The objective function provides only an indication of the number of correct bits in a candidate string, not the positions of the correct bits.

% algorithm
The Genetic Algorithm is implemented with a conservative configuration including binary tournament selection for the selection operator, uniform crossover for the recombination operator, and point mutations for the mutation operator.

% the listing
\lstinputlisting[firstline=7,language=ruby,caption=Genetic Algorithm in the Ruby Programming Language, label=genetic_algorithm]{../src/algorithms/evolutionary/genetic_algorithm.rb}

% References: Deeper understanding
% The references element description includes a listing of both primary sources of information about the technique as well as useful introductory sources for novices to gain a deeper understanding of the theory and application of the technique. The description consists of hand-selected reference material including books, peer reviewed conference papers, journal articles, and potentially websites. A bullet-pointed structure is suggested.
\subsection{References}
% What are the primary sources for a technique?
% What are the suggested reference sources for learning more about a technique?
% 
% Primary Sources
% 
\subsubsection{Primary Sources}
% adaptive plans - adaptive system framework
Holland is the grandfather of the field that became Genetic Algorithms. Holland investigated adaptive systems in the late 1960s proposing an adaptive system formalism and adaptive strategies referred to as `adaptive plans' \cite{Holland1962, Holland1962a, Holland1969}. 
% students
Holland's theoretical framework was investigated and elaborated by his Ph.D. students at the University of Michigan. Rosenberg investigated a chemical and molecular model of a biological inspired adaptive plan \cite{Rosenberg1967}. Bagley investigated meta-environments and a genetic adaptive plan referred to as a genetic algorithm applied to a simple game called hexapawn \cite{Bagley1967}. Cavicchio further elaborated the genetic adaptive plan by proposing numerous variations, referring to some as `reproductive plans' \cite{Cavicchio1970}. 

Other important contributions were made by Frantz who investigated what were referred to as genetic algorithms for search \cite{Frantz1972}, and Hollstien who investigated genetic plans for adaptive control and function optimization \cite{Hollstien1971}.
% de Jong
De Jong performed a seminal investigation of the genetic adaptive model (genetic plans) applied to continuous function optimization and his suite of test problems adopted are still commonly used \cite{Jong1975}.
% classical book
Holland wrote the the seminal book on his research focusing on the proposed adaptive systems formalism, the reproductive and genetic adaptive plans, and provided a theoretical framework for the mechanisms used and explanation for the capabilities of what would become genetic algorithms \cite{Holland1975}.


% 
% Learn More
% 
\subsubsection{Learn More}
% overview
The field of genetic algorithms is very large, resulting in large numbers of variations on the canonical technique.
% historical reviews
Goldberg provides a classical overview of the field in a review article \cite{Goldberg1994}, as does Mitchell \cite{Mitchell1995}.
% tutorial
Whitley describes a classical tutorial for the Genetic Algorithm covering both practical and theoretical concerns \cite{Whitley1994}.
% good introductions to the ga
The classical book on genetic algorithms as an optimization and machine learning technique was written by Goldberg and provides an in-depth review and practical study of the approach \cite{Goldberg1989}. Mitchell provides a contemporary reference text introducing the technique and the field \cite{Mitchell1998}. Finally, Goldberg provides a modern study of the field, the lessons learned, and reviews the broader toolset of optimization algorithms that the field has produced \cite{Goldberg2002}.


