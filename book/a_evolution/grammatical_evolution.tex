% The Clever Algorithms Project: http://www.CleverAlgorithms.com
% (c) Copyright 2010 Jason Brownlee. Some Rights Reserved. 
% This work is licensed under a Creative Commons Attribution-Noncommercial-Share Alike 2.5 Australia License.

% This is an algorithm description, see:
% Jason Brownlee. A Template for Standardized Algorithm Descriptions. Technical Report CA-TR-20100107-1, The Clever Algorithms Project http://www.CleverAlgorithms.com, January 2010.

% Name
% The algorithm name defines the canonical name used to refer to the technique, in addition to common aliases, abbreviations, and acronyms. The name is used in terms of the heading and sub-headings of an algorithm description.
\section{Grammatical Evolution} 
\label{sec:grammatical_evolution}
\index{Grammatical Evolution}

% other names
% What is the canonical name and common aliases for a technique?
% What are the common abbreviations and acronyms for a technique?
\emph{Grammatical Evolution, GE.}

% Taxonomy: Lineage and locality
% The algorithm taxonomy defines where a techniques fits into the field, both the specific subfields of Computational Intelligence and Biologically Inspired Computation as well as the broader field of Artificial Intelligence. The taxonomy also provides a context for determining the relation- ships between algorithms. The taxonomy may be described in terms of a series of relationship statements or pictorially as a venn diagram or a graph with hierarchical structure.
\subsection{Taxonomy}
% To what fields of study does a technique belong?
Grammatical Evolution is a Global Optimization technique and an instance of an Evolutionary Algorithm from the field of Evolutionary Computation. It may also be considered an algorithm for Automatic Programming.
% What are the closely related approaches to a technique?
Grammatical Evolution is related to other Evolutionary Algorithms for evolving programs such as Genetic Programming (Section~\ref{sec:genetic_programming}) and Gene Expression Programming (Section~\ref{sec:gene_expression_programming}), as well as the classical Genetic Algorithm that uses binary strings (Section~\ref{sec:genetic_algorithm}).

% Inspiration: Motivating system
% The inspiration describes the specific system or process that provoked the inception of the algorithm. The inspiring system may non-exclusively be natural, biological, physical, or social. The description of the inspiring system may include relevant domain specific theory, observation, nomenclature, and most important must include those salient attributes of the system that are somehow abstractly or conceptually manifest in the technique. The inspiration is described textually with citations and may include diagrams to highlight features and relationships within the inspiring system.
% Optional
\subsection{Inspiration}
% What is the system or process that motivated the development of a technique?
The Grammatical Evolution algorithm is inspired by the biological process used for generating a protein from genetic material as well as the broader genetic evolutionary process.
% Which features of the motivating system are relevant to a technique?
The genome is comprised of DNA as a string of building blocks that are transcribed to RNA. RNA codons are in turn translated into sequences of amino acids and used in the protein. The resulting protein in its environment is the phenotype. 

% Metaphor: Explanation via analogy
% The metaphor is a description of the technique in the context of the inspiring system or a different suitable system. The features of the technique are made apparent through an analogous description of the features of the inspiring system. The explanation through analogy is not expected to be literal scientific truth, rather the method is used as an allegorical communication tool. The inspiring system is not explicitly described, this is the role of the ‘inspiration’ element, which represents a loose dependency for this element. The explanation is textual and uses the nomenclature of the metaphorical system.
% Optional
\subsection{Metaphor}
% What is the explanation of a technique in the context of the inspiring system?
% mapping
The phenotype is a computer program that is created from a binary string-based genome. The genome is decoded into a sequence of integers that are in turn mapped onto pre-defined rules that makeup the program. 
% wrapping
The mapping from genotype to the phenotype is a one-to-many process that uses a wrapping feature. This is like the biological process observed in many bacteria, viruses, and mitochondria, where the same genetic material is used in the expression of different genes.
% What are the functionalities inferred for a technique from the analogous inspiring system?
The mapping adds robustness to the process both in the ability to adopt structure-agnostic genetic operators used during the evolutionary process on the sub-symbolic representation and the transcription of well-formed executable programs from the representation.

% Strategy: Problem solving plan
% The strategy is an abstract description of the computational model. The strategy describes the information processing actions a technique shall take in order to achieve an objective. The strategy provides a logical separation between a computational realization (procedure) and a analogous system (metaphor). A given problem solving strategy may be realized as one of a number specific algorithms or problem solving systems. The strategy description is textual using information processing and algorithmic terminology.
\subsection{Strategy}
% What is the information processing objective of a technique?
The objective of Grammatical Evolution is to adapt an executable program to a problem specific objective function.
% What is a techniques plan of action?
This is achieved through an iterative process with surrogates of evolutionary mechanisms such as descent with variation, genetic mutation and recombination, and genetic transcription and gene expression. A population of programs are evolved in a sub-symbolic form as variable length binary strings and mapped to a symbolic and well-structured form as a context free grammar for execution.

% Procedure: Abstract computation
% The algorithmic procedure summarizes the specifics of realizing a strategy as a systemized and parameterized computation. It outlines how the algorithm is organized in terms of the data structures and representations. The procedure may be described in terms of software engineering and computer science artifacts such as pseudo code, design diagrams, and relevant mathematical equations.
\subsection{Procedure}
% What are the data structures and representations used in a technique?
% grammar
A grammar is defined in Backus Normal Form (BNF), which is a context free grammar expressed as a series of production rules comprised of terminals and non-terminals.
% mapping
A variable-length binary string representation is used for the optimization process. Bits are read from the a candidate solutions genome in blocks of 8 called a codon, and decoded to an integer (in the range between 0 and $2^{8}-1$). If the end of the binary string is reached when reading integers, the reading process loops back to the start of the string, effectively creating a circular genome. The integers are mapped to expressions from the BNF until a complete syntactically correct expression is formed. This may not use a solutions entire genome, or use the decoded genome more than once given it's circular nature.
% What is the computational recipe for a technique?
Algorithm~\ref{alg:grammatical_evolution} provides a pseudocode listing of the Grammatical Evolution algorithm for minimizing a cost function.


\begin{algorithm}[ht]
	\SetLine  
	% params
	\SetKwData{Best}{$S_{best}$}
	\SetKwData{ProbabilityMutate}{$P_{mutation}$}
	\SetKwData{ProbabilityCrossover}{$P_{crossover}$}
	\SetKwData{ProbabilityDeleteCodon}{$P_{delete}$}
	\SetKwData{ProbabilityDuplicateCodon}{$P_{duplicate}$}
	\SetKwData{CodonBits}{$Codon_{numbits}$}
	\SetKwData{Grammar}{Grammar}
	\SetKwData{PopulationSize}{$Population_{size}$}
	% data
	\SetKwData{Population}{Population}
	\SetKwData{Parents}{Parents}
	\SetKwData{Children}{Children}	
	\SetKwData{ParentOne}{$Parent_{i}$}
	\SetKwData{ParentTwo}{$Parent_{j}$}
	\SetKwData{Solution}{$S_{i}$}	
	\SetKwData{SolutionBitstring}{$Si_{bitstring}$}
	\SetKwData{SolutionIntegers}{$Si_{integers}$}
	\SetKwData{SolutionProgram}{$Si_{program}$}
	\SetKwData{SolutionCost}{$Si_{cost}$}
	% functions
	\SetKwFunction{InitializePopulation}{InitializePopulation}  
	\SetKwFunction{EvaluatePopulation}{EvaluatePopulation} 
	\SetKwFunction{GetBestSolution}{GetBestSolution} 
	\SetKwFunction{SelectParents}{SelectParents}
	\SetKwFunction{Replace}{Replace}
	\SetKwFunction{StopCondition}{StopCondition}
	\SetKwFunction{Crossover}{Crossover}
	\SetKwFunction{Mutate}{Mutate}
	\SetKwFunction{CodonDeletion}{CodonDeletion}
	\SetKwFunction{CodonDuplication}{CodonDuplication}
	\SetKwFunction{Decode}{Decode}
	\SetKwFunction{Map}{Map}
	\SetKwFunction{Execute}{Execute}
  
	% I/O
	\KwIn{\Grammar, \CodonBits, \PopulationSize, \ProbabilityCrossover, \ProbabilityMutate, \ProbabilityDeleteCodon, \ProbabilityDuplicateCodon}		
	\KwOut{\Best}
  	% Algorithm
	% initialize	
	\Population $\leftarrow$ \InitializePopulation{\PopulationSize, \CodonBits}\;
	% evaluate
	\ForEach{\Solution $\in$ \Population}{
		\SolutionIntegers $\leftarrow$ \Decode{\SolutionBitstring, \CodonBits}\;
		\SolutionProgram $\leftarrow$ \Map{\SolutionIntegers, \Grammar}\;
		\SolutionCost $\leftarrow$ \Execute{\SolutionProgram}\;
	}
	% best
	\Best $\leftarrow$ \GetBestSolution{\Population}\;
	% loop
	\While{$\neg$\StopCondition{}} {
		% select
		\Parents $\leftarrow$ \SelectParents{\Population, \PopulationSize}\;
		% recombine
		\Children $\leftarrow 0$\;
		\ForEach{\ParentOne, \ParentTwo $\in$ \Parents}{
			% crossover
			\Solution $\leftarrow$ \Crossover{\ParentOne, \ParentTwo, \ProbabilityCrossover}\;
			% deletion
			\SolutionBitstring $\leftarrow$ \CodonDeletion{\SolutionBitstring, \ProbabilityDeleteCodon}\;
			% duplication
			\SolutionBitstring $\leftarrow$ \CodonDuplication{\SolutionBitstring, \ProbabilityDuplicateCodon}\;
			% mutation
			\SolutionBitstring $\leftarrow$ \Mutate{\SolutionBitstring, \ProbabilityMutate}\;
			% add
			\Children $\leftarrow$ \Solution\;
		}
		% evaluate
		\ForEach{\Solution $\in$ \Children}{
			\SolutionIntegers $\leftarrow$ \Decode{\SolutionBitstring, \CodonBits}\;
			\SolutionProgram $\leftarrow$ \Map{\SolutionIntegers, \Grammar}\;
			\SolutionCost $\leftarrow$ \Execute{\SolutionProgram}\;
		}
		% best
		\Best $\leftarrow$ \GetBestSolution{\Children}\;
		% replace
		\Population $\leftarrow$ \Replace{\Population, \Children}\;
	}
	\Return{\Best}\;
	% end
	\caption{Pseudo Code for the Grammatical Evolution algorithm.}
	\label{alg:grammatical_evolution}
\end{algorithm}


% Heuristics: Usage guidelines
% The heuristics element describe the commonsense, best practice, and demonstrated rules for applying and configuring a parameterized algorithm. The heuristics relate to the technical details of the techniques procedure and data structures for general classes of application (neither specific implementations not specific problem instances). The heuristics are described textually, such as a series of guidelines in a bullet-point structure.
\subsection{Heuristics}
% What are the suggested configurations for a technique?
% What are the guidelines for the application of a technique to a problem instance?
\begin{itemize}
	\item Grammatical Evolution was designed to optimize programs (such as mathematical equations) to specific cost functions.
	\item Classical genetic operators used by the Genetic Algorithm may be used in the Grammatical Evolution algorithm, such as point mutations and one-point crossover.
	\item Codons (groups of bits mapped to an integer) are commonly fixed at 8 bits, proving a range of integers $\in [0,2^{8}-1]$ that is scaled to the range of rules using a modulo function.
	\item Additional genetic operators may be used with variable-length representations such as codon segments, duplication (add to the end), number of codons selected at random, and deletion.
\end{itemize}

% The code description provides a minimal but functional version of the technique implemented with a programming language. The code description must be able to be typed into an appropriate computer, compiled or interpreted as need be, and provide a working execution of the technique. The technique implementation also includes a minimal problem instance to which it is applied, and both the problem and algorithm implementations are complete enough to demonstrate the techniques procedure. The description is presented as a programming source code listing.
\subsection{Code Listing}
% How is a technique implemented as an executable program?
% How is a technique applied to a concrete problem instance?
Listing~\ref{grammatical_evolution} provides an example of the Grammatical Evolution algorithm implemented in the Ruby Programming Language based on the version described by O'Neill and Ryan \cite{O'Neill2001}.
% problem
The demonstration problem is an instance of symbolic regression $f(x)=x^4+x^3+x^2+x$, where $x\in[-1,1]$. 
The grammar used in this problem is: 
\begin{itemize}
	\item Non-terminals: $N=\{expr,op,pre\_op\}$
	\item Terminals: $T=\{sin,cos,exp,log,+,-,/,*,x,1.0\}$
	\item Expression (program): $S=<expr>$
\end{itemize}

The production rules for the grammar in BNF are:
\begin{itemize}
	\item $<expr>$ ::= $<expr><op><expr>$, $(<expr><op><expr>)$, $<pre\_op>(<expr>)$, $<var>$
	\item $<op>$ ::= +, -, $\div$, $\times$
	\item $<pre\_op>$ ::= Sin, Cos, Exp, Log
	\item $<var>$ ::= x, 1.0
\end{itemize}

% algorithm
The algorithm uses point mutation and a codon-respecting one-point crossover operator. Binary tournament selection is used to determine the parent population's contribution to the subsequent generation. 
% eval
Binary strings are decoded to integers using an unsigned binary. Candidate solutions are then mapped directly into executable Ruby code and executed. A given candidate solution is evaluated by comparing its output against the target function and taking the sum of the absolute errors over a number of trials. The probabilities of point mutation, codon deletion, and codon duplication are hard coded as relative probabilities to each solution, although should be parameters of the algorithm. In this case they are heuristically defined as $\frac{1.0}{L}$, $\frac{0.5}{NC}$ and $\frac{1.0}{NC}$ respectively, where $L$ is the total number of bits, and $NC$ is the number of codons in a given candidate solution.

The implementation uses a maximum depth in the expression tree, whereas traditionally such deep expression trees are marked as invalid. Programs that resolve to a single expression that returns the output are penalized. 

% the listing
\lstinputlisting[firstline=7,language=ruby,caption=Grammatical Evolution algorithm in the Ruby Programming Language, label=grammatical_evolution]{../src/algorithms/evolutionary/grammatical_evolution.rb}


% References: Deeper understanding
% The references element description includes a listing of both primary sources of information about the technique as well as useful introductory sources for novices to gain a deeper understanding of the theory and application of the technique. The description consists of hand-selected reference material including books, peer reviewed conference papers, journal articles, and potentially websites. A bullet-pointed structure is suggested.
\subsection{References}
% What are the primary sources for a technique?
% What are the suggested reference sources for learning more about a technique?

% 
% Primary Sources
% 
\subsubsection{Primary Sources}
% seminal
Grammatical Evolution was proposed by Ryan, Collins and O'Neill in a seminal conference paper that applied the approach to a symbolic regression problem \cite{Ryan1998a}. 
% where did it come from
The approach was born out of the desire for syntax preservation while evolving programs using the Genetic Programming algorithm.
% application papers
This seminal work was followed by application papers for a symbolic integration problem \cite{O'Neill1998, O'Neill1998a} and solving trigonometric identities \cite{Ryan1998}.

% 
% Learn More
% 
\subsubsection{Learn More}
% overview
O'Neill and Ryan provide a high-level introduction to Grammatical Evolution and early demonstration applications \cite{O'Neill1999}. The same authors provide a through introduction to the technique and overview of the state of the field \cite{O'Neill2001}.
% books
O'Neill and Ryan present a seminal reference for Grammatical Evolution in their book \cite{O'Neill2003}. A second more recent book considers extensions to the approach improving its capability on dynamic problems \cite{Dempsey2009}.
