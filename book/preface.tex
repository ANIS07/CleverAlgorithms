% The Clever Algorithms Project: http://www.CleverAlgorithms.com
% (c) Copyright 2010 Jason Brownlee. Some Rights Reserved. 
% This work is licensed under a Creative Commons Attribution-Noncommercial-Share Alike 2.5 Australia License.


% A preface generally covers the story of how the book came into being, or how the idea for the book was developed; this is often followed by thanks and acknowledgments to people who were helpful to the author during the time of writing.
\chapter*{Preface}
\addcontentsline{toc}{chapter}{Preface}

% 
% About the book
% 
\section*{About the book}
% premise 
The idea for this project came out of frustration while working towards by Ph.D. I was investigating optimization algorithms and was implementing a large number of them for a software platform called the Optimization Algorithm Toolkit (OAT)\footnote{OAT located at \url{http://optalgtoolkit.sourceforge.net}}. Each algorithm required considerable effort to locate the relevant source material (such as books, papers, articles, and existing implementations), decipher and interpret the algorithm, then try to piece to a functional implementation of the approach. 

% problem summary
Taking a broader perspective, I realized that the communication of algorithmic techniques in the field of Artificial Intelligence was a difficult open problem. Algorithms were:

\begin{itemize}
	\item \emph{Incomplete}: many techniques are ambiguously described, partially described, or not described at all.
	\item \emph{Inconsistent}: a given technique may be described using a variety of formal and semi-formal methods that also vary across different techniques limiting the transferability of the skills an audience used to realize a technique (such as mathematics, pseudo code, program code, and narratives). An inconsistent representation for techniques mean that the skills used to understand and internalize one technique may not be transferable to realizing other techniques or even extensions of the same technique.
	\item \emph{Distributed}: the description of data structures, operations, and parameterization of a given technique may span an array of papers, articles, books, and source code published over a number of years, the access of which may be restricted and/or difficult to obtain.
\end{itemize}

% result
For the practitioner, an ill described algorithm may be a frustration where the gaps are filled with intuition and `best guess'. At the other end of the spectrum, a badly described algorithm may an example of bad science and the failure of the scientific method, where the inability to understand and implement a technique may prevent the replication of results or the investigation and extension of a technique. 

% solution summary
The software I produced provided a first step solution to this problem: a set of working algorithms implemented in a (somewhat) consistent set of algorithms and downloaded from a single location (a feature likely provided of any library of artificial intelligence techniques). The next step needed was a methodology for addressing this problem that anybody could follow: the strategy to address the open problem of poor technique communication is to present complete algorithm descriptions in a consistent manner in a centralized location.

% book
This book is the outcome of developing that strategy that not only provides a methodology for standardized algorithm descriptions, but provides a large corpus of complete and consistent algorithm descriptions in a single centralized location. 
% goal
These are practical, interesting, and fun, and the goal of this project was to promote these features by making algorithms from the field more accessible, usable, and understandable.
% project
This project was developed over a number years though a lot of writing, discussion, and revision. The content was developed and released publicly under a permissive license on the website \url{http://www.CleverAlgorithms.com}, where forerunning technical reports and the content of this book are freely available.
% take away
I hope that this project has succeeded in some small way and that you too can enjoy the applying, learning, and playing with Clever Algorithms. 



% 
% About the author
% 
\section*{About the author}
% education
Jason Brownlee has a Bachelors in Applied Science, a Masters in Information Technology and a Ph.D. in Computer Science from Swinburne University of Technology in Melbourne, Australia. The subject of Jason's Masters research was Niching Genetic Algorithms. Jason's Ph.D. work was in the area of Artificial Immune Systems and involved research into extending the state of Clonal Selection inspired machine learning algorithms and devising new techniques inspired by the structure and function of the acquired immune system.
% work
Jason has earned a living as a Consultant on numerous enterprise-level information technology projects in retail, energy, and information services sectors. Jason has also worked as Software Engineer investigating the use of intelligent agent technology in geospatial and information services domains in the defense sector.
% general
Jason has a long standing passion for both practical software engineering and basic research into machine learning and has developed and released many plug-ins and software tools. Jason also enjoys writing and maintains a blog located at \url{http://www.neverreadpassively.com} and can be followed on twitter via \url{http://twitter.com/jbrownlee}. 

% 
% Acknowledgments
% 
\section*{Acknowledgments}
% specific
Jason Brownlee would like to sincerely thank Daniel Angus for early discussions that lead to the inception of this book project.
% indirect
Jason would like to thank Ying Liu for her support and patience provided throughout the development of the project. 