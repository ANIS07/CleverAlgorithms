% The Clever Algorithms Project: http://www.CleverAlgorithms.com
% (c) Copyright 2010 Jason Brownlee. Some Rights Reserved. 
% This work is licensed under a Creative Commons Attribution-Noncommercial-Share Alike 2.5 Australia License.


% A preface generally covers the story of how the book came into being, or how the idea for the book was developed; this is often followed by thanks and acknowledgments to people who were helpful to the author during the time of writing.
\chapter*{Preface\markboth{Preface}{}}
\addcontentsline{toc}{chapter}{Preface}

% 
% About the book
% 
\section*{About the book}
% premise 
The need for this project was born of frustration while working towards my Ph.D. I was investigating optimization algorithms and was implementing a large number of them for a software platform called the Optimization Algorithm Toolkit (OAT)\footnote{OAT located at \url{http://optalgtoolkit.sourceforge.net}}. Each algorithm required considerable effort to locate the relevant source material (from books, papers, articles, and existing implementations), decipher and interpret the technique, then to finally attempt to piece together a functional implementation. 

% problem summary
Taking a broader perspective, I realized that the communication of algorithmic techniques in the field of Artificial Intelligence was clearly a difficult and outstanding open problem. Generally, algorithm descriptions are:

\begin{itemize}
	\item \emph{Incomplete}: many techniques are ambiguously described, partially described, or not described at all.
	\item \emph{Inconsistent}: a given technique may be described using a variety of formal and semi-formal methods that vary across different techniques, limiting the transferability of skills an audience requires (such as mathematics, pseudocode, program code, and narratives). An inconsistent representation for techniques mean that the skills used to understand and internalize one technique may not be transferable to realizing different techniques or even extensions of the same technique.
	\item \emph{Distributed}: the description of data structures, operations, and parameterization of a given technique may span a collection of papers, articles, books, and source code published over a number of years, the access to which may be restricted and difficult to obtain.
\end{itemize}

% result
For the practitioner, a badly described algorithm may be simply frustrating, where the gaps in available information are filled with intuition and `best guess'. At the other end of the spectrum, a badly described algorithm may an example of bad science and the failure of the scientific method, where the inability to understand and implement a technique may prevent the replication of results, the application, or the investigation and extension of a technique. 

% solution summary
The software I produced provided a first step solution to this problem: a set of working algorithms implemented in a (somewhat) consistent way and downloaded from a single location (features likely provided of any library of artificial intelligence techniques). The next logical step needed to address this problem is to develop a methodology that anybody can follow. The strategy to address the open problem of poor technique communication is to present complete algorithm descriptions (rather than implementations) in a consistent manner, and in a centralized location.
% book
This book is the outcome of developing such a strategy that not only provides a methodology for standardized algorithm descriptions, but provides a large corpus of complete and consistent algorithm descriptions in a single centralized location. 

% goal
The algorithms described in this work are practical, interesting, and fun, and the goal of this project was to promote these features by making algorithms from the field more accessible, usable, and understandable.
% project
This project was developed over a number years through a lot of writing, discussion, and revision. The content was developed and released publicly under a permissive license on the website \url{http://www.CleverAlgorithms.com}, where forerunning technical reports and the content of this book are freely available.
% take away
I hope that this project has succeeded in some small way and that you too can enjoy applying, learning, and playing with Clever Algorithms. 

\begin{flushright}
\vspace{1in}
Jason Brownlee
\end{flushright}

\begin{flushleft}
\vspace{0.2in}
Melbourne, Australia \\
2011
\end{flushleft}
