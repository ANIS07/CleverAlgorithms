% The Clever Algorithms Project: http://www.CleverAlgorithms.com
% (c) Copyright 2010 Jason Brownlee. Some Rights Reserved. 
% This work is licensed under a Creative Commons Attribution-Noncommercial-Share Alike 2.5 Australia License.

% This is a chapter

\renewcommand{\bibsection}{\subsection{\bibname}}
\begin{bibunit}

\chapter{Neural Algorithms}
\label{ch:neural}
\index{Neural Algorithms}
\index{Artificial Neural Networks}
\index{Neural Networks}
\index{Neural Computation}

\section{Overview}
This chapter describes Neural Algorithms.

% biological
\subsection{Biological Neural Networks}
A Biological Neural Network refers to the information processing elements of the nervous system, organized as a collection of neural cells (called neurons) that are interconnected in networks and interact with each other using electrochemical signals. A biological neuron is generally comprised of a dendrite which provides the input signals and is connected to other neurons via synapses. The neuron reacts to input signals and may produce an output signal on its output connection called an axon.

The study of biological neural networks falls within the domain of neuroscience which is a branch of biology concerned with the nervous system. 
Neuroanatomy is a subject that is concerned with the the structure and function of groups of neural networks both with regard to parts of the brain and the structures that lead from and to the brain from the rest of the body. 
Neuropsychology is another discipline concerned with the structure and function of the brain as they relate to abstract psychological behaviors.
For further information, refer to a good textbook on either any of these general topics.

% artificial
\subsection{Artificial Neural Networks}
The Artificial Neural Networks (ANN) is concerned with the investigation of computational models inspired by theories and observation of the structure and function of biological networks of neural cells in the brain. They are generally designed as models for addressing mathematical, computational, and engineering problems. As such, there is a lot of interdisciplinary research in mathematics, neurobiology and computer science. 

An Artificial Neural Network is generally comprised of a collection of artificial neurons that are interconnected in order to performs some computation of input patterns and create output patterns. They are adaptive systems capable of modifying their internal structure, typically the weights between nodes in the network, allowing them to be used for a variety of function approximation problems such as classification, regression, feature extraction and content addressable memory.

Given that the focus of the field is on performing computation with networks of discrete computing units, the field is traditionally called a `connectionist' paradigm of Artificial Intelligence and `Neural Computation'.

There are many types of neural networks, many of which fall into one of two categories:

\begin{itemize}
	\item \textbf{Feed-forward Networks} where input is provided on one side of the network and the signals are propagated forward (in one direction) through the network structure to the other side where output signals are read. These networks may be comprised of one cell, one layer or multiple layers of neurons. Some examples include the Perceptron, Radial Basis Function Networks, and the multi-layer perceptron networks.
	\item \textbf{Recurrent} where cycles in the network are permitted and the structure may be fully interconnected. Examples include the Hopfield Network and Bidirectional Associative Memory.
\end{itemize}

Artificial Neural Network structures are made up of nodes and weights which typically require training based on samples of patterns from a problem domain. Some examples of learning strategies include:

\begin{itemize}
	\item \textbf{Supervised Learning} where the network is exposed to the input that has a known expected answer. The internal state of the network is modified to better match the expected result. Examples of this learning method include the Back-propagation algorithm and the Hebb rule.
	\item \textbf{Unsupervised Learning} where the network is exposed to input patterns from which it must discern meaning and extract features. The most common type of unsupervised learning is competitive learning where neurons compete based in the input pattern to produce an output pattern. Examples include Neural Gas, Learning Vector Quantization, and the Self-Organizing Map.
\end{itemize}

Artificial Neural Networks are typically difficult to configure and slow to train, but once prepared are very fast in application. They are generally used for function approximation-based problem domains and prized for their capabilities of generalization and tolerance to noise. They are known to have the limitation of being opaque, meaning there is little explanation to the subject matter expert as to why decisions were made, only how.



% References
% \subsection{References}
% classical
% books
There are many excellent reference texts for the field of Artificial Neural Networks, some select texts include: ``Neural Networks for Pattern Recognition'' by Bishop \cite{Bishop1995}, ``Neural Smithing: Supervised Learning in Feedforward Artificial Neural Networks'' by Reed and Marks II \cite{Reed1999} and ``An Introduction to Neural Networks'' by Gurney \cite{Gurney1997}.


% 
% Extensions
% 
\subsection{Extensions}
There are many other algorithms and classes of algorithm that were not described from the field of Artificial Neural Networks, not limited to:

\begin{itemize}
	\item \textbf{Radial Basis Function Network}: A network where activation functions are controlled by Radial Basis Functions \cite{Howlett2001}.
	\item \textbf{Neural Gas}: Another self-organizing and unsupervised competitive learning algorithm. Unlike SOM (and more like LVQ), the nodes are not organized into a lower-dimensional structure, instead the competitive Hebbian-learning like rule is applied to connect, order, and adapt nodes in feature space \cite{Martinetz1991, Martinetz1993, Martinetz1994}.
	\item \textbf{Hierarchical Temporal Memory}: A neural network system based on models of some of the structural and algorithmic properties of the neocortex \cite{Hawkins2005}.
\end{itemize}

\putbib
\end{bibunit}

\newpage\begin{bibunit}% The Clever Algorithms Project: http://www.CleverAlgorithms.com
% (c) Copyright 2010 Jason Brownlee. Some Rights Reserved. 
% This work is licensed under a Creative Commons Attribution-Noncommercial-Share Alike 2.5 Australia License.

% This is an algorithm description, see:
% Jason Brownlee. A Template for Standardized Algorithm Descriptions. Technical Report CA-TR-20100107-1, The Clever Algorithms Project http://www.CleverAlgorithms.com, January 2010.

% Name
% The algorithm name defines the canonical name used to refer to the technique, in addition to common aliases, abbreviations, and acronyms. The name is used in terms of the heading and sub-headings of an algorithm description.
\section{Perceptron} 
\label{sec:perceptron}
\index{Perceptron}

% other names
% What is the canonical name and common aliases for a technique?
% What are the common abbreviations and acronyms for a technique?
\emph{Perceptron.}

% Taxonomy: Lineage and locality
% The algorithm taxonomy defines where a techniques fits into the field, both the specific subfields of Computational Intelligence and Biologically Inspired Computation as well as the broader field of Artificial Intelligence. The taxonomy also provides a context for determining the relation- ships between algorithms. The taxonomy may be described in terms of a series of relationship statements or pictorially as a venn diagram or a graph with hierarchical structure.
\subsection{Taxonomy}
% To what fields of study does a technique belong?
The Perceptron algorithm belongs to the field of Artificial Neural Networks and more broadly Computational Intelligence.
% What are the closely related approaches to a technique?
It is a single layer feedforward neural network (single cell network) that inspired many extensions and variants, not limited to Adalines and Widrow Hoff learning rules.

% Inspiration: Motivating system
% The inspiration describes the specific system or process that provoked the inception of the algorithm. The inspiring system may non-exclusively be natural, biological, physical, or social. The description of the inspiring system may include relevant domain specific theory, observation, nomenclature, and most important must include those salient attributes of the system that are somehow abstractly or conceptually manifest in the technique. The inspiration is described textually with citations and may include diagrams to highlight features and relationships within the inspiring system.
% Optional
\subsection{Inspiration}
% What is the system or process that motivated the development of a technique?
The Perceptron is inspired by the information processing of a single neural cell (called a neuron). 
% Which features of the motivating system are relevant to a technique?
A neuron accepts input signals via the dendrites, a chemical process occurs within the cell based on the input signals, and the cell may or may not produce an output signal on its axon. The point where one cells axon interfaces another cells dendrite is called the synapse, which may fire if the cell is activated.

% Metaphor: Explanation via analogy
% The metaphor is a description of the technique in the context of the inspiring system or a different suitable system. The features of the technique are made apparent through an analogous description of the features of the inspiring system. The explanation through analogy is not expected to be literal scientific truth, rather the method is used as an allegorical communication tool. The inspiring system is not explicitly described, this is the role of the ‘inspiration’ element, which represents a loose dependency for this element. The explanation is textual and uses the nomenclature of the metaphorical system.
% Optional
% \subsection{Metaphor}
% What is the explanation of a technique in the context of the inspiring system?
% What are the functionalities inferred for a technique from the analogous inspiring system?
% A textual description of the algorithm by analogy.

% Strategy: Problem solving plan
% The strategy is an abstract description of the computational model. The strategy describes the information processing actions a technique shall take in order to achieve an objective. The strategy provides a logical separation between a computational realization (procedure) and a analogous system (metaphor). A given problem solving strategy may be realized as one of a number specific algorithms or problem solving systems. The strategy description is textual using information processing and algorithmic terminology.
\subsection{Strategy}
% What is the information processing objective of a technique?
The information processing objective of the technique is to model a given function by modifying internal weightings of input signals to produce an expected output signal.
% What is a techniques plan of action?
The system is trained using a supervised learning method, where the error between the system's output and a known expected output is presented to the system and used to modify its internal state. State is maintained in a set of weightings on the input signals. The weights are used to represent an abstraction of the mapping of input vectors to the output signal for the examples that the system was exposed to during training.

% Procedure: Abstract computation
% The algorithmic procedure summarizes the specifics of realizing a strategy as a systemized and parameterized computation. It outlines how the algorithm is organized in terms of the data structures and representations. The procedure may be described in terms of software engineering and computer science artifacts such as Pseudocode, design diagrams, and relevant mathematical equations.
\subsection{Procedure}
% What are the data structures and representations used in a technique?
The Perceptron is comprised of a data structure (weights) and separate procedures for training and applying the structure. The structure is really just a vector of weights (one for each expected input) and a bias term.

% What is the computational recipe for a technique?
Algorithm~\ref{alg:train} provides a pseudocode for training the Perceptron. A weight is initialized for each input plus an additional weight for a fixed bias constant input that is almost always set to 1.0. The activation of the network to a given input pattern is calculated as follows:
\begin{equation}
	activation \leftarrow \sum_{k=1}^{n}\big( w_{k} \times x_{ki}\big) + w_{bias} \times 1.0
\end{equation}

where $n$ is the number of weights and inputs, $x_{ki}$ is the $k^{th}$ attribute on the $i^{th}$ input pattern, and $w_{bias}$ is the bias weight. The weights are updated as follows:

\begin{equation}
	w_{i}(t+1) = w_{i}(t) + \alpha \times (e(t)-a(t)) \times x_{i}(t)
\end{equation}

where $w_i$ is the $i^{th}$ weight at time $t$ and $t+1$, $\alpha$ is the learning rate, $e(t)$ and $a(t)$ are the expected and actual output at time $t$, and $x_i$ is the $i^{th}$ input. This update process is applied to each weight in turn (as well as the bias weight with its contact input).

\begin{algorithm}[ht]
	\SetLine  

	% data
	\SetKwData{ProblemSize}{ProblemSize}
	\SetKwData{MaxIterations}{$iterations_{max}$}
	\SetKwData{LearningRate}{$learn_{rate}$}
	\SetKwData{Weights}{Weights}
	\SetKwData{InputPatterns}{InputPatterns}
	\SetKwData{Pattern}{$Pattern_i$}
	\SetKwData{Activation}{$Activation_i$}
	\SetKwData{Output}{$Output_i$}

	% functions
	\SetKwFunction{InitializeWeights}{InitializeWeights}
	\SetKwFunction{SelectInputPattern}{SelectInputPattern}
	\SetKwFunction{ActivateNetwork}{ActivateNetwork}
	\SetKwFunction{TransferActivation}{TransferActivation}
	\SetKwFunction{UpdateWeights}{UpdateWeights}
	
	% I/O
	\KwIn{\ProblemSize, \InputPatterns, \MaxIterations, \LearningRate}		
	\KwOut{\Weights}
  
	% Algorithm
	\Weights $\leftarrow$ \InitializeWeights{\ProblemSize}\;
	% loop
	\For{$i=1$ \KwTo \MaxIterations} {
		\Pattern $\leftarrow$ \SelectInputPattern{\InputPatterns}\;
		\Activation $\leftarrow$ \ActivateNetwork{\Pattern, \Weights}\;
		\Output $\leftarrow$ \TransferActivation{\Activation}\;
		\UpdateWeights{\Pattern, \Output, \LearningRate}\;
	}
	\Return{\Weights}\;
	% end
	\caption{Pseudocode for the Perceptron.}
	\label{alg:train}
\end{algorithm}

% Heuristics: Usage guidelines
% The heuristics element describe the commonsense, best practice, and demonstrated rules for applying and configuring a parameterized algorithm. The heuristics relate to the technical details of the techniques procedure and data structures for general classes of application (neither specific implementations not specific problem instances). The heuristics are described textually, such as a series of guidelines in a bullet-point structure.
\subsection{Heuristics}
% What are the suggested configurations for a technique?
% What are the guidelines for the application of a technique to a problem instance?
\begin{itemize}
	\item The Perceptron can be used to approximate arbitrary linear functions and can be used for regression or classification problems.
	\item The Perceptron cannot learn a non-linear mapping between the input and output attributes. The XOR problem is a classical example of a problem that the Perceptron cannot learn.
	\item Input and output values should be normalized such that $x \in [0,1)$.
	\item The learning rate ($\alpha \in [0,1]$) controls the amount of change each error has on the system, lower learning rages are common such as 0.1.
	\item The weights can be updated in an online manner (after the exposure to each input pattern) or in batch (after a fixed number of patterns have been observed).
	\item Batch updates are expected to be more stable than online updates for some complex problems.
	\item A bias weight is used with a constant input signal to provide stability to the learning process. 
	\item A step transfer function is commonly used to transfer the activation to a binary output value $1 \leftarrow activation \geq 0$, otherwise $0$.
	\item It is good practice to expose the system to input patterns in a different random order each enumeration through the input set.
	\item The initial weights are typically small random values, typically $\in [0, 0.5]$.
\end{itemize}

% The code description provides a minimal but functional version of the technique implemented with a programming language. The code description must be able to be typed into an appropriate computer, compiled or interpreted as need be, and provide a working execution of the technique. The technique implementation also includes a minimal problem instance to which it is applied, and both the problem and algorithm implementations are complete enough to demonstrate the techniques procedure. The description is presented as a programming source code listing.
\subsection{Code Listing}
% How is a technique implemented as an executable program?
% How is a technique applied to a concrete problem instance?
Listing~\ref{perceptron} provides an example of the Perceptron algorithm implemented in the Ruby Programming Language. 
% problem
The problem is the classical OR boolean problem, where the inputs of the boolean truth table are provided as the two inputs and the result of the boolean OR operation is the expected as output.

% algorithm
The algorithm was implemented using an online learning method, meaning the weights are updated after each input pattern is observed. A step transfer function is used to convert the activation into a binary output $\in\{0,1\}$. Random samples are taken from the domain to train the weights, and similarly, random samples are drawn from the domain to demonstrate what the network has learned. A bias weight is used for stability with a constant input of 1.0.

% the listing
\lstinputlisting[firstline=7,language=ruby,caption=Perceptron algorithm in the Ruby Programming Language, label=perceptron]{../src/algorithms/neural/perceptron.rb}

% References: Deeper understanding
% The references element description includes a listing of both primary sources of information about the technique as well as useful introductory sources for novices to gain a deeper understanding of the theory and application of the technique. The description consists of hand-selected reference material including books, peer reviewed conference papers, journal articles, and potentially websites. A bullet-pointed structure is suggested.
\subsubsection{References}
% What are the primary sources for a technique?
% What are the suggested reference sources for learning more about a technique?

% 
% Primary Sources
% 
\subsubsection{Primary Sources}
% seminal
The Perceptron algorithm was proposed by Rosenblatt in 1958 \cite{Rosenblatt1958}. Rosenblatt proposed a range of neural network structures and methods. The `Perceptron' as it is known is in fact a simplification of Rosenblatt's models by Minsky and Papert for the purposes of analysis \cite{Minsky1969}.
% early
An early proof of convergence was provided by Novikoff \cite{Novikoff1962}

% 
% Learn More
% 
\subsection{Learn More}
% reviews
% books
Minsky and Papert wrote the classical text titled ``Perceptrons'' in 1969 that is known to have discredited the approach, suggesting it was limited to linear discrimination, which limited research in the area for decades afterward \cite{Minsky1969}.


\putbib\end{bibunit}
\newpage\begin{bibunit}% The Clever Algorithms Project: http://www.CleverAlgorithms.com
% (c) Copyright 2010 Jason Brownlee. Some Rights Reserved. 
% This work is licensed under a Creative Commons Attribution-Noncommercial-Share Alike 2.5 Australia License.

% This is an algorithm description, see:
% Jason Brownlee. A Template for Standardized Algorithm Descriptions. Technical Report CA-TR-20100107-1, The Clever Algorithms Project http://www.CleverAlgorithms.com, January 2010.

% Name
% The algorithm name defines the canonical name used to refer to the technique, in addition to common aliases, abbreviations, and acronyms. The name is used in terms of the heading and sub-headings of an algorithm description.
\section{Back-propagation} 
\label{sec:backpropagation}
\index{Back-propagation}
\index{Error Back-propagation}

% other names
% What is the canonical name and common aliases for a technique?
% What are the common abbreviations and acronyms for a technique?
\emph{Back-propagation, Backpropagation, Error Back Propagation, Backprop, Delta-rule.}

% Taxonomy: Lineage and locality
% The algorithm taxonomy defines where a techniques fits into the field, both the specific subfields of Computational Intelligence and Biologically Inspired Computation as well as the broader field of Artificial Intelligence. The taxonomy also provides a context for determining the relation- ships between algorithms. The taxonomy may be described in terms of a series of relationship statements or pictorially as a venn diagram or a graph with hierarchical structure.
\subsection{Taxonomy}
% To what fields of study does a technique belong?
The Back-propagation algorithm is a supervised learning method for multi-layer feed-forward networks from the field of Artificial Neural Networks and more broadly Computational Intelligence.
% What are the closely related approaches to a technique?
The name refers to the backward propagation of error during the training of the network. Back-propagation is the basis for many variations and extensions for training multi-layer feed-forward networks not limited to Vogl's Method (Bold Drive), Delta-Bar-Delta, Quickprop, and Rprop.

% Inspiration: Motivating system
% The inspiration describes the specific system or process that provoked the inception of the algorithm. The inspiring system may non-exclusively be natural, biological, physical, or social. The description of the inspiring system may include relevant domain specific theory, observation, nomenclature, and most important must include those salient attributes of the system that are somehow abstractly or conceptually manifest in the technique. The inspiration is described textually with citations and may include diagrams to highlight features and relationships within the inspiring system.
% Optional
\subsection{Inspiration}
% What is the system or process that motivated the development of a technique?
Feed-forward neural networks are inspired by the information processing of one or more neural cells (called a neuron). 
% Which features of the motivating system are relevant to a technique?
A neuron accepts input signals via the dendrites, a chemical process occurs within the cell based on the input signals, and the cell may or may not produce an output signal on its axon. The point where one cell's axon interfaces another cell's dendrite is called the synapse, which may fire if the cell is activated.
% backprop 
The Back-propagation algorithm is a training regime for multi-layer feed forward neural networks and is not directly inspired by the learning processes the biological system.

% Metaphor: Explanation via analogy
% The metaphor is a description of the technique in the context of the inspiring system or a different suitable system. The features of the technique are made apparent through an analogous description of the features of the inspiring system. The explanation through analogy is not expected to be literal scientific truth, rather the method is used as an allegorical communication tool. The inspiring system is not explicitly described, this is the role of the ‘inspiration’ element, which represents a loose dependency for this element. The explanation is textual and uses the nomenclature of the metaphorical system.
% Optional
% \subsection{Metaphor}
% What is the explanation of a technique in the context of the inspiring system?
% What are the functionalities inferred for a technique from the analogous inspiring system?
% A textual description of the algorithm by analogy.

% Strategy: Problem solving plan
% The strategy is an abstract description of the computational model. The strategy describes the information processing actions a technique shall take in order to achieve an objective. The strategy provides a logical separation between a computational realization (procedure) and a analogous system (metaphor). A given problem solving strategy may be realized as one of a number specific algorithms or problem solving systems. The strategy description is textual using information processing and algorithmic terminology.
\subsection{Strategy}
% What is the information processing objective of a technique?
The information processing objective of the technique is to model a given function by modifying internal weightings of input signals to produce an expected output signal.
% What is a techniques plan of action?
The system is trained using a supervised learning method, where the error between the system's output and a known expected output is presented to the system and used to modify its internal state. State is maintained in a set of weightings on the input signals. The weights are used to represent an abstraction of the mapping of input vectors to the output signal for the examples that the system was exposed to during training.
Each layer of the network provides an abstraction of the information processing of the previous layer, allowing the combination of sub-functions and higher order modeling.

% Procedure: Abstract computation
% The algorithmic procedure summarizes the specifics of realizing a strategy as a systemized and parameterized computation. It outlines how the algorithm is organized in terms of the data structures and representations. The procedure may be described in terms of software engineering and computer science artifacts such as Pseudocode, design diagrams, and relevant mathematical equations.
\subsection{Procedure}
% What are the data structures and representations used in a technique?
The Back-propagation algorithm is a method for training the weights in a multi-layer feed-forward neural network. As such, it requires a network structure to be defined of one or more layers where one layer is fully connected to the next layer. A standard network structure is one input layer, one hidden layer, and one output layer. The method is primarily concerned with adapting the weights to the calculated error in the presence of input patterns, and the method is applied backward from the network output layer through to the input layer.

% What is the computational recipe for a technique?
Algorithm~\ref{alg:train} provides a high-level pseudocode for preparing a network using the Back-propagation training method. A weight is initialized for each input plus an additional weight for a fixed bias constant input that is almost always set to 1.0. The activation of a single neuron to a given input pattern is calculated as follows:
\begin{equation}
	activation = \bigg(\sum_{k=1}^{n} w_{k} \times x_{ki}\bigg) + w_{bias} \times 1.0
\end{equation}

where $n$ is the number of weights and inputs, $x_{ki}$ is the $k^{th}$ attribute on the $i^{th}$ input pattern, and $w_{bias}$ is the bias weight. A logistic transfer function (sigmoid) is used to calculate the output for a neuron $\in [0,1]$ and provide nonlinearities between in the input and output signals: $\frac{1}{1+exp(-a)}$, where $a$ represents the neuron activation. 

The weight updates use the delta rule, specifically a changed delta rule where error is backwardly propagated through the network, starting at the output layer and weighted back through the previous layers. The following describes the back-propagation of error and weight updates for a single pattern.

An error signal is calculated for each node and propagated back through the network. For the output nodes this is the sum of the error between the node outputs and the expected outputs: 

\begin{equation}
	es_i = (c_i - o_i) \times td_i
\end{equation}

where $es_i$ is the error signal for the $i^{th}$ node, $c_i$ is the expected output and $o_i$ is the actual output for the $i^{th}$ node. The $td$ term is the derivative of the output of the $i^{th}$ node. If the sigmod transfer function is used, $td_i$ would be $o_i \times (1-o_i)$ For the hidden nodes, the error signal is the sum of the weighted error signals from the next layer.

\begin{equation}
	es_i = \bigg(\sum_{k=1}^n (w_{ik} \times es_k)\bigg) \times td_i
\end{equation}

where $es_i$ is the error signal for the $i^{th}$ node, $w_{ik}$ is the weight between the $i^{th}$ and the $k^{th}$ nodes, and $es_k$ is the error signal of the $k_th$ node.

The error derivatives for each weight are calculated by combining the input to each node and the error signal for the node.

\begin{equation}
	ed_i = \sum_{k=1}^n es_i \times x_k
\end{equation}

where $ed_i$ is the error derivative for the $i^{th}$ node, $es_i$ is the error signal for the $i^{th}$ node and $x_k$ is the input from the $k^{th}$ node in the previous layer. This process include the bias input that has a constant value.

Weights are updated in a direction that reduces the error derivative $ed_i$ (error assigned to the weight), metered by a learning coefficient.

\begin{equation}
	w_i(t+1) = w_i(t) + (ed_k \times learn_{rate})
\end{equation}

where $w_i(t+1)$ is the updated $i^{th}$ weight, $ed_k$ is the error derivative for the $k^{th}$ node and $learn_{rate}$ is an update coefficient parameter.

\begin{algorithm}[ht]
	\SetLine

	% data
	\SetKwData{ProblemSize}{ProblemSize}
	\SetKwData{MaxIterations}{$iterations_{max}$}
	\SetKwData{LearningRate}{$learn_{rate}$}
	\SetKwData{Network}{Network}
	\SetKwData{NetworkWeights}{$Network_{weights}$}
	\SetKwData{InputPatterns}{InputPatterns}
	\SetKwData{Pattern}{$Pattern_i$}
	\SetKwData{Output}{$Output_i$}

	% functions
	\SetKwFunction{ConstructNetworkLayers}{ConstructNetworkLayers}
	\SetKwFunction{SelectInputPattern}{SelectInputPattern}
	\SetKwFunction{InitializeWeights}{InitializeWeights}
	\SetKwFunction{ForwardPropagate}{ForwardPropagate}
	\SetKwFunction{BackwardPropagateError}{BackwardPropagateError}
	\SetKwFunction{UpdateWeights}{UpdateWeights}
	
	% I/O
	\KwIn{\ProblemSize, \InputPatterns, \MaxIterations, \LearningRate}		
	\KwOut{\Network}
  
	% Algorithm
	\Network $\leftarrow$ \ConstructNetworkLayers{}\;
	\NetworkWeights $\leftarrow$ \InitializeWeights{\Network, \ProblemSize}\;
	% loop
	\For{$i=1$ \KwTo \MaxIterations} {
		\Pattern $\leftarrow$ \SelectInputPattern{\InputPatterns}\;
		\Output $\leftarrow$ \ForwardPropagate{\Pattern, \Network}\;
		\BackwardPropagateError{\Pattern, \Output, \Network}\;		
		\UpdateWeights{\Pattern, \Output, \Network, \LearningRate}\;
	}
	\Return{\Network}\;
	% end
	\caption{Pseudocode for Back-propagation.}
	\label{alg:train}
\end{algorithm}

% Heuristics: Usage guidelines
% The heuristics element describe the commonsense, best practice, and demonstrated rules for applying and configuring a parameterized algorithm. The heuristics relate to the technical details of the techniques procedure and data structures for general classes of application (neither specific implementations not specific problem instances). The heuristics are described textually, such as a series of guidelines in a bullet-point structure.
\subsection{Heuristics}
% What are the suggested configurations for a technique?
% What are the guidelines for the application of a technique to a problem instance?
\begin{itemize}
	\item The Back-propagation algorithm can be used to train a multi-layer network to approximate arbitrary non-linear functions and can be used for regression or classification problems.
	\item Input and output values should be normalized such that $x \in [0,1)$.
	\item The weights can be updated in an online manner (after the exposure to each input pattern) or in batch (after a fixed number of patterns have been observed).
	\item Batch updates are expected to be more stable than online updates for some complex problems.
	\item A logistic function (sigmoid) transfer function is commonly used to transfer the activation to a binary output value, although other transfer functions can be used such as the hyperbolic tangent (tanh), Gaussian, and softmax.
	\item It is good practice to expose the system to input patterns in a different random order each enumeration through the input set.
	\item The initial weights are typically small random values, typically $\in [0, 0.5]$.
	\item Typically a small number of layers are used such as 2-4 given that the increase in layers result in an increase in the complexity of the system and the time required to train the weights.
	\item The learning rate can be varied during training, and it is common to introduce a momentum term to limit the rate of change.
	\item The weights of a give network can be initialized with a global optimization method before being refined using the Back-propagation algorithm.
	\item One output node is common for regression problems, where as one output node per class is common for classification problems.
\end{itemize}

% The code description provides a minimal but functional version of the technique implemented with a programming language. The code description must be able to be typed into an appropriate computer, compiled or interpreted as need be, and provide a working execution of the technique. The technique implementation also includes a minimal problem instance to which it is applied, and both the problem and algorithm implementations are complete enough to demonstrate the techniques procedure. The description is presented as a programming source code listing.
\subsection{Code Listing}
% How is a technique implemented as an executable program?
% How is a technique applied to a concrete problem instance?
Listing~\ref{backpropagation} provides an example of the Back-propagation algorithm implemented in the Ruby Programming Language. 
% problem
The problem is the classical XOR boolean problem, where the inputs of the boolean truth table are provided as inputs and the result of the boolean XOR operation is the expected as output. This is a classical problem for Back-Propagation because it was the problem instance referenced by Minsky and Papert in their analysis of the Perceptron highlighting the limitations of their simplified models of neural networks \cite{Minsky1969}.

% algorithm
The algorithm was implemented using an online learning method, meaning the weights are updated after each input pattern is observed. A logistic (sigmoid) transfer function is used to convert the activation into an output signal. Random samples are taken from the domain to train the weights, and similarly, random samples are drawn from the domain to demonstrate what the network has learned. 

A three layer network is demonstrated with 2 nodes in the input layer (two inputs), 2 nodes in the hidden layer and 1 node in the output layer, which is sufficient for the chosen problem. A bias weight is used on each neuron for stability with a constant input of 1.0. The learning process is separated into four steps: forward propagation, backward propagation of error, calculation of error derivatives (assigning blame to the weights) and the weight update. This separation facilities easy extensions such as adding a momentum term and/or weight decay to the update process.

Extensions to this implementation should cosider changes such as batch updates (rather than online learning) and the use of a weight decay and momentium (dampen large changes) coefficients. 

% the listing
\lstinputlisting[firstline=7,language=ruby,caption=Back-propagation in Ruby, label=backpropagation]{../src/algorithms/neural/backpropagation.rb}

% References: Deeper understanding
% The references element description includes a listing of both primary sources of information about the technique as well as useful introductory sources for novices to gain a deeper understanding of the theory and application of the technique. The description consists of hand-selected reference material including books, peer reviewed conference papers, journal articles, and potentially websites. A bullet-pointed structure is suggested.
\subsection{References}
% What are the primary sources for a technique?
% What are the suggested reference sources for learning more about a technique?

% 
% Primary Sources
% 
\subsubsection{Primary Sources}
% seminal
The backward propagation of error method is credited to Bryson and Ho in \cite{Bryson1969}. It was applied to the training of multi-layer networks and called back-propagation by Rumelhart, Hinton and Williams in 1986 \cite{Rumelhart1986b, Rumelhart1986c}. 
% early
This effort and the collection of studies edited by Rumelhart and McClelland that helped to define the field of Artificial Neural Networks in the late 1980's \cite{Rumelhart1986, Rumelhart1986a}.


% 
% Learn More
% 
\subsubsection{Learn More}
% reviews
% books
A seminal book on the approach was ``Backpropagation: theory, architectures, and applications'' by Chauvin and Rumelhart that provided an excellent introduction (chapter 1) but also a collection of studies applying and extending the approach \cite{Chauvin1995}.
Reed and Marks provide an excellent treatment of feed-forward neural networks called ``Neural Smithing'' that includes chapters dedicated to Back-propagation, the configuration of its parameters, error surface and speed improvements \cite{Reed1999}.


\putbib\end{bibunit}
\newpage\begin{bibunit}% The Clever Algorithms Project: http://www.CleverAlgorithms.com
% (c) Copyright 2010 Jason Brownlee. Some Rights Reserved. 
% This work is licensed under a Creative Commons Attribution-Noncommercial-Share Alike 2.5 Australia License.

% This is an algorithm description, see:
% Jason Brownlee. A Template for Standardized Algorithm Descriptions. Technical Report CA-TR-20100107-1, The Clever Algorithms Project http://www.CleverAlgorithms.com, January 2010.

% Name
% The algorithm name defines the canonical name used to refer to the technique, in addition to common aliases, abbreviations, and acronyms. The name is used in terms of the heading and sub-headings of an algorithm description.
\section{Hopfield Network} 
\label{sec:hopfield}
\index{Hopfield Network}

% other names
% What is the canonical name and common aliases for a technique?
% What are the common abbreviations and acronyms for a technique?
\emph{Hopfield Network, HN, Hopfield Model.}

% Taxonomy: Lineage and locality
% The algorithm taxonomy defines where a techniques fits into the field, both the specific subfields of Computational Intelligence and Biologically Inspired Computation as well as the broader field of Artificial Intelligence. The taxonomy also provides a context for determining the relation- ships between algorithms. The taxonomy may be described in terms of a series of relationship statements or pictorially as a venn diagram or a graph with hierarchical structure.
\subsection{Taxonomy}
% To what fields of study does a technique belong?
The Hopfield Network is a Neural Network and belongs to the field of Artificial Neural Networks and Neural Computation.
% What are the closely related approaches to a technique?
It is a Recurrent Neural Network and is related to other recurrent networks such as the Bidirectional Associative Memory (BAM).
It is is generally related to feedforward Artificial Neural Networks such as the Perceptron (Section~\ref{sec:perceptron}) and the Back-propagation algorithm (Section~\ref{sec:backpropagation}).

% Inspiration: Motivating system
% The inspiration describes the specific system or process that provoked the inception of the algorithm. The inspiring system may non-exclusively be natural, biological, physical, or social. The description of the inspiring system may include relevant domain specific theory, observation, nomenclature, and most important must include those salient attributes of the system that are somehow abstractly or conceptually manifest in the technique. The inspiration is described textually with citations and may include diagrams to highlight features and relationships within the inspiring system.
% Optional
\subsection{Inspiration}
% What is the system or process that motivated the development of a technique?
% Which features of the motivating system are relevant to a technique?
The Hopfield Network algorithm is inspired by the associated memory properties of the human brain.

% Metaphor: Explanation via analogy
% The metaphor is a description of the technique in the context of the inspiring system or a different suitable system. The features of the technique are made apparent through an analogous description of the features of the inspiring system. The explanation through analogy is not expected to be literal scientific truth, rather the method is used as an allegorical communication tool. The inspiring system is not explicitly described, this is the role of the ‘inspiration’ element, which represents a loose dependency for this element. The explanation is textual and uses the nomenclature of the metaphorical system.
% Optional
\subsection{Metaphor}
% What is the explanation of a technique in the context of the inspiring system?
% What are the functionalities inferred for a technique from the analogous inspiring system?
Through the training process, the weights in the network may be thought to minimize an energy function and slide down an energy surface. In a trained network, each pattern presented to the network provides an attractor, where progress is made towards the point of attraction by propagating information around the network.

% Strategy: Problem solving plan
% The strategy is an abstract description of the computational model. The strategy describes the information processing actions a technique shall take in order to achieve an objective. The strategy provides a logical separation between a computational realization (procedure) and a analogous system (metaphor). A given problem solving strategy may be realized as one of a number specific algorithms or problem solving systems. The strategy description is textual using information processing and algorithmic terminology.
\subsection{Strategy}
% What is the information processing objective of a technique?
% What is a techniques plan of action?
The information processing objective of the system is to associate the components of an input pattern with a holistic representation of the pattern called Content Addressable Memory (CAM). This means that once trained, the system will recall whole patterns, give a portion or a noisy version of the input pattern.

% Procedure: Abstract computation
% The algorithmic procedure summarizes the specifics of realizing a strategy as a systemized and parameterized computation. It outlines how the algorithm is organized in terms of the data structures and representations. The procedure may be described in terms of software engineering and computer science artifacts such as Pseudocode, design diagrams, and relevant mathematical equations.
\subsection{Procedure}
% What is the computational recipe for a technique?
% What are the data structures and representations used in a technique?
The Hopfield Network is comprised of a graphic data structure with weighted edges and separate procedures for training and applying the structure. The network structure is fully connected (a node connects to all other nodes except itself) and the edges (weights) between the nodes are bidirectional. 

The weights of the network can be learned via a one-shot method (one-iteration through the patterns) if all patterns to be memorized by the network are known. Alternatively, the weights can be updated incrementally using the Hebb rule where weights are increased or decreased based on the difference between the actual and the expected output. The one-shot calculation of the network weights for a single node occurs as follows:

\begin{equation}
	w_{i,j} = \sum_{k=1}^{N} v_k^i \times v_k^j
\end{equation}

where $w_{i,j}$ is the weight between neuron $i$ and $j$, $N$ is the number of input patterns, $v$ is the input pattern and $v_k^i$ is the $i^{th}$ attribute on the $k^{th}$ input pattern.

The propagation of the information through the network can be asynchronous where a random node is selected each iteration, or synchronously, where the output is calculated for each node before being applied for the whole network. Propagation of the information continues until no more changes are made or until a maximum number of iterations has completed, after which the output pattern from the network can be read. The activation for a single node is calculated as follows:

\begin{equation}
	n_i = \sum_{j=1}^n w_{i,j} \times n_j
\end{equation}

where $n_i$ is the activation of the $i^{th}$ neuron, $w_{i,j}$ with the weight between the nodes $i$ and $j$, and $n_j$ is the output of the $j^{th}$ neuron. The activation is transferred into an output using a transfer function, typically a step function as follows:

\[transfer(n_i) = \left\{ \begin{array}{l l} 1 & \quad if \geq \theta \\ -1 & \quad if < \theta \\ \end{array} \right. \]

where the threshold $\theta$ is typically fixed at 0.

% Heuristics: Usage guidelines
% The heuristics element describe the commonsense, best practice, and demonstrated rules for applying and configuring a parameterized algorithm. The heuristics relate to the technical details of the techniques procedure and data structures for general classes of application (neither specific implementations not specific problem instances). The heuristics are described textually, such as a series of guidelines in a bullet-point structure.
\subsection{Heuristics}
% What are the suggested configurations for a technique?
% What are the guidelines for the application of a technique to a problem instance?
\begin{itemize}
	\item The Hopfield network may be used to solve the recall problem of matching cues for an input pattern to an associated pre-learned pattern.
	\item The transfer function for turning the activation of a neuron into an output is typically a step function $f(a) \in \{-1,1\}$ (preferred), or more traditionally $f(a) \in \{0,1\}$.
	\item The input vectors are typically normalized to boolean values $x \in [-1,1]$.
	\item The network can be propagated asynchronously (where a random node is selected and output generated), or synchronously (where the output for all nodes are calculated before being applied).
	\item Weights can be learned in a one-shot or incremental method based on how much information is known about the patterns to be learned.
	\item All neurons in the network are typically both input and output neurons, although other network topologies have been investigated (such as the designation of input and output neurons).
	\item A Hopfield network has limits on the patterns it can store and retrieve accurately from memory, described by $N<0.15\times n$ where $N$ is the number of patterns that can be stored and retrieved and $n$ is the number of nodes in the network.
\end{itemize}

% The code description provides a minimal but functional version of the technique implemented with a programming language. The code description must be able to be typed into an appropriate computer, compiled or interpreted as need be, and provide a working execution of the technique. The technique implementation also includes a minimal problem instance to which it is applied, and both the problem and algorithm implementations are complete enough to demonstrate the techniques procedure. The description is presented as a programming source code listing.
\subsection{Code Listing}
% How is a technique implemented as an executable program?
% How is a technique applied to a concrete problem instance?
Listing~\ref{hopfield} provides an example of the Hopfield Network algorithm implemented in the Ruby Programming Language. 
% problem
The problem is an instance of a recall problem where patters are described in terms of a $3 \times 3$ matrix of binary values ($\in \{-1,1\}$). Once the network has learned the patterns, the system is exposed to perturbed versions of the patterns (with errors introduced) and must respond with the correct pattern. Two patterns are used in this example, specifically `T', and `U'.

% algorithm
The algorithm is an implementation of the Hopfield Network with a one-short training method for the network weights, given that all patterns are already known. The information is propagated through the network using an asynchronous method, which is repeated until no more changes in the node outputs are detected. The patterns are displayed to the console during the testing of the network, with the outputs converted from $\{-1,1\}$ to $\{0,1\}$ for readability.

% the listing
\lstinputlisting[firstline=7,language=ruby,caption=Hopfield Network in Ruby, label=hopfield]{../src/algorithms/neural/hopfield.rb}

% References: Deeper understanding
% The references element description includes a listing of both primary sources of information about the technique as well as useful introductory sources for novices to gain a deeper understanding of the theory and application of the technique. The description consists of hand-selected reference material including books, peer reviewed conference papers, journal articles, and potentially websites. A bullet-pointed structure is suggested.
\subsection{References}
% What are the primary sources for a technique?
% What are the suggested reference sources for learning more about a technique?

% 
% Primary Sources
% 
\subsubsection{Primary Sources}
% seminal
The Hopfield Network was proposed by Hopfield in 1982 where the basic model was described and related to an abstraction of the inspiring biological system \cite{Hopfield1982}.
% early
This early work was extend by Hopfield to `graded' neurons capable of outputting a continuous value through use of a logistic (sigmoid) transfer function \cite{Hopfield1984}.
% optimization
An innovative work by Hopfield and Tank considered the use of the Hopfield network for solving combinatorial optimization problems, with a specific study into the system applied to instances of the Traveling Salesman Problem \cite{Hopfield1985}. This was achieved with a large number of neurons and a representation that decoded the position of each position in the tour as a sub-problem on which a customized network energy function had to be minimized.

% 
% Learn More
% 
\subsubsection{Learn More}
% reviews
Popovici and Boncut provide a summary of the Hopfield Network algorithm with worked examples \cite{Popovici2005}.
Overviews of the Hopfield Network are provided in most good books on Artificial Neural Networks, such as \cite{Rojas1996}.
% books
Hertz, Krogh, and Palmer present an in depth study of the the field of Artificial Neural Networks with a detailed treatment of the Hopfield network from a statistical mechanics perspective \cite{Hertz1991}.


\putbib\end{bibunit}
\newpage\begin{bibunit}% The Clever Algorithms Project: http://www.CleverAlgorithms.com
% (c) Copyright 2010 Jason Brownlee. Some Rights Reserved. 
% This work is licensed under a Creative Commons Attribution-Noncommercial-Share Alike 2.5 Australia License.

% This is an algorithm description, see:
% Jason Brownlee. A Template for Standardized Algorithm Descriptions. Technical Report CA-TR-20100107-1, The Clever Algorithms Project http://www.CleverAlgorithms.com, January 2010.

% Name
% The algorithm name defines the canonical name used to refer to the technique, in addition to common aliases, abbreviations, and acronyms. The name is used in terms of the heading and sub-headings of an algorithm description.
\section{Learning Vector Quantization} 
\label{sec:lvq}
\index{Learning Vector Quantization}

% other names
% What is the canonical name and common aliases for a technique?
% What are the common abbreviations and acronyms for a technique?
\emph{Learning Vector Quantization, LVQ.}

% Taxonomy: Lineage and locality
% The algorithm taxonomy defines where a techniques fits into the field, both the specific subfields of Computational Intelligence and Biologically Inspired Computation as well as the broader field of Artificial Intelligence. The taxonomy also provides a context for determining the relation- ships between algorithms. The taxonomy may be described in terms of a series of relationship statements or pictorially as a venn diagram or a graph with hierarchical structure.
\subsection{Taxonomy}
The Learning Vector Quantization algorithm belongs to the field of Artificial Neural Networks and Neural Computation. More broadly to the field of Computational Intelligence.  
% What are the closely related approaches to a technique?
The Learning Vector Quantization algorithm is an supervised neural network that uses a competitive (winner-take-all) learning strategy.
It is related to other supervised neural networks such as the Perceptron (Section~\ref{sec:perceptron}) and the Back-propagation algorithm (Section~\ref{sec:backpropagation}).
It is related to other competitive learning neural networks such as the the Self-Organizing Map algorithm (Section~\ref{sec:som}) that is a similar algorithm for unsupervised learning with the addition of connections between the neurons.
Additionally, LVQ is a baseline technique that was defined with a few variants LVQ1, LVQ2, LVQ2.1, LVQ3, OLVQ1, and OLVQ3 as well as many third-party extensions and refinements too numerous to list.

% Inspiration: Motivating system
% The inspiration describes the specific system or process that provoked the inception of the algorithm. The inspiring system may non-exclusively be natural, biological, physical, or social. The description of the inspiring system may include relevant domain specific theory, observation, nomenclature, and most important must include those salient attributes of the system that are somehow abstractly or conceptually manifest in the technique. The inspiration is described textually with citations and may include diagrams to highlight features and relationships within the inspiring system.
% Optional
\subsection{Inspiration}
% What is the system or process that motivated the development of a technique?
% Which features of the motivating system are relevant to a technique?
The Learning Vector Quantization algorithm is related to the Self-Organizing Map which is in turn inspired by the self-organizing capabilities of neurons in the visual cortex. 

% Metaphor: Explanation via analogy
% The metaphor is a description of the technique in the context of the inspiring system or a different suitable system. The features of the technique are made apparent through an analogous description of the features of the inspiring system. The explanation through analogy is not expected to be literal scientific truth, rather the method is used as an allegorical communication tool. The inspiring system is not explicitly described, this is the role of the ‘inspiration’ element, which represents a loose dependency for this element. The explanation is textual and uses the nomenclature of the metaphorical system.
% Optional
% \subsection{Metaphor}
% What is the explanation of a technique in the context of the inspiring system?
% What are the functionalities inferred for a technique from the analogous inspiring system?
% A textual description of the algorithm by analogy.

% Strategy: Problem solving plan
% The strategy is an abstract description of the computational model. The strategy describes the information processing actions a technique shall take in order to achieve an objective. The strategy provides a logical separation between a computational realization (procedure) and a analogous system (metaphor). A given problem solving strategy may be realized as one of a number specific algorithms or problem solving systems. The strategy description is textual using information processing and algorithmic terminology.
\subsection{Strategy}
% What is the information processing objective of a technique?
The information processing objective of the algorithm is to prepare a set of codebook (or prototype) vectors in the domain of the observed input data samples and to use these vectors to classify unseen examples.
% What is a techniques plan of action?
An initially random pool of vectors is prepared which are then exposed to training samples. A winner-take-all strategy is employed where one or more of the most similar vectors to a given input pattern are selected and adjusted to be closer to the input vector, and in some cases, further away from the winner for runners up. The repetition of this process results in the distribution of codebook vectors in the input space which approximate the underlying distribution of samples from the test dataset.

% Procedure: Abstract computation
% The algorithmic procedure summarizes the specifics of realizing a strategy as a systemized and parameterized computation. It outlines how the algorithm is organized in terms of the data structures and representations. The procedure may be described in terms of software engineering and computer science artifacts such as Pseudocode, design diagrams, and relevant mathematical equations.
\subsection{Procedure}
% What are the data structures and representations used in a technique?
Vector Quantization is a technique from signal processing where density functions are approximated with prototype vectors for applications such as compression. Learning Vector Quantization is similar in principle, although the prototype vectors are learned through a supervised winner-take-all method.

% What is the computational recipe for a technique?
Algorithm~\ref{alg:train} provides a high-level pseudocode for preparing codebook vectors using the Learning Vector Quantization method. 
Codebook vectors are initialized to small floating point values, or sampled from an available dataset. The Best Matching Unit (BMU) is the codebook vector from the pool that has the minimum distance to an input vector. A distance measure between input patterns must be defined. For real-valued vectors, this is commonly the Euclidean distance:

\begin{equation}
	dist(x,c) = \sum_{i=1}^{n} (x_i - c_i)^2
\end{equation}

where $n$ is the number of attributes, $x$ is the input vector and $c$ is a given codebook vector.

\begin{algorithm}[ht]
	\SetLine

	% data
	\SetKwData{ProblemSize}{ProblemSize}
	\SetKwData{MaxIterations}{$iterations_{max}$}
	\SetKwData{LearningRate}{$learn_{rate}$}
	\SetKwData{CodebookVectors}{CodebookVectors}
	\SetKwData{InputPatterns}{InputPatterns}
	\SetKwData{Pattern}{$Pattern_i$}
	\SetKwData{PatternAttribute}{$Pattern_{i}^{attribute}$}
	\SetKwData{PatternClass}{$Pattern_{i}^{class}$}
	\SetKwData{Output}{$Output_i$}
	\SetKwData{NumCodebookVectors}{$CodebookVectors_{num}$}
	\SetKwData{Bmu}{$Bmu_i$}
	\SetKwData{BmuClass}{$Bmu_{i}^{class}$}
	\SetKwData{BmuAttribute}{$Bmu_{i}^{attribute}$}
	
	% functions
	\SetKwFunction{InitializeCodebookVectors}{InitializeCodebookVectors}
	\SetKwFunction{SelectInputPattern}{SelectInputPattern}
	\SetKwFunction{SelectBestMatchingUnit}{SelectBestMatchingUnit}

	
	% I/O
	\KwIn{\ProblemSize, \InputPatterns, \MaxIterations, \NumCodebookVectors, \LearningRate}		
	\KwOut{\CodebookVectors}
  
	% Algorithm
	\CodebookVectors $\leftarrow$ \InitializeCodebookVectors{\NumCodebookVectors, \ProblemSize}\;
	% loop
	\For{$i=1$ \KwTo \MaxIterations} {
		\Pattern $\leftarrow$ \SelectInputPattern{\InputPatterns}\;
		\Bmu $\leftarrow$ \SelectBestMatchingUnit{\Pattern, \CodebookVectors}\;
		\ForEach{\BmuAttribute $\in$ \Bmu}{
			\eIf{\BmuClass $\equiv$ \PatternClass} {
				\BmuAttribute $\leftarrow$ \BmuAttribute $+$ \LearningRate $\times$ (\PatternAttribute $-$ \BmuAttribute)
			}{
				\BmuAttribute $\leftarrow$ \BmuAttribute $-$ \LearningRate $\times$ (\PatternAttribute $-$ \BmuAttribute)
			}
		}
	}
	\Return{\CodebookVectors}\;
	% end
	\caption{Pseudocode for LVQ1.}
	\label{alg:train}
\end{algorithm}

% Heuristics: Usage guidelines
% The heuristics element describe the commonsense, best practice, and demonstrated rules for applying and configuring a parameterized algorithm. The heuristics relate to the technical details of the techniques procedure and data structures for general classes of application (neither specific implementations not specific problem instances). The heuristics are described textually, such as a series of guidelines in a bullet-point structure.
\subsection{Heuristics}
% What are the suggested configurations for a technique?
% What are the guidelines for the application of a technique to a problem instance?
\begin{itemize}
	\item Learning Vector Quantization was designed for classification problems that have existing data sets that can be used to supervise the learning by the system. The algorithm does not support regression problems.
	\item LVQ is non-parametric, meaning that it does not rely on assumptions about that structure of the function that it is approximating.
	\item Real-values in input vectors should be normalized such that $x \in [0,1)$. 
	\item Euclidean distance is commonly used to measure the distance between real-valued vectors, although other distance measures may be used (such as dot product), and data specific distance measures may be required for non-scalar attributes.
	\item There should be sufficient training iterations to expose all the training data to the model multiple times.
	\item The learning rate is typically linearly decayed over the training period from an initial value to close to zero.
	\item The more complex the class distribution, the more codebook vectors that will be required, some problems may need thousands.
	\item Multiple passes of the LVQ training algorithm are suggested for more robust usage, where the first pass has a large learning rate to prepare the codebook vectors and the second pass has a low learning rate and runs for a long time (perhaps 10-times more iterations).
\end{itemize}

% The code description provides a minimal but functional version of the technique implemented with a programming language. The code description must be able to be typed into an appropriate computer, compiled or interpreted as need be, and provide a working execution of the technique. The technique implementation also includes a minimal problem instance to which it is applied, and both the problem and algorithm implementations are complete enough to demonstrate the techniques procedure. The description is presented as a programming source code listing.
\subsection{Code Listing}
% How is a technique implemented as an executable program?
% How is a technique applied to a concrete problem instance?
Listing~\ref{lvq} provides an example of the Learning Vector Quantization algorithm implemented in the Ruby Programming Language. 
% problem
The problem is a contrived classification problem in a 2-dimensional domain $x\in[0,1], y\in[0,1]$ with two classes: `A' ($x\in[0,0.4999999], y\in[0,0.4999999]$) and `B' ($x\in[0.5,1], y\in[0.5,1]$).

% algorithm
The algorithm was implemented using the LVQ1 variant where the best matching codebook vector is located and moved toward the input vector if it is the same class, or away if the classes differ. A linear decay was used for the learning rate that was updated after each pattern was exposed to the model. The implementation can easily be extended to the other variants of the method.

% the listing
\lstinputlisting[firstline=7,language=ruby,caption=Learning Vector Quantization in Ruby, label=lvq]{../src/algorithms/neural/lvq.rb}

% References: Deeper understanding
% The references element description includes a listing of both primary sources of information about the technique as well as useful introductory sources for novices to gain a deeper understanding of the theory and application of the technique. The description consists of hand-selected reference material including books, peer reviewed conference papers, journal articles, and potentially websites. A bullet-pointed structure is suggested.
\subsection{References}
% What are the primary sources for a technique?
% What are the suggested reference sources for learning more about a technique?

% 
% Primary Sources
% 
\subsubsection{Primary Sources}
% seminal
The Learning Vector Quantization algorithm was described by Kohonen in 1988 \cite{Kohonen1988}, and was further described in the same year by Kohonen \cite{Kohonen1988a} and benchmarked by Kohonen, Barna, and Chrisley \cite{Kohonen1988b}.

% 
% Learn More
% 
\subsubsection{Learn More}
% reviews
Kohonen provides a detailed overview of the state of LVQ algorithms and variants (LVQ1, LVQ2, and LVQ2.1) \cite{Kohonen1990}. The technical report that comes with the LVQ\_PAK software (written by Kohonen and his students) provides both an excellent summary of the technique and its main variants, as well as summarizing the important considerations when applying the approach \cite{Kohonen1996}.
% books
The seminal book on Learning Vector Quantization and the Self-Organizing Map is ``Self-Organizing Maps'' by Kohonen, which includes a chapter (Chapter 6) dedicated to LVQ and its variants \cite{Kohonen1995}.


\putbib\end{bibunit}
\newpage\begin{bibunit}% The Clever Algorithms Project: http://www.CleverAlgorithms.com
% (c) Copyright 2010 Jason Brownlee. Some Rights Reserved. 
% This work is licensed under a Creative Commons Attribution-Noncommercial-Share Alike 2.5 Australia License.

% This is an algorithm description, see:
% Jason Brownlee. A Template for Standardized Algorithm Descriptions. Technical Report CA-TR-20100107-1, The Clever Algorithms Project http://www.CleverAlgorithms.com, January 2010.

% Name
% The algorithm name defines the canonical name used to refer to the technique, in addition to common aliases, abbreviations, and acronyms. The name is used in terms of the heading and sub-headings of an algorithm description.
\section{Self-Organizing Map} 
\label{sec:som}
\index{Self-Organizing Map}
\index{Kohonen Network}

% other names
% What is the canonical name and common aliases for a technique?
% What are the common abbreviations and acronyms for a technique?
\emph{Self-Organizing Map, SOM, Self-Organizing Feature Map, SOFM, Kohonen Map, Kohonen Network.}

% Taxonomy: Lineage and locality
% The algorithm taxonomy defines where a techniques fits into the field, both the specific subfields of Computational Intelligence and Biologically Inspired Computation as well as the broader field of Artificial Intelligence. The taxonomy also provides a context for determining the relation- ships between algorithms. The taxonomy may be described in terms of a series of relationship statements or pictorially as a venn diagram or a graph with hierarchical structure.
\subsection{Taxonomy}
% To what fields of study does a technique belong?
The Self-Organizing Map algorithm belongs to the field of Artificial Neural Networks and Neural Computation. More broadly it belongs to the field of Computational Intelligence.  
% What are the closely related approaches to a technique?
The Self-Organizing Map is an unsupervised neural network that uses a competitive (winner-take-all) learning strategy.
It is related to other unsupervised neural networks such as the Adaptive Resonance Theory (ART) method.
It is related to other competitive learning neural networks such as the the Neural Gas Algorithm, and the Learning Vector Quantization algorithm (Section~\ref{sec:lvq}), which is a similar algorithm for classification without connections between the neurons. 
Additionally, SOM is a baseline technique that has inspired many variations and extensions, not limited to the Adaptive-Subspace Self-Organizing Map (ASSOM).

% Inspiration: Motivating system
% The inspiration describes the specific system or process that provoked the inception of the algorithm. The inspiring system may non-exclusively be natural, biological, physical, or social. The description of the inspiring system may include relevant domain specific theory, observation, nomenclature, and most important must include those salient attributes of the system that are somehow abstractly or conceptually manifest in the technique. The inspiration is described textually with citations and may include diagrams to highlight features and relationships within the inspiring system.
% Optional
\subsection{Inspiration}
% What is the system or process that motivated the development of a technique?
% Which features of the motivating system are relevant to a technique?
The Self-Organizing Map is inspired by postulated feature maps of neurons in the brain comprised of feature-sensitive cells that provide ordered projections between neuronal layers, such as those that may exist in the retina and cochlea. For example, there are acoustic feature maps that respond to sounds to which an animal is most frequently exposed, and tonotopic maps that may be responsible for the order preservation of acoustic resonances.

% Metaphor: Explanation via analogy
% The metaphor is a description of the technique in the context of the inspiring system or a different suitable system. The features of the technique are made apparent through an analogous description of the features of the inspiring system. The explanation through analogy is not expected to be literal scientific truth, rather the method is used as an allegorical communication tool. The inspiring system is not explicitly described, this is the role of the ‘inspiration’ element, which represents a loose dependency for this element. The explanation is textual and uses the nomenclature of the metaphorical system.
% Optional
% \subsection{Metaphor}
% What is the explanation of a technique in the context of the inspiring system?
% What are the functionalities inferred for a technique from the analogous inspiring system?
% A textual description of the algorithm by analogy.

% Strategy: Problem solving plan
% The strategy is an abstract description of the computational model. The strategy describes the information processing actions a technique shall take in order to achieve an objective. The strategy provides a logical separation between a computational realization (procedure) and a analogous system (metaphor). A given problem solving strategy may be realized as one of a number specific algorithms or problem solving systems. The strategy description is textual using information processing and algorithmic terminology.
\subsection{Strategy}
% What is the information processing objective of a technique?
The information processing objective of the algorithm is to optimally place a topology (grid or lattice) of codebook or prototype vectors in the domain of the observed input data samples.
% What is a techniques plan of action?
An initially random pool of vectors is prepared which are then exposed to training samples. A winner-take-all strategy is employed where the most similar vector to a given input pattern is selected, then the selected vector and neighbors of the selected vector are updated to closer resemble the input pattern. The repetition of this process results in the distribution of codebook vectors in the input space which approximate the underlying distribution of samples from the test dataset. The result is the mapping of the topology of codebook vectors to the underlying structure in the input samples which may be summarized or visualized to reveal topologically preserved features from the input space in a low-dimensional projection.

% Procedure: Abstract computation
% The algorithmic procedure summarizes the specifics of realizing a strategy as a systemized and parameterized computation. It outlines how the algorithm is organized in terms of the data structures and representations. The procedure may be described in terms of software engineering and computer science artifacts such as Pseudocode, design diagrams, and relevant mathematical equations.
\subsection{Procedure}
% What are the data structures and representations used in a technique?
The Self-Organizing map is comprised of a collection of codebook vectors connected together in a topological arrangement, typically a one dimensional line or a two dimensional grid. The codebook vectors themselves represent prototypes (points) within the domain, whereas the topological structure imposes an ordering between the vectors during the training process. The result is a low dimensional projection or approximation of the problem domain which may be visualized, or from which clusters may be extracted.

% What is the computational recipe for a technique?
Algorithm~\ref{alg:train} provides a high-level pseudocode for preparing codebook vectors using the Self-Organizing Map method. 
Codebook vectors are initialized to small floating point values, or sampled from the domain. The Best Matching Unit (BMU) is the codebook vector from the pool that has the minimum distance to an input vector. A distance measure between input patterns must be defined. For real-valued vectors, this is commonly the Euclidean distance:

\begin{equation}
	dist(x,c) = \sum_{i=1}^{n} (x_i - c_i)^2
\end{equation}

where $n$ is the number of attributes, $x$ is the input vector and $c$ is a given codebook vector.

The neighbors of the BMU in the topological structure of the network are selected using a neighborhood size that is linearly decreased during the training of the network. The BMU and all selected neighbors are then adjusted toward the input vector using a learning rate that too is decreased linearly with the training cycles:

\begin{equation}
	c_i(t+1) = learn_{rate}(t) \times (c_i(t) - x_i)
\end{equation}

where $c_i(t)$ is the $i^{th}$ attribute of a codebook vector at time $t$, $learn_{rate}$ is the current learning rate, an $x_i$ is the $i^{th}$ attribute of a input vector.

The neighborhood is typically square (called bubble) where all neighborhood nodes are updated using the same learning rate for the iteration, or Gaussian where the learning rate is proportional to the neighborhood distance using a Gaussian distribution (neighbors further away from the BMU are updated less). 

\begin{algorithm}[ht]
	\SetLine

	% parametrs
	\SetKwData{InputPatterns}{InputPatterns}
	\SetKwData{MaxIterations}{$iterations_{max}$}
	\SetKwData{LearningRate}{$learn_{rate}^{init}$}	
	\SetKwData{NeighborhoodSize}{$neighborhood_{size}^{init}$}
	\SetKwData{GridWidth}{$Grid_{width}$}
	\SetKwData{GridHeight}{$Grid_{height}$}
	% data
	\SetKwData{CodebookVectors}{CodebookVectors}
	\SetKwData{Neighborhood}{Neighborhood}
	\SetKwData{Lrate}{$learn_{rate}^{i}$}	
	\SetKwData{Nsize}{$neighborhood_{size}^{i}$}
	\SetKwData{Pattern}{$Pattern_i$}	
	\SetKwData{Bmu}{$Bmu_i$}
	\SetKwData{Vector}{$Vector_i$}
	\SetKwData{PatternAttribute}{$Pattern_{i}^{attribute}$}
	\SetKwData{VectorAttribute}{$Vector_{i}^{attribute}$}
	
	% functions
	\SetKwFunction{InitializeCodebookVectors}{InitializeCodebookVectors}
	\SetKwFunction{CalculateLearningRate}{CalculateLearningRate}
	\SetKwFunction{CalculateNeighborhoodSize}{CalculateNeighborhoodSize}
	\SetKwFunction{SelectInputPattern}{SelectInputPattern}
	\SetKwFunction{SelectBestMatchingUnit}{SelectBestMatchingUnit}
	\SetKwFunction{SelectNeighbors}{SelectNeighbors}
	
	% I/O
	\KwIn{\InputPatterns, \MaxIterations, \LearningRate, \NeighborhoodSize, \GridWidth, \GridHeight}		
	\KwOut{\CodebookVectors}
  
	% Algorithm
	\CodebookVectors $\leftarrow$ \InitializeCodebookVectors{\GridWidth, \GridHeight, \InputPatterns}\;
	% loop
	\For{$i=1$ \KwTo \MaxIterations} {
		\Lrate $\leftarrow$ \CalculateLearningRate{$i$, \LearningRate}\;
		\Nsize $\leftarrow$ \CalculateNeighborhoodSize{$i$, \NeighborhoodSize}\;
		\Pattern $\leftarrow$ \SelectInputPattern{\InputPatterns}\;
		\Bmu $\leftarrow$ \SelectBestMatchingUnit{\Pattern, \CodebookVectors}\;
		\Neighborhood $\leftarrow$ \Bmu\;
		\Neighborhood $\leftarrow$ \SelectNeighbors{\Bmu, \CodebookVectors, \Nsize}\;
		\ForEach{\Vector $\in $\Neighborhood} {
			\ForEach{\VectorAttribute $\in$ \Vector}{
				\VectorAttribute $\leftarrow$ \VectorAttribute $+$ \Lrate $\times$ (\PatternAttribute - \VectorAttribute)
			}
		}
	}
	\Return{\CodebookVectors}\;
	% end
	\caption{Pseudocode for the SOM.}
	\label{alg:train}
\end{algorithm}

% Heuristics: Usage guidelines
% The heuristics element describe the commonsense, best practice, and demonstrated rules for applying and configuring a parameterized algorithm. The heuristics relate to the technical details of the techniques procedure and data structures for general classes of application (neither specific implementations not specific problem instances). The heuristics are described textually, such as a series of guidelines in a bullet-point structure.
\subsection{Heuristics}
% What are the suggested configurations for a technique?
% What are the guidelines for the application of a technique to a problem instance?
\begin{itemize}
	\item The Self-Organizing Map was designed for unsupervised learning problems such as feature extraction, visualization and clustering. Some extensions of the approach can label the prepared codebook vectors which can be used for classification.
	\item SOM is non-parametric, meaning that it does not rely on assumptions about that structure of the function that it is approximating.
	\item Real-values in input vectors should be normalized such that $x \in [0,1)$. 
	\item Euclidean distance is commonly used to measure the distance between real-valued vectors, although other distance measures may be used (such as dot product), and data specific distance measures may be required for non-scalar attributes.
	\item There should be sufficient training iterations to expose all the training data to the model multiple times.
	\item The more complex the class distribution, the more codebook vectors that will be required, some problems may need thousands.
	\item Multiple passes of the SOM training algorithm are suggested for more robust usage, where the first pass has a large learning rate to prepare the codebook vectors and the second pass has a low learning rate and runs for a long time (perhaps 10-times more iterations).
	\item The SOM can be visualized by calculating a Unified Distance Matrix (U-Matrix) shows highlights the relationships between the nodes in the chosen topology. A Principle Component Analysis (PCA) or Sammon's Mapping can be used to visualize just the nodes of the network without their inter-relationships.
	\item A rectangular 2D grid topology is typically used for a SOM, although toroidal and sphere topologies can be used. Hexagonal grids have demonstrated better results on some problems and grids with higher dimensions have been investigated. 
	\item The neuron positions can be updated incrementally or in a batch model (each epoch of being exposed to all training samples). Batch-mode training is generally expected to result in a more stable network.
	\item The learning rate and neighborhood size parameters typically decrease linearly with the training iterations, although non-linear functions may be used.
\end{itemize}

% The code description provides a minimal but functional version of the technique implemented with a programming language. The code description must be able to be typed into an appropriate computer, compiled or interpreted as need be, and provide a working execution of the technique. The technique implementation also includes a minimal problem instance to which it is applied, and both the problem and algorithm implementations are complete enough to demonstrate the techniques procedure. The description is presented as a programming source code listing.
\subsection{Code Listing}
% How is a technique implemented as an executable program?
% How is a technique applied to a concrete problem instance?
Listing~\ref{som} provides an example of the Self-Organizing Map algorithm implemented in the Ruby Programming Language. 
% problem
The problem is a feature detection problem, where the network is expected to learn a predefined shape based on being exposed to samples in the domain. The domain is two-dimensional $x,y \in [0,1]$, where a shape is pre-defined as a square in the middle of the domain $x,y \in [0.3,0.6]$. The system is initialized to vectors within the domain although is only exposed to samples within the pre-defined shape during training. The expectation is that the system will model the shape based on the observed samples.

% algorithm
The algorithm is an implementation of the basic Self-Organizing Map algorithm based on the description in Chapter~3 of the seminal book on the technique \cite{Kohonen1995}. The implementation is configured with a $4 \times 5$ grid of nodes, the Euclidean distance measure is used to determine the BMU and neighbors, a Bubble neighborhood function is used. Error rates are presented to the console, and the codebook vectors themselves are described before and after training. The learning process is incremental rather than batch, for simplicity. 

An extension to this implementation would be to visualize the resulting network structure in the domain - shrinking from a mesh that covers the whole domain, down to a mesh that only covers the pre-defined shape within the domain.

% the listing
\lstinputlisting[firstline=7,language=ruby,caption=Self-Organizing Map in Ruby, label=som]{../src/algorithms/neural/som.rb}

% References: Deeper understanding
% The references element description includes a listing of both primary sources of information about the technique as well as useful introductory sources for novices to gain a deeper understanding of the theory and application of the technique. The description consists of hand-selected reference material including books, peer reviewed conference papers, journal articles, and potentially websites. A bullet-pointed structure is suggested.
\subsection{References}
% What are the primary sources for a technique?
% What are the suggested reference sources for learning more about a technique?

% 
% Primary Sources
% 
\subsubsection{Primary Sources}
% seminal
The Self-Organizing Map was proposed by Kohonen in 1982 in a study that included the mathematical basis for the approach, summary of related physiology, and simulation on demonstration problem domains using one and two dimensional topological structures \cite{Kohonen1982}.
% early
This work was tightly related two other papers published at close to the same time on topological maps and self-organization \cite{Kohonen1981, Kohonen1982b}.

% 
% Learn More
% 
\subsubsection{Learn More}
% reviews
Kohonen provides a detailed introduction and summary of the Self-Organizing Map in a journal article \cite{Kohonen1990a}.
Kohonen et al.\ provide a practical presentation of the algorithm and heuristics for configuration in the technical report written to accompany the released SOM-PAK implementation of the algorithm for academic research \cite{Kohonen1996a}.
% books
The seminal book on the technique is ``Self-Organizing Maps'' by Kohonen, which includes chapters dedicated to the description of the basic approach, physiological interpretations of the algorithm, variations, and summaries of application areas \cite{Kohonen1995}.


\putbib\end{bibunit}

