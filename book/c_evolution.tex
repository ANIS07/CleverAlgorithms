% The Clever Algorithms Project: http://www.CleverAlgorithms.com
% (c) Copyright 2010 Jason Brownlee. Some Rights Reserved. 
% This work is licensed under a Creative Commons Attribution-Noncommercial-Share Alike 2.5 Australia License.

% This is a chapter

\renewcommand{\bibsection}{\subsection{\bibname}}
\begin{bibunit}

\chapter{Evolutionary Algorithms}
\label{ch:evolutionary}
\index{Evolutionary Algorithms}
\index{Evolutionary Computation}

\section{Overview}
This chapter describes Evolutionary Algorithms.

% Evolution
\subsection{Evolution}
Evolutionary Algorithms belong to the Evolutionary Computation field of study concerned with computational methods inspired by the process and mechanisms of biological evolution. The process of evolution by means of natural selection (descent with modification) was proposed by Darwin to account for the variety of life and its suitability (adaptive fit) for its environment. The mechanisms of evolution describe how evolution actually takes place through the modification and propagation of genetic material (proteins). Evolutionary Algorithms are concerned with investigating computations that resemble simplified versions of the processes and mechanisms of evolution toward achieving the effects of these processes and mechanisms, namely the development of adaptive systems.
% related
Additional subject areas that fall within the realm of Evolutionary Computation are algorithms that seek to exploit the properties from the related fields of Population Genetics, Population Ecology, Coevolutionary Biology, and Developmental Biology. 

% References
\subsection{References}
Evolutionary Algorithms share properties of adaptation through an iterative process that accumulates and amplifies beneficial variation through trial and error. Candidate solutions represent members of a virtual population striving to survive in an environment defined by a problem specific objective function. In each case, the evolutionary process refines the adaptive fit of the population of candidate solutions in the environment, typically using surrogates for the mechanisms of evolution such as genetic recombination and mutation.

% references
There are many excellent texts on the theory of evolution, although Darwin's original source can be an interesting and surprisingly enjoyable read \cite{Darwin1859}. Huxley's book defined the modern synthesis in evolutionary biology that combined Darwin's natural selection with Mendel's genetic mechanisms \cite{Huxley1942}, although any good textbook on evolution will suffice (such as Futuyma's `\emph{Evolution}' \cite{Futuyma2009}). Popular science books on evolution are an easy place to start, such as Dawkins' `\emph{The Selfish Gene}' that presents on a gene-centric perspective on evolution \cite{Dawkins1976}, and Dennett's `\emph{Darwin's Dangerous Idea}' that considers on the algorithmic properties of the process \cite{Dennett1995}.

% classical
Goldberg's classic text is still a valuable resource for the Genetic Algorithm \cite{Goldberg1989}, and Holland's text is interesting for those looking to learn about the research into adaptive systems that became the Genetic Algorithm \cite{Holland1975}. Additionally, the seminal work by Koza should be considered for those interested in Genetic Programming \cite{Koza1992}, and Schwefel's seminal work should be considered for those with an interest in Evolution Strategies \cite{Schwefel1981}. For an indepth review of the history of research into the use of simulated evolutionary processed for problem solving, see Fogel \cite{Fogel1998}
% modern
For a rounded and modern review of the field of Evolutionary Computation B\"ack, Fogel, and Michalewicz's two volumes of `\emph{Evolutionary Computation}' are an excellent resource covering the major techniques, theory, and application specific concerns \cite{Baeck2000, Baeck2000a}.
% other books
For some additional modern books on the unified field of Evolutionary Computation and Evolutionary Algorithms, see De Jong \cite{Jong2006}, a recent edition of Fogel \cite{Fogel1995}, and Eiben and Smith \cite{Eiben2003}.

% 
% Extensions
% 
\section{Extensions}
There are many other algorithms and classes of algorithm that were not described from the field of Evolutionary Computation, not limited to:

\begin{itemize}
	\item \textbf{Distributed Evolutionary Computation}: that are designed to partition a population across computer networks or computational units such as the Distributed or `Island Population' Genetic Algorithm \cite{Tanese1989, Cantu-Paz2000} and Diffusion Genetic Algorithms (also known as Cellular Genetic Algorithms) \cite{Alba2008}.
	\item \textbf{Niching Genetic Algorithms}: that form groups or sub-populations automatically within a population such as the Deterministic Crowding Genetic Algorithm \cite{Mahfoud1992, Mahfoud1995}, Restricted Tournament Selection \cite{Harik1994, Harik1995}, and Fitness Sharing Genetic Algorithm \cite{Goldberg1987, Deb1989}.
	\item \textbf{Evolutionary Multiple Objective Optimization Algorithms}: such as Vector-Evaluated Genetic Algorithm, Pareto Archived Evolution Strategy, and the Niched Pareto Genetic Algorithm.
	\item \textbf{Classical Techniques}: such as GENITOR \cite{Whitley1989}, and the CHC Genetic Algorithm.
	\item \textbf{Competent Genetic Algorithms}: (so-called \cite{Goldberg2002}) such as the Messy Genetic Algorithm, Fast Messy Genetic Algorithm, Gene Expression Messy Genetic Algorithm, and the Linkage-Learning Genetic Algorithm.
\end{itemize}


\putbib
\end{bibunit}

\newpage\begin{bibunit}% The Clever Algorithms Project: http://www.CleverAlgorithms.com
% (c) Copyright 2010 Jason Brownlee. Some Rights Reserved. 
% This work is licensed under a Creative Commons Attribution-Noncommercial-Share Alike 2.5 Australia License.

% This is an algorithm description, see:
% Jason Brownlee. A Template for Standardized Algorithm Descriptions. Technical Report CA-TR-20100107-1, The Clever Algorithms Project http://www.CleverAlgorithms.com, January 2010.

% Name
% The algorithm name defines the canonical name used to refer to the technique, in addition to common aliases, abbreviations, and acronyms. The name is used in terms of the heading and sub-headings of an algorithm description.
\section{Genetic Algorithm} 
\label{sec:genetic_algorithm}
\index{Genetic Algorithm}

% other names
% What is the canonical name and common aliases for a technique?
% What are the common abbreviations and acronyms for a technique?
\emph{Genetic Algorithm, GA, Simple Genetic Algorithm, SGA, Canonical Genetic Algorithm, CGA.}

% Taxonomy: Lineage and locality
% The algorithm taxonomy defines where a techniques fits into the field, both the specific subfields of Computational Intelligence and Biologically Inspired Computation as well as the broader field of Artificial Intelligence. The taxonomy also provides a context for determining the relation- ships between algorithms. The taxonomy may be described in terms of a series of relationship statements or pictorially as a venn diagram or a graph with hierarchical structure.
\subsection{Taxonomy}
% To what fields of study does a technique belong?
The Genetic Algorithm is an Adaptive Strategy and a Global Optimization technique. It is an Evolutionary Algorithm and belongs to the broader study of Evolutionary Computation.
% What are the closely related approaches to a technique?
The Genetic Algorithm is a sibling of other Evolutionary Algorithms such as Genetic Programming (Section~\ref{sec:genetic_programming}), Evolution Strategies (Section~\ref{sec:evolution_strategies}), Evolutionary Programming (Section~\ref{sec:evolutionary_programming}), and Learning Classifier Systems (Section~\ref{sec:learning_classifier_system}). The Genetic Algorithm is a parent of a large number of variant techniques and sub-fields too numerous to list.

% Inspiration: Motivating system
% The inspiration describes the specific system or process that provoked the inception of the algorithm. The inspiring system may non-exclusively be natural, biological, physical, or social. The description of the inspiring system may include relevant domain specific theory, observation, nomenclature, and most important must include those salient attributes of the system that are somehow abstractly or conceptually manifest in the technique. The inspiration is described textually with citations and may include diagrams to highlight features and relationships within the inspiring system.
% Optional
\subsection{Inspiration}
% What is the system or process that motivated the development of a technique?
% Which features of the motivating system are relevant to a technique?
The Genetic Algorithm is inspired by population genetics (including heredity and gene frequencies), and evolution at the population level, as well as the Mendelian understanding of the structure (such as chromosomes, genes, alleles) and mechanisms (such as recombination and mutation). This is the so-called new or modern synthesis of evolutionary biology. 

% Metaphor: Explanation via analogy
% The metaphor is a description of the technique in the context of the inspiring system or a different suitable system. The features of the technique are made apparent through an analogous description of the features of the inspiring system. The explanation through analogy is not expected to be literal scientific truth, rather the method is used as an allegorical communication tool. The inspiring system is not explicitly described, this is the role of the ‘inspiration’ element, which represents a loose dependency for this element. The explanation is textual and uses the nomenclature of the metaphorical system.
% Optional
\subsection{Metaphor}
% What is the explanation of a technique in the context of the inspiring system?
% What are the functionalities inferred for a technique from the analogous inspiring system?
Individuals of a population contribute their genetic material (called the genotype) proportional to their suitability of their expressed genome (called their phenotype) to their environment, in the form of offspring. The next generation is created through a process of mating that involves recombination of two individuals genomes in the population with the introduction of random copying errors (called mutation). This iterative process may result in an improved adaptive-fit between the phenotypes of individuals in a population and the environment.

% Strategy: Problem solving plan
% The strategy is an abstract description of the computational model. The strategy describes the information processing actions a technique shall take in order to achieve an objective. The strategy provides a logical separation between a computational realization (procedure) and a analogous system (metaphor). A given problem solving strategy may be realized as one of a number specific algorithms or problem solving systems. The strategy description is textual using information processing and algorithmic terminology.
\subsection{Strategy}
% What is the information processing objective of a technique?
The objective of the Genetic Algorithm is to maximize the payoff of candidate solutions in the population against a cost function from the problem domain. 
% What is a techniques plan of action?
The strategy for the Genetic Algorithm is to repeatedly employ surrogates for the recombination and mutation genetic mechanisms on the population of candidate solutions, where the cost function (also known as objective or fitness function) applied to a decoded representation of a candidate governs the probabilistic contributions a given candidate solution can make to the subsequent generation of candidate solutions.

% Procedure: Abstract computation
% The algorithmic procedure summarizes the specifics of realizing a strategy as a systemized and parameterized computation. It outlines how the algorithm is organized in terms of the data structures and representations. The procedure may be described in terms of software engineering and computer science artifacts such as Pseudocode, design diagrams, and relevant mathematical equations.
\subsection{Procedure}
% What is the computational recipe for a technique?
% What are the data structures and representations used in a technique?
Algorithm~\ref{alg:genetic_algorithm} provides a pseudocode listing of the Genetic Algorithm for minimizing a cost function. 

\begin{algorithm}[htp]
	\SetLine  

	% data
	\SetKwData{Best}{$S_{best}$}
	\SetKwData{ProbabilityMutate}{$P_{mutation}$}
	\SetKwData{ProbabilityCrossover}{$P_{crossover}$}
	\SetKwData{Parents}{Parents}
	\SetKwData{Children}{Children}
	\SetKwData{ProblemSize}{$Problem_{size}$}
	\SetKwData{Population}{Population}
	\SetKwData{PopulationSize}{$Population_{size}$}
	\SetKwData{ParentOne}{$Parent_{1}$}
	\SetKwData{ParentTwo}{$Parent_{2}$}
	\SetKwData{ChildOne}{$Child_{1}$}
	\SetKwData{ChildTwo}{$Child_{2}$}	
	% functions
	\SetKwFunction{InitializePopulation}{InitializePopulation}  
	\SetKwFunction{EvaluatePopulation}{EvaluatePopulation} 
	\SetKwFunction{GetBestSolution}{GetBestSolution} 
	\SetKwFunction{SelectParents}{SelectParents}
	\SetKwFunction{Replace}{Replace}
	\SetKwFunction{StopCondition}{StopCondition}
	\SetKwFunction{Crossover}{Crossover}
	\SetKwFunction{Mutate}{Mutate}
  
	% I/O
	\KwIn{\PopulationSize, \ProblemSize, \ProbabilityCrossover, \ProbabilityMutate}		
	\KwOut{\Best}
  	% Algorithm
	% initialize	
	\Population $\leftarrow$ \InitializePopulation{\PopulationSize, \ProblemSize}\;
	% evaluate
	\EvaluatePopulation{\Population}\;
	% best
	\Best $\leftarrow$ \GetBestSolution{\Population}\;
	% loop
	\While{$\neg$\StopCondition{}} {
		% select
		\Parents $\leftarrow$ \SelectParents{\Population, \PopulationSize}\;
		% recombine
		\Children $\leftarrow 0$\;
		\ForEach{\ParentOne, \ParentTwo $\in$ \Parents}{
			\ChildOne, \ChildTwo $\leftarrow$ \Crossover{\ParentOne, \ParentTwo, \ProbabilityCrossover}\;
			\Children $\leftarrow$ \Mutate{\ChildOne, \ProbabilityMutate}\;
			\Children $\leftarrow$ \Mutate{\ChildTwo, \ProbabilityMutate}\;
		}
		% evaluate
		\EvaluatePopulation{\Children}\;
		% best
		\Best $\leftarrow$ \GetBestSolution{\Children}\;
		% replace
		\Population $\leftarrow$ \Replace{\Population, \Children}\;
	}
	\Return{\Best}\;
	% end
	\caption{Pseudocode for the Genetic Algorithm.}
	\label{alg:genetic_algorithm}
\end{algorithm}

% Heuristics: Usage guidelines
% The heuristics element describe the commonsense, best practice, and demonstrated rules for applying and configuring a parameterized algorithm. The heuristics relate to the technical details of the techniques procedure and data structures for general classes of application (neither specific implementations not specific problem instances). The heuristics are described textually, such as a series of guidelines in a bullet-point structure.
\subsection{Heuristics}
% What are the suggested configurations for a technique?
% What are the guidelines for the application of a technique to a problem instance?
\begin{itemize}
	\item Binary strings (referred to as `bitstrings') are the classical representation as they can be decoded to almost any desired representation. Real-valued and integer variables can be decoded using the binary coded decimal method, one's or two's complement methods, or the gray code method, the latter of which is generally preferred.
	\item Problem specific representations and customized genetic operators should be adopted, incorporating as much prior information about the problem domain as possible.
	\item The algorithm is highly-modular and a sub-field exists to study each sub-process, specifically: selection, recombination, mutation, and representation. 
	\item The Genetic Algorithm is most commonly used as an optimization technique, although it should also be considered a general adaptive strategy \cite{Jong1992}.
	\item The schema theorem is a classical explanation for the power of the Genetic Algorithm proposed by Holland  \cite{Holland1975}, and investigated by Goldberg under the name of the building block hypothesis \cite{Goldberg1989}.
	\item The size of the population must be large enough to provide sufficient coverage of the domain and mixing of the useful sub-components of the solution  \cite{Goldberg1992}.
	\item The Genetic Algorithm is classically configured with a high probability of recombination (such as 95\%-99\% of the selected population) and a low probability of mutation (such as $\frac{1}{L}$ where $L$ is the number of components in a solution) \cite{Muhlenbein1992, Back1993}.
	\item The fitness-proportionate selection of candidate solutions to contribute to the next generation should be neither too greedy (to avoid the takeover of fitter candidate solutions) nor too random.
\end{itemize}

% Code Listing
% The code description provides a minimal but functional version of the technique implemented with a programming language. The code description must be able to be typed into an appropriate computer, compiled or interpreted as need be, and provide a working execution of the technique. The technique implementation also includes a minimal problem instance to which it is applied, and both the problem and algorithm implementations are complete enough to demonstrate the techniques procedure. The description is presented as a programming source code listing.
\subsection{Code Listing}
% How is a technique implemented as an executable program?
% How is a technique applied to a concrete problem instance?
Listing~\ref{genetic_algorithm} provides an example of the Genetic Algorithm implemented in the Ruby Programming Language. 
% problem
The demonstration problem is a maximizing binary optimization problem called OneMax that seeks a binary string of unity (all `1' bits). The objective function provides only an indication of the number of correct bits in a candidate string, not the positions of the correct bits.

% algorithm
The Genetic Algorithm is implemented with a conservative configuration including binary tournament selection for the selection operator, uniform crossover for the recombination operator, and point mutations for the mutation operator.

% the listing
\lstinputlisting[firstline=7,language=ruby,caption=Genetic Algorithm in the Ruby Programming Language, label=genetic_algorithm]{../src/algorithms/evolutionary/genetic_algorithm.rb}

% References: Deeper understanding
% The references element description includes a listing of both primary sources of information about the technique as well as useful introductory sources for novices to gain a deeper understanding of the theory and application of the technique. The description consists of hand-selected reference material including books, peer reviewed conference papers, journal articles, and potentially websites. A bullet-pointed structure is suggested.
\subsection{References}
% What are the primary sources for a technique?
% What are the suggested reference sources for learning more about a technique?
% 
% Primary Sources
% 
\subsubsection{Primary Sources}
% adaptive plans - adaptive system framework
Holland is the grandfather of the field that became Genetic Algorithms. Holland investigated adaptive systems in the late 1960s proposing an adaptive system formalism and adaptive strategies referred to as `adaptive plans' \cite{Holland1962, Holland1962a, Holland1969}. 
% students
Holland's theoretical framework was investigated and elaborated by his Ph.D. students at the University of Michigan. Rosenberg investigated a chemical and molecular model of a biological inspired adaptive plan \cite{Rosenberg1967}. Bagley investigated meta-environments and a genetic adaptive plan referred to as a genetic algorithm applied to a simple game called hexapawn \cite{Bagley1967}. Cavicchio further elaborated the genetic adaptive plan by proposing numerous variations, referring to some as `reproductive plans' \cite{Cavicchio1970}. 

Other important contributions were made by Frantz who investigated what were referred to as genetic algorithms for search \cite{Frantz1972}, and Hollstien who investigated genetic plans for adaptive control and function optimization \cite{Hollstien1971}.
% de Jong
De Jong performed a seminal investigation of the genetic adaptive model (genetic plans) applied to continuous function optimization and his suite of test problems adopted are still commonly used \cite{Jong1975}.
% classical book
Holland wrote the the seminal book on his research focusing on the proposed adaptive systems formalism, the reproductive and genetic adaptive plans, and provided a theoretical framework for the mechanisms used and explanation for the capabilities of what would become genetic algorithms \cite{Holland1975}.


% 
% Learn More
% 
\subsubsection{Learn More}
% overview
The field of genetic algorithms is very large, resulting in large numbers of variations on the canonical technique.
% historical reviews
Goldberg provides a classical overview of the field in a review article \cite{Goldberg1994}, as does Mitchell \cite{Mitchell1995}.
% tutorial
Whitley describes a classical tutorial for the Genetic Algorithm covering both practical and theoretical concerns \cite{Whitley1994}.
% good introductions to the ga
The classical book on genetic algorithms as an optimization and machine learning technique was written by Goldberg and provides an in-depth review and practical study of the approach \cite{Goldberg1989}. Mitchell provides a contemporary reference text introducing the technique and the field \cite{Mitchell1998}. Finally, Goldberg provides a modern study of the field, the lessons learned, and reviews the broader toolset of optimization algorithms that the field has produced \cite{Goldberg2002}.


\putbib\end{bibunit}
\newpage\begin{bibunit}% The Clever Algorithms Project: http://www.CleverAlgorithms.com
% (c) Copyright 2010 Jason Brownlee. Some Rights Reserved. 
% This work is licensed under a Creative Commons Attribution-Noncommercial-Share Alike 2.5 Australia License.

% This is an algorithm description, see:
% Jason Brownlee. A Template for Standardized Algorithm Descriptions. Technical Report CA-TR-20100107-1, The Clever Algorithms Project http://www.CleverAlgorithms.com, January 2010.

% Name
% The algorithm name defines the canonical name used to refer to the technique, in addition to common aliases, abbreviations, and acronyms. The name is used in terms of the heading and sub-headings of an algorithm description.
\section{Genetic Programming} 
\label{sec:genetic_programming}
\index{Genetic Programming}

% other names
% What is the canonical name and common aliases for a technique?
% What are the common abbreviations and acronyms for a technique?
\emph{Genetic Programming, GP.}

% Taxonomy: Lineage and locality
% The algorithm taxonomy defines where a techniques fits into the field, both the specific subfields of Computational Intelligence and Biologically Inspired Computation as well as the broader field of Artificial Intelligence. The taxonomy also provides a context for determining the relation- ships between algorithms. The taxonomy may be described in terms of a series of relationship statements or pictorially as a venn diagram or a graph with hierarchical structure.
\subsection{Taxonomy}
% To what fields of study does a technique belong?
The Genetic Programming algorithm is an example of a Evolutionary Algorithm and belongs to the field of Evolutionary Computation and more broadly Computational Intelligence and Biologically Inspired Computation. 
% What are the closely related approaches to a technique?
The Genetic Programming algorithm is a sibling to other Evolutionary Algorithms such as the Genetic Algorithm (Section~\ref{sec:genetic_algorithm}), Evolution Strategies (Section~\ref{sec:evolution_strategies}), Evolutionary Programming (Section~\ref{sec:evolutionary_programming}), and Learning Classifier Systems (Section~\ref{sec:learning_classifier_system}). Technically, the Genetic Programming algorithm is an extension of the Genetic Algorithm. The Genetic Algorithm is a parent to a host of variations and extensions.

% Inspiration: Motivating system
% The inspiration describes the specific system or process that provoked the inception of the algorithm. The inspiring system may non-exclusively be natural, biological, physical, or social. The description of the inspiring system may include relevant domain specific theory, observation, nomenclature, and most important must include those salient attributes of the system that are somehow abstractly or conceptually manifest in the technique. The inspiration is described textually with citations and may include diagrams to highlight features and relationships within the inspiring system.
% Optional
\subsection{Inspiration}
% What is the system or process that motivated the development of a technique?
% Which features of the motivating system are relevant to a technique?
The Genetic Programming algorithm is inspired by population genetics (including heredity and gene frequencies), and evolution at the population level, as well as the Mendelian understanding of the structure (such as chromosomes, genes, alleles) and mechanisms (such as recombination and mutation). This is the so-called new or modern synthesis of evolutionary biology. 

% Metaphor: Explanation via analogy
% The metaphor is a description of the technique in the context of the inspiring system or a different suitable system. The features of the technique are made apparent through an analogous description of the features of the inspiring system. The explanation through analogy is not expected to be literal scientific truth, rather the method is used as an allegorical communication tool. The inspiring system is not explicitly described, this is the role of the ‘inspiration’ element, which represents a loose dependency for this element. The explanation is textual and uses the nomenclature of the metaphorical system.
% Optional
\subsection{Metaphor}
% What is the explanation of a technique in the context of the inspiring system?
% What are the functionalities inferred for a technique from the analogous inspiring system?
Individuals of a population contribute their genetic material (called the genotype) proportional to their suitability of their expressed genome (called their phenotype) to their environment. The next generation is created through a process of mating that involves genetic operators such as recombination of two individuals genomes in the population and the introduction of random copying errors (called mutation). This iterative process may result in an improved adaptive-fit between the phenotypes of individuals in a population and the environment.

% advanced
Programs may be evolved and used in a secondary adaptive process, where an assessment of candidates at the end of that secondary adaptive process is used for differential reproductive success in the first evolutionary process. This system may be understood as the inter-dependencies experienced in evolutionary development where evolution operates upon an embryo that in turn develops into an individual in an environment that eventually may reproduce.

% Strategy: Problem solving plan
% The strategy is an abstract description of the computational model. The strategy describes the information processing actions a technique shall take in order to achieve an objective. The strategy provides a logical separation between a computational realization (procedure) and a analogous system (metaphor). A given problem solving strategy may be realized as one of a number specific algorithms or problem solving systems. The strategy description is textual using information processing and algorithmic terminology.
\subsection{Strategy}
% What is the information processing objective of a technique?
The objective of the Genetic Programming algorithm is to use induction to devise a computer program. 
% What is a techniques plan of action?
This is achieved by using evolutionary operators on candidate programs with a tree structure to improve the adaptive fit between the population of candidate programs and an objective function. An assessment of a candidate solution involves its execution.

% Procedure: Abstract computation
% The algorithmic procedure summarizes the specifics of realizing a strategy as a systemized and parameterized computation. It outlines how the algorithm is organized in terms of the data structures and representations. The procedure may be described in terms of software engineering and computer science artifacts such as pseudo code, design diagrams, and relevant mathematical equations.
\subsection{Procedure}
% What is the computational recipe for a technique?
% What are the data structures and representations used in a technique?
Algorithm~\ref{alg:genetic_programming} provides a pseudo-code listing of the Genetic Programming algorithm for minimizing a cost function, based on Koza and Poli's tutorial \cite{Koza2005}. 

% more
The Genetic Program uses LISP-like symbolic expressions called S-expressions that represent the graph of a program with function nodes and terminal nodes. While the algorithm is running, the programs are treated like data, and when they are evaluated they are executed. The traversal of a program graph is always depth first, and functions must always return a value.

\begin{algorithm}[htp]
	\SetLine  

	% data
	\SetKwData{Best}{$S_{best}$}
	\SetKwData{ProbabilityMutate}{$P_{mutation}$}
	\SetKwData{ProbabilityCrossover}{$P_{crossover}$}
	\SetKwData{ProbabilityReproduction}{$P_{reproduction}$}
	\SetKwData{ProbabilityAlteration}{$P_{alteration}$}
	\SetKwData{Parents}{Parents}
	\SetKwData{Children}{Children}
	\SetKwData{FunctionNodes}{$nodes_{func}$}
	\SetKwData{TerminalNodes}{$nodes_{term}$}
	\SetKwData{Population}{Population}
	\SetKwData{PopulationSize}{$Population_{size}$}
	\SetKwData{ParentOne}{$Parent_{1}$}
	\SetKwData{ParentTwo}{$Parent_{2}$}
	\SetKwData{ChildOne}{$Child_{1}$}
	\SetKwData{ChildTwo}{$Child_{2}$}
	\SetKwData{Operator}{Operator}	
	\SetKwData{CrossoverOperator}{CrossoverOperator}
	\SetKwData{MutationOperator}{MutationOperator}
	\SetKwData{ReproductionOperator}{ReproductionOperator}
	\SetKwData{AlterationOperator}{AlterationOperator}
	% functions
	\SetKwFunction{InitializePopulation}{InitializePopulation}  
	\SetKwFunction{EvaluatePopulation}{EvaluatePopulation} 
	\SetKwFunction{GetBestSolution}{GetBestSolution} 
	\SetKwFunction{SelectParents}{SelectParents}
	\SetKwFunction{Replace}{Replace}
	\SetKwFunction{StopCondition}{StopCondition}
	\SetKwFunction{Crossover}{Crossover}
	\SetKwFunction{Mutate}{Mutate}
	\SetKwFunction{Size}{Size}
	\SetKwFunction{SelectGeneticOperator}{SelectGeneticOperator}
  	\SetKwFunction{Reproduce}{Reproduce}
	\SetKwFunction{AlterArchitecture}{AlterArchitecture}

	% I/O
	\KwIn{\PopulationSize, \FunctionNodes, \TerminalNodes, \ProbabilityCrossover, \ProbabilityMutate, \ProbabilityReproduction, \ProbabilityAlteration}		
	\KwOut{\Best}
  	% Algorithm
	% initialize	
	\Population $\leftarrow$ \InitializePopulation{\PopulationSize, \FunctionNodes, \TerminalNodes}\;
	% evaluate
	\EvaluatePopulation{\Population}\;
	% best
	\Best $\leftarrow$ \GetBestSolution{\Population}\;
	% loop
	\While{$\neg$\StopCondition{}} {
		% build up child generation
		\Children $\leftarrow 0$\;
		\While{\Size{\Children} $<$ \PopulationSize} {
			\Operator $\leftarrow$ \SelectGeneticOperator{\ProbabilityCrossover, \ProbabilityMutate, \ProbabilityReproduction, \ProbabilityAlteration}\;
		
			\If{\Operator $\equiv$ \CrossoverOperator}{
				\ParentOne, \ParentTwo $\leftarrow$ \SelectParents{\Population, \PopulationSize}\;
				\ChildOne, \ChildTwo $\leftarrow$ \Crossover{\ParentOne, \ParentTwo}\;
				\Children $\leftarrow$ \ChildOne\;
				\Children $\leftarrow$ \ChildTwo\;
			}\ElseIf{\Operator $\equiv$ \MutationOperator}{
				\ParentOne $\leftarrow$ \SelectParents{\Population, \PopulationSize}\;
				\ChildOne $\leftarrow$ \Mutate{\ParentOne}\;
				\Children $\leftarrow$ \ChildOne\;
			}\ElseIf{\Operator $\equiv$ \ReproductionOperator}{
				\ParentOne $\leftarrow$ \SelectParents{\Population, \PopulationSize}\;
				\ChildOne $\leftarrow$ \Reproduce{\ParentOne}\;
				\Children $\leftarrow$ \ChildOne\;
			}\ElseIf{\Operator $\equiv$ \AlterationOperator}{
				\ParentOne $\leftarrow$ \SelectParents{\Population, \PopulationSize}\;
				\ChildOne $\leftarrow$ \AlterArchitecture{\ParentOne}\;
				\Children $\leftarrow$ \ChildOne\;
			}
		}
		% evaluate
		\EvaluatePopulation{\Children}\;
		% best
		\Best $\leftarrow$ \GetBestSolution{\Children, \Best}\;
		% replace
		\Population $\leftarrow$ \Children\;
	}
	\Return{\Best}\;
	% end
	\caption{Pseudo Code for the Genetic Programming algorithm.}
	\label{alg:genetic_programming}
\end{algorithm}


% Heuristics: Usage guidelines
% The heuristics element describe the commonsense, best practice, and demonstrated rules for applying and configuring a parameterized algorithm. The heuristics relate to the technical details of the techniques procedure and data structures for general classes of application (neither specific implementations not specific problem instances). The heuristics are described textually, such as a series of guidelines in a bullet-point structure.
\subsection{Heuristics}
% What are the suggested configurations for a technique?
% What are the guidelines for the application of a technique to a problem instance?
\begin{itemize}
	\item The Genetic Programming algorithm was designed for inductive automatic programming and is well suited to symbolic regression, controller design, and machine learning tasks under the broader name of function approximation.
	\item Traditionally Lisp symbolic expressions are evolved and evaluated in a virtual machine, although the approach has been applied with compiled programming languages.
	\item The evaluation (fitness assignment) of a candidate solution typically takes the structure of the program into account, rewarding parsimony.
	\item The selection process should be balanced between random selection and greedy selection to bias the search towards fitter candidate solutions (exploitation), whilst promoting useful diversity into the population (exploration).
	\item A program may respond to zero or more input values and may produce one or more outputs.
	\item All functions used in the function node set must return a usable result. For example, the division function must return a sensible value (such as zero or one) when a division by zero occurs.
	\item All genetic operations ensure (or should ensure) that syntactically valid and executable programs are produced as a result of their application.
	\item The Genetic Programming algorithm is commonly configured with a high-probability of crossover ($\geq 90\%$) and a low-probability of mutation ($\leq 1\%$). Other operators such as reproduction and architecture alterations are used with moderate-level probabilities and fill in the probabilistic gap.
	\item Architecture altering operations are not limited to the duplication and deletion of sub-structures of a given program. 
	\item The crossover genetic operator in the algorithm is commonly configured to select a function as a the cross-point with a high-probability ($\geq 90\%$) and low-probability of selecting a terminal as a cross-point ($\leq 10\%$).
	\item The function set may also include control structures such as conditional statements and loop constructs.
	\item The Genetic Programing algorithm can be realized as a stack-based virtual machine as opposed to a call graph \cite{Perkis1994}.
	\item The Genetic Programming algorithm can make use of Automatically Defined Functions (ADFs) that are sub-graphs and are promoted to the status of functions for reuse and are co-evolved with the programs. 
	 \item The genetic operators employed during reproduction in the algorithm may be considered transformation programs for candidate solutions and may themselves be co-evolved in the algorithm \cite{Angeline1996}.
\end{itemize}

% The code description provides a minimal but functional version of the technique implemented with a programming language. The code description must be able to be typed into an appropriate computer, compiled or interpreted as need be, and provide a working execution of the technique. The technique implementation also includes a minimal problem instance to which it is applied, and both the problem and algorithm implementations are complete enough to demonstrate the techniques procedure. The description is presented as a programming source code listing.
\subsection{Code Listing}
% How is a technique implemented as an executable program?
% How is a technique applied to a concrete problem instance?
Listing~\ref{genetic_programming} provides an example of the Genetic Programming algorithm implemented in the Ruby Programming Language based on Koza and Poli's tutorial \cite{Koza2005}.
 
% problem
The demonstration problem is an instance of a symbolic regression, where a function must be devised to match a set of observations. In this case the target function is a quadratic polynomial $x^2+x+1$ where $x \in [-1,1]$. The observations are generated directly from the target function without noise for the purposes of this example. In practical problems, if one knew and had access to the target function then the genetic program would not be required.

% algorithm
The algorithm is configured to search for a program with the function set $\{ +, -, \times, \div \}$ and the terminal set $\{ X, R \}$, where $X$ is the input value, and $R$ is a static random variable generated for a program $X \in [-5,5]$. A division by zero returns a value of one. 
% fitness
The fitness of a candidate solution is calculated by evaluating the program on range of random input values and calculating the Root Mean Squared Error (RMSE).
% config
The algorithm is configured with a 90\% probability of crossover, 8\% probability of reproduction (copying), and a 2\% probability of mutation.
% missing
For brevity, the algorithm does not implement the architecture altering genetic operation and does not bias crossover points towards functions over terminals.

% the listing
\lstinputlisting[firstline=7,language=ruby,caption=Genetic Programming algorithm in the Ruby Programming Language, label=genetic_programming]{../src/algorithms/evolutionary/genetic_programming.rb}

% References: Deeper understanding
% The references element description includes a listing of both primary sources of information about the technique as well as useful introductory sources for novices to gain a deeper understanding of the theory and application of the technique. The description consists of hand-selected reference material including books, peer reviewed conference papers, journal articles, and potentially websites. A bullet-pointed structure is suggested.
\subsection{References}
% What are the primary sources for a technique?
% What are the suggested reference sources for learning more about a technique?

% 
% Primary Sources
% 
\subsubsection{Primary Sources}
% early work
An early work by Cramer involved the study of a Genetic Algorithm using an expression tree structure for representing computer programs for primitive mathematical operations \cite{Cramer1985}.
% koza
Koza is credited with the development of the field of Genetic Programming.
% early koza
An early paper by Koza referred to his hierarchical genetic algorithms as an extension to the simple genetic algorithm that use symbolic expressions (S-expressions) as a representation and were applied to a range of induction-style problems \cite{Koza1989}.
% seminal
The seminal reference for the field is Koza's 1992 book on Genetic Programming \cite{Koza1992}.


% 
% Learn More
% 
\subsubsection{Learn More}
% scope
The field of Genetic Programming is vast, including many books, dedicated conferences and uncounted thousands of publications.
% koza
Koza is generally credited with the development and popularizing of the field, publishing a large number of books and papers himself. 
% intro papers 
Koza provides a practical introduction to the field as a tutorial and provides recent overview of the broader field and usage of the technique \cite{Koza2005}.

% books
In addition his the seminal 1992 book, Koza has released three more volumes in the series including volume II on Automatically Defined Functions (ADFs) \cite{Koza1994}, volume III that considered the Genetic Programming Problem Solver (GPPS) for automatically defining the function set and program structure for a given problem \cite{Koza1999}, and volume IV that focuses on the human competitive results the technique is able to achieve in a routine manner \cite{Koza2003}. All books are rich with targeted and practical demonstration problem instances.

% additional books
Some additional excellent books include a text by Banzhaf et al. that provides an introduction to the field \cite{Banzhaf1998}, Langdon and Poli's detailed look at the technique \cite{Langdon2002}, and Poli, Langdon, and McPhee's contemporary and practical field guide to Genetic Programming \cite{Poli2008}.


\putbib\end{bibunit}
\newpage\begin{bibunit}% The Clever Algorithms Project: http://www.CleverAlgorithms.com
% (c) Copyright 2010 Jason Brownlee. Some Rights Reserved. 
% This work is licensed under a Creative Commons Attribution-Noncommercial-Share Alike 2.5 Australia License.

% This is an algorithm description, see:
% Jason Brownlee. A Template for Standardized Algorithm Descriptions. Technical Report CA-TR-20100107-1, The Clever Algorithms Project http://www.CleverAlgorithms.com, January 2010.

% Name
% The algorithm name defines the canonical name used to refer to the technique, in addition to common aliases, abbreviations, and acronyms. The name is used in terms of the heading and sub-headings of an algorithm description.
\section{Evolution Strategies} 
\label{sec:evolution_strategies}
\index{Evolution Strategies}

% other names
% What is the canonical name and common aliases for a technique?
% What are the common abbreviations and acronyms for a technique?
\emph{Evolution Strategies, Evolution Strategy, Evolutionary Strategies, ES.}

% Taxonomy: Lineage and locality
% The algorithm taxonomy defines where a techniques fits into the field, both the specific subfields of Computational Intelligence and Biologically Inspired Computation as well as the broader field of Artificial Intelligence. The taxonomy also provides a context for determining the relation- ships between algorithms. The taxonomy may be described in terms of a series of relationship statements or pictorially as a venn diagram or a graph with hierarchical structure.
\subsection{Taxonomy}
% To what fields of study does a technique belong?
Evolution Strategies is a global optimization algorithm and is an instance of an Evolutionary Algorithm from the field of Evolutionary Computation.
% What are the closely related approaches to a technique?
Evolution Strategies is a sibling technique to other Evolutionary Algorithms such as Genetic Algorithms (Section~\ref{sec:genetic_algorithm}), Genetic Programming (Section~\ref{sec:genetic_programming}), Learning Classifier Systems (Section~\ref{sec:learning_classifier_system}), and Evolutionary Programming (Section~\ref{sec:evolutionary_programming}). A popular descendant of the Evolution Strategies algorithm is the Covariance Matrix Adaptation Evolution Strategies (CMA-ES).

% Inspiration: Motivating system
% The inspiration describes the specific system or process that provoked the inception of the algorithm. The inspiring system may non-exclusively be natural, biological, physical, or social. The description of the inspiring system may include relevant domain specific theory, observation, nomenclature, and most important must include those salient attributes of the system that are somehow abstractly or conceptually manifest in the technique. The inspiration is described textually with citations and may include diagrams to highlight features and relationships within the inspiring system.
% Optional
\subsection{Inspiration}
% What is the system or process that motivated the development of a technique?
Evolution Strategies is inspired by the theory of evolution by means of natural selection.
% Which features of the motivating system are relevant to a technique?
Specifically, the technique is inspired by macro-level or the species-level process of evolution (phenotype, hereditary, variation) and is not concerned with the genetic mechanisms of evolution (genome, chromosomes, genes, alleles).

% Metaphor: Explanation via analogy
% The metaphor is a description of the technique in the context of the inspiring system or a different suitable system. The features of the technique are made apparent through an analogous description of the features of the inspiring system. The explanation through analogy is not expected to be literal scientific truth, rather the method is used as an allegorical communication tool. The inspiring system is not explicitly described, this is the role of the ‘inspiration’ element, which represents a loose dependency for this element. The explanation is textual and uses the nomenclature of the metaphorical system.
% Optional
% \subsection{Metaphor}
% What is the explanation of a technique in the context of the inspiring system?
% What are the functionalities inferred for a technique from the analogous inspiring system?
% Evolution Strategies only briefly flirted with explanation via metaphor, and is less preferred to grounded probabilistic explanations.

% Strategy: Problem solving plan
% The strategy is an abstract description of the computational model. The strategy describes the information processing actions a technique shall take in order to achieve an objective. The strategy provides a logical separation between a computational realization (procedure) and a analogous system (metaphor). A given problem solving strategy may be realized as one of a number specific algorithms or problem solving systems. The strategy description is textual using information processing and algorithmic terminology.
\subsection{Strategy}
% What is the information processing objective of a technique?
The objective of the Evolution Strategies algorithm is to maximize the suitability of collection of candidate solutions in the context of an objective function from a domain.
% What is a techniques plan of action?
The objective was classically achieved through the adoption of dynamic variation, a surrogate for descent with modification, where the amount of variation was adapted dynamically with performance-based heuristics. Contemporary approaches co-adapt parameters that control the amount and bias of variation with the candidate solutions.

% Procedure: Abstract computation
% The algorithmic procedure summarizes the specifics of realizing a strategy as a systemized and parameterized computation. It outlines how the algorithm is organized in terms of the data structures and representations. The procedure may be described in terms of software engineering and computer science artifacts such as pseudo code, design diagrams, and relevant mathematical equations.
\subsection{Procedure}
% What is the computational recipe for a technique?
% What are the data structures and representations used in a technique?
% terminology
Instances of Evolution Strategies algorithms may be concisely described with a custom terminology in the form $(\mu,\lambda)-ES$, where $\mu$ is number of candidate solutions in the parent generation, and $\lambda$ is the number of candidate solutions generated from the parent generation. In this configuration, the best $\mu$ are kept if $\lambda > \mu$, where $\lambda$ must be great or equal to $\mu$. In addition to the so-called comma-selection Evolution Strategies algorithm, a plus-selection variation may be defined $(\mu + \lambda)-ES$, where the best members of the union of the $\mu$ and $\lambda$ generations compete based on objective fitness for a position in the next generation. The simplest configuration is the $(1+1)-ES$, which is a type of greedy hill climbing algorithm.
% algo
Algorithm~\ref{alg:evolution_strategies} provides a pseudocode listing of the $(\mu,\lambda)-ES$ algorithm for minimizing a cost function. The algorithm shows the adaptation of candidate solutions that co-adapt their own strategy parameters that influence the amount of mutation applied to a candidate solutions descendants. 

\begin{algorithm}[htp]
	\SetLine  
	% data
	\SetKwData{Best}{$S_{best}$}
	\SetKwData{Children}{Children}
	\SetKwData{ProblemSize}{ProblemSize}
	\SetKwData{Population}{Population}
	\SetKwData{PopulationSize}{$\mu$}
	\SetKwData{ChildPopulationSize}{$\lambda$}
	\SetKwData{Parent}{$Parent_{i}$}
	\SetKwData{Candidate}{$S_{i}$}
	\SetKwData{CandidateObjective}{$Si_{problem}$}
	\SetKwData{CandidateStrategy}{$Si_{strategy}$}
	\SetKwData{ParentObjective}{$Pi_{problem}$}
	\SetKwData{ParentStrategy}{$Pi_{strategy}$}
	% functions
	\SetKwFunction{InitializePopulation}{InitializePopulation}  
	\SetKwFunction{EvaluatePopulation}{EvaluatePopulation} 
	\SetKwFunction{SelectBest}{SelectBest} 
	\SetKwFunction{GetBest}{GetBest}
	\SetKwFunction{StopCondition}{StopCondition}
	\SetKwFunction{Mutate}{Mutate}
  \SetKwFunction{RandomSelection}{RandomSelection}
	\SetKwFunction{Cost}{Cost}
	\SetKwFunction{Size}{Size}
	\SetKwFunction{GetParent}{GetParent}
	% I/O
	\KwIn{\PopulationSize, \ChildPopulationSize, \ProblemSize}		
	\KwOut{\Best}
  	% Algorithm
	% initialize	
	\Population $\leftarrow$ \InitializePopulation{\PopulationSize, \ProblemSize}\;
	% evaluate
	\EvaluatePopulation{\Population}\;
	% best
	\Best $\leftarrow$ \GetBest{\Population, 1}\;
	% loop
	\While{$\neg$\StopCondition{}} {
		% mutate
		\Children $\leftarrow \emptyset$\;
		\For{$i=0$ \KwTo \ChildPopulationSize} {
			\Parent $\leftarrow$ \GetParent{\Population, $i$}\;	
			\Candidate $\leftarrow \emptyset$\;			
			\CandidateObjective $\leftarrow$ \Mutate{\ParentObjective, \ParentStrategy}\;
			\CandidateStrategy $\leftarrow$ \Mutate{\ParentStrategy}\;
			\Children $\leftarrow$ \Candidate\;
		}
		\EvaluatePopulation{\Children}\;
		% best
		\Best $\leftarrow$ \GetBest{\Children$+$\Best, 1}\;
		% replace
		\Population $\leftarrow$ \SelectBest{\Population, \Children, \PopulationSize}\;
	}
	\Return{\Best}\;
	% end
	\caption{Pseudocode for $(\mu,\lambda)$ Evolution Strategies.}
	\label{alg:evolution_strategies}
\end{algorithm}

% Heuristics: Usage guidelines
% The heuristics element describe the commonsense, best practice, and demonstrated rules for applying and configuring a parameterized algorithm. The heuristics relate to the technical details of the techniques procedure and data structures for general classes of application (neither specific implementations not specific problem instances). The heuristics are described textually, such as a series of guidelines in a bullet-point structure.
\subsection{Heuristics}
% What are the suggested configurations for a technique?
% What are the guidelines for the application of a technique to a problem instance?
\begin{itemize}
	\item Evolution Strategies uses problem specific representations, such as real values for continuous function optimization.
	\item The algorithm is commonly configured such that $1 < \mu < \lambda < \infty$.
	\item The ratio of $\mu$ to $\lambda$ influences the amount of selection pressure (greediness) exerted by the algorithm.
	\item A contemporary update to the algorithms notation includes a $\rho$ as $(\mu/\rho,\lambda)-ES$ that specifies the number of parents that will contribute to each new candidate solution using a recombination operator. 
	\item A classical rule used to govern the amount of mutation (standard deviation used in mutation for continuous function optimization) was the $\frac{1}{5}$-rule, where the ratio of successful mutations should be $\frac{1}{5}$ of all mutations. If it is greater the variance is increased, otherwise if the ratio is is less, the variance is decreased.
	\item The comma-selection variation of the algorithm can be good for dynamic problem instances given its capability for continued exploration of the search space, whereas the plus-selection variation can be good for refinement and convergence.
\end{itemize}

% The code description provides a minimal but functional version of the technique implemented with a programming language. The code description must be able to be typed into an appropriate computer, compiled or interpreted as need be, and provide a working execution of the technique. The technique implementation also includes a minimal problem instance to which it is applied, and both the problem and algorithm implementations are complete enough to demonstrate the techniques procedure. The description is presented as a programming source code listing.
\subsection{Code Listing}
% How is a technique implemented as an executable program?
% How is a technique applied to a concrete problem instance?
Listing~\ref{evolution_strategies} provides an example of the Evolution Strategies algorithm implemented in the Ruby Programming Language.
% problem
The demonstration problem is an instance of a continuous function optimization that seeks $\min f(x)$ where $f=\sum_{i=1}^n x_{i}^2$, $-5.0\leq x_i \leq 5.0$ and $n=2$. The optimal solution for this basin function is $(v_0,\ldots,v_{n-1})=0.0$.
% algorithm
The algorithm is a implementation of Evolution Strategies based on simple version described by B\"ack and Schwefel \cite{Back1993b}, which was also used as the basis of a detailed empirical study \cite{Yao1997}.
%  cfg
The algorithm is an $(30+20)-ES$ that adapts both the problem and strategy (standard deviations) variables. 
% a good guide
More contemporary implementations may modify the strategy variables differently, and include an additional set of adapted strategy parameters to influence the direction of mutation (see \cite{Rudolph2000} for a concise description).

% the listing
\lstinputlisting[firstline=7,language=ruby,caption=Evolution Strategies algorithm in the Ruby Programming Language, label=evolution_strategies]{../src/algorithms/evolutionary/evolution_strategies.rb}


% References: Deeper understanding
% The references element description includes a listing of both primary sources of information about the technique as well as useful introductory sources for novices to gain a deeper understanding of the theory and application of the technique. The description consists of hand-selected reference material including books, peer reviewed conference papers, journal articles, and potentially websites. A bullet-pointed structure is suggested.
\subsection{References}
% What are the primary sources for a technique?
% What are the suggested reference sources for learning more about a technique?

% 
% Primary Sources
% 
\subsubsection{Primary Sources}
% background
Evolution Strategies was developed by three students (Bienert, Rechenberg, Schwefel) at the Technical University in Berlin in 1964 in an effort to robotically optimize an aerodynamics design problem.
% seminal
The seminal work in Evolution Strategies was Rechenberg's PhD thesis  \cite{Rechenberg1971} that was later published as a book \cite{Rechenberg1973}, both in German.
% papers
Many technical reports and papers were published by Schwefel and Rechenberg, although the seminal paper published in English was by Klockgether and Schwefel on the two--phase nozzle design problem \cite{Klockgether1970}.

% Learn More
% 
\subsubsection{Learn More}
% Schwefel
Schwefel published his PhD dissertation \cite{Schwefel1975} not long after Rechenberg, which was also published as a book \cite{Schwefel1977}, both in German. Schwefel's book was later translated into English and represents a classical reference for the technique \cite{Schwefel1981}. 
% Back
B\"ack et al.\ provide a classical introduction to the technique, covering the history, development of the algorithm, and the steps that lead it to where it was in 1991 \cite{Back1991}.
% Beyer and Schwefel
Beyer and Schwefel provide a contemporary introduction to the field that includes a detailed history of the approach, the developments and improvements since its inception, and an overview of the theoretical findings that have been made \cite{Beyer2002}.
\putbib\end{bibunit}
\newpage\begin{bibunit}% The Clever Algorithms Project: http://www.CleverAlgorithms.com
% (c) Copyright 2010 Jason Brownlee. Some Rights Reserved. 
% This work is licensed under a Creative Commons Attribution-Noncommercial-Share Alike 2.5 Australia License.

% This is an algorithm description, see:
% Jason Brownlee. A Template for Standardized Algorithm Descriptions. Technical Report CA-TR-20100107-1, The Clever Algorithms Project http://www.CleverAlgorithms.com, January 2010.

% Name
% The algorithm name defines the canonical name used to refer to the technique, in addition to common aliases, abbreviations, and acronyms. The name is used in terms of the heading and sub-headings of an algorithm description.
\section{Differential Evolution} 
\label{sec:differential_evolution}
\index{Differential Evolution}

% other names
% What is the canonical name and common aliases for a technique?
% What are the common abbreviations and acronyms for a technique?
\emph{Differential Evolution, DE.}

% Taxonomy: Lineage and locality
% The algorithm taxonomy defines where a techniques fits into the field, both the specific subfields of Computational Intelligence and Biologically Inspired Computation as well as the broader field of Artificial Intelligence. The taxonomy also provides a context for determining the relation- ships between algorithms. The taxonomy may be described in terms of a series of relationship statements or pictorially as a venn diagram or a graph with hierarchical structure.
\subsection{Taxonomy}
% To what fields of study does a technique belong?
Differential Evolution is a Stochastic Direct Search and Global Optimization algorithm, and is an instance of an Evolutionary Algorithm from the field of Evolutionary Computation.
% What are the closely related approaches to a technique?
It is related to sibling Evolutionary Algorithms such as the Genetic Algorithm and Evolutionary Programming, and Evolution Strategies, and shows some similarities to Particle Swarm Optimization.

% Strategy: Problem solving plan
% The strategy is an abstract description of the computational model. The strategy describes the information processing actions a technique shall take in order to achieve an objective. The strategy provides a logical separation between a computational realization (procedure) and a analogous system (metaphor). A given problem solving strategy may be realized as one of a number specific algorithms or problem solving systems. The strategy description is textual using information processing and algorithmic terminology.
\subsection{Strategy}
% What is the information processing objective of a technique?
% What is a techniques plan of action?
The Differential Evolution algorithm involves maintaining a population of candidate solutions subjected to iterations of recombination, evaluation, and selection. The recombination approach involves the creation of new candidate solution components based on the weighted difference between two randomly selected population members added to a third population member. This perturbs population members relative to the spread of the broader population. In conjunction with selection, the perturbation effect self-organizes the sampling of the problem space, bounding it to known areas of interest.

% Procedure: Abstract computation
% The algorithmic procedure summarizes the specifics of realizing a strategy as a systemized and parameterized computation. It outlines how the algorithm is organized in terms of the data structures and representations. The procedure may be described in terms of software engineering and computer science artifacts such as pseudo code, design diagrams, and relevant mathematical equations.
\subsection{Procedure}
% What are the data structures and representations used in a technique?
Differential Evolution has a specialized nomenclature that describes the adopted configuration. This takes the form of DE$/x/y/z$, where $x$ represents the solution to be perturbed (such a random or best). The $y$ signifies the number of difference vectors used in the perturbation of $x$, where a difference vectors is the difference between two randomly selected although distinct members of the population. Finally, $z$ signifies the recombination operator performed such as bin for binomial and exp for exponential. 

% What is the computational recipe for a technique?
Algorithm~\ref{alg:differential_evolution} provides a pseudo-code listing of the Differential Evolution algorithm for minimizing a cost function, specifically a DE/rand/1/bin configuration. Algorithm~\ref{alg:new_sample} provides a pseudo-code listing of the $NewSample$ function from the Differential Evolution algorithm.

\begin{algorithm}[htp]
	\SetLine  

	% data
	\SetKwData{Best}{$S_{best}$}
	\SetKwData{Member}{$P_{i}$}
	\SetKwData{Sample}{$S_{i}$}
	\SetKwData{Population}{Population}
	\SetKwData{NewPopulation}{NewPopulation}	
	\SetKwData{PopulationSize}{G}
	\SetKwData{NumParameters}{NP}
	\SetKwData{WeightingFactor}{F}
	\SetKwData{CrossoverRate}{CR}
	
	% functions
	\SetKwFunction{StopCondition}{StopCondition}
	\SetKwFunction{InitializePopulation}{InitializePopulation}
	\SetKwFunction{EvaluateCost}{EvaluateCost}
	\SetKwFunction{GetBestSolution}{GetBestSolution}
	\SetKwFunction{NewSample}{NewSample}
	\SetKwFunction{Cost}{Cost}
  
	% I/O
	\KwIn{\PopulationSize, \NumParameters, \WeightingFactor, \CrossoverRate}		
	\KwOut{\Best}
  	% Algorithm
	% initialize	
	\Population $\leftarrow$ \InitializePopulation{\PopulationSize, \NumParameters}\;
	% evaluate
	\EvaluateCost{\Population}\;
	% best
	\Best $\leftarrow$ \GetBestSolution{\Population}\;
	% loop
	\While{$\neg$\StopCondition{}} {
		\NewPopulation $\leftarrow 0$\;
		% create new 
		\ForEach{\Member $\in$ \Population}{
			\Sample $\leftarrow$ \NewSample{\Member, \Population, \NumParameters, \WeightingFactor, \CrossoverRate}\;
			\eIf{\Cost{\Sample} $\leq$ \Cost{\Member}}{
				\NewPopulation $\leftarrow$ \Sample\;
			}
			{
				\NewPopulation $\leftarrow$ \Member\;
			}
		}
		\Population $\leftarrow$ \NewPopulation\;
		% eval
		\EvaluateCost{\Population}\;
		\Best $\leftarrow$ \GetBestSolution{\Population}\;
	}
	\Return{\Best}\;
	% end
	\caption{Pseudo Code for the Differential Evolution algorithm.}
	\label{alg:differential_evolution}
\end{algorithm}


\begin{algorithm}[htp]
	\SetLine  

	% data
	\SetKwData{Membera}{$P_{0}$}
	\SetKwData{Memberb}{$P_{1}$}
	\SetKwData{Memberc}{$P_{2}$}
	\SetKwData{Memberd}{$P_{3}$}
	\SetKwData{Memberai}{$P0_{i}$}
	\SetKwData{Memberbi}{$P1_{i}$}
	\SetKwData{Memberci}{$P2_{i}$}
	\SetKwData{Memberdi}{$P3_{i}$}
	\SetKwData{Sample}{$S$}
	\SetKwData{SampleVar}{$S_{i}$}
	\SetKwData{Population}{Population}
	\SetKwData{NumParameters}{NP}
	\SetKwData{WeightingFactor}{F}
	\SetKwData{CrossoverRate}{CR}
	\SetKwData{CutPoint}{CutPoint}
	
	% functions
	\SetKwFunction{RandomMemeber}{RandomMemeber}
	\SetKwFunction{RandomPosition}{RandomPosition}
	\SetKwFunction{Rand}{Rand}
  
	% I/O
	\KwIn{\Membera, \Population, \NumParameters, \WeightingFactor, \CrossoverRate}		
	\KwOut{\Sample}
  	% Algorithm
	\Repeat{\Memberb $\neq$ \Membera}{\Memberb $\leftarrow$ \RandomMemeber{\Population}\;}
	\Repeat{\Memberc $\neq$ \Membera $\vee$ \Memberc $\neq$ \Memberb}{\Memberc $\leftarrow$ \RandomMemeber{\Population}\;}
	\Repeat{\Memberd $\neq$ \Membera $\vee$ \Memberd $\neq$ \Memberb $\vee$ \Memberd $\neq$ \Memberc}{\Memberd $\leftarrow$ \RandomMemeber{\Population}\;}
	\CutPoint $\leftarrow$ \RandomPosition{\NumParameters}\;
	\Sample $\leftarrow0$\;
	\For{$i$ $\KwTo$ \NumParameters}{
		\eIf{$i \equiv$  \CutPoint $\wedge$ \Rand{} $<$ \CrossoverRate}{
			\SampleVar $\leftarrow$ \Memberdi + \WeightingFactor $\times$ (\Memberbi - \Memberci)\;
		}{
			\SampleVar $\leftarrow$ \Memberai\;
		}
	}
	
	\Return{\Sample}\;
	% end
	\caption{Pseudo Code for the NewSample function in the Differential Evolution algorithm.}
	\label{alg:new_sample}
\end{algorithm}

% Heuristics: Usage guidelines
% The heuristics element describe the commonsense, best practice, and demonstrated rules for applying and configuring a parameterized algorithm. The heuristics relate to the technical details of the techniques procedure and data structures for general classes of application (neither specific implementations not specific problem instances). The heuristics are described textually, such as a series of guidelines in a bullet-point structure.
\subsection{Heuristics}
% What are the suggested configurations for a technique?
% What are the guidelines for the application of a technique to a problem instance?
\begin{itemize}
	\item Differential evolution was designed for nonlinear, non-differentiable continuous function optimization.
	\item The weighting factor $F \in [0,2]$ controls the amplification of differential variation, a value of 0.8 is suggested.
	\item the crossover weight $CR \in [0,1]$ probabilistically controls the amount of recombination, a value of 0.9 is suggested.
	\item The initial population of candidate solutions should be randomly generated from within the space of valid solutions.
	\item The popular configurations are DE/rand/1/* and DE/best/2/*. 
\end{itemize}

% The code description provides a minimal but functional version of the technique implemented with a programming language. The code description must be able to be typed into an appropriate computer, compiled or interpreted as need be, and provide a working execution of the technique. The technique implementation also includes a minimal problem instance to which it is applied, and both the problem and algorithm implementations are complete enough to demonstrate the techniques procedure. The description is presented as a programming source code listing.
\subsection{Code Listing}
% How is a technique implemented as an executable program?
% How is a technique applied to a concrete problem instance?
Listing~\ref{differential_evolution} provides an example of the Differential Evolution algorithm implemented in the Ruby Programming Language.
% problem
The demonstration problem is an instance of a continuous function optimization that seeks $min f(x)$ where $f=\sum_{i=1}^n x_{i}^2$, $-5.0\leq x_i \leq 5.0$ and $n=3$. The optimal solution for this basin function is $(v_0,\ldots,v_{n-1})=0.0$.
% algorithm
The algorithm is an implementation of Differential Evolution with the DE/rand/1/bin configuration proposed by Storn and Price \cite{Storn1997}.

% the listing
\lstinputlisting[firstline=7,language=ruby,caption=Differential Evolution algorithm in the Ruby Programming Language, label=differential_evolution]{../src/algorithms/evolutionary/differential_evolution.rb}


% References: Deeper understanding
% The references element description includes a listing of both primary sources of information about the technique as well as useful introductory sources for novices to gain a deeper understanding of the theory and application of the technique. The description consists of hand-selected reference material including books, peer reviewed conference papers, journal articles, and potentially websites. A bullet-pointed structure is suggested.
\subsection{References}
% What are the primary sources for a technique?
% What are the suggested reference sources for learning more about a technique?

% 
% Primary Sources
% 
\subsubsection{Primary Sources}
% seminal
The Differential Evolution algorithm was presented by Storn and Price in a technical report that considered DE1 and DE2 variants of the approach applied to a suite of continuous function optimization problems \cite{Storn1995}. 
% early papers, maturation
An early paper by Storn applied the approach to the optimization of an IIR-filter (Infinite Impulse Response) \cite{Storn1996a}. A second early paper applied the approach to a second suite of benchmark problem instances, adopting the contemporary nomenclature for describing the approach, including the DE/rand/1 and DE/best/2 variations \cite{Storn1996b}.
% seminal
The early work including technical reports and conference papers by Storn and Price culminated in a seminal journal article \cite{Storn1997}.

% 
% Learn More
% 
\subsubsection{Learn More}
% historical reviews
A classical overview of Differential Evolution is presented by Price and Storn \cite{Price1997}, and terse introduction to the approach for function optimization is presented by Storn \cite{Storn1996}. A seminal extended description of the algorithm with sample applications was presented by Storn and Price as a book chapter \cite{Price1999}.
% books
Price, Storn, and Lampinen release a contemporary book dedicated to Differential Evolution including theory, benchmarks, sample code and numerous application demonstrations \cite{Price2005}. Chakraborty also released a book considering extensions to the approach to address complexities such as rotation invariance and stopping criteria  \cite{Chakraborty2008}.


\putbib\end{bibunit}
\newpage\begin{bibunit}% The Clever Algorithms Project: http://www.CleverAlgorithms.com
% (c) Copyright 2010 Jason Brownlee. Some Rights Reserved. 
% This work is licensed under a Creative Commons Attribution-Noncommercial-Share Alike 2.5 Australia License.

% This is an algorithm description, see:
% Jason Brownlee. A Template for Standardized Algorithm Descriptions. Technical Report CA-TR-20100107-1, The Clever Algorithms Project http://www.CleverAlgorithms.com, January 2010.

% Name
% The algorithm name defines the canonical name used to refer to the technique, in addition to common aliases, abbreviations, and acronyms. The name is used in terms of the heading and sub-headings of an algorithm description.
\section{Evolutionary Programming} 
\label{sec:evolutionary_programming}
\index{Evolutionary Programming}

% other names
% What is the canonical name and common aliases for a technique?
% What are the common abbreviations and acronyms for a technique?
\emph{Evolutionary Programming, EP.}

% Taxonomy: Lineage and locality
% The algorithm taxonomy defines where a techniques fits into the field, both the specific subfields of Computational Intelligence and Biologically Inspired Computation as well as the broader field of Artificial Intelligence. The taxonomy also provides a context for determining the relation- ships between algorithms. The taxonomy may be described in terms of a series of relationship statements or pictorially as a venn diagram or a graph with hierarchical structure.
\subsection{Taxonomy}
% To what fields of study does a technique belong?
Evolutionary Programming is a Global Optimization algorithm and is an instance of an Evolutionary Algorithm from the field of Evolutionary Computation.
% What are the closely related approaches to a technique?
The approach is a sibling of other Evolutionary Algorithms such as the Genetic Algorithm (Section~\ref{sec:genetic_algorithm}), and Learning Classifier Systems (Section~\ref{sec:learning_classifier_system}). It is sometimes confused with Genetic Programming given the similarity in name (Section~\ref{sec:genetic_programming}), and more recently it shows a strong functional similarity to Evolution Strategies (Section~\ref{sec:evolution_strategies}). 

% Inspiration: Motivating system
% The inspiration describes the specific system or process that provoked the inception of the algorithm. The inspiring system may non-exclusively be natural, biological, physical, or social. The description of the inspiring system may include relevant domain specific theory, observation, nomenclature, and most important must include those salient attributes of the system that are somehow abstractly or conceptually manifest in the technique. The inspiration is described textually with citations and may include diagrams to highlight features and relationships within the inspiring system.
% Optional
\subsection{Inspiration}
% What is the system or process that motivated the development of a technique?
Evolutionary Programming is inspired by the theory of evolution by means of natural selection.
% Which features of the motivating system are relevant to a technique?
Specifically, the technique is inspired by macro-level or the species-level process of evolution (phenotype, hereditary, variation) and is not concerned with the genetic mechanisms of evolution (genome, chromosomes, genes, alleles).

% Metaphor: Explanation via analogy
% The metaphor is a description of the technique in the context of the inspiring system or a different suitable system. The features of the technique are made apparent through an analogous description of the features of the inspiring system. The explanation through analogy is not expected to be literal scientific truth, rather the method is used as an allegorical communication tool. The inspiring system is not explicitly described, this is the role of the ‘inspiration’ element, which represents a loose dependency for this element. The explanation is textual and uses the nomenclature of the metaphorical system.
% Optional
\subsection{Metaphor}
% What is the explanation of a technique in the context of the inspiring system?
% What are the functionalities inferred for a technique from the analogous inspiring system?
A population of a species reproduce, creating progeny with small phenotypical variation. The progeny and the parents compete based on their suitability to the environment, where the generally more fit members constitute the subsequent generation and are provided with the opportunity to reproduce themselves. This process repeats, improving the adaptive fit between the species and the environment.

% Strategy: Problem solving plan
% The strategy is an abstract description of the computational model. The strategy describes the information processing actions a technique shall take in order to achieve an objective. The strategy provides a logical separation between a computational realization (procedure) and a analogous system (metaphor). A given problem solving strategy may be realized as one of a number specific algorithms or problem solving systems. The strategy description is textual using information processing and algorithmic terminology.
\subsection{Strategy}
% What is the information processing objective of a technique?
The objective of the Evolutionary Programming algorithm is to maximize the suitability of a collection of candidate solutions in the context of an objective function from the domain.
% What is a techniques plan of action?
This objective is pursued by using an adaptive model with surrogates for the processes of evolution, specifically hereditary (reproduction with variation) under competition. The representation used for candidate solutions is directly assessable by a cost or objective function from the domain. 

% Procedure: Abstract computation
% The algorithmic procedure summarizes the specifics of realizing a strategy as a systemized and parameterized computation. It outlines how the algorithm is organized in terms of the data structures and representations. The procedure may be described in terms of software engineering and computer science artifacts such as pseudo code, design diagrams, and relevant mathematical equations.
\subsection{Procedure}
% What is the computational recipe for a technique?
% What are the data structures and representations used in a technique?
Algorithm~\ref{alg:evolutionary_programming} provides a pseudocode listing of the Evolutionary Programming algorithm for minimizing a cost function. 

\begin{algorithm}[ht]
	\SetLine  
	% data
	\SetKwData{Best}{$S_{best}$}
	\SetKwData{Children}{Children}
	\SetKwData{Union}{Union}
	\SetKwData{ProblemSize}{ProblemSize}
	\SetKwData{Population}{Population}
	\SetKwData{PopulationSize}{$Population_{size}$}
	\SetKwData{Parent}{$Parent_{i}$}
	\SetKwData{Child}{$Child_{i}$}
	\SetKwData{Candidate}{$S_{i}$}
	\SetKwData{CandidateWins}{$Si_{wins}$}
	\SetKwData{CandidateOther}{$S_{j}$}
	\SetKwData{BoutSize}{BoutSize}
	% functions
	\SetKwFunction{InitializePopulation}{InitializePopulation}  
	\SetKwFunction{EvaluatePopulation}{EvaluatePopulation} 
	\SetKwFunction{GetBestSolution}{GetBestSolution} 
	\SetKwFunction{StopCondition}{StopCondition}
	\SetKwFunction{Mutate}{Mutate}
  	\SetKwFunction{RandomSelection}{RandomSelection}
	\SetKwFunction{SelectBestByWins}{SelectBestByWins}
	\SetKwFunction{Cost}{Cost}
	% I/O
	\KwIn{\PopulationSize, \ProblemSize, \BoutSize}		
	\KwOut{\Best}
  	% Algorithm
	% initialize	
	\Population $\leftarrow$ \InitializePopulation{\PopulationSize, \ProblemSize}\;
	% evaluate
	\EvaluatePopulation{\Population}\;
	% best
	\Best $\leftarrow$ \GetBestSolution{\Population}\;
	% loop
	\While{$\neg$\StopCondition{}} {
		% mutate
		\Children $\leftarrow \emptyset$\;
		\ForEach{\Parent $\in$ \Population}{
			\Child $\leftarrow$ \Mutate{\Parent}\;
			\Children $\leftarrow$ \Child\;
		}
		\EvaluatePopulation{\Children}\;
		% best
		\Best $\leftarrow$ \GetBestSolution{\Children, \Best}\;
		% tournaments
		\Union $\leftarrow$ \Population $+$ \Children\;
		\ForEach{\Candidate $\in$ \Union}{
			\For{$1$ \KwTo \BoutSize}{
				\CandidateOther $\leftarrow$ \RandomSelection{\Union}\;
				\If{\Cost{\Candidate} $<$ \Cost{\CandidateOther}} {
					\CandidateWins $\leftarrow$ \CandidateWins $+$ 1\;
				}
			}
		}
		\Population $\leftarrow$ \SelectBestByWins{\Union, \PopulationSize}\;
	}
	\Return{\Best}\;
	% end
	\caption{Pseudocode for Evolutionary Programming.}
	\label{alg:evolutionary_programming}
\end{algorithm}

% Heuristics: Usage guidelines
% The heuristics element describe the commonsense, best practice, and demonstrated rules for applying and configuring a parameterized algorithm. The heuristics relate to the technical details of the techniques procedure and data structures for general classes of application (neither specific implementations not specific problem instances). The heuristics are described textually, such as a series of guidelines in a bullet-point structure.
\subsection{Heuristics}
% What are the suggested configurations for a technique?
% What are the guidelines for the application of a technique to a problem instance?
\begin{itemize}
	\item The representation for candidate solutions should be domain specific, such as real numbers for continuous function optimization.
	\item The sample size (bout size) for tournament selection during competition is commonly between 5\% and 10\% of the population size.
	\item Evolutionary Programming traditionally only uses the mutation operator to create new candidate solutions from existing candidate solutions. The crossover operator that is used in some other Evolutionary Algorithms is not employed in Evolutionary Programming.
	\item Evolutionary Programming is concerned with the linkage between parent and child candidate solutions and is not concerned with surrogates for genetic mechanisms.
	\item Continuous function optimization is a popular application for the approach, where real-valued representations are used  with a Gaussian-based mutation operator.
	\item The mutation-specific parameters used in the application of the algorithm to continuous function optimization can be adapted in concert with the candidate solutions \cite{Fogel1991a}.
\end{itemize}

% The code description provides a minimal but functional version of the technique implemented with a programming language. The code description must be able to be typed into an appropriate computer, compiled or interpreted as need be, and provide a working execution of the technique. The technique implementation also includes a minimal problem instance to which it is applied, and both the problem and algorithm implementations are complete enough to demonstrate the techniques procedure. The description is presented as a programming source code listing.
\subsection{Code Listing}
% How is a technique implemented as an executable program?
% How is a technique applied to a concrete problem instance?
Listing~\ref{evolutionary_programming} provides an example of the Evolutionary Programming algorithm implemented in the Ruby Programming Language.
% problem
The demonstration problem is an instance of a continuous function optimization that seeks $\min f(x)$ where $f=\sum_{i=1}^n x_{i}^2$, $-5.0\leq x_i \leq 5.0$ and $n=2$. The optimal solution for this basin function is $(v_0,\ldots,v_{n-1})=0.0$.
% algorithm
The algorithm is an implementation of Evolutionary Programming based on the classical implementation for continuous function optimization by Fogel et al.\ \cite{Fogel1991a} with per-variable adaptive variance based on Fogel's description for a self-adaptive variation on page 160 of his 1995 book \cite{Fogel1995}.

% the listing
\lstinputlisting[firstline=7,language=ruby,caption=Evolutionary Programming algorithm in the Ruby Programming Language, label=evolutionary_programming]{../src/algorithms/evolutionary/evolutionary_programming.rb}


% References: Deeper understanding
% The references element description includes a listing of both primary sources of information about the technique as well as useful introductory sources for novices to gain a deeper understanding of the theory and application of the technique. The description consists of hand-selected reference material including books, peer reviewed conference papers, journal articles, and potentially websites. A bullet-pointed structure is suggested.
\subsection{References}
% What are the primary sources for a technique?
% What are the suggested reference sources for learning more about a technique?


% 
% Primary Sources
% 
\subsubsection{Primary Sources}
% early work
Evolutionary Programming was developed by Lawrence Fogel, outlined in early papers (such as \cite{Fogel1962}) and later became the focus of his PhD dissertation \cite{Fogel1964}. Fogel focused on the use of an evolutionary process for the development of control systems using Finite State Machine (FSM) representations. 
% book
Fogel's early work on Evolutionary Programming culminated in a book (co-authored with Owens and Walsh) that elaborated the approach, focusing on the evolution of state machines for the prediction of symbols in time series data \cite{Fogel1966}.

% 
% Learn More
% 
\subsubsection{Learn More}
% revival
The field of Evolutionary Programming lay relatively dormant for 30 years until it was revived by Fogel's son, David. Early works considered the application of Evolutionary Programming to control systems \cite{Sebald1990}, and later function optimization (system identification) culminating in a book on the approach \cite{Fogel1991}, and David Fogel's PhD dissertation \cite{Fogel1992}.
% Lawrence
Lawrence Fogel collaborated in the revival of the technique, including reviews \cite{Fogel1990, Fogel1994} and extensions on what became the focus of the approach on function optimization \cite{Fogel1991a}.

% recent
Yao et al.\ provide a seminal study of Evolutionary Programming proposing an extension and racing it against the classical approach on a large number of test problems \cite{Yao1999}. Finally, Porto provides an excellent contemporary overview of the field and the technique \cite{Porto2000}.


\putbib\end{bibunit}
\newpage\begin{bibunit}% The Clever Algorithms Project: http://www.CleverAlgorithms.com
% (c) Copyright 2010 Jason Brownlee. Some Rights Reserved. 
% This work is licensed under a Creative Commons Attribution-Noncommercial-Share Alike 2.5 Australia License.

% This is an algorithm description, see:
% Jason Brownlee. A Template for Standardized Algorithm Descriptions. Technical Report CA-TR-20100107-1, The Clever Algorithms Project http://www.CleverAlgorithms.com, January 2010.

% Name
% The algorithm name defines the canonical name used to refer to the technique, in addition to common aliases, abbreviations, and acronyms. The name is used in terms of the heading and sub-headings of an algorithm description.
\section{Grammatical Evolution} 
\label{sec:grammatical_evolution}
\index{Grammatical Evolution}

% other names
% What is the canonical name and common aliases for a technique?
% What are the common abbreviations and acronyms for a technique?
\emph{Grammatical Evolution, GE.}

% Taxonomy: Lineage and locality
% The algorithm taxonomy defines where a techniques fits into the field, both the specific subfields of Computational Intelligence and Biologically Inspired Computation as well as the broader field of Artificial Intelligence. The taxonomy also provides a context for determining the relation- ships between algorithms. The taxonomy may be described in terms of a series of relationship statements or pictorially as a venn diagram or a graph with hierarchical structure.
\subsection{Taxonomy}
% To what fields of study does a technique belong?
Grammatical Evolution is a Global Optimization technique and an instance of an Evolutionary Algorithm from the field of Evolutionary Computation. It may also be considered an algorithm for Automatic Programming.
% What are the closely related approaches to a technique?
Grammatical Evolution is related to other Evolutionary Algorithms for evolving programs such as Genetic Programming (Section~\ref{sec:genetic_programming}) and Gene Expression Programming (Section~\ref{sec:gene_expression_programming}), as well as the classical Genetic Algorithm that uses binary strings (Section~\ref{sec:genetic_algorithm}).

% Inspiration: Motivating system
% The inspiration describes the specific system or process that provoked the inception of the algorithm. The inspiring system may non-exclusively be natural, biological, physical, or social. The description of the inspiring system may include relevant domain specific theory, observation, nomenclature, and most important must include those salient attributes of the system that are somehow abstractly or conceptually manifest in the technique. The inspiration is described textually with citations and may include diagrams to highlight features and relationships within the inspiring system.
% Optional
\subsection{Inspiration}
% What is the system or process that motivated the development of a technique?
The Grammatical Evolution algorithm is inspired by the biological process used for generating a protein from genetic material as well as the broader genetic evolutionary process.
% Which features of the motivating system are relevant to a technique?
The genome is comprised of DNA as a string of building blocks that are transcribed to RNA. RNA codons are in turn translated into sequences of amino acids and used in the protein. The resulting protein in its environment is the phenotype. 

% Metaphor: Explanation via analogy
% The metaphor is a description of the technique in the context of the inspiring system or a different suitable system. The features of the technique are made apparent through an analogous description of the features of the inspiring system. The explanation through analogy is not expected to be literal scientific truth, rather the method is used as an allegorical communication tool. The inspiring system is not explicitly described, this is the role of the ‘inspiration’ element, which represents a loose dependency for this element. The explanation is textual and uses the nomenclature of the metaphorical system.
% Optional
\subsection{Metaphor}
% What is the explanation of a technique in the context of the inspiring system?
% mapping
The phenotype is a computer program that is created from a binary string-based genome. The genome is decoded into a sequence of integers that are in turn mapped onto pre-defined rules that makeup the program. 
% wrapping
The mapping from genotype to the phenotype is a one-to-many process that uses a wrapping feature. This is like the biological process observed in many bacteria, viruses, and mitochondria, where the same genetic material is used in the expression of different genes.
% What are the functionalities inferred for a technique from the analogous inspiring system?
The mapping adds robustness to the process both in the ability to adopt structure-agnostic genetic operators used during the evolutionary process on the sub-symbolic representation and the transcription of well-formed executable programs from the representation.

% Strategy: Problem solving plan
% The strategy is an abstract description of the computational model. The strategy describes the information processing actions a technique shall take in order to achieve an objective. The strategy provides a logical separation between a computational realization (procedure) and a analogous system (metaphor). A given problem solving strategy may be realized as one of a number specific algorithms or problem solving systems. The strategy description is textual using information processing and algorithmic terminology.
\subsection{Strategy}
% What is the information processing objective of a technique?
The objective of Grammatical Evolution is to adapt an executable program to a problem specific objective function.
% What is a techniques plan of action?
This is achieved through an iterative process with surrogates of evolutionary mechanisms such as descent with variation, genetic mutation and recombination, and genetic transcription and gene expression. A population of programs are evolved in a sub-symbolic form as variable length binary strings and mapped to a symbolic and well-structured form as a context free grammar for execution.

% Procedure: Abstract computation
% The algorithmic procedure summarizes the specifics of realizing a strategy as a systemized and parameterized computation. It outlines how the algorithm is organized in terms of the data structures and representations. The procedure may be described in terms of software engineering and computer science artifacts such as Pseudocode, design diagrams, and relevant mathematical equations.
\subsection{Procedure}
% What are the data structures and representations used in a technique?
% grammar
A grammar is defined in Backus Normal Form (BNF), which is a context free grammar expressed as a series of production rules comprised of terminals and non-terminals.
% mapping
A variable-length binary string representation is used for the optimization process. Bits are read from the a candidate solutions genome in blocks of 8 called a codon, and decoded to an integer (in the range between 0 and $2^{8}-1$). If the end of the binary string is reached when reading integers, the reading process loops back to the start of the string, effectively creating a circular genome. The integers are mapped to expressions from the BNF until a complete syntactically correct expression is formed. This may not use a solutions entire genome, or use the decoded genome more than once given it's circular nature.
% What is the computational recipe for a technique?
Algorithm~\ref{alg:grammatical_evolution} provides a pseudocode listing of the Grammatical Evolution algorithm for minimizing a cost function.


\begin{algorithm}[ht]
	\SetLine  
	% params
	\SetKwData{Best}{$S_{best}$}
	\SetKwData{ProbabilityMutate}{$P_{mutation}$}
	\SetKwData{ProbabilityCrossover}{$P_{crossover}$}
	\SetKwData{ProbabilityDeleteCodon}{$P_{delete}$}
	\SetKwData{ProbabilityDuplicateCodon}{$P_{duplicate}$}
	\SetKwData{CodonBits}{$Codon_{numbits}$}
	\SetKwData{Grammar}{Grammar}
	\SetKwData{PopulationSize}{$Population_{size}$}
	% data
	\SetKwData{Population}{Population}
	\SetKwData{Parents}{Parents}
	\SetKwData{Children}{Children}	
	\SetKwData{ParentOne}{$Parent_{i}$}
	\SetKwData{ParentTwo}{$Parent_{j}$}
	\SetKwData{Solution}{$S_{i}$}	
	\SetKwData{SolutionBitstring}{$Si_{bitstring}$}
	\SetKwData{SolutionIntegers}{$Si_{integers}$}
	\SetKwData{SolutionProgram}{$Si_{program}$}
	\SetKwData{SolutionCost}{$Si_{cost}$}
	% functions
	\SetKwFunction{InitializePopulation}{InitializePopulation}  
	\SetKwFunction{EvaluatePopulation}{EvaluatePopulation} 
	\SetKwFunction{GetBestSolution}{GetBestSolution} 
	\SetKwFunction{SelectParents}{SelectParents}
	\SetKwFunction{Replace}{Replace}
	\SetKwFunction{StopCondition}{StopCondition}
	\SetKwFunction{Crossover}{Crossover}
	\SetKwFunction{Mutate}{Mutate}
	\SetKwFunction{CodonDeletion}{CodonDeletion}
	\SetKwFunction{CodonDuplication}{CodonDuplication}
	\SetKwFunction{Decode}{Decode}
	\SetKwFunction{Map}{Map}
	\SetKwFunction{Execute}{Execute}
  
	% I/O
	\KwIn{\Grammar, \CodonBits, \PopulationSize, \ProbabilityCrossover, \ProbabilityMutate, \ProbabilityDeleteCodon, \ProbabilityDuplicateCodon}		
	\KwOut{\Best}
  	% Algorithm
	% initialize	
	\Population $\leftarrow$ \InitializePopulation{\PopulationSize, \CodonBits}\;
	% evaluate
	\ForEach{\Solution $\in$ \Population}{
		\SolutionIntegers $\leftarrow$ \Decode{\SolutionBitstring, \CodonBits}\;
		\SolutionProgram $\leftarrow$ \Map{\SolutionIntegers, \Grammar}\;
		\SolutionCost $\leftarrow$ \Execute{\SolutionProgram}\;
	}
	% best
	\Best $\leftarrow$ \GetBestSolution{\Population}\;
	% loop
	\While{$\neg$\StopCondition{}} {
		% select
		\Parents $\leftarrow$ \SelectParents{\Population, \PopulationSize}\;
		% recombine
		\Children $\leftarrow \emptyset$\;
		\ForEach{\ParentOne, \ParentTwo $\in$ \Parents}{
			% crossover
			\Solution $\leftarrow$ \Crossover{\ParentOne, \ParentTwo, \ProbabilityCrossover}\;
			% deletion
			\SolutionBitstring $\leftarrow$ \CodonDeletion{\SolutionBitstring, \ProbabilityDeleteCodon}\;
			% duplication
			\SolutionBitstring $\leftarrow$ \CodonDuplication{\SolutionBitstring, \ProbabilityDuplicateCodon}\;
			% mutation
			\SolutionBitstring $\leftarrow$ \Mutate{\SolutionBitstring, \ProbabilityMutate}\;
			% add
			\Children $\leftarrow$ \Solution\;
		}
		% evaluate
		\ForEach{\Solution $\in$ \Children}{
			\SolutionIntegers $\leftarrow$ \Decode{\SolutionBitstring, \CodonBits}\;
			\SolutionProgram $\leftarrow$ \Map{\SolutionIntegers, \Grammar}\;
			\SolutionCost $\leftarrow$ \Execute{\SolutionProgram}\;
		}
		% best
		\Best $\leftarrow$ \GetBestSolution{\Children}\;
		% replace
		\Population $\leftarrow$ \Replace{\Population, \Children}\;
	}
	\Return{\Best}\;
	% end
	\caption{Pseudocode for the Grammatical Evolution algorithm.}
	\label{alg:grammatical_evolution}
\end{algorithm}


% Heuristics: Usage guidelines
% The heuristics element describe the commonsense, best practice, and demonstrated rules for applying and configuring a parameterized algorithm. The heuristics relate to the technical details of the techniques procedure and data structures for general classes of application (neither specific implementations not specific problem instances). The heuristics are described textually, such as a series of guidelines in a bullet-point structure.
\subsection{Heuristics}
% What are the suggested configurations for a technique?
% What are the guidelines for the application of a technique to a problem instance?
\begin{itemize}
	\item Grammatical Evolution was designed to optimize programs (such as mathematical equations) to specific cost functions.
	\item Classical genetic operators used by the Genetic Algorithm may be used in the Grammatical Evolution algorithm, such as point mutations and one-point crossover.
	\item Codons (groups of bits mapped to an integer) are commonly fixed at 8 bits, proving a range of integers $\in [0,2^{8}-1]$ that is scaled to the range of rules using a modulo function.
	\item Additional genetic operators may be used with variable-length representations such as codon segments, duplication (add to the end), number of codons selected at random, and deletion.
\end{itemize}

% The code description provides a minimal but functional version of the technique implemented with a programming language. The code description must be able to be typed into an appropriate computer, compiled or interpreted as need be, and provide a working execution of the technique. The technique implementation also includes a minimal problem instance to which it is applied, and both the problem and algorithm implementations are complete enough to demonstrate the techniques procedure. The description is presented as a programming source code listing.
\subsection{Code Listing}
% How is a technique implemented as an executable program?
% How is a technique applied to a concrete problem instance?
Listing~\ref{grammatical_evolution} provides an example of the Grammatical Evolution algorithm implemented in the Ruby Programming Language based on the version described by O'Neill and Ryan \cite{O'Neill2001}.
% problem
The demonstration problem is an instance of symbolic regression $f(x)=x^3+x^2+x$, where $x\in[-1,1]$. 
The grammar used in this problem is: 
\begin{itemize}
	\item Non-terminals: $N=\{expr,op,pre\_op\}$
	\item Terminals: $T=\{+,-,/,*,x,1.0\}$
	\item Expression (program): $S=<expr>$
\end{itemize}

The production rules for the grammar in BNF are:
\begin{itemize}
	\item $<expr>$ ::= $<expr><op><expr>$, $(<expr><op><expr>)$, $<pre\_op>(<expr>)$, $<var>$
	\item $<op>$ ::= +, -, $\div$, $\times$
	\item $<var>$ ::= x, 1.0
\end{itemize}

% algorithm
The algorithm uses point mutation and a codon-respecting one-point crossover operator. Binary tournament selection is used to determine the parent population's contribution to the subsequent generation. 
% eval
Binary strings are decoded to integers using an unsigned binary. Candidate solutions are then mapped directly into executable Ruby code and executed. A given candidate solution is evaluated by comparing its output against the target function and taking the sum of the absolute errors over a number of trials. The probabilities of point mutation, codon deletion, and codon duplication are hard coded as relative probabilities to each solution, although should be parameters of the algorithm. In this case they are heuristically defined as $\frac{1.0}{L}$, $\frac{0.5}{NC}$ and $\frac{1.0}{NC}$ respectively, where $L$ is the total number of bits, and $NC$ is the number of codons in a given candidate solution.

Solutions are evaluated by generating a number of random samples from the domain and calculating the mean error of the program to the expected outcome. Programs that contain a single term or those that return an invalid (NaN) or infinite result are penalized with an enormous error value.
The implementation uses a maximum depth in the expression tree, whereas traditionally such deep expression trees are marked as invalid. Programs that resolve to a single expression that returns the output are penalized. 

% the listing
\lstinputlisting[firstline=7,language=ruby,caption=Grammatical Evolution algorithm in the Ruby Programming Language, label=grammatical_evolution]{../src/algorithms/evolutionary/grammatical_evolution.rb}


% References: Deeper understanding
% The references element description includes a listing of both primary sources of information about the technique as well as useful introductory sources for novices to gain a deeper understanding of the theory and application of the technique. The description consists of hand-selected reference material including books, peer reviewed conference papers, journal articles, and potentially websites. A bullet-pointed structure is suggested.
\subsection{References}
% What are the primary sources for a technique?
% What are the suggested reference sources for learning more about a technique?

% 
% Primary Sources
% 
\subsubsection{Primary Sources}
% seminal
Grammatical Evolution was proposed by Ryan, Collins and O'Neill in a seminal conference paper that applied the approach to a symbolic regression problem \cite{Ryan1998a}. 
% where did it come from
The approach was born out of the desire for syntax preservation while evolving programs using the Genetic Programming algorithm.
% application papers
This seminal work was followed by application papers for a symbolic integration problem \cite{O'Neill1998, O'Neill1998a} and solving trigonometric identities \cite{Ryan1998}.

% 
% Learn More
% 
\subsubsection{Learn More}
% overview
O'Neill and Ryan provide a high-level introduction to Grammatical Evolution and early demonstration applications \cite{O'Neill1999}. The same authors provide a through introduction to the technique and overview of the state of the field \cite{O'Neill2001}.
% books
O'Neill and Ryan present a seminal reference for Grammatical Evolution in their book \cite{O'Neill2003}. A second more recent book considers extensions to the approach improving its capability on dynamic problems \cite{Dempsey2009}.
\putbib\end{bibunit}
\newpage\begin{bibunit}% The Clever Algorithms Project: http://www.CleverAlgorithms.com
% (c) Copyright 2010 Jason Brownlee. Some Rights Reserved. 
% This work is licensed under a Creative Commons Attribution-Noncommercial-Share Alike 2.5 Australia License.

% This is an algorithm description, see:
% Jason Brownlee. A Template for Standardized Algorithm Descriptions. Technical Report CA-TR-20100107-1, The Clever Algorithms Project http://www.CleverAlgorithms.com, January 2010.

% Name
% The algorithm name defines the canonical name used to refer to the technique, in addition to common aliases, abbreviations, and acronyms. The name is used in terms of the heading and sub-headings of an algorithm description.
\section{Gene Expression Programming} 
\label{sec:gene_expression_programming}
\index{Gene Expression Programming}

% other names
% What is the canonical name and common aliases for a technique?
% What are the common abbreviations and acronyms for a technique?
\emph{Gene Expression Programming, GEP.}

% Taxonomy: Lineage and locality
% The algorithm taxonomy defines where a techniques fits into the field, both the specific subfields of Computational Intelligence and Biologically Inspired Computation as well as the broader field of Artificial Intelligence. The taxonomy also provides a context for determining the relation- ships between algorithms. The taxonomy may be described in terms of a series of relationship statements or pictorially as a venn diagram or a graph with hierarchical structure.
\subsection{Taxonomy}
% To what fields of study does a technique belong?
Gene Expression Programming is a Global Optimization algorithm and an Automatic Programming technique, and it is an instance of an Evolutionary Algorithm from the field of Evolutionary Computation.
% What are the closely related approaches to a technique?
It is a sibling of other Evolutionary Algorithms such as a the Genetic Algorithm as well as other Evolutionary Automatic Programming techniques such as Genetic Programming and Grammatical Evolution.

% Inspiration: Motivating system
% The inspiration describes the specific system or process that provoked the inception of the algorithm. The inspiring system may non-exclusively be natural, biological, physical, or social. The description of the inspiring system may include relevant domain specific theory, observation, nomenclature, and most important must include those salient attributes of the system that are somehow abstractly or conceptually manifest in the technique. The inspiration is described textually with citations and may include diagrams to highlight features and relationships within the inspiring system.
% Optional
\subsection{Inspiration}
% What is the system or process that motivated the development of a technique?
Gene Expression Programming is inspired by the replication and expression of the DNA molecule, specifically at the gene level. 
% expression
The expression of a gene involves the transcription of its DNA to RNA which in turn forms amino acids that make up proteins in the phenotype of an organism. 
% replication
The DNA building blocks are subjected to mechanisms of variation (mutations such as coping errors) as well as recombination during sexual reproduction.

% Which features of the motivating system are relevant to a technique?

% Metaphor: Explanation via analogy
% The metaphor is a description of the technique in the context of the inspiring system or a different suitable system. The features of the technique are made apparent through an analogous description of the features of the inspiring system. The explanation through analogy is not expected to be literal scientific truth, rather the method is used as an allegorical communication tool. The inspiring system is not explicitly described, this is the role of the ‘inspiration’ element, which represents a loose dependency for this element. The explanation is textual and uses the nomenclature of the metaphorical system.
% Optional
\subsection{Metaphor}
% What is the explanation of a technique in the context of the inspiring system?
% What are the functionalities inferred for a technique from the analogous inspiring system?
Gene Expression Programming uses a linear genome as the basis for genetic operators such as mutation, recombination, inversion, and transposition. The genome is comprised of chromosomes and each chromosome is comprised of genes that are translated into an expression tree to solve a given problem. The robust gene definition means that genetic operators can be applied to the sub-symbolic representation without concern for the structure of the resultant gene expression, providing separation of genotype and phenotype.

% Strategy: Problem solving plan
% The strategy is an abstract description of the computational model. The strategy describes the information processing actions a technique shall take in order to achieve an objective. The strategy provides a logical separation between a computational realization (procedure) and a analogous system (metaphor). A given problem solving strategy may be realized as one of a number specific algorithms or problem solving systems. The strategy description is textual using information processing and algorithmic terminology.
\subsection{Strategy}
% What is the information processing objective of a technique?
The objective of the Gene Expression Programming algorithm is to improve the adaptive fit of an expressed program in the context of a problem specific cost function.
% What is a techniques plan of action?
This is achieved through the use of an evolutionary process that operates on a sub-symbolic representation of candidate solutions using surrogates for the processes (descent with modification) and mechanisms (genetic recombination, mutation, inversion, transposition, and gene expression) of evolution.

% Procedure: Abstract computation
% The algorithmic procedure summarizes the specifics of realizing a strategy as a systemized and parameterized computation. It outlines how the algorithm is organized in terms of the data structures and representations. The procedure may be described in terms of software engineering and computer science artifacts such as pseudo code, design diagrams, and relevant mathematical equations.
\subsection{Procedure}
% What are the data structures and representations used in a technique?
A candidate solution is represented as a linear string of symbols called Karva notation or a K-expression, where each symbol maps to a function or terminal node. The linear representation is mapped to an expression tree in a breadth-first manner. 
A K-expression has fixed length and is comprised of one or more sub-expressions (genes), which are also defined with a fixed length. A gene is comprised of two sections, a head which may contain any function or terminal symbols, and a tail section that may only contain terminal symbols. Each gene will always translate to a syntactically correct expression tree, where the tail portion of the gene provides a genetic buffer which ensures closure of the expression.

% What is the computational recipe for a technique?
Algorithm~\ref{alg:gene_expression_programming} provides a pseudo-code listing of the Gene Expression Programming algorithm for minimizing a cost function. 

\begin{algorithm}[ht]
	\SetLine  

	% data
	\SetKwData{Best}{$S_{best}$}
	\SetKwData{ProbabilityMutate}{$P_{mutation}$}
	\SetKwData{ProbabilityCrossover}{$P_{crossover}$}
	\SetKwData{Parents}{Parents}
	\SetKwData{Children}{Children}
	\SetKwData{Population}{Population}
	\SetKwData{PopulationSize}{$Population_{size}$}
	\SetKwData{ParentOne}{$Parent_{1}$}
	\SetKwData{ParentTwo}{$Parent_{2}$}
	\SetKwData{Solution}{$S_{i}$}
	\SetKwData{SolutionGenome}{$Si_{genome}$}
	\SetKwData{SolutionProgram}{$Si_{program}$}
	\SetKwData{SolutionCost}{$Si_{cost}$}	
	\SetKwData{Grammar}{Grammar}
	\SetKwData{HeadLength}{$Head_{length}$}
	\SetKwData{TailLength}{$Tail_{length}$}
	
	% functions
	\SetKwFunction{InitializePopulation}{InitializePopulation}  
	\SetKwFunction{EvaluatePopulation}{EvaluatePopulation} 
	\SetKwFunction{GetBestSolution}{GetBestSolution} 
	\SetKwFunction{SelectParents}{SelectParents}
	\SetKwFunction{Replace}{Replace}
	\SetKwFunction{StopCondition}{StopCondition}
	\SetKwFunction{Crossover}{Crossover}
	\SetKwFunction{Mutate}{Mutate}
	\SetKwFunction{Map}{DecodeBreadthFirst}
	\SetKwFunction{Execute}{Execute}
  
	% I/O
	\KwIn{\Grammar, \PopulationSize, \HeadLength, \TailLength, \ProbabilityCrossover, \ProbabilityMutate}		
	\KwOut{\Best}
  	% Algorithm
	% initialize	
	\Population $\leftarrow$ \InitializePopulation{\PopulationSize, \Grammar, \HeadLength, \TailLength}\;
	% evaluate
	\ForEach{\Solution $\in$ \Population}{
		\SolutionProgram $\leftarrow$ \Map{\SolutionGenome, \Grammar}\;
		\SolutionCost $\leftarrow$ \Execute{\SolutionProgram}\;
	}
	% best
	\Best $\leftarrow$ \GetBestSolution{\Population}\;
	% loop
	\While{$\neg$\StopCondition{}} {
		% select
		\Parents $\leftarrow$ \SelectParents{\Population, \PopulationSize}\;
		% recombine
		\Children $\leftarrow 0$\;
		\ForEach{\ParentOne, \ParentTwo $\in$ \Parents}{
			% crossover
			\SolutionGenome $\leftarrow$ \Crossover{\ParentOne, \ParentTwo, \ProbabilityCrossover}\;
			% mutation
			\SolutionGenome $\leftarrow$ \Mutate{\SolutionGenome, \ProbabilityMutate}\;
			% add
			\Children $\leftarrow$ \Solution\;
		}
		% evaluate
		\ForEach{\Solution $\in$ \Children}{
			\SolutionProgram $\leftarrow$ \Map{\SolutionGenome, \Grammar}\;
			\SolutionCost $\leftarrow$ \Execute{\SolutionProgram}\;
		}
		% replace
		\Population $\leftarrow$ \Replace{\Population, \Children}\;
		% best
		\Best $\leftarrow$ \GetBestSolution{\Children}\;
	}
	\Return{\Best}\;
	% end
	\caption{Pseudo Code for the Gene Expression Programming algorithm.}
	\label{alg:gene_expression_programming}
\end{algorithm}


% Heuristics: Usage guidelines
% The heuristics element describe the commonsense, best practice, and demonstrated rules for applying and configuring a parameterized algorithm. The heuristics relate to the technical details of the techniques procedure and data structures for general classes of application (neither specific implementations not specific problem instances). The heuristics are described textually, such as a series of guidelines in a bullet-point structure.
\subsection{Heuristics}
% What are the suggested configurations for a technique?
% What are the guidelines for the application of a technique to a problem instance?
\begin{itemize}
	\item The length of a chromosome is defined by the number of genes, where a gene length is defined by $h + t$. The $h$ is a user defined parameter (such as 10), and $t$ is defined as $t = h (n-1) + 1$, where the $n$ represents the maximum arity of functional nodes in the expression (such as 2 if the arithmetic functions $\times, \div, -, +$ are used).
	\item The mutation operator substitutes expressions along the genome, although must respect the gene rules such that function and terminal nodes are mutated in the head of genes, whereas only terminal nodes are substituted in the tail of genes.
	\item Crossover occurs between two selected parents from the population and can occur based on a one-point cross, two point cross, or a gene-based approach where genes are selected from the parents with uniform probability.
	\item An inversion operator may be used with a low probability that reverses a small sequence of symbols (1-3) within a section of a gene (tail or head). 
	\item A transposition operator may be used that has a number of different modes, including: duplicate a small sequences (1-3) from somewhere on a gene to the head, small sequences on a gene to the root of the gene, and moving of entire genes in the chromosome. In the case of intra-gene transpositions, the sequence in the head of the gene is moved down to accommodate the copied sequence and the length of the head is truncated to maintain consistent gene sizes.
	\item A `?' is included in the terminal set that represents a numeric constant from an array that is evolved on the end of the genome. The constants are read from the end of the genome and are substituted for `?' as the expression tree is created (in breadth first order). Finally the numeric constants are used as array indices in yet another chromosome of numerical values which are substituted into the expression tree.
	\item Mutation is low (such as $\frac{1}{L}$), selection can be any of the classical approaches (such as roulette wheel or tournament), and crossover rates are typically high (0.7 of offspring)
	\item Use multiple sub-expressions linked together on hard problems when one gene is not sufficient to address the problem. The sub-expressions are linked using link expressions which are function nodes that are either statically defined (such as a conjunction) or evolved on the genome with the genes.
\end{itemize}

% The code description provides a minimal but functional version of the technique implemented with a programming language. The code description must be able to be typed into an appropriate computer, compiled or interpreted as need be, and provide a working execution of the technique. The technique implementation also includes a minimal problem instance to which it is applied, and both the problem and algorithm implementations are complete enough to demonstrate the techniques procedure. The description is presented as a programming source code listing.
\subsection{Code Listing}
% How is a technique implemented as an executable program?
% How is a technique applied to a concrete problem instance?
Listing~\ref{gene_expression_programming} provides an example of the Gene Expression Programming algorithm implemented in the Ruby Programming Language based on the seminal version proposed by Ferreira \cite{Ferreira2001}.
% problem
The demonstration problem is an instance of symbolic regression $f(x)=x^4+x^3+x^2+x$, where $x\in[-1,1]$. The grammar used in this problem is: Functions: $F=\{+,-,\div,\times,\}$ and Terminals: $T=\{x\}$.
% algorithm
The algorithm uses binary tournament selection, uniform crossover and point mutations. The K-expression is decoded to an expression tree in a breadth-first manner, which is then parsed depth first as a Ruby expression string for display and direct evaluation.

% the listing
\lstinputlisting[firstline=7,language=ruby,caption=Gene Expression Programming algorithm in the Ruby Programming Language, label=gene_expression_programming]{../src/algorithms/evolutionary/gene_expression_programming.rb}


% References: Deeper understanding
% The references element description includes a listing of both primary sources of information about the technique as well as useful introductory sources for novices to gain a deeper understanding of the theory and application of the technique. The description consists of hand-selected reference material including books, peer reviewed conference papers, journal articles, and potentially websites. A bullet-pointed structure is suggested.
\subsection{References}
% What are the primary sources for a technique?
% What are the suggested reference sources for learning more about a technique?

% 
% Primary Sources
% 
\subsubsection{Primary Sources}
% seminal
The Gene Expression Programming algorithm was proposed by Ferreira in a paper that detailed the approach, provided a careful walkthrough of the process and operators and demonstrated the the algorithm on a number of benchmark problem instances such as symbolic regression \cite{Ferreira2001}.

% 
% Learn More
% 
\subsubsection{Learn More}
% reviews
Ferreira provided an early and detailed introduction and overview of the approach as book chapter, providing a step-by-step walkthrough of the procedure and sample applications \cite{Ferreira2002}. A more contemporary and detailed introduction is provided in a later book chapter \cite{Ferreira2005}.
% books
Ferreira published a book on the approach in 2002 covering background, the algorithm, and demonstration applications which is now in its second edition \cite{Ferreira2006}.


\putbib\end{bibunit}
\newpage\begin{bibunit}% The Clever Algorithms Project: http://www.CleverAlgorithms.com
% (c) Copyright 2010 Jason Brownlee. Some Rights Reserved. 
% This work is licensed under a Creative Commons Attribution-Noncommercial-Share Alike 2.5 Australia License.

% This is an algorithm description, see:
% Jason Brownlee. A Template for Standardized Algorithm Descriptions. Technical Report CA-TR-20100107-1, The Clever Algorithms Project http://www.CleverAlgorithms.com, January 2010.

% Name
% The algorithm name defines the canonical name used to refer to the technique, in addition to common aliases, abbreviations, and acronyms. The name is used in terms of the heading and sub-headings of an algorithm description.
\section{Learning Classifier System} 
\label{sec:learning_classifier_system}
\index{Learning Classifier System}

% other names
% What is the canonical name and common aliases for a technique?
% What are the common abbreviations and acronyms for a technique?
\emph{Learning Classifier System, LCS.}

% Taxonomy: Lineage and locality
% The algorithm taxonomy defines where a techniques fits into the field, both the specific subfields of Computational Intelligence and Biologically Inspired Computation as well as the broader field of Artificial Intelligence. The taxonomy also provides a context for determining the relation- ships between algorithms. The taxonomy may be described in terms of a series of relationship statements or pictorially as a venn diagram or a graph with hierarchical structure.
\subsection{Taxonomy}
% To what fields of study does a technique belong?
The Learning Classifier System algorithm is both an instance of an Evolutionary Algorithm from the field of Evolutionary Computation and an instance of a Reinforcement Learning algorithm from Machine Learning. Internally, Learning Classifier Systems make use of a Genetic Algorithm (Section~\ref{sec:genetic_algorithm}).
% What are the closely related approaches to a technique?
The Learning Classifier System is a theoretical system with a number of implementations. Two streams of classifier are the Pittsburgh-style that seeks to optimize whole classifier, and the Michigan-style that optimize responsive rulesets. 
% common 
The Michigan-style Learning Classifier is the most common and is comprised of two versions: the ZCS (zeroth-level classifier system) and the XCS (accuracy-based classifier system).

% Strategy: Problem solving plan
% The strategy is an abstract description of the computational model. The strategy describes the information processing actions a technique shall take in order to achieve an objective. The strategy provides a logical separation between a computational realization (procedure) and a analogous system (metaphor). A given problem solving strategy may be realized as one of a number specific algorithms or problem solving systems. The strategy description is textual using information processing and algorithmic terminology.
\subsection{Strategy}
% What is the information processing objective of a technique?
The objective of the Learning Classifier System algorithm is to optimize payoff based on exposure to stimuli from a problem-specific environment.
% What is a techniques plan of action?
This is achieved by managing credit assignment for those rules that prove useful and searching for new rules and new variations on existing rules using an evolutionary process.

% Procedure: Abstract computation
% The algorithmic procedure summarizes the specifics of realizing a strategy as a systemized and parameterized computation. It outlines how the algorithm is organized in terms of the data structures and representations. The procedure may be described in terms of software engineering and computer science artifacts such as Pseudocode, design diagrams, and relevant mathematical equations.
\subsection{Procedure}
% What are the data structures and representations used in a technique?
% actors
The actors of the system include detectors, messages, effectors, feedback, and classifiers. Detectors are used by the system to perceive the state of the environment. Messages are the discrete information packets passed from the detectors into the system. The system performs information processing on messages, and messages may directly result in actions in the environment. Effectors control the actions of the system on and within the environment. In addition to the system actively perceiving via its detections, it may also receive directed feedback from the environment (payoff). Classifiers are condition-action rules that provide a filter for messages. If a message satisfies the conditional part of the classifier, the action of the classifier triggers. Rules act as message processors.
% data
Message a fixed length bitstring. A classifier is defined as a ternary string with an alphabet $\in \{1, 0, \#\}$, where the $\#$ represents do not care (matching either 1 or 0). 

% What is the computational recipe for a technique?
The processing loop for the Learning Classifier system is as follows: i) Messages from the environment are placed on the message list. ii) The conditions of each classifier are checked to see if they are satisfied by at least one message in the message list. iii) All classifiers that are satisfied participate in a competition, those that win post their action to the message list. iv) All messages directed to the effectors are executed (causing actions in the environment). v) All messages on the message list from the previous cycle are deleted (messages persist for a single cycle).
% algorithm
The algorithm may be described in terms of the main processing loop and two sub-algorithms: a reinforcement learning algorithm such as the bucket brigade algorithm or Q-learning, and a genetic algorithm for optimization of the system.
% presentated
Algorithm~\ref{alg:learning_classifier_system} provides a pseudocode listing of the high-level processing loop of the Learning Classifier System, specifically the XCS as described by Butz and Wilson \cite{Butz2002a}. 

\begin{algorithm}[h!t]
	\SetLine  

	% data
	\SetKwData{Population}{Population}
	\SetKwData{Environment}{env}
	\SetKwData{EnvironmentDetails}{EnvironmentDetails}
	\SetKwData{Input}{$Input_{t}$}
	\SetKwData{Matchset}{Matchset}
	\SetKwData{Prediction}{Prediction}
	\SetKwData{Action}{Action}
	\SetKwData{Actionset}{$ActionSet_{t}$}
	\SetKwData{Reward}{$Reward_{t}$}
	\SetKwData{Payoff}{$Payoff_{t}$}
	\SetKwData{LastActionset}{$ActionSet_{t-1}$}
	\SetKwData{LastInput}{$Input_{t-1}$}
	\SetKwData{LastReward}{$Reward_{t-1}$}

	% functions
	\SetKwFunction{StopCondition}{StopCondition}
	\SetKwFunction{InitializeEnvironment}{InitializeEnvironment}
	\SetKwFunction{InitializePopulation}{InitializePopulation}
	\SetKwFunction{GenerateMatchSet}{GenerateMatchSet}
	\SetKwFunction{GeneratePrediction}{GeneratePrediction}
	\SetKwFunction{SelectionAction}{SelectionAction}
	\SetKwFunction{GenerateActionSet}{GenerateActionSet}
	\SetKwFunction{ExecuteAction}{ExecuteAction}
	\SetKwFunction{CalculatePayoff}{CalculatePayoff}
	\SetKwFunction{PerformLearning}{PerformLearning}
	\SetKwFunction{RunGeneticAlgorithm}{RunGeneticAlgorithm}
	\SetKwFunction{LastStepOfTask}{LastStepOfTask}
  
	% I/O
	\KwIn{\EnvironmentDetails}		
	\KwOut{\Population}
  	% Algorithm
	
	\Environment $\leftarrow$ \InitializeEnvironment{\EnvironmentDetails}\;
	\Population $\leftarrow$ \InitializePopulation{}\;
	\LastActionset $\leftarrow$ $\emptyset$\;
	\LastInput $\leftarrow$ $\emptyset$\;
	\LastReward $\leftarrow$ $\emptyset$\;
		
	\While{$\neg$\StopCondition{}}{
		\Input $\leftarrow$ \Environment\;
		\Matchset $\leftarrow$ \GenerateMatchSet{\Population, \Input}\;
		\Prediction $\leftarrow$ \GeneratePrediction{\Matchset}\;
		\Action $\leftarrow$ \SelectionAction{\Prediction}\;
		\Actionset $\leftarrow$ \GenerateActionSet{\Action, \Matchset}\;
		\Reward $\leftarrow$ \ExecuteAction{\Action, \Environment}\;
		\If{\LastActionset $\neq$ $\emptyset$}{
			\Payoff $\leftarrow$ \CalculatePayoff{\LastReward, \Prediction}\;
			\PerformLearning{\LastActionset, \Payoff, \Population}\;
			\RunGeneticAlgorithm{\LastActionset, \LastInput, \Population}\;
		}
		
		\eIf{\LastStepOfTask{\Environment, \Action}}{
			\Payoff $\leftarrow$ \Reward\;
			\PerformLearning{\Actionset, \Payoff, \Population}\;
			\RunGeneticAlgorithm{\Actionset, \Input, \Population}\;
			\LastActionset $\leftarrow$ $\emptyset$\;
		}{
			\LastActionset $\leftarrow$ \Actionset\;
			\LastInput $\leftarrow$ \Input\;
			\LastReward $\leftarrow$ \Reward\;
		}
	}
	% end
	\caption{Pseudocode for the LCS.}
	\label{alg:learning_classifier_system}
\end{algorithm}


% Heuristics: Usage guidelines
% The heuristics element describe the commonsense, best practice, and demonstrated rules for applying and configuring a parameterized algorithm. The heuristics relate to the technical details of the techniques procedure and data structures for general classes of application (neither specific implementations not specific problem instances). The heuristics are described textually, such as a series of guidelines in a bullet-point structure.
\subsection{Heuristics}
% What are the suggested configurations for a technique?
% What are the guidelines for the application of a technique to a problem instance?
The majority of the heuristics in this section are specific to the XCS Learning Classifier System as described by Butz and Wilson \cite{Butz2002a}.

\begin{itemize}
	\item Learning Classifier Systems are suited for problems with the following characteristics: perpetually novel events with significant noise, continual real-time requirements for action, implicitly or inexactly defined goals, and sparse payoff or reinforcement obtainable only through long sequences of tasks.
	\item The learning rate $\beta$ for a classifier's expected payoff, error, and fitness are typically in the range $[0.1,0.2]$.
	\item The frequency of running the genetic algorithm $\theta_{GA}$ should be in the range $[25,50]$.
	\item The discount factor used in multi-step programs $\gamma$ are typically in the around $0.71$.
	\item The minimum error whereby classifiers are considered to have equal accuracy $\epsilon_{0}$ is typically 10\% of the maximum reward.
	\item The probability of crossover in the genetic algorithm $\chi$ is typically in the range $[0.5,1.0]$.
	\item The probability of mutating a single position in a classifier in the genetic algorithm $\mu$ is typically in the range $[0.01,0.05]$.
	\item The experience threshold during classifier deletion $\theta_{del}$ is typically about 20.
	\item The experience threshold for a classifier during subsumption $\theta_{sub}$ is typically around 20.
	\item The initial values for a classifier's expected payoff $p_1$, error $\epsilon_1$, and fitness $f_1$ are typically small and close to zero.
	\item The probability of selecting a random action for the purposes of exploration $p_{exp}$ is typically close to 0.5.
	\item The minimum number of different actions that must be specified in a match set $\theta_{mna}$ is usually the total number of possible actions in the environment for the input.
	\item Subsumption should be used on problem domains that are known contain well defined rules for mapping inputs to outputs.
\end{itemize}

% The code description provides a minimal but functional version of the technique implemented with a programming language. The code description must be able to be typed into an appropriate computer, compiled or interpreted as need be, and provide a working execution of the technique. The technique implementation also includes a minimal problem instance to which it is applied, and both the problem and algorithm implementations are complete enough to demonstrate the techniques procedure. The description is presented as a programming source code listing.
\subsection{Code Listing}
% How is a technique implemented as an executable program?
% How is a technique applied to a concrete problem instance?
Listing~\ref{learning_classifier_system} provides an example of the Learning Classifier System algorithm implemented in the Ruby Programming Language. 
% problem
The problem is an instance of a Boolean multiplexer called the 6-multiplexer. It can be described as a classification problem, where each of the $2^6$ patterns of bits is associated with a boolean class $\{1,0\}$. For this problem instance, the first two bits may be decoded as an address into the remaining four bits that specify the class (for example in 100011, `10' decode to the index of `2' in the remaining 4 bits making the class `1'). In propositional logic this problem instance may be described as $F=(\neg x_0) (\neg x_1) x_2 + (\neg x_0) x_1 x_3 + x_0 (\neg x_1) x_4 + x_0 x_1 x_5$. 
% algorithm
The algorithm is an instance of XCS based on the description provided by Butz and Wilson \cite{Butz2002a} with the parameters based on the application of XCS to Boolean multiplexer problems by Wilson \cite{Wilson1995, Wilson1998}.
% specifics
The population is grown as needed, and subsumption which would be appropriate for the Boolean multiplexer problem was not used for brevity. The multiplexer problem is a single step problem, so the complexities of delayed payoff are not required. A number of parameters were hard coded to recommended values, specifically: $\alpha=0.1, v=5, \delta=0.1$ and $P_{\#}=\frac{1}{3}$.

% the listing
\lstinputlisting[firstline=7,language=ruby,caption=Learning Classifier System algorithm in the Ruby Programming Language, label=learning_classifier_system]{../src/algorithms/evolutionary/learning_classifier_system.rb}


% References: Deeper understanding
% The references element description includes a listing of both primary sources of information about the technique as well as useful introductory sources for novices to gain a deeper understanding of the theory and application of the technique. The description consists of hand-selected reference material including books, peer reviewed conference papers, journal articles, and potentially websites. A bullet-pointed structure is suggested.
\subsection{References}
% What are the primary sources for a technique?
% What are the suggested reference sources for learning more about a technique?

% 
% Primary Sources
% 
\subsubsection{Primary Sources}
% seminal
Early ideas on the theory of Learning Classifier Systems were proposed by Holland \cite{Holland1976, Holland1977}, culminating in a standardized presentation a few years later \cite{Holland1980}.
% taxonomy
A number of implementations of the theoretical system were investigated, although a taxonomy of the two main streams was proposed by De Jong \cite{Jong1988}: 1) Pittsburgh-style proposed by Smith \cite{Smith1980, Smith1983} and 2) Holland-style or Michigan-style Learning classifiers that are further comprised of the Zeroth-level classifier (ZCS) \cite{Wilson1994} and the accuracy-based classifier (XCS) \cite{Wilson1995}.

% 
% Learn More
% 
\subsubsection{Learn More}
% classical
Booker, Goldberg, and Holland provide a classical introduction to Learning Classifier Systems including an overview of the state of the field and the algorithm in detail \cite{Booker1989}. Wilson and Goldberg also provide an introduction and review of the approach, taking a more critical stance \cite{Wilson1989}.
% review
Holmes et al.\ provide a contemporary review of the field focusing both on a description of the method and application areas to which the approach has been demonstrated successfully \cite{Holmes2002}.
% books
Lanzi, Stolzmann, and Wilson provide a seminal book in the field as a collection of papers covering the basics, advanced topics, and demonstration applications. A particular highlight from this book is the first section that provides a concise description of Learning Classifier Systems by many leaders and major contributors to the field \cite{Holland2000}, providing rare insight. Another paper from this book by Lanzi and Riolo provides a detailed review of the development of the approach as it matured throughout the 1990s.
% other book
Bull and Kovacs provide  a second book introductory book to the field focusing on the theory of the approach and its practical application \cite{Bull2005}.


\putbib\end{bibunit}
\newpage\begin{bibunit}% The Clever Algorithms Project: http://www.CleverAlgorithms.com
% (c) Copyright 2010 Jason Brownlee. Some Rights Reserved. 
% This work is licensed under a Creative Commons Attribution-Noncommercial-Share Alike 2.5 Australia License.

% This is an algorithm description, see:
% Jason Brownlee. A Template for Standardized Algorithm Descriptions. Technical Report CA-TR-20100107-1, The Clever Algorithms Project http://www.CleverAlgorithms.com, January 2010.

% Name
% The algorithm name defines the canonical name used to refer to the technique, in addition to common aliases, abbreviations, and acronyms. The name is used in terms of the heading and sub-headings of an algorithm description.
\section{Non-dominated Sorting Genetic Algorithm} 
\label{sec:nsga}
\index{Non-dominated Sorting Genetic Algorithm}
\index{NSGA-II}

% other names
% What is the canonical name and common aliases for a technique?
% What are the common abbreviations and acronyms for a technique?
\emph{Non-dominated Sorting Genetic Algorithm, Nondominated Sorting Genetic Algorithm, Fast Elitist Non-dominated Sorting Genetic Algorithm, NSGA, NSGA-II, NSGAII.}

% Taxonomy: Lineage and locality
% The algorithm taxonomy defines where a techniques fits into the field, both the specific subfields of Computational Intelligence and Biologically Inspired Computation as well as the broader field of Artificial Intelligence. The taxonomy also provides a context for determining the relation- ships between algorithms. The taxonomy may be described in terms of a series of relationship statements or pictorially as a venn diagram or a graph with hierarchical structure.
\subsection{Taxonomy}
% To what fields of study does a technique belong?
The Non-dominated Sorting Genetic Algorithm is a Multiple Objective Optimization (MOO) algorithm and is an instance of an Evolutionary Algorithm (EA) from the field of Evolutionary Computation (EC). 
% What are the closely related approaches to a technique?
NSGA is an extension of the Genetic Algorithm (GA) for multiple objective function optimization.
% related
It is related to other Evolutionary Multiple Objective Optimization Algorithms (EMOO) (or Multiple Objective Evolutionary Algorithms MOEA) such as the Vector-Evaluated Genetic Algorithm (VEGA), Strength Pareto Evolutionary Algorithm (SPEA), and Pareto Archived Evolution Strategy (PAES).
% taxonomy
There are two versions of the algorithm, the classical NSGA and the updated and currently canonical form NSGA-II.

% Strategy: Problem solving plan
% The strategy is an abstract description of the computational model. The strategy describes the information processing actions a technique shall take in order to achieve an objective. The strategy provides a logical separation between a computational realization (procedure) and a analogous system (metaphor). A given problem solving strategy may be realized as one of a number specific algorithms or problem solving systems. The strategy description is textual using information processing and algorithmic terminology.
\subsection{Strategy}
% What is the information processing objective of a technique?
The objective of the NSGA algorithm is to improve the adaptive fit of a population of candidate solutions to a Pareto front constrained by a set of objective functions.
% What is a techniques plan of action?
The algorithm uses an evolutionary process with surrogates for evolutionary operators including selection, genetic crossover, and genetic mutation. 
% fronts
The population is sorted into a hierarchy of sub-populations based on the ordering of Pareto dominance. Similarity between members of each sub-group is evaluated on the Pareto front, and the resulting groups and similarity measures are used to promote a diverse front of non-dominated solutions.

% Procedure: Abstract computation
% The algorithmic procedure summarizes the specifics of realizing a strategy as a systemized and parameterized computation. It outlines how the algorithm is organized in terms of the data structures and representations. The procedure may be described in terms of software engineering and computer science artifacts such as pseudo code, design diagrams, and relevant mathematical equations.
\subsection{Procedure}
% What is the computational recipe for a technique?
% What are the data structures and representations used in a technique?
Algorithm~\ref{alg:nsga} provides a pseudo-code listing of the Non-dominated Sorting Genetic Algorithm II (NSGA-II) for minimizing a cost function. 
% explaination
The \texttt{SortByRankAndDistance} function orders the population into a hierarchy of non-dominated Pareto fronts. The \texttt{CrowdingDistanceAssignment} calculates the average distance between members of each front on the front itself. Refer to Deb et al. for a clear presentation of the pseudo code and explanation of these functions \cite{Deb2002}. The \texttt{CrossoverAndMutation} function performs the classical crossover and mutation genetic operators of the Genetic Algorithm. Both the \texttt{SelectParentsByRankAndDistance} and \texttt{SortByRankAndDistance} functions discriminate members of the population first by rank (order of dominated precedence of the front to which the solution belongs) and then distance within the front (calculated by \texttt{CrowdingDistanceAssignment}).

\begin{algorithm}[htp]
	\SetLine  

	% data
	\SetKwData{Best}{$S_{best}$}
	\SetKwData{ProbabilityMutate}{$P_{mutation}$}
	\SetKwData{ProbabilityCrossover}{$P_{crossover}$}
	\SetKwData{Selected}{Selected}
	\SetKwData{Children}{Children}
	\SetKwData{ProblemSize}{ProblemSize}
	\SetKwData{Population}{Population}
	\SetKwData{PopulationSize}{$Population_{size}$}
	\SetKwData{Union}{Union}
	\SetKwData{Fronts}{Fronts}
	\SetKwData{Front}{$Front_i$}
	\SetKwData{Parents}{Parents}
	\SetKwData{LastFront}{$Front_L$}

	% functions
	\SetKwFunction{InitializePopulation}{InitializePopulation}  
	\SetKwFunction{EvaluateAgainstObjectiveFunctions}{EvaluateAgainstObjectiveFunctions} 
	\SetKwFunction{StopCondition}{StopCondition}	
	\SetKwFunction{FastNondominatedSort}{FastNondominatedSort}
	\SetKwFunction{SelectParentsByRank}{SelectParentsByRank}
	\SetKwFunction{SelectParentsByRankAndDistance}{SelectParentsByRankAndDistance}	
	\SetKwFunction{Merge}{Merge}
	\SetKwFunction{Size}{Size}
	\SetKwFunction{Break}{Break}
	\SetKwFunction{CrossoverAndMutation}{CrossoverAndMutation}
	\SetKwFunction{CrowdingDistanceAssignment}{CrowdingDistanceAssignment}
	\SetKwFunction{SortByRankAndDistance}{SortByRankAndDistance}
	
	% I/O
	\KwIn{\PopulationSize, \ProblemSize, \ProbabilityCrossover, \ProbabilityMutate}		
	\KwOut{\Best}
	% Algorithm
	% initialize	
	\Population $\leftarrow$ \InitializePopulation{\PopulationSize, \ProblemSize}\;
	% evaluate
	\EvaluateAgainstObjectiveFunctions{\Population}\;
	\FastNondominatedSort{\Population}\;
	\Selected $\leftarrow$ \SelectParentsByRank{\Population, \PopulationSize}\;
	\Children $\leftarrow$ \CrossoverAndMutation{\Selected, \ProbabilityCrossover, \ProbabilityMutate}\;
	% loop
	\While{$\neg$\StopCondition{}} {
	\EvaluateAgainstObjectiveFunctions{\Children}\;
		\Union $\leftarrow$ \Merge{\Population, \Children}\;
		\Fronts $\leftarrow$ \FastNondominatedSort{\Union}\;		
		% classical loop in paper
		\Parents $\leftarrow$ 0\;
		\LastFront $\leftarrow$ 0\;
		\ForEach{\Front $\in$ \Fronts}{
			\CrowdingDistanceAssignment{\Front}\;
			\eIf{\Size{\Parents}$+$\Size{\Front} $>$ \PopulationSize}{
			\LastFront $\leftarrow$ $i$\;
				\Break{}\;
			}{
				\Parents $\leftarrow$ \Merge{\Parents, \Front}\;
			}
		}
		\If{\Size{\Parents}$<$\PopulationSize}{
			\LastFront $\leftarrow$ \SortByRankAndDistance{\LastFront}\;
			\For{$P_1$ \KwTo $P_{\PopulationSize-\Size{LastFront}}$}{
				\Parents $\leftarrow$ $Pi$\;
			}
		}		
		% sample old GA
		\Selected $\leftarrow$ \SelectParentsByRankAndDistance{\Parents, \PopulationSize}\;
		\Population  $\leftarrow$ \Children\;
		\Children $\leftarrow$ \CrossoverAndMutation{\Selected, \ProbabilityCrossover, \ProbabilityMutate}\;
	}
	\Return{\Children}\;
	% end
	\caption{Pseudo Code for the Non-dominated Sorting Genetic Algorithm II.}
	\label{alg:nsga}
\end{algorithm}

% Heuristics: Usage guidelines
% The heuristics element describe the commonsense, best practice, and demonstrated rules for applying and configuring a parameterized algorithm. The heuristics relate to the technical details of the techniques procedure and data structures for general classes of application (neither specific implementations not specific problem instances). The heuristics are described textually, such as a series of guidelines in a bullet-point structure.
\subsection{Heuristics}
% What are the suggested configurations for a technique?
% What are the guidelines for the application of a technique to a problem instance?
\begin{itemize}
	\item NSGA was designed for and is suited to continuous function multiple objective optimization problem instances.
	\item A binary representation can be used in conjunction with classical genetic operators such as one-point crossover and point mutation.
	\item A real-valued representation is recommended for continuous function optimization problems, in turn requiring representation specific genetic operators such as Simulated Binary Crossover (SBX) and polynomial mutation \cite{Deb1995}.
\end{itemize}

% The code description provides a minimal but functional version of the technique implemented with a programming language. The code description must be able to be typed into an appropriate computer, compiled or interpreted as need be, and provide a working execution of the technique. The technique implementation also includes a minimal problem instance to which it is applied, and both the problem and algorithm implementations are complete enough to demonstrate the techniques procedure. The description is presented as a programming source code listing.
\subsection{Code Listing}
% How is a technique implemented as an executable program?
% How is a technique applied to a concrete problem instance?
Listing~\ref{nsga} provides an example of the Non-dominated Sorting Genetic Algorithm II (NSGA-II) implemented in the Ruby Programming Language.
% problem
The demonstration problem is an instance of continuous multiple objective function optimization called SCH (problem one in \cite{Deb2002}). The problem seeks the minimum of two functions: $f1=\sum_{i=1}^n x_{i}^2$ and $f2=\sum_{i=1}^n (x_{i}-2)^2$, $-10^3\leq x_i \leq 10^3$ and $n=1$. The optimal solution for this function are $x \in [0,2]$.
% algorithm
The algorithm is an implementation of NSGA-II based on the presentation by Deb, et al. \cite{Deb2002}.
%  cfg
The algorithm uses a binary string representation (16 bits per objective function parameter) that is decoded using the binary coded decimal method and rescaled to the function domain. The implementation uses a uniform crossover operator and point mutations with a fixed mutation rate of $\frac{1}{L}$, where $L$ is the number of bits in a solution's binary string. 

% the listing
\lstinputlisting[firstline=7,language=ruby,caption=Non-dominated Sorting Genetic Algorithm II (NSGA-II) in the Ruby Programming Language, label=nsga]{../src/algorithms/evolutionary/nsga_ii.rb}

% References: Deeper understanding
% The references element description includes a listing of both primary sources of information about the technique as well as useful introductory sources for novices to gain a deeper understanding of the theory and application of the technique. The description consists of hand-selected reference material including books, peer reviewed conference papers, journal articles, and potentially websites. A bullet-pointed structure is suggested.
\subsection{References}
% What are the primary sources for a technique?
% What are the suggested reference sources for learning more about a technique?

% 
% Primary Sources
% 
\subsubsection{Primary Sources}
% seminal
Srinivas and Deb proposed the NSGA algorithm inspired by Goldberg's notion of a non-dominated sorting procedure \cite{Srinivas1994}. Goldberg proposed a non-dominated sorting procedure in his book in considering the biases in the Pareto optimal solutions provided by VEGA \cite{Goldberg1989}. Srinivas and Deb's NSGA used the sorting procedure as a ranking selection method, and a fitness sharing niching method to maintain stable sub-populations across the Pareto front.
% extension
Deb, et al. later extended NSGA to address three criticism of the approach: i) the $O(mN^3)$ time complexity, the lack of elitism, and the need for a sharing parameter for the fitness sharing niching method \cite{Deb2000, Deb2002}.

% 
% Learn More
% 
\subsubsection{Learn More}
% reviews
% books
Deb provides in depth coverage of Evolutionary Multiple Objective Optimization algorithms in his book, including a detailed description of the NSGA in Chapter 5 \cite{Deb2001}.\putbib\end{bibunit}
\newpage\begin{bibunit}% The Clever Algorithms Project: http://www.CleverAlgorithms.com
% (c) Copyright 2010 Jason Brownlee. All Rights Reserved. 
% This work is licensed under a Creative Commons Attribution-Noncommercial-Share Alike 2.5 Australia License.

% This is an algorithm description, see:
% Jason Brownlee. A Template for Standardized Algorithm Descriptions. Technical Report CA-TR-20100107-1, The Clever Algorithms Project http://www.CleverAlgorithms.com, January 2010.

% Name
% The algorithm name defines the canonical name used to refer to the technique, in addition to common aliases, abbreviations, and acronyms. The name is used in terms of the heading and sub-headings of an algorithm description.
\section{Strength Pareto Evolutionary Algorithm} 
\label{sec:spea}

% other names
% What is the canonical name and common aliases for a technique?
% What are the common abbreviations and acronyms for a technique?
\emph{The heading and alternate headings for the algorithm description.}

% Taxonomy: Lineage and locality
% The algorithm taxonomy defines where a techniques fits into the field, both the specific subfields of Computational Intelligence and Biologically Inspired Computation as well as the broader field of Artificial Intelligence. The taxonomy also provides a context for determining the relation- ships between algorithms. The taxonomy may be described in terms of a series of relationship statements or pictorially as a venn diagram or a graph with hierarchical structure.
\subsection{Taxonomy}
% To what fields of study does a technique belong?
% What are the closely related approaches to a technique?
A small tree diagram showing related fields and algorithms.

% Inspiration: Motivating system
% The inspiration describes the specific system or process that provoked the inception of the algorithm. The inspiring system may non-exclusively be natural, biological, physical, or social. The description of the inspiring system may include relevant domain specific theory, observation, nomenclature, and most important must include those salient attributes of the system that are somehow abstractly or conceptually manifest in the technique. The inspiration is described textually with citations and may include diagrams to highlight features and relationships within the inspiring system.
% Optional
\subsection{Inspiration}
% What is the system or process that motivated the development of a technique?
% Which features of the motivating system are relevant to a technique?
A textual description of the inspiring system.

% Metaphor: Explanation via analogy
% The metaphor is a description of the technique in the context of the inspiring system or a different suitable system. The features of the technique are made apparent through an analogous description of the features of the inspiring system. The explanation through analogy is not expected to be literal scientific truth, rather the method is used as an allegorical communication tool. The inspiring system is not explicitly described, this is the role of the ‘inspiration’ element, which represents a loose dependency for this element. The explanation is textual and uses the nomenclature of the metaphorical system.
% Optional
\subsection{Metaphor}
% What is the explanation of a technique in the context of the inspiring system?
% What are the functionalities inferred for a technique from the analogous inspiring system?
A textual description of the algorithm by analogy.

% Strategy: Problem solving plan
% The strategy is an abstract description of the computational model. The strategy describes the information processing actions a technique shall take in order to achieve an objective. The strategy provides a logical separation between a computational realization (procedure) and a analogous system (metaphor). A given problem solving strategy may be realized as one of a number specific algorithms or problem solving systems. The strategy description is textual using information processing and algorithmic terminology.
\subsection{Strategy}
% What is the information processing objective of a technique?
% What is a techniques plan of action?
A textual description of the information processing strategy.

% Procedure: Abstract computation
% The algorithmic procedure summarizes the specifics of realizing a strategy as a systemized and parameterized computation. It outlines how the algorithm is organized in terms of the data structures and representations. The procedure may be described in terms of software engineering and computer science artifacts such as pseudo code, design diagrams, and relevant mathematical equations.
\subsection{Procedure}
% What is the computational recipe for a technique?
% What are the data structures and representations used in a technique?
A pseudo code description of the algorithms procedure.

% Heuristics: Usage guidelines
% The heuristics element describe the commonsense, best practice, and demonstrated rules for applying and configuring a parameterized algorithm. The heuristics relate to the technical details of the techniques procedure and data structures for general classes of application (neither specific implementations not specific problem instances). The heuristics are described textually, such as a series of guidelines in a bullet-point structure.
\subsection{Heuristics}
% What are the suggested configurations for a technique?
% What are the guidelines for the application of a technique to a problem instance?
A bullet-point listing of best practice usage.

% Tutorial: Guided implementation
% The tutorial description provides a guide to realizing the technique using a programming lan- guage. The result of completing the tutorial is a minimal yet complete implementation of the technique applied to a problem, similar or the same to the source code description. The tuto- rial description provides explanations as to the design decisions and rationale for the way the technique is implemented. The tutorial description is textual, providing a narrative with an objective, a series of steps, and outcome that may be interspersed with source code examples.
\subsection{Tutorial}
% What is the rationale when implementing a technique as an executable program?
% What is the rationale when applying a technique to a concrete problem instance?
A textural narrative for realizing the algorithm with complete source code.

% References: Deeper understanding
% The references element description includes a listing of both primary sources of information about the technique as well as useful introductory sources for novices to gain a deeper understanding of the theory and application of the technique. The description consists of hand-selected reference material including books, peer reviewed conference papers, journal articles, and potentially websites. A bullet-pointed structure is suggested.
\subsection{References}
% What are the primary sources for a technique?
% What are the suggested reference sources for learning more about a technique?
An bullet-point annotated reference list of primary sources and useful resources.


\putbib\end{bibunit}

