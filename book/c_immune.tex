% The Clever Algorithms Project: http://www.CleverAlgorithms.com
% (c) Copyright 2010 Jason Brownlee. Some Rights Reserved. 
% This work is licensed under a Creative Commons Attribution-Noncommercial-Share Alike 2.5 Australia License.

% This is a chapter

\renewcommand{\bibsection}{\subsection{\bibname}}
\begin{bibunit}

\chapter{Immune Algorithms}
\label{ch:immune}
\index{Immune Algorithms}
\index{Artificial Immune Systems}
\index{Adaptive Immune System}

\section{Overview}
This chapter describes Immune Algorithms.

% Immune
\subsection{Immune System}
Immune Algorithms belong to the Artificial Immune Systems field of study concerned with computational methods inspired by the process and mechanisms of the biological immune system. 

A simplified description of the immune system is an organ system intended to protect the host organism from the threats posed to it from pathogens and toxic substances. Pathogens encompass a range of micro-organisms such as bacteria, viruses, parasites and pollen. The traditional perspective regarding the role of the immune system is divided into two primary tasks: the \emph{detection} and \emph{elimination} of pathogen. This behavior is typically referred to as the differentiation of self (molecules and cells that belong to the host organisms) from potentially harmful non-self. More recent perspectives on the role of the system include a maintenance system \cite{Cohen2001a}, and a cognitive system \cite{Varela1994}.

The architecture of the immune system is such that a series of defensive layers protect the host. Once a pathogen makes it inside the host, it must contend with the \emph{innate} and \emph{acquired} immune system. These interrelated immunological sub-systems are comprised of many types of cells  and molecules produced by specialized organs and processes to address the self-nonself problem at the lowest level using chemical bonding, where the surfaces of cells and molecules interact with the surfaces of pathogen.

The adaptive immune system, also referred to as the acquired immune system, is named such because it is responsible for specializing a defense for the host organism based on the \emph{specific} pathogen to which it is exposed. Unlike the innate immune system, the acquired immune system is present only in vertebrates (animals with a spinal column). The system retains a \emph{memory} of exposures which it has encountered. This memory is \emph{recalled} on reinfection exhibiting a \emph{learned} pathogen identification. This learning process may be divided into two types of response. The first or \emph{primary response} occurs when the system encounters a novel pathogen. The system is slow to respond, potentially taking a number of weeks to clear the infection. On re-encountering the same pathogen again, the system exhibits a \emph{secondary response}, applying what was learned in the primary response and clearing up the infection rapidly. The \emph{memory} the system acquires in the primary response is typically long lasting, providing pathogenic immunity for the lifetime of the host, two common examples of which are the chickenpox and measles. White blood cells called lymphocytes (or leukocytes) are the most important cell in the acquired immune system. Lymphocytes are involved in both the identification and elimination of pathogen, and recirculate within the host organisms body in the blood and lymph (the fluid that permeates tissue). 

% AIS
\subsection{Artificial Immune Systems}
Artificial Immune Systems (AIS) is a sub-field of Computational Intelligence motivated by immunology (primarily mammalian immunology) that emerged in the early 1990s (for example \cite{Bersini1990, Ishida1990}), based on the proposal in the late 1980s to apply theoretical immunological models to machine learning and automated problem solving (such as \cite{Hoffmann1986, Farmer1986}). The early works in the field were inspired by exotic theoretical models (immune network theory) and were applied to machine learning, control and optimization problems. The approaches were reminiscent of paradigms such as Artificial Neural Networks, Genetic Algorithms, Reinforcement Learning, and Learning Classifier Systems. The most formative works in giving the field an identity were those that proposed the immune system as an analogy for information protection systems in the field of computer security. The classical examples include Forrest et~al.'s Computer Immunity \cite{Forrest1994, Forrest1997a} and Kephart's Immune Anti-Virus \cite{Kephart1994, Kephart1995}. These works were formative for the field because they provided an intuitive application domain that captivated a broader audience and assisted in differentiating the work as an independent sub-field.

Modern Artificial Immune systems are inspired by one of three sub-fields: clonal selection, negative selection and immune network algorithms. The techniques are commonly used for clustering, pattern recognition, classification, optimization, and other similar machine learning problem domains.

% References
% \subsubsection{References}
% classical
The seminal reference for those interested in the field is the text book by de Castro and Timmis ``\emph{Artificial Immune Systems: A New Computational Intelligence Approach}'' \cite{Castro2002}. This reference text provides an introduction to immunology with a level of detail appropriate for a computer scientist, followed by a summary of the state of the art, algorithms, application areas, and case studies.

% 
% Extensions
% 
\subsection{Extensions}
There are many other algorithms and classes of algorithm that were not described from the field of Artificial Immune Systems, not limited to:

\begin{itemize}
	\item \textbf{Clonal Selection Algorithms}: such as the B-Cell Algorithm \cite{Kelsey2003}, the Multi-objective Immune System Algorithm (MSIRA) \cite{Coello2002, Cortes2003} and the the Optimization Immune Algorithm (opt-IA, opt-IMMALG) \cite{Cutello2002a, Cutello2002} and the Simple Immunological Algorithm \cite{Cutello2005b}.
	\item \textbf{Immune Network Algorithms}: such as the approach by Timmis used for clustering called the Artificial Immune Network (AIN) \cite{Timmis2000}  (later extended and renamed the Resource Limited Artificial Immune System \cite{Timmis2001, Timmis2000a}.
	\item \textbf{Negative Selection Algorithms}: such as an adaptive framework called the \emph{ARTificial Immune System} (ARTIS), with the application to intrusion detection renamed the \emph{Lightweight Intrusion Detection System} (LISYS) \cite{Hofmeyr1999, Hofmeyr2000}.
\end{itemize}


\putbib
\end{bibunit}

\newpage\begin{bibunit}% The Clever Algorithms Project: http://www.CleverAlgorithms.com
% (c) Copyright 2010 Jason Brownlee. Some Rights Reserved. 
% This work is licensed under a Creative Commons Attribution-Noncommercial-Share Alike 2.5 Australia License.

% This is an algorithm description, see:
% Jason Brownlee. A Template for Standardized Algorithm Descriptions. Technical Report CA-TR-20100107-1, The Clever Algorithms Project http://www.CleverAlgorithms.com, January 2010.

% Name
% The algorithm name defines the canonical name used to refer to the technique, in addition to common aliases, abbreviations, and acronyms. The name is used in terms of the heading and sub-headings of an algorithm description.
\section{Clonal Selection Algorithm} 
\label{sec:clonal_selection_algorithm}
\index{Clonal Selection Algorithm}
\index{CLONALG}

% other names
% What is the canonical name and common aliases for a technique?
% What are the common abbreviations and acronyms for a technique?
\emph{Clonal Selection Algorithm, CSA, CLONALG.}

% Taxonomy: Lineage and locality
% The algorithm taxonomy defines where a techniques fits into the field, both the specific subfields of Computational Intelligence and Biologically Inspired Computation as well as the broader field of Artificial Intelligence. The taxonomy also provides a context for determining the relation- ships between algorithms. The taxonomy may be described in terms of a series of relationship statements or pictorially as a venn diagram or a graph with hierarchical structure.
\subsection{Taxonomy}
% To what fields of study does a technique belong?
The Clonal Selection Algorithm (CLONALG) belongs to the field of Artificial Immune Systems.
% What are the closely related approaches to a technique?
It is related to other Clonal Selection algorithms such as the Artificial Immune Recognition System (Section~\ref{sec:airs}), the B-Cell Algorithm (BCA), and the Multi-objective Immune System Algorithm (MISA).
% extensions
There are numerious extensions to CLONALG including tweaks such as the CLONALG1 and CLONALG2 approaches, a version for classification called CLONCLAS,  and an adaptive version called Adaptive Clonal Selection (ACS).

% Inspiration: Motivating system
% The inspiration describes the specific system or process that provoked the inception of the algorithm. The inspiring system may non-exclusively be natural, biological, physical, or social. The description of the inspiring system may include relevant domain specific theory, observation, nomenclature, and most important must include those salient attributes of the system that are somehow abstractly or conceptually manifest in the technique. The inspiration is described textually with citations and may include diagrams to highlight features and relationships within the inspiring system.
% Optional
\subsection{Inspiration}
% What is the system or process that motivated the development of a technique?
The Clonal Selection algorithm is inspired by the Clonal Selection theory of acquired immunity.
% Which features of the motivating system are relevant to a technique?
The clonal selection theory credited to Burnet was proposed to account for the behavior and capabilities of antibodies in the acquired immune system \cite{Burnet1957, Burnet1959}. Inspired itself by the principles of Darwinian natural selection theory of evolution, the theory proposes that antigens select-for lymphocytes (both B and T-cells). When a lymphocyte is selected and binds to an antigenic determinant, the cell proliferates making many thousands more copies of itself and differentiates into different cell types (plasma and memory cells). Plasma cells have a short lifespan and produce vast quantities of antibody molecules, whereas memory cells live for an extended period in the host anticipating future recognition of the same determinant. The important feature of the theory is that when a cell is selected and proliferates, it is subjected to small copying errors (changes to the genome called somatic hypermutation) that change the shape of the expressed receptors and subsequent determinant recognition capabilities of both the antibodies bound to the lymphocytes cells surface, and the antibodies that plasma cells produce.

% Metaphor: Explanation via analogy
% The metaphor is a description of the technique in the context of the inspiring system or a different suitable system. The features of the technique are made apparent through an analogous description of the features of the inspiring system. The explanation through analogy is not expected to be literal scientific truth, rather the method is used as an allegorical communication tool. The inspiring system is not explicitly described, this is the role of the ‘inspiration’ element, which represents a loose dependency for this element. The explanation is textual and uses the nomenclature of the metaphorical system.
% Optional
\subsection{Metaphor}
% What is the explanation of a technique in the context of the inspiring system?
% What are the functionalities inferred for a technique from the analogous inspiring system?
The theory suggests that starting with an initial repertoire of general immune cells, the system is able to change itself (the compositions and densities of cells and their receptors) in response to experience with the environment. Through a blind process of selection and accumulated variation on the large scale of many billions of cells, the acquired immune system is capable of acquiring the necessary information to protect the host organism from the specific pathogenic dangers of the environment. It also suggests that the system must anticipate (guess) at the pathogen to which it will be exposed, and requires exposure to pathogen that may harm the host before it can acquire the necessary information to provide a defense.

% Strategy: Problem solving plan
% The strategy is an abstract description of the computational model. The strategy describes the information processing actions a technique shall take in order to achieve an objective. The strategy provides a logical separation between a computational realization (procedure) and a analogous system (metaphor). A given problem solving strategy may be realized as one of a number specific algorithms or problem solving systems. The strategy description is textual using information processing and algorithmic terminology.
\subsection{Strategy}
% What is the information processing objective of a technique?
The information processing principles of the clonal selection theory describe a general learning strategy.
% What is a techniques plan of action?
This strategy involves a population of adaptive information units (each representing a problem-solution or component) subjected to a competitive processes for selection, which together with the resultant duplication and variation ultimately improves the adaptive fit the information units to their environment.

% Procedure: Abstract computation
% The algorithmic procedure summarizes the specifics of realizing a strategy as a systemized and parameterized computation. It outlines how the algorithm is organized in terms of the data structures and representations. The procedure may be described in terms of software engineering and computer science artifacts such as pseudo code, design diagrams, and relevant mathematical equations.
\subsection{Procedure}
% What is the computational recipe for a technique?
% What are the data structures and representations used in a technique?
Algorithm~\ref{alg:clonalg} provides a pseudocode listing of the Clonal Selection Algorithm (CLONALG) for minimizing a cost function. 
% description
The general CLONALG model involves the selection of antibodies (candidate solutions) based on affinity either by matching against an antigen pattern or via evaluation of a pattern by a cost function. Selected antibodies are subjected to cloning proportional to affinity, and the hypermutation of clones inversely-proportional to clone affinity. The resultant clonal-set competes with the existent antibody population for membership in the next generation. In addition, low-affinity population members are replaced by randomly generated antibodies. The pattern recognition variation of the algorithm includes the maintenance of a memory solution set which in its entirety represents a solution to the problem. A binary-encoding scheme is employed for the binary-pattern recognition and continuous function optimization examples, and an integer permutation scheme is employed for the Traveling Salesman Problem (TSP).

\begin{algorithm}[ht]
  \SetLine  
  % data
  \SetKwData{Pop}{Population}
  \SetKwData{Length}{$Problem_{size}$}
  \SetKwData{Selectsize}{$Selection_{size}$}
  \SetKwData{RandomCells}{$RandomCells_{num}$}
  \SetKwData{PopSize}{$Population_{size}$}
  \SetKwData{CloneRate}{$Clone_{rate}$}
  \SetKwData{MutationRate}{$Mutation_{rate}$}

	 % functions
  \SetKwFunction{StopCondition}{StopCondition}
  \SetKwFunction{Hypermutate}{Hypermutate}
  \SetKwFunction{Affinity}{Affinity}
  \SetKwFunction{Select}{Select}
  \SetKwFunction{Clone}{Clone}
  \SetKwFunction{Replace}{Replace}
  \SetKwFunction{CreateRandomCells}{CreateRandomCells}  
  
  \KwIn{\PopSize, \Selectsize, \Length, \RandomCells, \CloneRate, \MutationRate}		
  \KwOut{\Pop}
  
  % create cells	
	\Pop $\leftarrow$ \CreateRandomCells{\PopSize, \Length}\;
	
	\While{$\neg$\StopCondition{}}
	{
	 \ForEach{$p_i \in$ \Pop}		%// presentation
	 {
	 	\Affinity{$p_i$}\;
	 }
	 $Population_{select} \leftarrow$ \Select{\Pop, \Selectsize}\;		%// clonal selection
	 $Population_{clones} \leftarrow$ 0\;
	 \ForEach{$p_i \in Population_{select}$}	%	// clonal expansion
	 {
	 	$Population_{clones} \leftarrow$ \Clone{$p_i$, \CloneRate}\;
	 }	 
	 \ForEach{$p_i \in Population_{clones}$}		%// affinity maturation
	 {
    \Hypermutate{$p_i$, \MutationRate}\;
	  \Affinity{$p_i$}\;
	 }	 
	\Pop $\leftarrow$ \Select{\Pop, $Population_{clones}$, \PopSize}\;		%// greedy selection
	$Population_{rand} \leftarrow$ \CreateRandomCells{\RandomCells}\;
	\Replace{\Pop, $Population_{rand}$}\;	%// random replacement
	}
	\Return{\Pop}\;
	
	\caption{Pseudo Code for the Clonal Selection Algorithm (CLONALG).}
	\label{alg:clonalg}
\end{algorithm}

% Heuristics: Usage guidelines
% The heuristics element describe the commonsense, best practice, and demonstrated rules for applying and configuring a parameterized algorithm. The heuristics relate to the technical details of the techniques procedure and data structures for general classes of application (neither specific implementations not specific problem instances). The heuristics are described textually, such as a series of guidelines in a bullet-point structure.
\subsection{Heuristics}
% What are the suggested configurations for a technique?
% What are the guidelines for the application of a technique to a problem instance?
\begin{itemize}
	\item The CLONALG was designed as a general machine learning approach and has been applied to pattern recognition, function optimization, and combinatorial optimization problem domains.
	\item Binary string representations are used and decoded to a representation suitable for a specific problem domain.
	\item The number of clones created for each selected member is calculated as a function of the repertoire size $N_c=round(\beta \cdot N)$, where $\beta$ is the user parameter $Clone_{rate}$. 
	\item A rank-based affinity-proportionate function is used to determine the number of clones created for selected members of the population for pattern recognition problem instances.
	\item The number of random antibodies inserted each iteration is typically very low (1-2).
	\item Point mutations (bit-flips) are used in the hypermutation operation.
	\item The function $exp(-\rho \cdot f)$ is used to determine the probability of individual component mutation for a given candidate solution, where $f$ is the candidates affinity (normalized maximizing cost value), and $\rho$ is the user parameter $Mutation_{rate}$.
\end{itemize}

% The code description provides a minimal but functional version of the technique implemented with a programming language. The code description must be able to be typed into an appropriate computer, compiled or interpreted as need be, and provide a working execution of the technique. The technique implementation also includes a minimal problem instance to which it is applied, and both the problem and algorithm implementations are complete enough to demonstrate the techniques procedure. The description is presented as a programming source code listing.
\subsection{Code Listing}
% How is a technique implemented as an executable program?
% How is a technique applied to a concrete problem instance?
Listing~\ref{clonal_selection_algorithm} provides an example of the Clonal Selection Algorithm (CLONALG) implemented in the Ruby Programming Language.
% problem
The demonstration problem is an instance of a continuous function optimization that seeks $\min f(x)$ where $f=\sum_{i=1}^n x_{i}^2$, $-5.0\leq x_i \leq 5.0$ and $n=3$. The optimal solution for this basin function is $(v_0,\ldots,v_{n-1})=0.0$.
% algorithm
The algorithm is implemented as described by de Castro and Von Zuben for function optimization \cite{Castro2002a}.

% the listing
\lstinputlisting[firstline=7,language=ruby,caption=Clonal Selection Algorithm (CLONALG) in the Ruby Programming Language, label=clonal_selection_algorithm]{../src/algorithms/immune/clonal_selection_algorithm.rb}


% References: Deeper understanding
% The references element description includes a listing of both primary sources of information about the technique as well as useful introductory sources for novices to gain a deeper understanding of the theory and application of the technique. The description consists of hand-selected reference material including books, peer reviewed conference papers, journal articles, and potentially websites. A bullet-pointed structure is suggested.
\subsection{References}
% What are the primary sources for a technique?
% What are the suggested reference sources for learning more about a technique?

% 
% Primary Sources
% 
\subsubsection{Primary Sources}
% seminal
Hidden at the back of a technical report on the applications of Artificial Immune Systems de~Castro and Von~Zuben \cite{Castro1999} proposed the Clonal Selection Algorithm (CSA) as a computational realization of the clonal selection principle for pattern matching and optimization.
% early
The algorithm was later published \cite{Castro2000}, and investigated where it was renamed to CLONALG (CLONal selection ALGorithm) \cite{Castro2002a}.

% 
% Learn More
% 
\subsubsection{Learn More}
% parallel
Watkins, et~al. proposed to exploit the \emph{inherent distributedness} of the CLONALG and proposed a parallel version of the pattern recognition version of the algorithm \cite{Watkins2003}.
% clonclas
White and Garret also investigated the pattern recognition version of CLONALG and generalized the approach for the task of binary pattern classification renaming it to Clonal Classification (CLONCLAS) where their approach was compared to a number of simple Hamming distance based heuristics \cite{White2003}.
% acs
In an attempt to address concerns of algorithm efficiency, parameterization, and representation selection for continuous function optimization Garrett proposed an updated version of CLONALG called Adaptive Clonal Selection (ACS) \cite{Garrett2004}.
% reviews
In their book, de Castro and Timmis provide a detailed treatment of CLONALG including a description of the approach (starting page 79) and a step through of the algorithm (starting page 99) \cite{Castro2002b}.
Cutello and Nicosia provide a study of the clonal selection principle and algorithms inspired by the theory \cite{Cutello2005}.
Brownlee provides a review of Clonal Selection algorithms providing a taxonomy, algorithm reviews, and a broader bibliography \cite{Brownlee2007b}.
\putbib\end{bibunit}
\newpage\begin{bibunit}% The Clever Algorithms Project: http://www.CleverAlgorithms.com
% (c) Copyright 2010 Jason Brownlee. Some Rights Reserved. 
% This work is licensed under a Creative Commons Attribution-Noncommercial-Share Alike 2.5 Australia License.

% This is an algorithm description, see:
% Jason Brownlee. A Template for Standardized Algorithm Descriptions. Technical Report CA-TR-20100107-1, The Clever Algorithms Project http://www.CleverAlgorithms.com, January 2010.

% Name
% The algorithm name defines the canonical name used to refer to the technique, in addition to common aliases, abbreviations, and acronyms. The name is used in terms of the heading and sub-headings of an algorithm description.
\section{Negative Selection Algorithm} 
\label{sec:negative_selection_algorithm}

% other names
% What is the canonical name and common aliases for a technique?
% What are the common abbreviations and acronyms for a technique?
\emph{The heading and alternate headings for the algorithm description.}

% Taxonomy: Lineage and locality
% The algorithm taxonomy defines where a techniques fits into the field, both the specific subfields of Computational Intelligence and Biologically Inspired Computation as well as the broader field of Artificial Intelligence. The taxonomy also provides a context for determining the relation- ships between algorithms. The taxonomy may be described in terms of a series of relationship statements or pictorially as a venn diagram or a graph with hierarchical structure.
\subsection{Taxonomy}
% To what fields of study does a technique belong?
% What are the closely related approaches to a technique?
A small tree diagram showing related fields and algorithms.

% Inspiration: Motivating system
% The inspiration describes the specific system or process that provoked the inception of the algorithm. The inspiring system may non-exclusively be natural, biological, physical, or social. The description of the inspiring system may include relevant domain specific theory, observation, nomenclature, and most important must include those salient attributes of the system that are somehow abstractly or conceptually manifest in the technique. The inspiration is described textually with citations and may include diagrams to highlight features and relationships within the inspiring system.
% Optional
\subsection{Inspiration}
% What is the system or process that motivated the development of a technique?
% Which features of the motivating system are relevant to a technique?
A textual description of the inspiring system.

% Metaphor: Explanation via analogy
% The metaphor is a description of the technique in the context of the inspiring system or a different suitable system. The features of the technique are made apparent through an analogous description of the features of the inspiring system. The explanation through analogy is not expected to be literal scientific truth, rather the method is used as an allegorical communication tool. The inspiring system is not explicitly described, this is the role of the ‘inspiration’ element, which represents a loose dependency for this element. The explanation is textual and uses the nomenclature of the metaphorical system.
% Optional
\subsection{Metaphor}
% What is the explanation of a technique in the context of the inspiring system?
% What are the functionalities inferred for a technique from the analogous inspiring system?
A textual description of the algorithm by analogy.

% Strategy: Problem solving plan
% The strategy is an abstract description of the computational model. The strategy describes the information processing actions a technique shall take in order to achieve an objective. The strategy provides a logical separation between a computational realization (procedure) and a analogous system (metaphor). A given problem solving strategy may be realized as one of a number specific algorithms or problem solving systems. The strategy description is textual using information processing and algorithmic terminology.
\subsection{Strategy}
% What is the information processing objective of a technique?
% What is a techniques plan of action?
A textual description of the information processing strategy.

% Procedure: Abstract computation
% The algorithmic procedure summarizes the specifics of realizing a strategy as a systemized and parameterized computation. It outlines how the algorithm is organized in terms of the data structures and representations. The procedure may be described in terms of software engineering and computer science artifacts such as pseudo code, design diagrams, and relevant mathematical equations.
\subsection{Procedure}
% What is the computational recipe for a technique?
% What are the data structures and representations used in a technique?
A pseudo code description of the algorithms procedure.

% Heuristics: Usage guidelines
% The heuristics element describe the commonsense, best practice, and demonstrated rules for applying and configuring a parameterized algorithm. The heuristics relate to the technical details of the techniques procedure and data structures for general classes of application (neither specific implementations not specific problem instances). The heuristics are described textually, such as a series of guidelines in a bullet-point structure.
\subsection{Heuristics}
% What are the suggested configurations for a technique?
% What are the guidelines for the application of a technique to a problem instance?
A bullet-point listing of best practice usage.

% Code Listing
% The code description provides a minimal but functional version of the technique implemented with a programming language. The code description must be able to be typed into an appropriate computer, compiled or interpreted as need be, and provide a working execution of the technique. The technique implementation also includes a minimal problem instance to which it is applied, and both the problem and algorithm implementations are complete enough to demonstrate the techniques procedure. The description is presented as a programming source code listing.
\subsection{Code Listing}
% How is a technique implemented as an executable program?
% How is a technique applied to a concrete problem instance?
A code listing and a terse description of the listing.

% References: Deeper understanding
% The references element description includes a listing of both primary sources of information about the technique as well as useful introductory sources for novices to gain a deeper understanding of the theory and application of the technique. The description consists of hand-selected reference material including books, peer reviewed conference papers, journal articles, and potentially websites. A bullet-pointed structure is suggested.
\subsection{References}
% What are the primary sources for a technique?
% What are the suggested reference sources for learning more about a technique?
An bullet-point annotated reference list of primary sources and useful resources.


\putbib\end{bibunit}
\newpage\begin{bibunit}% The Clever Algorithms Project: http://www.CleverAlgorithms.com
% (c) Copyright 2010 Jason Brownlee. Some Rights Reserved. 
% This work is licensed under a Creative Commons Attribution-Noncommercial-Share Alike 2.5 Australia License.

% This is an algorithm description, see:
% Jason Brownlee. A Template for Standardized Algorithm Descriptions. Technical Report CA-TR-20100107-1, The Clever Algorithms Project http://www.CleverAlgorithms.com, January 2010.

% Name
% The algorithm name defines the canonical name used to refer to the technique, in addition to common aliases, abbreviations, and acronyms. The name is used in terms of the heading and sub-headings of an algorithm description.
\section{Artificial Immune Recognition System} 
\label{sec:airs}
\index{Artificial Immune Recognition System}
\index{AIRS}

% other names
% What is the canonical name and common aliases for a technique?
% What are the common abbreviations and acronyms for a technique?
\emph{Artificial Immune Recognition System, AIRS.}

% Taxonomy: Lineage and locality
% The algorithm taxonomy defines where a techniques fits into the field, both the specific subfields of Computational Intelligence and Biologically Inspired Computation as well as the broader field of Artificial Intelligence. The taxonomy also provides a context for determining the relation- ships between algorithms. The taxonomy may be described in terms of a series of relationship statements or pictorially as a venn diagram or a graph with hierarchical structure.
\subsection{Taxonomy}
% To what fields of study does a technique belong?
The Artificial Immune Recognition System belongs to the field of Artificial Immune Systems, and more broadly to the field of Computational Intelligence.
% What are the closely related approaches to a technique?
It is was extended early to the canonical version called the  Artificial Immune Recognition System 2 (AIRS2) and provides the basis for extensions such as the Parallel Artificial Immune Recognition System \cite{Watkins2004}.
It is related to other Artificial Immune System algorithms such as the Dendritic Cell Algorithm (Section~\ref{sec:dca}), the Clonal Selection Algorithm (Section~\ref{sec:clonal_selection_algorithm}), and the Negative Selection Algorithm (Section~\ref{sec:negative_selection_algorithm}).

% Inspiration: Motivating system
% The inspiration describes the specific system or process that provoked the inception of the algorithm. The inspiring system may non-exclusively be natural, biological, physical, or social. The description of the inspiring system may include relevant domain specific theory, observation, nomenclature, and most important must include those salient attributes of the system that are somehow abstractly or conceptually manifest in the technique. The inspiration is described textually with citations and may include diagrams to highlight features and relationships within the inspiring system.
% Optional
\subsection{Inspiration}
% What is the system or process that motivated the development of a technique?
The Artificial Immune Recognition System is inspired by the Clonal Selection theory of acquired immunity.
% Which features of the motivating system are relevant to a technique?
The clonal selection theory credited to Burnet was proposed to account for the behavior and capabilities of antibodies in the acquired immune system \cite{Burnet1957, Burnet1959}. Inspired itself by the principles of Darwinian natural selection theory of evolution, the theory proposes that antigens select-for lymphocytes (both B and T-cells). When a lymphocyte is selected and binds to an antigenic determinant, the cell proliferates making many thousands more copies of itself and differentiates into different cell types (plasma and memory cells). Plasma cells have a short lifespan and produce vast quantities of antibody molecules, whereas memory cells live for an extended period in the host anticipating future recognition of the same determinant. The important feature of the theory is that when a cell is selected and proliferates, it is subjected to small copying errors (changes to the genome called somatic hypermutation) that change the shape of the expressed receptors. It also affects the  subsequent determinant recognition capabilities of both the antibodies bound to the lymphocytes cells surface, and the antibodies that plasma cells produce.

% Metaphor: Explanation via analogy
% The metaphor is a description of the technique in the context of the inspiring system or a different suitable system. The features of the technique are made apparent through an analogous description of the features of the inspiring system. The explanation through analogy is not expected to be literal scientific truth, rather the method is used as an allegorical communication tool. The inspiring system is not explicitly described, this is the role of the ‘inspiration’ element, which represents a loose dependency for this element. The explanation is textual and uses the nomenclature of the metaphorical system.
% Optional
\subsection{Metaphor}
% What is the explanation of a technique in the context of the inspiring system?
% What are the functionalities inferred for a technique from the analogous inspiring system?
The theory suggests that starting with an initial repertoire of general immune cells, the system is able to change itself (the compositions and densities of cells and their receptors) in response to experience with the environment. Through a blind process of selection and accumulated variation on the large scale of many billions of cells, the acquired immune system is capable of acquiring the necessary information to protect the host organism from the specific pathogenic dangers of the environment. It also suggests that the system must anticipate (guess) at the pathogen to which it will be exposed, and requires exposure to pathogen that may harm the host before it can acquire the necessary information to provide a defense.

% Strategy: Problem solving plan
% The strategy is an abstract description of the computational model. The strategy describes the information processing actions a technique shall take in order to achieve an objective. The strategy provides a logical separation between a computational realization (procedure) and a analogous system (metaphor). A given problem solving strategy may be realized as one of a number specific algorithms or problem solving systems. The strategy description is textual using information processing and algorithmic terminology.
\subsection{Strategy}
% What is the information processing objective of a technique?
The information processing objective of the technique is to prepare a set of real-valued vectors to classify patterns. 
% What is a techniques plan of action?
The Artificial Immune Recognition System maintains a pool of memory cells that are prepared by exposing the system to a single iteration of the training data. Candidate memory cells are prepared when the memory cells are insufficiently stimulated for a given input pattern. A process of cloning and mutation of cells occurs for the most stimulated memory cell. The clones compete with each other for entry into the memory pool based on stimulation and on the amount of resources each cell is using. This concept of resources comes from prior work on Artificial Immune Networks, where a single cell (an Artificial Recognition Ball or ARB) represents a set of similar cells. Here, a cell's resources are a function of its stimulation to a given input pattern and the number of clones it may create.

% Procedure: Abstract computation
% The algorithmic procedure summarizes the specifics of realizing a strategy as a systemized and parameterized computation. It outlines how the algorithm is organized in terms of the data structures and representations. The procedure may be described in terms of software engineering and computer science artifacts such as Pseudocode, design diagrams, and relevant mathematical equations.
\subsection{Procedure}
% What is the computational recipe for a technique?
Algorithm~\ref{alg:train} provides a high-level pseudocode for preparing memory cell vectors using the Artificial Immune Recognition System, specifically the canonical AIRS2. 
An affinity (distance) measure between input patterns must be defined. For real-valued vectors, this is commonly the Euclidean distance:

\begin{equation}
	dist(x,c) = \sum_{i=1}^{n} (x_i - c_i)^2
\end{equation}

where $n$ is the number of attributes, $x$ is the input vector and $c$ is a given cell vector. The variation of cells during cloning (somatic hypermutation) occurs inversely proportional to the stimulation of a given cell to an input pattern.

\begin{algorithm}[htp]
	\SetLine

	% paramd
	\SetKwData{InputPatterns}{InputPatterns}
	\SetKwData{CloneRate}{$clone_{rate}$}
	\SetKwData{MutateRate}{$mutate_{rate}$}
	\SetKwData{StimulationThreshold}{$stim_{thresh}$}
	\SetKwData{AffinityThreshold}{$affinity_{thresh}$}
	\SetKwData{MaxResources}{$resources_{max}$}
	% data
	\SetKwData{MemoryPool}{$Cells_{memory}$}
	\SetKwData{Clones}{$Cells_{clones}$}
	\SetKwData{Clone}{$Cell_{i}$}
	\SetKwData{Candidate}{$Cell_{c}$}
	\SetKwData{CandidateStimulation}{$Cell_{c}^{stim}$}
	\SetKwData{InputPattern}{$InputPattern_i$}
	\SetKwData{BestMatch}{$Cell_{best}$}
	\SetKwData{BestMatchClass}{$Cell_{best}^{class}$}
	\SetKwData{InputPatternClass}{$InputPattern_{i}^{class}$}
	\SetKwData{NumClones}{$Clones_{num}$}
	\SetKwData{BestMatchStimulation}{$Cell_{best}^{stim}$}
	% functions
	\SetKwFunction{InitializeMemoryPool}{InitializeMemoryPool}
	\SetKwFunction{Stimulate}{Stimulate}
	\SetKwFunction{GetMostStimulated}{GetMostStimulated}
	\SetKwFunction{CreateNewMemoryCell}{CreateNewMemoryCell}
	\SetKwFunction{CloneAndMutate}{CloneAndMutate}
	\SetKwFunction{ReducePoolToMaximumResources}{ReducePoolToMaximumResources}
	\SetKwFunction{Affinity}{Affinity}
	\SetKwFunction{DeleteCell}{DeleteCell}
	\SetKwFunction{AverageStimulation}{AverageStimulation}
	
	% I/O
	\KwIn{\InputPatterns, \CloneRate, \MutateRate, \StimulationThreshold, \MaxResources, \AffinityThreshold}		
	\KwOut{\MemoryPool}
  
	% Algorithm
	\MemoryPool $\leftarrow$ \InitializeMemoryPool{\InputPatterns}\;
	% loop
	\ForEach{\InputPattern $\in$ \InputPatterns}{
		\Stimulate{\MemoryPool, \InputPatterns}\;
		\BestMatch $\leftarrow$ \GetMostStimulated{\InputPattern, \MemoryPool}\;
		\eIf{\BestMatchClass $\neq$ \InputPatternClass} {
			\MemoryPool $\leftarrow$ \CreateNewMemoryCell{\InputPattern}\;
		}{
			\NumClones $\leftarrow$ \BestMatchStimulation $\times$ \CloneRate $\times$ \MutateRate\;
			\Clones $\leftarrow$ \BestMatch\;
			\For{$i$ \KwTo \NumClones} {
				\Clones $\leftarrow$ \CloneAndMutate{\BestMatch}\;
			}
			\While{\AverageStimulation{\Clones} $\leq$ \StimulationThreshold} {
				\ForEach{\Clone $\in$ \Clones} {
					\Clones $\leftarrow$ \CloneAndMutate{\Clone}\;
				}
				\Stimulate{\Clones, \InputPatterns}\;
				\ReducePoolToMaximumResources{\Clones, \MaxResources}\;
			}
			\Candidate $\leftarrow$ \GetMostStimulated{\InputPattern, \Clones}\;
			\If{\CandidateStimulation $>$ \BestMatchStimulation} {
				\MemoryPool $\leftarrow$ \Candidate\;
				\If{\Affinity{\Candidate, \BestMatch} $\leq$ \AffinityThreshold} {
					\DeleteCell{\BestMatch, \MemoryPool}\;
				}
			}			
		}
	}
	\Return{\MemoryPool}\;
	% end
	\caption{Pseudocode for training the Artificial Immune Recognition System (AIRS2).}
	\label{alg:train}
\end{algorithm}

% Heuristics: Usage guidelines
% The heuristics element describe the commonsense, best practice, and demonstrated rules for applying and configuring a parameterized algorithm. The heuristics relate to the technical details of the techniques procedure and data structures for general classes of application (neither specific implementations not specific problem instances). The heuristics are described textually, such as a series of guidelines in a bullet-point structure.
\subsection{Heuristics}
% What are the suggested configurations for a technique?
% What are the guidelines for the application of a technique to a problem instance?
\begin{itemize}
	\item The AIRS was designed as a supervised algorithm for classification problem domains.
	\item The AIRS is non-parametric, meaning that it does not rely on assumptions about that structure of the function that is is approximating.
	\item Real-values in input vectors should be normalized such that $x \in [0,1)$. 
	\item Euclidean distance is commonly used to measure the distance between real-valued vectors (affinity calculation), although other distance measures may be used (such as dot product), and data specific distance measures may be required for non-scalar attributes.
	\item Cells may be initialized with small random values or more commonly with values from instances in the training set.
	\item A cell's affinity is typically minimizing, where as a cells stimulation is maximizing and typically $\in [0,1]$.
\end{itemize}

% Code Listing
% The code description provides a minimal but functional version of the technique implemented with a programming language. The code description must be able to be typed into an appropriate computer, compiled or interpreted as need be, and provide a working execution of the technique. The technique implementation also includes a minimal problem instance to which it is applied, and both the problem and algorithm implementations are complete enough to demonstrate the techniques procedure. The description is presented as a programming source code listing.
\subsection{Code Listing}
% How is a technique implemented as an executable program?
% How is a technique applied to a concrete problem instance?
Listing~\ref{airs} provides an example of the Artificial Immune Recognition System implemented in the Ruby Programming Language. 
% problem
The problem is a contrived classification problem in a 2-dimensional domain $x\in[0,1], y\in[0,1]$ with two classes: `A' ($x\in[0,0.4999999], y\in[0,0.4999999]$) and `B' ($x\in[0.5,1], y\in[0.5,1]$).

% algorithm
The algorithm is an implementation of the AIRS2 algorithm \cite{Watkins2002b}. An initial pool of memory cells is created, one cell for each class. Euclidean distance divided by the maximum possible distance in the domain is taken as the affinity and stimulation is taken as $1.0-affinity$. The meta-dynamics for memory cells (competition for input patterns) is not performed and may be added into the implementation as an extension.

% the listing
\lstinputlisting[firstline=7,language=ruby,caption=Artificial Immune Recognition System in the Ruby Programming Language, label=airs]{../src/algorithms/immune/airs.rb}

% References: Deeper understanding
% The references element description includes a listing of both primary sources of information about the technique as well as useful introductory sources for novices to gain a deeper understanding of the theory and application of the technique. The description consists of hand-selected reference material including books, peer reviewed conference papers, journal articles, and potentially websites. A bullet-pointed structure is suggested.
\subsection{References}
% What are the primary sources for a technique?
% What are the suggested reference sources for learning more about a technique?

% 
% Primary Sources
% 
\subsubsection{Primary Sources}
% seminal
The Artificial Immune Recognition System was proposed in the Masters work by Watkins \cite{Watkins2001}, and later published \cite{Watkins2002a}.
% early
Early works included the application the AIRS by Watkins and Boggess to a suite of benchmark classification problems \cite{Watkins2002}, and a similar study by Goodman and Boggess comparing to a conceptually similar approach called Learning Vector Quantization \cite{Goodman2002}.

% 
% Learn More
% 
\subsubsection{Learn More}
% reviews
Marwah and Boggess investigated the algorithm seeking issues that affect the algorithms performance \cite{Marwah2002}. They compared various variations of the algorithm with modified resource allocation schemes, tie-handling within the ARB pool, and ARB pool organization.
Watkins and Timmis proposed a new version of the algorithm called AIRS2 which became the replacement for AIRS1 \cite{Watkins2002b}. The updates reduced the complexity of the approach while maintaining the accuracy of the results. An investigation by Goodman, et~al. into the so called `\emph{source of power}' in AIRS indicated that perhaps the memory cell maintenance procedures played an important role \cite{Goodman2003}.
% books
Watkins et al.\ provide a detailed review of the technique and its application \cite{Watkins2004a}.


\putbib\end{bibunit}
\newpage\begin{bibunit}% The Clever Algorithms Project: http://www.CleverAlgorithms.com
% (c) Copyright 2010 Jason Brownlee. Some Rights Reserved. 
% This work is licensed under a Creative Commons Attribution-Noncommercial-Share Alike 2.5 Australia License.

% This is an algorithm description, see:
% Jason Brownlee. A Template for Standardized Algorithm Descriptions. Technical Report CA-TR-20100107-1, The Clever Algorithms Project http://www.CleverAlgorithms.com, January 2010.

% Name
% The algorithm name defines the canonical name used to refer to the technique, in addition to common aliases, abbreviations, and acronyms. The name is used in terms of the heading and sub-headings of an algorithm description.
\section{Immune Network Algorithm} 
\label{sec:immune_network_algorithm}
\index{Artificial Immune Network}
\index{aiNet}
\index{opt-aiNet}

% other names
% What is the canonical name and common aliases for a technique?
% What are the common abbreviations and acronyms for a technique?
\emph{Artificial Immune Network, aiNet, Optimization Artificial Immune Network, opt-aiNet.}

% Taxonomy: Lineage and locality
% The algorithm taxonomy defines where a techniques fits into the field, both the specific subfields of Computational Intelligence and Biologically Inspired Computation as well as the broader field of Artificial Intelligence. The taxonomy also provides a context for determining the relation- ships between algorithms. The taxonomy may be described in terms of a series of relationship statements or pictorially as a venn diagram or a graph with hierarchical structure.
\subsection{Taxonomy}
% To what fields of study does a technique belong?
The Artificial Immune Network algorithm (aiNet) is a Immune Network Algorithm from the field of Artificial Immune Systems.
% What are the closely related approaches to a technique?
It is related to other Artificial Immune System algorithms such as the Clonal Selection Algorithm (Section~\ref{sec:clonal_selection_algorithm}), the Negative Selection Algorithm (Section~\ref{sec:negative_selection_algorithm}), and the Dendritic Cell Algorithm (Section~\ref{sec:dca}).
% others
Artificial Immune Network algorithm includes the base version and the extension for optimization problems called the Optimization Artificial Immune Network algorithm (opt-aiNet).

% Inspiration: Motivating system
% The inspiration describes the specific system or process that provoked the inception of the algorithm. The inspiring system may non-exclusively be natural, biological, physical, or social. The description of the inspiring system may include relevant domain specific theory, observation, nomenclature, and most important must include those salient attributes of the system that are somehow abstractly or conceptually manifest in the technique. The inspiration is described textually with citations and may include diagrams to highlight features and relationships within the inspiring system.
% Optional
\subsection{Inspiration}
% What is the system or process that motivated the development of a technique?
The Artificial Immune Network algorithm is inspired by the Immune Network theory of the acquired immune system.
% Which features of the motivating system are relevant to a technique?
The clonal selection theory of acquired immunity accounts for the adaptive behavior of the immune system including the ongoing selection and proliferation of cells that select-for potentially harmful (and typically foreign) material in the body.
A concern of the clonal selection theory is that it presumes that the repertoire of reactive cells remains idle when there are no pathogen to which to respond. Jerne proposed an Immune Network Theory (Idiotypic Networks) where immune cells are not at rest in the absence of pathogen, instead antibody and immune cells recognize and respond to each other \cite{Jerne1974, Jerne1974a, Jerne1984}. 

The Immune Network theory proposes that antibody (both free floating and surface bound) possess idiotopes (surface features) to which the receptors of other antibody can bind. As a result of receptor interactions, the repertoire becomes dynamic, where receptors continually both inhibit and excite each other in complex regulatory networks (chains of receptors). The theory suggests that the clonal selection process may be triggered by the idiotopes of other immune cells and molecules in addition to the surface characteristics of pathogen, and that the maturation process applies both to the receptors themselves the idiotopes which they expose.

% Metaphor: Explanation via analogy
% The metaphor is a description of the technique in the context of the inspiring system or a different suitable system. The features of the technique are made apparent through an analogous description of the features of the inspiring system. The explanation through analogy is not expected to be literal scientific truth, rather the method is used as an allegorical communication tool. The inspiring system is not explicitly described, this is the role of the ‘inspiration’ element, which represents a loose dependency for this element. The explanation is textual and uses the nomenclature of the metaphorical system.
% Optional
\subsection{Metaphor}
% What is the explanation of a technique in the context of the inspiring system?
% What are the functionalities inferred for a technique from the analogous inspiring system?
The immune network theory has interesting resource maintenance and signaling information processing properties.
The classical clonal selection and negative selection paradigms integrate the accumulative and filtered learning of the acquired immune system, whereas the immune network theory proposes an additional order of complexity between the cells and molecules under selection. In addition to cells that interact directly with pathogen, there are cells that interact with those reactive cells and with pathogen indirectly, in successive layers such that networks of activity for higher-order structures such as internal images of pathogen (promotion), and regulatory networks (so-called anti-idiotopes and anti-anti-idiotopes).

% Strategy: Problem solving plan
% The strategy is an abstract description of the computational model. The strategy describes the information processing actions a technique shall take in order to achieve an objective. The strategy provides a logical separation between a computational realization (procedure) and a analogous system (metaphor). A given problem solving strategy may be realized as one of a number specific algorithms or problem solving systems. The strategy description is textual using information processing and algorithmic terminology.
\subsection{Strategy}
% What is the information processing objective of a technique?
The objective of the immune network process is to prepare a repertoire of discrete pattern detectors for a given problem domain, where better performing cells suppress low-affinity (similar) cells in the network.
% What is a techniques plan of action?
This principle is achieved through an interactive process of exposing the population to external information to which it responds with both a clonal selection response and internal meta-dynamics of intra-population responses that stabilizes the responses of the population to the external stimuli.

% Procedure: Abstract computation
% The algorithmic procedure summarizes the specifics of realizing a strategy as a systemized and parameterized computation. It outlines how the algorithm is organized in terms of the data structures and representations. The procedure may be described in terms of software engineering and computer science artifacts such as Pseudocode, design diagrams, and relevant mathematical equations.
\subsection{Procedure}
% What is the computational recipe for a technique?
% What are the data structures and representations used in a technique?
Algorithm~\ref{alg:opt_ainet} provides a pseudocode listing of the Optimization Artificial Immune Network algorithm (opt-aiNet) for minimizing a cost function. 

\begin{algorithm}[ht]
	\SetLine  

	% data
	\SetKwData{Best}{$S_{best}$}
	\SetKwData{NumClones}{$N_{clones}$}
	\SetKwData{NumRandomCells}{$N_{random}$}
	\SetKwData{Progeny}{Progeny}
	\SetKwData{Clones}{Clones}
	\SetKwData{ProblemSize}{ProblemSize}
	\SetKwData{Population}{Population}
	\SetKwData{PopulationSize}{$Population_{size}$}
	\SetKwData{Cell}{$Cell_{i}$}
	\SetKwData{Clone}{$Clone_{i}$}
	\SetKwData{Cost}{$Cost_{avg}$}
	\SetKwData{AffinityThreshold}{AffinityThreshold}
	% functions
	\SetKwFunction{InitializePopulation}{InitializePopulation}  
	\SetKwFunction{EvaluatePopulation}{EvaluatePopulation} 
	\SetKwFunction{GetBestSolution}{GetBestSolution} 
	\SetKwFunction{CreateRandomCells}{CreateRandomCells}
	\SetKwFunction{StopCondition}{StopCondition}
	\SetKwFunction{MutateRelativeToFitnessOfParent}{MutateRelativeToFitnessOfParent}
	\SetKwFunction{CalculateAveragePopulationCost}{CalculateAveragePopulationCost}
	\SetKwFunction{CreateClones}{CreateClones}  
	\SetKwFunction{SupressLowAffinityCells}{SupressLowAffinityCells}

	% I/O
	\KwIn{\PopulationSize, \ProblemSize, \NumClones, \NumRandomCells, \AffinityThreshold}		
	\KwOut{\Best}
  % Algorithm
	% initialize	
	\Population $\leftarrow$ \InitializePopulation{\PopulationSize, \ProblemSize}\;
	% loop
	\While{$\neg$\StopCondition{}} {
		% evaluate
		\EvaluatePopulation{\Population}\;
		% best
		\Best $\leftarrow$ \GetBestSolution{\Population}\;
		% state of population
		\Progeny $\leftarrow \emptyset$\;
		\Cost $\leftarrow$ \CalculateAveragePopulationCost{\Population}\;
		\Repeat{\CalculateAveragePopulationCost{\Population} $\leq$ \Cost} {
			clone
			\ForEach{\Cell $\in$ \Population}{
				% clone
				\Clones $\leftarrow$ \CreateClones{\Cell, \NumClones}\;
				\ForEach{\Clone $\in$ \Clones} {
					% mutate
					\Clone $\leftarrow$ \MutateRelativeToFitnessOfParent{\Clone, \Cell}\;				
				}
				% select best from the clone
				\EvaluatePopulation{\Clones}\;
				\Progeny $\leftarrow$ \GetBestSolution{\Clones}\;			
			}
		}
		% network effects
		\SupressLowAffinityCells{\Progeny, \AffinityThreshold}\;
		% random cells
		\Progeny $\leftarrow$ \CreateRandomCells{\NumRandomCells}\;		
		% replace
		\Population $\leftarrow$ \Progeny\;
	}
	\Return{\Best}\;
	% end
	\caption{Pseudocode for opt-aiNet.}
	\label{alg:opt_ainet}
\end{algorithm}

% Heuristics: Usage guidelines
% The heuristics element describe the commonsense, best practice, and demonstrated rules for applying and configuring a parameterized algorithm. The heuristics relate to the technical details of the techniques procedure and data structures for general classes of application (neither specific implementations not specific problem instances). The heuristics are described textually, such as a series of guidelines in a bullet-point structure.
\subsection{Heuristics}
% What are the suggested configurations for a technique?
% What are the guidelines for the application of a technique to a problem instance?
\begin{itemize}
	\item aiNet is designed for unsupervised clustering, where as the opt-aiNet extension was designed for pattern recognition and optimization, specifically multi-modal function optimization.
	\item The amount of mutation of clones is proportionate to the affinity of the parent cell with the cost function (better fitness, lower mutation).
	\item The addition of random cells each iteration adds a random-restart like capability to the algorithms.
	\item Suppression based on cell similarity provides a mechanism for reducing redundancy.
	\item The population size is dynamic, and if it continues to grow it may be an indication of a problem with many local optima or that the affinity threshold may needs to be increased.
	\item Affinity proportionate mutation is performed using $c' = c + \alpha \times N(1,0)$ where $\alpha = \frac{1}{\beta} \times exp(-f)$, $N$ is a Guassian random number, and $f$ is the fitness of the parent cell, $\beta$ controls the decay of the function and can be set to 100. 
	\item The affinity threshold is problem and representation specific, for example a $AffinityThreshold$ may be set to an arbitrary value such as 0.1 on a continuous function domain, or calculated as a percentage of the size of the problem space.
	\item The number of random cells inserted may be 40\% of the population size. 
	\item The number of clones created for a cell may be small, such as 10.
\end{itemize}

% Code Listing
% The code description provides a minimal but functional version of the technique implemented with a programming language. The code description must be able to be typed into an appropriate computer, compiled or interpreted as need be, and provide a working execution of the technique. The technique implementation also includes a minimal problem instance to which it is applied, and both the problem and algorithm implementations are complete enough to demonstrate the techniques procedure. The description is presented as a programming source code listing.
\subsection{Code Listing}
% How is a technique implemented as an executable program?
% How is a technique applied to a concrete problem instance?
Listing~\ref{optainet} provides an example of the Optimization Artificial Immune Network (opt-aiNet) implemented in the Ruby Programming Language.
% problem
The demonstration problem is an instance of a continuous function optimization that seeks $\min f(x)$ where $f=\sum_{i=1}^n x_{i}^2$, $-5.0\leq x_i \leq 5.0$ and $n=2$. The optimal solution for this basin function is $(v_0,\ldots,v_{n-1})=0.0$.
% algorithm
The algorithm is an implementation based on the specification by de~Castro and Von Zuben \cite{Castro2002c}.

% the listing
\lstinputlisting[firstline=7,language=ruby,caption=Optimization Artificial Immune Network in Ruby, label=optainet]{../src/algorithms/immune/optainet.rb}

% References: Deeper understanding
% The references element description includes a listing of both primary sources of information about the technique as well as useful introductory sources for novices to gain a deeper understanding of the theory and application of the technique. The description consists of hand-selected reference material including books, peer reviewed conference papers, journal articles, and potentially websites. A bullet-pointed structure is suggested.
\subsection{References}
% What are the primary sources for a technique?
% What are the suggested reference sources for learning more about a technique?

% 
% Primary Sources
% 
\subsubsection{Primary Sources}
% early
Early works, such as Farmer, et~al. \cite{Farmer1986} suggested at the exploitation of the information processing properties of network theory for machine learning.
% pre-cursor
A seminal network theory based algorithm was proposed by Timmis, et~al. for clustering problems called the Artificial Immune Network (AIN) \cite{Timmis2000} that was later extended and renamed the Resource Limited Artificial Immune System \cite{Timmis2001} and Artificial Immune Network (AINE) \cite{Knight2001}.
% seminal
The Artificial Immune Network (aiNet) algorithm was proposed by de~Castro and Von Zuben that extended the principles of the Artificial Immune Network (AIN) and the Clonal Selection Algorithm (CLONALG) and was applied to clustering \cite{Castro2000a}. The aiNet algorithm was later further extended to optimization domains and renamed opt-aiNet \cite{Castro2002c}.

% 
% Learn More
% 
\subsubsection{Learn More}
% extensions
The authors de~Castro and Von Zuben provide a detailed presentation of the aiNet algorithm as a book chapter that includes immunological theory, a description of the algorithm, and demonstration application to clustering problem instances \cite{Castro2001}.
% fixes
Timmis and Edmonds provide a careful examination of the opt-aiNet algorithm and propose some modifications and augmentations to improve its applicability and performance for multimodal function optimization problem domains \cite{Timmis2004}.
% dynamic
The authors de~Franca, Von~Zuben, and de~Castro proposed an extension to opt-aiNet that provided a number of enhancements and adapted its capability for for dynamic function optimization problems called dopt-aiNet \cite{Franca2005}.


\putbib\end{bibunit}
\newpage\begin{bibunit}% The Clever Algorithms Project: http://www.CleverAlgorithms.com
% (c) Copyright 2010 Jason Brownlee. Some Rights Reserved. 
% This work is licensed under a Creative Commons Attribution-Noncommercial-Share Alike 2.5 Australia License.

% This is an algorithm description, see:
% Jason Brownlee. A Template for Standardized Algorithm Descriptions. Technical Report CA-TR-20100107-1, The Clever Algorithms Project http://www.CleverAlgorithms.com, January 2010.

% Name
% The algorithm name defines the canonical name used to refer to the technique, in addition to common aliases, abbreviations, and acronyms. The name is used in terms of the heading and sub-headings of an algorithm description.
\section{Dendritic Cell Algorithm} 
\label{sec:dca}
\index{Dendritic Cell Algorithm}

% other names
% What is the canonical name and common aliases for a technique?
% What are the common abbreviations and acronyms for a technique?
\emph{Dendritic Cell Algorithm, DCA.}

% Taxonomy: Lineage and locality
% The algorithm taxonomy defines where a techniques fits into the field, both the specific subfields of Computational Intelligence and Biologically Inspired Computation as well as the broader field of Artificial Intelligence. The taxonomy also provides a context for determining the relation- ships between algorithms. The taxonomy may be described in terms of a series of relationship statements or pictorially as a venn diagram or a graph with hierarchical structure.
\subsection{Taxonomy}
% To what fields of study does a technique belong?
The Dendritic Cell Algorithm belongs to the field of Artificial Immune Systems, and more broadly to the field of Computational Intelligence.
% What are the closely related approaches to a technique?
The Dendritic Cell Algorithm is the basis for extensions such as the Deterministic Dendritic Cell Algorithm (dDCA) \cite{Greensmith2008}.
It is generally related to other Artificial Immune System algorithms such as the Clonal Selection Algorithm (Section~\ref{sec:clonal_selection_algorithm}), and the Immune Network Algorithm (Section~\ref{sec:immune_network_algorithm}).

% Inspiration: Motivating system
% The inspiration describes the specific system or process that provoked the inception of the algorithm. The inspiring system may non-exclusively be natural, biological, physical, or social. The description of the inspiring system may include relevant domain specific theory, observation, nomenclature, and most important must include those salient attributes of the system that are somehow abstractly or conceptually manifest in the technique. The inspiration is described textually with citations and may include diagrams to highlight features and relationships within the inspiring system.
% Optional
\subsection{Inspiration}
% What is the system or process that motivated the development of a technique?
The Dendritic Cell Algorithm is inspired by the Danger Theory of the mammalian immune system, and specifically the role and function of dendritic cells. 
% Which features of the motivating system are relevant to a technique?
The Danger Theory was proposed by Matzinger and suggests that the roles of the acquired immune system is to respond to signals of danger, rather than discriminating self from non-self \cite{Matzinger1994, Matzinger2002}. The theory suggests that antigen presenting cells (such as helper T-cells) activate an alarm signal providing the necessarily co-stimulation of antigen-specific cells to respond. Dendritic cells are a type of cell from the innate immune system that respond to some specific forms of danger signals. There are three main types of dendritic cells: `immature' that collect parts of the antigen and the signals, `semi-mature' that are immature cells that internally decide that the local signals represent safe and present the antigen to T-cells resulting in tolerance, and `mature' cells that internally decide that the local signals represent danger and present the antigen to T-cells resulting in a reactive response.

% Metaphor: Explanation via analogy
% The metaphor is a description of the technique in the context of the inspiring system or a different suitable system. The features of the technique are made apparent through an analogous description of the features of the inspiring system. The explanation through analogy is not expected to be literal scientific truth, rather the method is used as an allegorical communication tool. The inspiring system is not explicitly described, this is the role of the ‘inspiration’ element, which represents a loose dependency for this element. The explanation is textual and uses the nomenclature of the metaphorical system.
% Optional
% \subsection{Metaphor}
% What is the explanation of a technique in the context of the inspiring system?
% What are the functionalities inferred for a technique from the analogous inspiring system?
% TODO

% Strategy: Problem solving plan
% The strategy is an abstract description of the computational model. The strategy describes the information processing actions a technique shall take in order to achieve an objective. The strategy provides a logical separation between a computational realization (procedure) and a analogous system (metaphor). A given problem solving strategy may be realized as one of a number specific algorithms or problem solving systems. The strategy description is textual using information processing and algorithmic terminology.
\subsection{Strategy}
% What is the information processing objective of a technique?
The information processing objective of the algorithm is to prepare a set of mature dendritic cells (prototypes) that provide context specific information about how to classify normal and anomalous input patterns.
% What is a techniques plan of action?
This is achieved as a system of three asynchronous processes of 1) migrating sufficiently stimulated immature cells, 2) promoting migrated cells to semi-mature (safe) or mature (danger) status depending in their accumulated response, and 3) labeling observed patterns as safe or dangerous based on the composition of the sub-population of cells that respond to each pattern.

% Procedure: Abstract computation
% The algorithmic procedure summarizes the specifics of realizing a strategy as a systemized and parameterized computation. It outlines how the algorithm is organized in terms of the data structures and representations. The procedure may be described in terms of software engineering and computer science artifacts such as Pseudocode, design diagrams, and relevant mathematical equations.
\subsection{Procedure}
% What is the computational recipe for a technique?
% What are the data structures and representations used in a technique?
Algorithm~\ref{alg:dca} provides pseudocode for training a pool of cells in the Dendritic Cell Algorithm, specifically the Deterministic Dendritic Cell Algorithm. Mature migrated cells associate their collected input patterns with anomalies, whereas semi-mature migrated cells associate their collected input patterns as normal.
The resulting migrated cells can then be used to classify input patterns as normal or anomalous. This can be done through sampling the cells and using a voting mechanism, or more elaborate methods such as a `mature context antigen value' (MCAV) which is $\frac{M}{Ag}$ (where $M$ is the number of mature cells with the antigen and $Ag$ is the sum of the exposures to the antigen by those mature cells), which gives a probability of a pattern being an anomaly.

\begin{algorithm}[htp]
	\SetLine

	% data
	\SetKwData{MaxIterations}{$iterations_{max}$}
	\SetKwData{NumCells}{$cells_{num}$}
	\SetKwData{MigrationThresholdBounds}{$MigrationThresh_{bounds}$}
	\SetKwData{InputPatterns}{InputPatterns}
	\SetKwData{Pattern}{$P_i$}
	\SetKwData{PatternDanger}{$Pi_{danger}$}
	\SetKwData{PatternSafe}{$Pi_{safe}$}
	\SetKwData{PatternAntigen}{$Pi_{antigen}$}
	\SetKwData{K}{$k_i$}
	\SetKwData{CSM}{$cms_i$}
	\SetKwData{Cell}{$Cell_i$}
	\SetKwData{CellLifespan}{$Celli_{lifespan}$}
	\SetKwData{CellMigrationThreshold}{$Celli_{thresh}$}
	\SetKwData{CellType}{$Celli_{type}$}
	\SetKwData{CellCSM}{$Celli_{csm}$}
	\SetKwData{CellK}{$Celli_{k}$}
	\SetKwData{ImmatureCells}{ImmatureCells}
	\SetKwData{MigratedCells}{MigratedCells}
	\SetKwData{Mature}{Mature}
	\SetKwData{Semimature}{Semimature}
	
	% functions
	\SetKwFunction{InitializeCells}{InitializeCells}
	\SetKwFunction{SelectInputPattern}{SelectInputPattern}
	\SetKwFunction{StoreAntigen}{StoreAntigen}
	\SetKwFunction{UpdateCellOutputSignals}{UpdateCellOutputSignals}
	\SetKwFunction{ReInitializeCell}{ReInitializeCell}
	\SetKwFunction{RemoveCell}{RemoveCell}
	\SetKwFunction{CreateNewCell}{CreateNewCell}
	
	% I/O
	\KwIn{\InputPatterns, \MaxIterations, \NumCells, \MigrationThresholdBounds}		
	\KwOut{\MigratedCells}
  
	% Algorithm
	\ImmatureCells $\leftarrow$ \InitializeCells{\NumCells, \MigrationThresholdBounds}\;
	\MigratedCells $\leftarrow$ $\emptyset$\;
	% loop
	\For{$i=1$ \KwTo \MaxIterations} {
		\Pattern $\leftarrow$ \SelectInputPattern{\InputPatterns}\;
		\K $\leftarrow$ (\PatternDanger $-$ 2 $\times$ \PatternSafe)\;
		\CSM $\leftarrow$ (\PatternDanger + \PatternSafe)\;
		\ForEach{\Cell $\in$ \ImmatureCells} {
			\UpdateCellOutputSignals{\Cell, \K, \CSM}\;
			\StoreAntigen{\Cell, \PatternAntigen}\;
			\uIf{\CellLifespan $\leq$ $0$} {
				\ReInitializeCell{\Cell}\;
			}
			\ElseIf{\CellCSM $\geq$ \CellMigrationThreshold}{
				\RemoveCell{\ImmatureCells, \Cell}\;
				\ImmatureCells $\leftarrow$ \CreateNewCell{\MigrationThresholdBounds}\;				
				\eIf{\CellK $<$ $0$} {
					\CellType $\leftarrow$ \Mature\;
				} {
					\CellType $\leftarrow$ \Semimature\;
				}
				\MigratedCells $\leftarrow$ \Cell\;
			}
		}
	}
	\Return{\MigratedCells}\;
	% end
	\caption{Pseudocode for the Dendritic Cell Algorithm.}
	\label{alg:dca}
\end{algorithm}

% Heuristics: Usage guidelines
% The heuristics element describe the commonsense, best practice, and demonstrated rules for applying and configuring a parameterized algorithm. The heuristics relate to the technical details of the techniques procedure and data structures for general classes of application (neither specific implementations not specific problem instances). The heuristics are described textually, such as a series of guidelines in a bullet-point structure.
\subsection{Heuristics}
% What are the suggested configurations for a technique?
% What are the guidelines for the application of a technique to a problem instance?
\begin{itemize}
	\item The Dendritic Cell Algorithm is not specifically a classification algorithm, it may be considered a data filtering method for use in anomaly detection problems.
	\item The canonical algorithm is designed to operate on a single discrete, categorical or ordinal input and two probabilistic specific signals indicating the heuristic danger or safety of the input.
	\item The \texttt{danger} and \texttt{safe} signals are problem specific signals of the risk that the input pattern is an anomaly or is normal, both typically  $\in [0,100]$.
	\item The \texttt{danger} and \texttt{safe} signals do not have to be reciprocal, meaning they may provide conflicting information.
	\item The system was designed be used in real-time anomaly detection problems, not just static problem.
	\item Each cells migration threshold is set separately, typically $\in [5,15]$
\end{itemize}

% Code Listing
% The code description provides a minimal but functional version of the technique implemented with a programming language. The code description must be able to be typed into an appropriate computer, compiled or interpreted as need be, and provide a working execution of the technique. The technique implementation also includes a minimal problem instance to which it is applied, and both the problem and algorithm implementations are complete enough to demonstrate the techniques procedure. The description is presented as a programming source code listing.
\subsection{Code Listing}
% How is a technique implemented as an executable program?
% How is a technique applied to a concrete problem instance?
Listing~\ref{dca} provides an example of the Dendritic Cell Algorithm implemented in the Ruby Programming Language, specifically the Deterministic Dendritic Cell Algorithm (dDCA).
% problem
The problem is a contrived anomaly-detection problem with ordinal inputs $\in [0,50]$ , where values that divide by 10 with no remainder are considered anomalies. Probabilistic safe and danger signal functions are provided, suggesting danger signals correctly with $P(danger)=0.70$, and safe signals correctly with $P(safe)=0.95$.

% algorithm
The algorithm is an implementation of the Deterministic Dendritic Cell Algorithm (dDCA) as described in \cite{Stibor2009, Greensmith2008}, with verification from \cite{Greensmith2006a}. The algorithm was designed to be executed as three asynchronous processes in a real-time or semi-real time environment. For demonstration purposes, the implementation separated out the three main processes and executed the sequentially as a training and cell promotion phase followed by a test (labeling phase).

% the listing
\lstinputlisting[firstline=7,language=ruby,caption=Deterministic Dendritic Cell Algorithm (dDCA) in the Ruby Programming Language, label=dca]{../src/algorithms/immune/dendritic_cell_algorithm.rb}

% References: Deeper understanding
% The references element description includes a listing of both primary sources of information about the technique as well as useful introductory sources for novices to gain a deeper understanding of the theory and application of the technique. The description consists of hand-selected reference material including books, peer reviewed conference papers, journal articles, and potentially websites. A bullet-pointed structure is suggested.
\subsection{References}
% What are the primary sources for a technique?
% What are the suggested reference sources for learning more about a technique?

% 
% Primary Sources
% 
\subsubsection{Primary Sources}
% seminal
The Dendritic Cell Algorithm was proposed by Greensmith, Aickelin and Cayzer describing the inspiring biological system and providing experimental results on a classification problem \cite{Greensmith2005}.
% early
This work was followed shortly by a second study into the algorithm by Greensmith, Twycross, and Aickelin, focusing on computer security instances of anomaly detection and classification problems \cite{Greensmith2006}.

% 
% Learn More
% 
\subsubsection{Learn More}
% reviews
The Dendritic Cell Algorithm was the focus of Greensmith's thesis, which provides a detailed discussion of the methods abstraction from the inspiring biological system, and a review of the technique's limitations \cite{Greensmith2007}. 
A formal presentation of the algorithm is provided by Greenwmith et al.\ \cite{Greensmith2006a}.
Greensmith and Aickelin proposed the Deterministic Dendritic Cell Algorithm (dDCA) that seeks to remove some of the stochastic decisions from the method, reduce the complexity and to make it more amenable to analysis \cite{Greensmith2008}.
Stibor et al.\ provide a theoretical analysis of the Deterministic Dendritic Cell Algorithm, considering the discrimination boundaries of single dendrite cells in the system \cite{Stibor2009}. 
% books
Greensmith and Aickelin provide a detailed overview of the Dendritic Cell Algorithm focusing in the information processing principles of the inspiring biological systems as a book chapter \cite{Greensmith2009}.


\putbib\end{bibunit}
