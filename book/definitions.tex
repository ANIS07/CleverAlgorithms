% The Clever Algorithms Project: http://www.CleverAlgorithms.com
% (c) Copyright 2010 Jason Brownlee. Some Rights Reserved. 
% This work is licensed under a Creative Commons Attribution-Noncommercial-Share Alike 2.5 Australia License.


% 
% Definitions
% 

% The main title of the book
\newcommand{\mybooktitle}{Clever Algorithms}
% The sub title of the book
\newcommand{\mybooksubtitle}{Nature-Inspired Programming Recipes}
% title
\newcommand{\mybookauthor}{Jason Brownlee}
% date
\newcommand{\mybookdate}{2011}


% new macro for starting a new page and changing the style to empty
% \newpage == ends the current page. 
% \thispagestyle == works in the same manner as the \pagestyle, except that it changes the style for the current page only. 
% empty == Produces empty heads and feet - no page numbers
\newcommand{\blanknonumber}{\newpage\thispagestyle{empty}}


% 
% Packages
% 

% a replacement for fancyheadings
% http://www.ctan.org/tex-archive/macros/latex/contrib/fancyhdr/
\usepackage{fancyhdr}
% \pagestyle{fancy}
\fancyhead[LE,RO]{\thepage}
\fancyhead[LO]{\slshape \leftmark}
\fancyhead[RE]{\slshape \rightmark}
\renewcommand{\headrulewidth}{0.4pt}
\renewcommand{\sectionmark}[1]{\markright{\thesection.\ #1}}
\renewcommand{\chaptermark}[1]{\markboth{\chaptername\ \thechapter.\ #1}{}}


% If you want to generate an index, automatically
% http://en.wikibooks.org/wiki/LaTeX/Indexing
\usepackage{makeidx} 

% Flexible and easy interface to page dimensions
% http://www.ctan.org/tex-archive/macros/latex/contrib/geometry/
% also, bigger pages by default
\usepackage[pdftex]{geometry}

% Supports the Text Companion fonts which provide many text symbols (such as baht, bullet, copyright, musicalnote, onequarter, section, and yen) in the TS1 encoding.
% http://www.ctan.org/tex-archive/help/Catalogue/entries/textcomp.html
% needed for listings
\usepackage{textcomp}

% http://www.maths.adelaide.edu.au/anthony.roberts/LaTeX/ltxusecol.html
% needed for listings - lots of pretty colors
\usepackage[usenames,dvipsnames]{color}

% bold in ttfamily
\usepackage{bold-extra}

% code listings (lots of languages)
% http://mirror.aarnet.edu.au/pub/CTAN/macros/latex/contrib/listings/
\usepackage{listings} 
% define the look of the ruby code
% \lstset{language=ruby, 
%   basicstyle=\footnotesize\ttfamily, 
%   numbers=left, 
%   numberstyle=\tiny, 
%   frame=single, 
%   columns=flexible, 
%   upquote=true, 
%   showstringspaces=false, 
%   tabsize=2, 
%   captionpos=b,
%   breaklines=true,
%   breakatwhitespace=true,
%   keywordstyle=\color{blue}, 
%   stringstyle=\color{ForestGreen}, 
%   commentstyle=\color{Gray}}

\lstset{language=ruby, 
	basicstyle=\footnotesize\ttfamily, 
  numbers=left, 
  numberstyle=\tiny, 
	keywordstyle=\bfseries\ttfamily,
  frame=single, 
  columns=flexible, 
  upquote=true, 
  showstringspaces=false, 
  tabsize=2, 
  captionpos=b,
  breaklines=true,
  breakatwhitespace=true}

% for ebooks, turn cross references into links
% http://www.tug.org/applications/hyperref/manual.html
\usepackage[pdftex,
	breaklinks=true,
	colorlinks=true,
	urlcolor=blue,
	linkcolor=blue,
	citecolor=blue]{hyperref}

% modifies the widths of certain columns, rather than the inter column space, to set a table with the requested total width
% http://www.cs.brown.edu/system/software/latex/doc/tabularx.pdf
\usepackage{tabularx}

% This package provide some additional commands to enhance the quality of tables in LaTeX.
% http://www.ctan.org/tex-archive/macros/latex/contrib/booktabs/
\usepackage{booktabs}

% a form of verbatim command that allows linebreaks at certain characters or combinations of characters
% http://www.tex.ac.uk/tex-archive/help/Catalogue/entries/url.html
% works well with hyperref
\usepackage{url}

% Algorithm2e is an environment for writing algorithms in LaTeX2e
% http://www.ctan.org/tex-archive/macros/latex/contrib/algorithm2e/
\usepackage[noline, algoruled, linesnumbered, algosection]{styles/algorithm2e}

% for adding in bib for each section or chapter
% http://merkel.zoneo.net/Latex/natbib.php
\usepackage[numbers, sort&compress]{natbib}
\usepackage{styles/bibunits}

% maths
\usepackage{amsmath}
\usepackage{latexsym}

% graphics
\usepackage{graphicx}

