% The Clever Algorithms Project: http://www.CleverAlgorithms.com
% (c) Copyright 2010 Jason Brownlee. Some Rights Reserved. 
% This work is licensed under a Creative Commons Attribution-Noncommercial-Share Alike 2.5 Australia License.

% This is a chapter

\renewcommand{\bibsection}{\subsection{\bibname}}
\begin{bibunit}

\chapter{Physical Algorithms}
\label{ch:physical}
\index{Physical Algorithms}

\section{Overview}
This chapter describes Physical Algorithms.


% biological
\subsection{Physical Properties}
Physical algorithms are those algorithms inspired by a physical process. The described physical algorithm generally belong to the fields of Metaheustics and Computational Intelligence, although do not fit neatly into the existing categories of the biological inspired techniques (such as Swarm, Immune, Neural, and Evolution). In this vein, they could just as easily be referred to as nature inspired algorithms.

The inspiring physical systems range from metallurgy, music, the interplay between culture and evolution, and complex dynamic systems such as avalanches. They are generally stochastic optimization algorithms with a mixtures of local (neighborhood-based) and global search techniques.

% 
% Extensions
% 
\subsection{Extensions}
There are many other algorithms and classes of algorithm that were not described inspired by natural systems, not limited to:

\begin{itemize}
	\item \textbf{More Annealing}: Extensions to the classical Simulated Annealing algorithm, such as Adaptive Simulated Annealing (formally Very Fast Simulated Re-annealing) \cite{Ingber1989, Ingber1996}, and Quantum Annealing \cite{Apolloni1989, Das2005}.
	\item \textbf{Stochastic tunneling}: based on the physical idea of a particle tunneling through structures \cite{Wenzel1999}.
\end{itemize}

\putbib
\end{bibunit}

\newpage\begin{bibunit}% The Clever Algorithms Project: http://www.CleverAlgorithms.com
% (c) Copyright 2010 Jason Brownlee. Some Rights Reserved. 
% This work is licensed under a Creative Commons Attribution-Noncommercial-Share Alike 2.5 Australia License.

% This is an algorithm description, see:
% Jason Brownlee. A Template for Standardized Algorithm Descriptions. Technical Report CA-TR-20100107-1, The Clever Algorithms Project http://www.CleverAlgorithms.com, January 2010.

% Name
% The algorithm name defines the canonical name used to refer to the technique, in addition to common aliases, abbreviations, and acronyms. The name is used in terms of the heading and sub-headings of an algorithm description.
\section{Simulated Annealing} 
\label{sec:simulated_annealing}
\index{Simulated Annealing}

% other names
% What is the canonical name and common aliases for a technique?
% What are the common abbreviations and acronyms for a technique?
\emph{Simulated Annealing, SA.}

% Taxonomy: Lineage and locality
% The algorithm taxonomy defines where a techniques fits into the field, both the specific subfields of Computational Intelligence and Biologically Inspired Computation as well as the broader field of Artificial Intelligence. The taxonomy also provides a context for determining the relation- ships between algorithms. The taxonomy may be described in terms of a series of relationship statements or pictorially as a venn diagram or a graph with hierarchical structure.
\subsection{Taxonomy}
% To what fields of study does a technique belong?
Simulated Annealing is a global optimization that belongs to the field of Stochastic Optimization and Metaheuristics.
% What are the closely related approaches to a technique?
Simulated Annealing is an adaptation of the Metropolis-Hastings Monte Carlo algorithm and is used in function optimization. Like the Genetic Algorithm (Section~\ref{sec:genetic_algorithm}), it provides a basis for a large variety of extensions and specialization's of the general method not limited to Parallel Simulated Annealing, Fast Simulated Annealing, and Adaptive Simulated Annealing.

% Inspiration: Motivating system
% The inspiration describes the specific system or process that provoked the inception of the algorithm. The inspiring system may non-exclusively be natural, biological, physical, or social. The description of the inspiring system may include relevant domain specific theory, observation, nomenclature, and most important must include those salient attributes of the system that are somehow abstractly or conceptually manifest in the technique. The inspiration is described textually with citations and may include diagrams to highlight features and relationships within the inspiring system.
% Optional
\subsection{Inspiration}
% What is the system or process that motivated the development of a technique?
Simulated Annealing is inspired by the process of annealing in metallurgy. In this natural process a material is heated and slowly cooled under controlled conditions to increase the size of the crystals in the material and reduce their defects. This has the effect of improving the strength and durability of the material. The heat increases the energy of the atoms allowing them to move freely, and the slow cooling schedule allows a new low-energy configuration to be discovered and exploited.
% Which features of the motivating system are relevant to a technique?

% Metaphor: Explanation via analogy
% The metaphor is a description of the technique in the context of the inspiring system or a different suitable system. The features of the technique are made apparent through an analogous description of the features of the inspiring system. The explanation through analogy is not expected to be literal scientific truth, rather the method is used as an allegorical communication tool. The inspiring system is not explicitly described, this is the role of the ‘inspiration’ element, which represents a loose dependency for this element. The explanation is textual and uses the nomenclature of the metaphorical system.
% Optional
\subsection{Metaphor}
% What is the explanation of a technique in the context of the inspiring system?
Each configuration of a solution in the search space represents a different internal energy of the system. Heating the system results in a relaxation of the acceptance criteria of the samples taken from the search space. As the system is cooled, the acceptance criteria of samples is narrowed to focus on improving movements. Once the system has cooled, the configuration will represent a sample at or close to a global optimum. 
% What are the functionalities inferred for a technique from the analogous inspiring system?

% Strategy: Problem solving plan
% The strategy is an abstract description of the computational model. The strategy describes the information processing actions a technique shall take in order to achieve an objective. The strategy provides a logical separation between a computational realization (procedure) and a analogous system (metaphor). A given problem solving strategy may be realized as one of a number specific algorithms or problem solving systems. The strategy description is textual using information processing and algorithmic terminology.
\subsection{Strategy}
% What is the information processing objective of a technique?
The information processing objective of the technique is to locate the minimum cost configuration in the search space.
% What is a techniques plan of action?
The algorithms plan of action is to probabilistically re-sample the problem space where the acceptance of new samples into the currently held sample is managed by a probabilistic function that becomes more discerning of the cost of samples it accepts over the execution time of the algorithm. This probabilistic decision is based on the Metropolis-Hastings algorithm for simulating samples from a thermodynamic system.

% Procedure: Abstract computation
% The algorithmic procedure summarizes the specifics of realizing a strategy as a systemized and parameterized computation. It outlines how the algorithm is organized in terms of the data structures and representations. The procedure may be described in terms of software engineering and computer science artifacts such as Pseudocode, design diagrams, and relevant mathematical equations.
\subsection{Procedure}
% What is the computational recipe for a technique?
% What are the data structures and representations used in a technique?
Algorithm~\ref{alg:sa} provides a pseudocode listing of the main Simulated Annealing algorithm for minimizing a cost function.

\begin{algorithm}[ht]
	\SetLine  

	% data
	\SetKwData{ProblemSize}{ProblemSize}
	\SetKwData{MaxIterations}{$iterations_{max}$}
	\SetKwData{InitialTemperature}{$temp_{max}$}
	\SetKwData{Temp}{$temp_{curr}$}
	\SetKwData{ProblemSize}{ProblemSize}
	\SetKwData{Best}{$S_{best}$}
	\SetKwData{Current}{$S_{current}$}
	\SetKwData{Candidate}{$S_{i}$}
	
	% functions
	\SetKwFunction{Cost}{Cost}
	\SetKwFunction{Rand}{Rand}
	\SetKwFunction{StopCondition}{StopCondition}
	\SetKwFunction{CreateInitialSolution}{CreateInitialSolution}
	\SetKwFunction{CreateNeighborSolution}{CreateNeighborSolution}
	\SetKwFunction{CalculateTemperature}{CalculateTemperature}
	
	% I/O
	\KwIn{\ProblemSize, \MaxIterations, \InitialTemperature}		
	\KwOut{\Best}
  
	% Algorithm
	\Current $\leftarrow$ \CreateInitialSolution{\ProblemSize}\;
	\Best $\leftarrow$ \Current\;
	% loop
	\For{$i=1$ \KwTo \MaxIterations} {
		\Candidate $\leftarrow$ \CreateNeighborSolution{\Current}\;
		\Temp $\leftarrow$ \CalculateTemperature{$i$, \InitialTemperature}\;
		\uIf{\Cost{\Candidate} $\leq$ \Cost{\Current}} {
			\Current $\leftarrow$ \Candidate\;
			\If{\Cost{\Candidate} $\leq$ \Cost{\Best}} {
				\Best $\leftarrow$ \Candidate\;
			}
		}
		\ElseIf{$\exp(\frac{\Cost{\Current}-\Cost{\Candidate}}{\Temp})$ $>$ \Rand{}} {
			\Current $\leftarrow$ \Candidate\;
		}
	}
	\Return{\Best}\;
	% end
	\caption{Pseudocode for Simulated Annealing.}
	\label{alg:sa}
\end{algorithm}

% Heuristics: Usage guidelines
% The heuristics element describe the commonsense, best practice, and demonstrated rules for applying and configuring a parameterized algorithm. The heuristics relate to the technical details of the techniques procedure and data structures for general classes of application (neither specific implementations not specific problem instances). The heuristics are described textually, such as a series of guidelines in a bullet-point structure.
\subsection{Heuristics}
% What are the suggested configurations for a technique?
% What are the guidelines for the application of a technique to a problem instance?
\begin{itemize}
	\item Simulated Annealing was designed for use with combinatorial optimization problems, although it has been adapted for continuous function optimization problems.
	\item The convergence proof suggests that with a long enough cooling period, the system will always converge to the global optimum. The downside of this theoretical finding is that the number of samples taken for optimum convergence to occur on some problems may be more than a complete enumeration of the search space. 
	\item Performance improvements can be given with the selection of a candidate move generation scheme (neighborhood) that is less likely to generate candidates of significantly higher cost.
	\item Restarting the cooling scheduling using the best found solution so far can lead to an improved outcome on some problems.
	\item A common acceptance method is to always accept improving solutions and accept worse solutions with a probability of $P(accept) \leftarrow \exp(\frac{e-e\prime}{T})$, where $T$ is the current temperature, $e$ is the energy (or cost) of the current solution and $e\prime$ is the energy of a candidate solution being considered.
	\item The size of the neighborhood considered in generating candidate solutions may also change over time or be influenced by the temperature, starting initially broad and narrowing with the execution of the algorithm.
	\item A problem specific heuristic method can be used to provide the starting point for the search.
\end{itemize}

% Code Listing
% The code description provides a minimal but functional version of the technique implemented with a programming language. The code description must be able to be typed into an appropriate computer, compiled or interpreted as need be, and provide a working execution of the technique. The technique implementation also includes a minimal problem instance to which it is applied, and both the problem and algorithm implementations are complete enough to demonstrate the techniques procedure. The description is presented as a programming source code listing.
\subsection{Code Listing}
% How is a technique implemented as an executable program?
% How is a technique applied to a concrete problem instance?
Listing~\ref{simulated_annealing} provides an example of the Simulated Annealing algorithm implemented in the Ruby Programming Language. 
% problem
The algorithm is applied to the Berlin52 instance of the Traveling Salesman Problem (TSP), taken from the TSPLIB. The problem seeks a permutation of the order to visit cities (called a tour) that minimized the total distance traveled. The optimal tour distance for Berlin52 instance is 7542 units.

% algorithm
The algorithm implementation uses a two-opt procedure for the neighborhood function and the classical $P(accept) \leftarrow \exp(\frac{e-e\prime}{T})$ as the acceptance function. A simple linear cooling regime is used with a large initial temperature which is decreased each iteration. The initial solution is created using a nearest neighbor heuristic to provide a good starting point for the search.

% the listing
\lstinputlisting[firstline=7,language=ruby,caption=Simulated Annealing in Ruby, label=simulated_annealing]{../src/algorithms/physical/simulated_annealing.rb}

% References: Deeper understanding
% The references element description includes a listing of both primary sources of information about the technique as well as useful introductory sources for novices to gain a deeper understanding of the theory and application of the technique. The description consists of hand-selected reference material including books, peer reviewed conference papers, journal articles, and potentially websites. A bullet-pointed structure is suggested.
\subsection{References}
% What are the primary sources for a technique?
% What are the suggested reference sources for learning more about a technique?

% 
% Primary Sources
% 
\subsubsection{Primary Sources}
% seminal
Simulated Annealing is credited to Kirkpatrick, Gelatt, and Vecchi in 1983 \cite{Kirkpatrick1983}. Granville, Krivanek, and Rasson provided the proof for convergence for Simulated Annealing in 1994 \cite{Granville1994}.
% early
There were a number of early studies and application papers such as Kirkpatrick's investigation into the TSP and minimum cut problems \cite{Kirkpatrick1983a}, and a study by Vecchi and Kirkpatrick on Simulated Annealing applied to the global wiring problem \cite{Vecchi1983}.

% 
% Learn More
% 
\subsubsection{Learn More}
% reviews
There are many excellent reviews of Simulated Annealing, not limited to the review by Ingber that describes improved methods such as Adaptive Simulated Annealing, Simulated Quenching, and hybrid methods \cite{Ingber1993}.
% books
There are books dedicated to Simulated Annealing, applications and variations. Two examples of good texts include ``Simulated Annealing: Theory and Applications'' by Laarhoven and Aarts \cite{Laarhoven1988} that provides an introduction to the technique and applications, and ``Simulated Annealing: Parallelization Techniques'' by Robert Azencott \cite{Azencott1992} that focuses of the theory and applications of parallel methods for Simulated Annealing.


\putbib\end{bibunit}
\newpage\begin{bibunit}% The Clever Algorithms Project: http://www.CleverAlgorithms.com
% (c) Copyright 2010 Jason Brownlee. Some Rights Reserved. 
% This work is licensed under a Creative Commons Attribution-Noncommercial-Share Alike 2.5 Australia License.

% This is an algorithm description, see:
% Jason Brownlee. A Template for Standardized Algorithm Descriptions. Technical Report CA-TR-20100107-1, The Clever Algorithms Project http://www.CleverAlgorithms.com, January 2010.

% Name
% The algorithm name defines the canonical name used to refer to the technique, in addition to common aliases, abbreviations, and acronyms. The name is used in terms of the heading and sub-headings of an algorithm description.
\section{Extremal Optimization} 
\label{sec:extremal_optimization}
\index{Extremal Optimization}

% other names
% What is the canonical name and common aliases for a technique?
% What are the common abbreviations and acronyms for a technique?
\emph{Extremal Optimization, EO.}

% Taxonomy: Lineage and locality
% The algorithm taxonomy defines where a techniques fits into the field, both the specific subfields of Computational Intelligence and Biologically Inspired Computation as well as the broader field of Artificial Intelligence. The taxonomy also provides a context for determining the relation- ships between algorithms. The taxonomy may be described in terms of a series of relationship statements or pictorially as a venn diagram or a graph with hierarchical structure.
\subsection{Taxonomy}
% To what fields of study does a technique belong?
Extremal Optimization is a stochastic search technique that has properties of being a local and global search method.
% What are the closely related approaches to a technique?
It is generally related to hill-climbing algorithms and provides the basis for extensions such as Generalized Extremal Optimization.

% Inspiration: Motivating system
% The inspiration describes the specific system or process that provoked the inception of the algorithm. The inspiring system may non-exclusively be natural, biological, physical, or social. The description of the inspiring system may include relevant domain specific theory, observation, nomenclature, and most important must include those salient attributes of the system that are somehow abstractly or conceptually manifest in the technique. The inspiration is described textually with citations and may include diagrams to highlight features and relationships within the inspiring system.
% Optional
\subsection{Inspiration}
% What is the system or process that motivated the development of a technique?
Extremal Optimization is inspired by the Bak-Sneppen self-organized criticality model of co-evolution from the field of statistical physics. 
% Which features of the motivating system are relevant to a technique?
The self-organized criticality model suggests that some dynamical systems have a critical point as an attractor, whereby the systems exhibit periods of slow movement or accumulation followed by short periods of avalanche or instability. Examples of such systems include land formation, earthquakes, and the dynamics of sand piles. The Bak-Sneppen considers these dynamics in co-evolutionary systems and in the punctuated equilibrium model, which is described as long periods of status followed by short periods of extension and large evolutionary change.

% Metaphor: Explanation via analogy
% The metaphor is a description of the technique in the context of the inspiring system or a different suitable system. The features of the technique are made apparent through an analogous description of the features of the inspiring system. The explanation through analogy is not expected to be literal scientific truth, rather the method is used as an allegorical communication tool. The inspiring system is not explicitly described, this is the role of the ‘inspiration’ element, which represents a loose dependency for this element. The explanation is textual and uses the nomenclature of the metaphorical system.
% Optional
\subsection{Metaphor}
% What is the explanation of a technique in the context of the inspiring system?
The dynamics of the system result in the steady improvement of a candidate solution with sudden and large crashes in the quality of the candidate solution. These dynamics allow two main phases of activity in the system: 1) to exploit higher quality solutions in a local search like manner, and 2) escape possible local optima with a population crash and explore the search space for a new area of high quality solutions.
% What are the functionalities inferred for a technique from the analogous inspiring system?

% Strategy: Problem solving plan
% The strategy is an abstract description of the computational model. The strategy describes the information processing actions a technique shall take in order to achieve an objective. The strategy provides a logical separation between a computational realization (procedure) and a analogous system (metaphor). A given problem solving strategy may be realized as one of a number specific algorithms or problem solving systems. The strategy description is textual using information processing and algorithmic terminology.
\subsection{Strategy}
% What is the information processing objective of a technique?
The objective of the information processing strategy is to iteratively identify the worst performing components of a given solution and replace or swap them with other components.
% What is a techniques plan of action?
This is achieved through the allocation of cost to the components of the solution based on their contribution to the overall cost of the solution in the problem domain. Once components are assessed they can be ranked and the weaker components replaced or switched with a randomly selected component.

% Procedure: Abstract computation
% The algorithmic procedure summarizes the specifics of realizing a strategy as a systemized and parameterized computation. It outlines how the algorithm is organized in terms of the data structures and representations. The procedure may be described in terms of software engineering and computer science artifacts such as Pseudocode, design diagrams, and relevant mathematical equations.
\subsection{Procedure}
% What is the computational recipe for a technique?
% What are the data structures and representations used in a technique?
Algorithm~\ref{alg:eo} provides a pseudocode listing of the Extremal Optimization algorithm for minimizing a cost function. The deterministic selection of the worst component in the \\ \texttt{Select\-Weak\-Component} function and replacement in the \texttt{Select\-Replacement\-Component} function is classical EO. If these decisions are probabilistic making use of $\tau$ parameter, this is referred to as $\tau$-Extremal Optimization.

\begin{algorithm}[ht]
	\SetLine  

	% data
	\SetKwData{ProblemSize}{ProblemSize}
	\SetKwData{MaxIterations}{$iterations_{max}$}
	\SetKwData{TauParam}{$\tau$}
	\SetKwData{Best}{$S_{best}$}
	\SetKwData{Current}{$S_{current}$}
	\SetKwData{Candidate}{$S_{candidate}$}	
	\SetKwData{CandidateComponents}{$Si_{components}$}
	\SetKwData{Component}{$Component_{i}$}	
	\SetKwData{OtherComponent}{$Component_{j}$}	
	\SetKwData{ComponentCost}{$Component_{i}^{cost}$}
	\SetKwData{RankedComponents}{RankedComponents}

	% functions
	\SetKwFunction{Cost}{Cost}
	\SetKwFunction{CreateInitialSolution}{CreateInitialSolution}
	\SetKwFunction{Rank}{Rank}
	\SetKwFunction{SelectWeakComponent}{SelectWeakComponent}
	\SetKwFunction{SelectReplacementComponent}{SelectReplacementComponent}
	\SetKwFunction{Replace}{Replace}
	
	% I/O
	\KwIn{\ProblemSize, \MaxIterations, \TauParam}		
	\KwOut{\Best}
  
	% Algorithm
	\Current $\leftarrow$ \CreateInitialSolution{\ProblemSize}\;
	\Best $\leftarrow$ \Current\;
	% loop
	\For{$i=1$ \KwTo \MaxIterations} {
		\ForEach{\Component $\in$ \Current}{
			\ComponentCost $\leftarrow$ \Cost{\Component, \Current}\;
		}
		\RankedComponents $\leftarrow$ \Rank{\CandidateComponents}
		
		\Component $\leftarrow$ \SelectWeakComponent{\RankedComponents, \Component, \TauParam}\;
		\OtherComponent $\leftarrow$ \SelectReplacementComponent{\RankedComponents, \TauParam}\;
		\Candidate $\leftarrow$ \Replace{\Current, \Component, \OtherComponent}\;
	
		\If{\Cost{\Candidate} $\leq$ \Cost{\Best}} {
			\Best $\leftarrow$ \Candidate\;
		}
	}
	\Return{\Best}\;
	% end
	\caption{Pseudocode for the Extremal Optimization algorithm.}
	\label{alg:eo}
\end{algorithm}

% Heuristics: Usage guidelines
% The heuristics element describe the commonsense, best practice, and demonstrated rules for applying and configuring a parameterized algorithm. The heuristics relate to the technical details of the techniques procedure and data structures for general classes of application (neither specific implementations not specific problem instances). The heuristics are described textually, such as a series of guidelines in a bullet-point structure.
\subsection{Heuristics}
% What are the suggested configurations for a technique?
% What are the guidelines for the application of a technique to a problem instance?
\begin{itemize}
	\item Extremal Optimization was designed for combinatorial optimization problems, although variations have been applied to continuous function optimization.
	\item The selection of the worst component and the replacement component each iteration can be deterministic or probabilistic, the latter of which is referred to as $\tau$-Extremal Optimization given the use of a $\tau$ parameter.
	\item The selection of an appropriate scoring function of the components of a solution is the most difficult part in the application of the technique.
	\item For $\tau$-Extremal Optimization, low $\tau$ values are used (such as $\tau \in [1.2,1.6]$) have been found to be effective for the TSP.
\end{itemize}

% Code Listing
% The code description provides a minimal but functional version of the technique implemented with a programming language. The code description must be able to be typed into an appropriate computer, compiled or interpreted as need be, and provide a working execution of the technique. The technique implementation also includes a minimal problem instance to which it is applied, and both the problem and algorithm implementations are complete enough to demonstrate the techniques procedure. The description is presented as a programming source code listing.
\subsection{Code Listing}
% How is a technique implemented as an executable program?
% How is a technique applied to a concrete problem instance?
Listing~\ref{extremal_optimization} provides an example of the Extremal Optimization algorithm implemented in the Ruby Programming Language. 
% problem
The algorithm is applied to the Berlin52 instance of the Traveling Salesman Problem (TSP), taken from the TSPLIB. The problem seeks a permutation of the order to visit cities (called a tour) that minimized the total distance traveled. The optimal tour distance for Berlin52 instance is 7542 units.

% algorithm
The algorithm implementation is based on the seminal work by Boettcher and Percus \cite{Boettcher1999}. A solution is comprised of a permutation of city components. Each city can potentially form a connection to any other city, and the connections to other cities ordered by distance may be considered its neighborhood. For a given candidate solution, the city components of a solution are scored based on the neighborhood rank of the cities to which they are connected: $fitness_k \leftarrow \frac{3}{r_i + r_j}$, where $r_i$ and $r_j$ are the neighborhood ranks of cities $i$ and $j$ against city $k$. A city is selected for modification probabilistically where the probability of selecting a given city is proportional to $n_i^{-\tau}$, where $n$ is the rank of city $i$. The longest connection is borken, and the city is connected with another neighbouring city that is also probabilistically selected.

% the listing
\lstinputlisting[firstline=7,language=ruby,caption=Extremal Optimization algorithm in the Ruby Programming Language, label=extremal_optimization]{../src/algorithms/physical/extremal_optimization.rb}

% References: Deeper understanding
% The references element description includes a listing of both primary sources of information about the technique as well as useful introductory sources for novices to gain a deeper understanding of the theory and application of the technique. The description consists of hand-selected reference material including books, peer reviewed conference papers, journal articles, and potentially websites. A bullet-pointed structure is suggested.
\subsection{References}
% What are the primary sources for a technique?
% What are the suggested reference sources for learning more about a technique?

% 
% Primary Sources
% 
\subsubsection{Primary Sources}
% seminal
Extremal Optimization was proposed as an optimization heuristic by Boettcher and Percus applied to graph partitioning and the Traveling Salesman Problem \cite{Boettcher1999}. The approach was inspired by the Bak-Sneppen self-organized criticality model of co-evolution \cite{Bak1987, Bak1993}.

% 
% Learn More
% 
\subsubsection{Learn More}
% reviews
A number of detailed reviews of Extremal Optimization have been presented, including a review and studies by Boettcher and Percus \cite{Boettcher2000}, an accessible review by Boettcher \cite{Boettcher2000a}, and a focused study on the Spin Glass problem by Boettcher and Percus \cite{Boettcher2001}.
% books


\putbib\end{bibunit}
\newpage\begin{bibunit}% The Clever Algorithms Project: http://www.CleverAlgorithms.com
% (c) Copyright 2010 Jason Brownlee. Some Rights Reserved. 
% This work is licensed under a Creative Commons Attribution-Noncommercial-Share Alike 2.5 Australia License.

% This is an algorithm description, see:
% Jason Brownlee. A Template for Standardized Algorithm Descriptions. Technical Report CA-TR-20100107-1, The Clever Algorithms Project http://www.CleverAlgorithms.com, January 2010.

% Name
% The algorithm name defines the canonical name used to refer to the technique, in addition to common aliases, abbreviations, and acronyms. The name is used in terms of the heading and sub-headings of an algorithm description.
\section{Harmony Search} 
\label{sec:harmony_search}
\index{Harmony Search}

% other names
% What is the canonical name and common aliases for a technique?
% What are the common abbreviations and acronyms for a technique?
\emph{Harmony Search, HS.}

% Taxonomy: Lineage and locality
% The algorithm taxonomy defines where a techniques fits into the field, both the specific subfields of Computational Intelligence and Biologically Inspired Computation as well as the broader field of Artificial Intelligence. The taxonomy also provides a context for determining the relation- ships between algorithms. The taxonomy may be described in terms of a series of relationship statements or pictorially as a venn diagram or a graph with hierarchical structure.
\subsection{Taxonomy}
% To what fields of study does a technique belong?
% What are the closely related approaches to a technique?
Harmony Search belongs to the fields of Computational Intelligence and Metaheuristics.

% Inspiration: Motivating system
% The inspiration describes the specific system or process that provoked the inception of the algorithm. The inspiring system may non-exclusively be natural, biological, physical, or social. The description of the inspiring system may include relevant domain specific theory, observation, nomenclature, and most important must include those salient attributes of the system that are somehow abstractly or conceptually manifest in the technique. The inspiration is described textually with citations and may include diagrams to highlight features and relationships within the inspiring system.
% Optional
\subsection{Inspiration}
% What is the system or process that motivated the development of a technique?
Harmony Search was inspired by the improvisation of Jazz musicians. Specifically, the process by which the musicians (who may have never played together before) rapidly refine their individual improvisation though variation resulting in an aesthetic harmony.  
% Which features of the motivating system are relevant to a technique?


% Metaphor: Explanation via analogy
% The metaphor is a description of the technique in the context of the inspiring system or a different suitable system. The features of the technique are made apparent through an analogous description of the features of the inspiring system. The explanation through analogy is not expected to be literal scientific truth, rather the method is used as an allegorical communication tool. The inspiring system is not explicitly described, this is the role of the ‘inspiration’ element, which represents a loose dependency for this element. The explanation is textual and uses the nomenclature of the metaphorical system.
% Optional
\subsection{Metaphor}
% What is the explanation of a technique in the context of the inspiring system?
% What are the functionalities inferred for a technique from the analogous inspiring system?
Each musician corresponds to an attribute in a candidate solution from a problem domain, and each instrument's pitch and range corresponds to the bounds and constraints on the decision variable. The harmony between the musicians is taken a complete candidate solution at a given time, and the audiences aesthetic appreciation of the harmony represent the problem specific cost function. The musicians seek harmony over time through small variations and improvisations, which results in an improvement against the cost function.

% Strategy: Problem solving plan
% The strategy is an abstract description of the computational model. The strategy describes the information processing actions a technique shall take in order to achieve an objective. The strategy provides a logical separation between a computational realization (procedure) and a analogous system (metaphor). A given problem solving strategy may be realized as one of a number specific algorithms or problem solving systems. The strategy description is textual using information processing and algorithmic terminology.
\subsection{Strategy}
% What is the information processing objective of a technique?
The information processing objective of the technique is to use good candidate solutions already discovered to influence the creation of new candidate solutions toward locating the problems optima.
% What is a techniques plan of action?
This is achieved by stochastically creating candidate solutions in a step-wise manner, where each component is either drawn randomly from a memory of high-quality solutions, adjusted from the memory of high-quality solution, or assigned randomly within the bounds of the problem. The memory of candidate solutions is initially random, and a greedy acceptance criteria is used to admit new candidate solutions only if they have an improved objective value, replacing an existing member.

% Procedure: Abstract computation
% The algorithmic procedure summarizes the specifics of realizing a strategy as a systemized and parameterized computation. It outlines how the algorithm is organized in terms of the data structures and representations. The procedure may be described in terms of software engineering and computer science artifacts such as pseudo code, design diagrams, and relevant mathematical equations.
\subsection{Procedure}
% What is the computational recipe for a technique?
% What are the data structures and representations used in a technique?
Algorithm~\ref{alg:harmony_search} provides a pseudo-code listing of the Harmony Search algorithm for minimizing a cost function. 
The adjustment of a pitch selected from the harmony memory is typically linear, for example for continuous function optimization: 
\begin{equation}
	x\prime \leftarrow x + range \times \epsilon
\end{equation}
where $range$ is a the user parameter (pitch bandwidth) to control the size of the changes, and $\epsilon$ is a uniformly random number $\in [-1,1]$.

\begin{algorithm}[ht]
	\SetLine

	% params
	\SetKwData{NumPitches}{$Pitch_{num}$}
	\SetKwData{PitchBounds}{$Pitch_{bounds}$}
	\SetKwData{MemorySize}{$Memory_{size}$}
	\SetKwData{ConsolidationRate}{$Consolidation_{rate}$}
	\SetKwData{AdjustRate}{$PitchAdjust_{rate}$}
	\SetKwData{MaxImprovisations}{$Improvisation_{max}$}	
	% data
	\SetKwData{Harmonies}{Harmonies}
	\SetKwData{Best}{$Harmony_{best}$}
	\SetKwData{CandidateHarmony}{$Harmony$}
	\SetKwData{CandidateHarmonyPitch}{$Harmony_{pitch}^i$}
	\SetKwData{RandomHarmonyPitch}{$RandomHarmony_{pitch}^i$}
	\SetKwData{Pitch}{$Pitch_{i}$}
	% functions
	\SetKwFunction{InitializeHarmonyMemory}{InitializeHarmonyMemory} 
	\SetKwFunction{EvaluateHarmonies}{EvaluateHarmonies} 
	\SetKwFunction{SelectRandomHarmonyPitch}{SelectRandomHarmonyPitch} 
  \SetKwFunction{Rand}{Rand} 
	\SetKwFunction{AdjustPitch}{AdjustPitch} 
	\SetKwFunction{RandomPitch}{RandomPitch} 
	\SetKwFunction{Cost}{Cost} 
	\SetKwFunction{Worst}{Worst} 
	
	% I/O
	\KwIn{\NumPitches, \PitchBounds, \MemorySize, \ConsolidationRate, \AdjustRate, \MaxImprovisations}		
	\KwOut{\Best}

  % Algorithm
	\Harmonies $\leftarrow$ \InitializeHarmonyMemory{\NumPitches, \PitchBounds, \MemorySize}\;
	\EvaluateHarmonies{\Harmonies}\;
	\For{$i$ \KwTo \MaxImprovisations} {
		\CandidateHarmony $\leftarrow$ $0$\;		
		\ForEach{\Pitch $\in$ \NumPitches} {			
			\eIf{\Rand{} $\leq$ \ConsolidationRate} {
				\RandomHarmonyPitch $\leftarrow$ \SelectRandomHarmonyPitch{\Harmonies, \Pitch}\;
				\eIf{\Rand{} $\leq$ \AdjustRate} {
					\CandidateHarmonyPitch $\leftarrow$ \AdjustPitch{\RandomHarmonyPitch}\;
				} {
					\CandidateHarmonyPitch $\leftarrow$ \RandomHarmonyPitch\;
				}
			} {
				\CandidateHarmonyPitch $\leftarrow$ \RandomPitch{\PitchBounds}\;
			}		
		}
		\EvaluateHarmonies{\CandidateHarmony}\;
		\If{\Cost{\CandidateHarmony} $\leq$ \Cost{\Worst{\Harmonies}}} {
			\Worst{\Harmonies} $\leftarrow$ \CandidateHarmony\;
		}
	}
	\Return{\Best}\;
	% end
	\caption{Pseudo Code for the Harmony Search algorithm.}
	\label{alg:harmony_search}
\end{algorithm}

% Heuristics: Usage guidelines
% The heuristics element describe the commonsense, best practice, and demonstrated rules for applying and configuring a parameterized algorithm. The heuristics relate to the technical details of the techniques procedure and data structures for general classes of application (neither specific implementations not specific problem instances). The heuristics are described textually, such as a series of guidelines in a bullet-point structure.
\subsection{Heuristics}
% What are the suggested configurations for a technique?
% What are the guidelines for the application of a technique to a problem instance?
\begin{itemize}
	\item Harmony Search was designed as a generalized optimization method for continuous, discrete, and constrained optimization and has been applied to numerous types of optimization problems.
	\item The harmony memory considering rate (HMCR) $\in [0,1]$ controls the use of information from the harmony memory or the generation of a random pitch. As such, it controls the rate of convergence of the algorithm and is typically configured $\in [0.7,0.95]$.
	\item The pitch adjustment rate (PAR) $\in [0,1]$ controls the frequency of adjustment of pitches selected from harmony memory, typically configured $\in [0.1,0.5]$. High values can result in the premature convergence of the search.
	\item The pitch adjustment rate and the adjustment method (amount of adjustment or fret width) are typically fixed, having a linear effect through time. Non-linear methods have been considered, for example refer to Geem \cite{Geem2010a}.
	\item When creating a new harmony, aggregations of pitches can be taken from across musicians in the harmony memory.
	\item The harmony memory update is typically a greedy process, although other considerations such as diversity may be used where the most similar harmony is replaced.
\end{itemize}

% Code Listing
% The code description provides a minimal but functional version of the technique implemented with a programming language. The code description must be able to be typed into an appropriate computer, compiled or interpreted as need be, and provide a working execution of the technique. The technique implementation also includes a minimal problem instance to which it is applied, and both the problem and algorithm implementations are complete enough to demonstrate the techniques procedure. The description is presented as a programming source code listing.
\subsection{Code Listing}
% How is a technique implemented as an executable program?
% How is a technique applied to a concrete problem instance?
Listing~\ref{harmony_search} provides an example of the Harmony Search algorithm implemented in the Ruby Programming Language. 
% problem
The demonstration problem is an instance of a continuous function optimization that seeks $min f(x)$ where $f=\sum_{i=1}^n x_{i}^2$, $-5.0\leq x_i \leq 5.0$ and $n=3$. The optimal solution for this basin function is $(v_0,\ldots,v_{n-1})=0.0$.
% algorithm
The algorithm implementation and parameterization are based on the description by Yang \cite{Yang2009}, with refinement from Geem \cite{Geem2010a}.

% the listing
\lstinputlisting[firstline=7,language=ruby,caption=Harmony Search algorithm in the Ruby Programming Language, label=harmony_search]{../src/algorithms/physical/harmony_search.rb}

% References: Deeper understanding
% The references element description includes a listing of both primary sources of information about the technique as well as useful introductory sources for novices to gain a deeper understanding of the theory and application of the technique. The description consists of hand-selected reference material including books, peer reviewed conference papers, journal articles, and potentially websites. A bullet-pointed structure is suggested.
\subsection{References}
% What are the primary sources for a technique?
% What are the suggested reference sources for learning more about a technique?

% 
% Primary Sources
% 
\subsubsection{Primary Sources}
% seminal
Geem et al.\ proposed the Harmony Search algorithm in 2001, which was applied to a range of optimization problems including a constraint optimization, the Traveling Salesman problem, and the design of a water supply network \cite{Geem2001}.
% early

% 
% Learn More
% 
\subsubsection{Learn More}
% reviews
% books
A book on Harmony Search, edited by Geem provides a collection of papers on the technique and its applications \cite{Geem2009}, chapter 1 provides a useful summary of the method heuristics for its configuration \cite{Yang2009}. Similarly a second edited volume by Geem focuses on studies that provide more advanced applications of the approach \cite{Geem2010}, and chapter 1 provides a detailed walkthrough of the technique itself \cite{Geem2010a}. Geem also provides a treatment of Harmony Search applied to the optimal design of water distribution networks \cite{Geem2009a} and edits yet a third volume on papers related to the application of the technique to structural design optimization problems \cite{Geem2009b}.


\putbib\end{bibunit}
\newpage\begin{bibunit}% The Clever Algorithms Project: http://www.CleverAlgorithms.com
% (c) Copyright 2010 Jason Brownlee. Some Rights Reserved. 
% This work is licensed under a Creative Commons Attribution-Noncommercial-Share Alike 2.5 Australia License.

% This is an algorithm description, see:
% Jason Brownlee. A Template for Standardized Algorithm Descriptions. Technical Report CA-TR-20100107-1, The Clever Algorithms Project http://www.CleverAlgorithms.com, January 2010.

% Name
% The algorithm name defines the canonical name used to refer to the technique, in addition to common aliases, abbreviations, and acronyms. The name is used in terms of the heading and sub-headings of an algorithm description.
\section{Cultural Algorithm} 
\label{sec:cultural_algorithm}
\index{Cultural Algorithm}

% other names
% What is the canonical name and common aliases for a technique?
% What are the common abbreviations and acronyms for a technique?
\emph{Cultural Algorithm, CA.}

% Taxonomy: Lineage and locality
% The algorithm taxonomy defines where a techniques fits into the field, both the specific subfields of Computational Intelligence and Biologically Inspired Computation as well as the broader field of Artificial Intelligence. The taxonomy also provides a context for determining the relation- ships between algorithms. The taxonomy may be described in terms of a series of relationship statements or pictorially as a venn diagram or a graph with hierarchical structure.
\subsection{Taxonomy}
% To what fields of study does a technique belong?
The Cultural Algorithm is an extension to the field of Evolutionary Computation and may be considered a Meta-Evolutionary Algorithm. It more broadly belongs to the field of Computational Intelligence and Metaheuristics.
% What are the closely related approaches to a technique?
It is related to other high-order extensions of Evolutionary Computation such as the Memetic Algorithm (Section~\ref{sec:memetic_algorithm}).

% Inspiration: Motivating system
% The inspiration describes the specific system or process that provoked the inception of the algorithm. The inspiring system may non-exclusively be natural, biological, physical, or social. The description of the inspiring system may include relevant domain specific theory, observation, nomenclature, and most important must include those salient attributes of the system that are somehow abstractly or conceptually manifest in the technique. The inspiration is described textually with citations and may include diagrams to highlight features and relationships within the inspiring system.
% Optional
\subsection{Inspiration}
% What is the system or process that motivated the development of a technique?
The Cultural Algorithm is inspired by the principle of cultural evolution.
% Which features of the motivating system are relevant to a technique?
Culture includes the habits, knowledge, beliefs, customs, and morals of a member of society. Culture does not exist independent of the environment, and can interact with the environment via positive or negative feedback cycles. the study of the interaction of culture in the environment is referred to as Cultural Ecology.

% Metaphor: Explanation via analogy
% The metaphor is a description of the technique in the context of the inspiring system or a different suitable system. The features of the technique are made apparent through an analogous description of the features of the inspiring system. The explanation through analogy is not expected to be literal scientific truth, rather the method is used as an allegorical communication tool. The inspiring system is not explicitly described, this is the role of the ‘inspiration’ element, which represents a loose dependency for this element. The explanation is textual and uses the nomenclature of the metaphorical system.
% Optional
\subsection{Metaphor}
% What is the explanation of a technique in the context of the inspiring system?
The Cultural Algorithm may be explained in the context of the inspiring system. As the evolutionary process unfolds, individuals accumulate information about the world which is communicated to other individuals in the population. Collectively this corpus of information is a knowledge base that members of the population may tap-into and exploit. Positive feedback mechanisms can occur where cultural knowledge indicates useful areas of the environment, information which is passed down between generations, exploited, refined, and adapted as situations change. Additionally, areas of potential hazard may also be communicated through the cultural knowledge base.
% What are the functionalities inferred for a technique from the analogous inspiring system?

% Strategy: Problem solving plan
% The strategy is an abstract description of the computational model. The strategy describes the information processing actions a technique shall take in order to achieve an objective. The strategy provides a logical separation between a computational realization (procedure) and a analogous system (metaphor). A given problem solving strategy may be realized as one of a number specific algorithms or problem solving systems. The strategy description is textual using information processing and algorithmic terminology.
\subsection{Strategy}
% What is the information processing objective of a technique?
The information processing objective of the algorithm is to improve the learning or convergence of an embedded search technique (typically an evolutionary algorithm) using a higher-order cultural evolution. 
% What is a techniques plan of action?
The algorithm operates at two levels: population level and a cultural level. The population level is like an evolutionary search, where individuals represent candidate solutions, are mostly distinct and their characteristics are translated into an objective or cost function in the problem domain. The second level is the knowledge or believe space where information acquired by generations  is stored, and which is accessible to the current generation. A communication protocol is used to allow the two spaces to interact and the types of of information that can be exchanged.

% Procedure: Abstract computation
% The algorithmic procedure summarizes the specifics of realizing a strategy as a systemized and parameterized computation. It outlines how the algorithm is organized in terms of the data structures and representations. The procedure may be described in terms of software engineering and computer science artifacts such as pseudo code, design diagrams, and relevant mathematical equations.
\subsection{Procedure}
% What are the data structures and representations used in a technique?
The focus of the algorithm is the \texttt{KnowledgeBase} data structure that records different knowledge types based on the nature of the problem. For example, the structure may be used to record the best candidate solution found as well as generalized information about areas of the search space that are expected to payoff (result in good candidate solutions). This cultural knowledge is discovered by the population-based evolutionary search, and is in turn used to influence subsequent generations. The accept functions constrain the communication of knowledge from the population to the knowledge base.

% What is the computational recipe for a technique?
Algorithm~\ref{alg:cultural_algorithm} provides a pseudo-code listing of the Cultural Algorithm. The algorithm is abstract, providing flexibility in the interpretation of the processes such as the acceptance of information, the structure of the knowledge base, and the specific embedded evolutionary algorithm.


\begin{algorithm}[ht]
	\SetLine

	% params
	\SetKwData{ProblemSize}{$Problem_{size}$}
	\SetKwData{PopulationSize}{$Population_{num}$}	
	% data
	\SetKwData{Population}{Population}
	\SetKwData{KnowledgeBase}{KnowledgeBase}
	\SetKwData{CandidateSituational}{$SituationalKnowledge_{candidate}$}
	\SetKwData{CandidateNormative}{$NormativeKnowledge_{candidate}$}
	\SetKwData{Children}{Children}
	
	% functions
	\SetKwFunction{InitializePopulation}{InitializePopulation}  
	\SetKwFunction{InitializeKnowledgebase}{InitializeKnowledgebase} 
	\SetKwFunction{StopCondition}{StopCondition} 
	\SetKwFunction{Evaluate}{Evaluate}
	\SetKwFunction{AcceptSituationalKnowledge}{AcceptSituationalKnowledge}
	\SetKwFunction{UpdateSituationalKnowledge}{UpdateSituationalKnowledge}
	\SetKwFunction{ReproduceWithInfluence}{ReproduceWithInfluence}
	\SetKwFunction{Select}{Select}
	\SetKwFunction{AcceptNormativeKnowledge}{AcceptNormativeKnowledge}
	\SetKwFunction{UpdateNormativeKnowledge}{UpdateNormativeKnowledge}
  
	% I/O
	\KwIn{\ProblemSize, \PopulationSize}		
	\KwOut{\KnowledgeBase}
	
  % Algorithm
	\Population $\leftarrow$ \InitializePopulation{\ProblemSize, \PopulationSize}\;
	\KnowledgeBase $\leftarrow$ \InitializeKnowledgebase{\ProblemSize, \PopulationSize}\;

	\While{$\neg$\StopCondition{}} {
		% eval
		\Evaluate{\Population}\;
		\CandidateSituational $\leftarrow$ \AcceptSituationalKnowledge{\Population}\;
		% situational
		\UpdateSituationalKnowledge{\KnowledgeBase, \CandidateSituational}\;
		% reproduce
		\Children $\leftarrow$ \ReproduceWithInfluence{\Population, \KnowledgeBase}\;
		\Population $\leftarrow$ \Select{\Children, \Population}\;
		% normative
		\CandidateNormative $\leftarrow$ \AcceptNormativeKnowledge{\Population}\;
		\UpdateNormativeKnowledge{\KnowledgeBase, \CandidateNormative}\;
	}
	\Return{\KnowledgeBase}\;
	% end
	\caption{Pseudo Code for the Cultural Algorithm.}
	\label{alg:cultural_algorithm}
\end{algorithm}

% Heuristics: Usage guidelines
% The heuristics element describe the commonsense, best practice, and demonstrated rules for applying and configuring a parameterized algorithm. The heuristics relate to the technical details of the techniques procedure and data structures for general classes of application (neither specific implementations not specific problem instances). The heuristics are described textually, such as a series of guidelines in a bullet-point structure.
\subsection{Heuristics}
% What are the suggested configurations for a technique?
% What are the guidelines for the application of a technique to a problem instance?
\begin{itemize}
	\item The Cultural Algorithm was initially used as a simulation tool to investigate Cultural Ecology. It has been adapted for use as an optimization algorithm for a wide variety of domains not-limited to constraint optimization, combinatorial optimization, and continuous function optimization.
	\item The knowledge base structure provides a mechanism for incorporating problem-specific information into the execution of an evolutionary search.
	\item The acceptance functions that control the flow of information into the knowledge base are typically greedy, only including the best information from the current generation, and not replacing existing knowledge unless it is an improvement.
	\item Acceptance functions are traditionally deterministic, although probabilistic and fuzzy acceptance functions have been investigated.
\end{itemize}

% Code Listing
% The code description provides a minimal but functional version of the technique implemented with a programming language. The code description must be able to be typed into an appropriate computer, compiled or interpreted as need be, and provide a working execution of the technique. The technique implementation also includes a minimal problem instance to which it is applied, and both the problem and algorithm implementations are complete enough to demonstrate the techniques procedure. The description is presented as a programming source code listing.
\subsection{Code Listing}
% How is a technique implemented as an executable program?
% How is a technique applied to a concrete problem instance?
Listing~\ref{cultural_algorithm} provides an example of the Cultural Algorithm implemented in the Ruby Programming Language. 
% problem
The demonstration problem is an instance of a continuous function optimization that seeks $\min f(x)$ where $f=\sum_{i=1}^n x_{i}^2$, $-5.0\leq x_i \leq 5.0$ and $n=2$. The optimal solution for this basin function is $(v_0,\ldots,v_{n-1})=0.0$.

% algorithm
The Cultural Algorithm was implemented based on the description of the Cultural Algorithm Evolutionary Program (CAEP) presented by Reynolds \cite{Reynolds1999}. 
A real-valued Genetic Algorithm was used as the embedded evolutionary algorithm.
The overall best solution is taken as the `situational' cultural knowledge, whereas the bounds of the top 20\% of the best solutions each generation are taken as the `normative' cultural knowledge. 
The situational knowledge is returned as the result of the search, whereas the normative knowledge is used to influence the evolutionary process. Specifically, vector the bounds in the normative knowledge are used to define a subspace from which new candidate solutions are uniformly sampled during the reproduction step of the evolutionary algorithm's variation mechanism. 
A real-valued representation and a binary tournament selection strategy are used by the evolutionary algorithm.

% the listing
\lstinputlisting[firstline=7,language=ruby,caption=Cultural Algorithm in the Ruby Programming Language, label=cultural_algorithm]{../src/algorithms/physical/cultural_algorithm.rb}

% References: Deeper understanding
% The references element description includes a listing of both primary sources of information about the technique as well as useful introductory sources for novices to gain a deeper understanding of the theory and application of the technique. The description consists of hand-selected reference material including books, peer reviewed conference papers, journal articles, and potentially websites. A bullet-pointed structure is suggested.
\subsection{References}
% What are the primary sources for a technique?
% What are the suggested reference sources for learning more about a technique?

% 
% Primary Sources
% 
\subsubsection{Primary Sources}
% seminal
The Cultural Algorithm was proposed by Reynolds in 1994 that combined the method with the Version Space Algorithm (a binary string based Genetic Algorithm), where generalizations of individual solutions were communicated as cultural knowledge in the form of schema patterns (strings of 1's, 0's and \#'s, where `\#' represents either a 1 or a 0) \cite{Reynolds1994}. 
% early

% 
% Learn More
% 
\subsubsection{Learn More}
% reviews
Chung and Reynolds provide a study of the Cultural Algorithm on a testbed of constraint satisfaction problems \cite{Chung1996}.
Reynolds provides a detailed overview of the history of the technique as a book chapter that presents the state of the art and summaries of application areas including concept learning and continuous function optimization \cite{Reynolds1999}.
Coello Coello and Becerra proposed a variation of the Cultural Algorithm that uses Evolutionary Programming as the embedded weak search method, for use with Multi-Objective Optimization problems \cite{CoelloCoello2003}.
% books


\putbib\end{bibunit}
\newpage\begin{bibunit}% The Clever Algorithms Project: http://www.CleverAlgorithms.com
% (c) Copyright 2010 Jason Brownlee. Some Rights Reserved. 
% This work is licensed under a Creative Commons Attribution-Noncommercial-Share Alike 2.5 Australia License.

% This is an algorithm description, see:
% Jason Brownlee. A Template for Standardized Algorithm Descriptions. Technical Report CA-TR-20100107-1, The Clever Algorithms Project http://www.CleverAlgorithms.com, January 2010.

% Name
% The algorithm name defines the canonical name used to refer to the technique, in addition to common aliases, abbreviations, and acronyms. The name is used in terms of the heading and sub-headings of an algorithm description.
\section{Memetic Algorithm} 
\label{sec:memetic_algorithm}
\index{Memetic Algorithm}

% other names
% What is the canonical name and common aliases for a technique?
% What are the common abbreviations and acronyms for a technique?
\emph{Memetic Algorithm, MA.}

% Taxonomy: Lineage and locality
% The algorithm taxonomy defines where a techniques fits into the field, both the specific subfields of Computational Intelligence and Biologically Inspired Computation as well as the broader field of Artificial Intelligence. The taxonomy also provides a context for determining the relation- ships between algorithms. The taxonomy may be described in terms of a series of relationship statements or pictorially as a venn diagram or a graph with hierarchical structure.
\subsection{Taxonomy}
% To what fields of study does a technique belong?
Memetic Algorithms have elements of Metaheuristics and Computational Intelligence. Although they have principles of Evolutionary Algorithms, they may not strictly be considered an Evolutionary Technique.
% What are the closely related approaches to a technique?
Memetic Algorithms have functional similarities to Baldwinian Evolutionary Algorithms, Lamarckian Evolutionary Algorithms, Hybrid Evolutionary Algorithms, and Cultural Algorithms (Section~\ref{sec:cultural_algorithm}). Using ideas of memes and Memetic Algorithms in optimization may be referred to as Memetic Computing.

% Inspiration: Motivating system
% The inspiration describes the specific system or process that provoked the inception of the algorithm. The inspiring system may non-exclusively be natural, biological, physical, or social. The description of the inspiring system may include relevant domain specific theory, observation, nomenclature, and most important must include those salient attributes of the system that are somehow abstractly or conceptually manifest in the technique. The inspiration is described textually with citations and may include diagrams to highlight features and relationships within the inspiring system.
% Optional
\subsection{Inspiration}
% What is the system or process that motivated the development of a technique?
Memetic Algorithms are inspired by the interplay of genetic evolution and memetic evolution. Universal Darwinism is the generalization of genes beyond biological-based systems to any system where discrete units of information can be inherited and be subjected to evolutionary forces of selection and variation. The term `meme' is used to refer to a piece of discrete cultural information, suggesting at the interplay of genetic and cultural evolution.
% Which features of the motivating system are relevant to a technique?

% Metaphor: Explanation via analogy
% The metaphor is a description of the technique in the context of the inspiring system or a different suitable system. The features of the technique are made apparent through an analogous description of the features of the inspiring system. The explanation through analogy is not expected to be literal scientific truth, rather the method is used as an allegorical communication tool. The inspiring system is not explicitly described, this is the role of the ‘inspiration’ element, which represents a loose dependency for this element. The explanation is textual and uses the nomenclature of the metaphorical system.
% Optional
\subsection{Metaphor}
% What is the explanation of a technique in the context of the inspiring system?
The genotype is evolved based on the interaction the phenotype has with the environment. This interaction is metered by cultural phenomena that influence the selection mechanisms, and even the pairing and recombination mechanisms. Cultural information is shared between individuals, spreading through the population as memes relative to their fitness or fitness the memes impart to the individuals. Collectively, the interplay of the geneotype and the memeotype strengthen the fitness of population in the environment.
% What are the functionalities inferred for a technique from the analogous inspiring system?

% Strategy: Problem solving plan
% The strategy is an abstract description of the computational model. The strategy describes the information processing actions a technique shall take in order to achieve an objective. The strategy provides a logical separation between a computational realization (procedure) and a analogous system (metaphor). A given problem solving strategy may be realized as one of a number specific algorithms or problem solving systems. The strategy description is textual using information processing and algorithmic terminology.
\subsection{Strategy}
% What is the information processing objective of a technique?
The objective of the information processing strategy is to exploit a population based global search technique to broadly locate good areas of the search space, combined with the repeated usage of a local search heuristic by individual solutions to locate local optimum.
% What is a techniques plan of action?
Ideally, memetic algorithms embrace the duality of genetic and cultural evolution, allowing the transmission, selection, inheritance, and variation of memes as well as genes.

% Procedure: Abstract computation
% The algorithmic procedure summarizes the specifics of realizing a strategy as a systemized and parameterized computation. It outlines how the algorithm is organized in terms of the data structures and representations. The procedure may be described in terms of software engineering and computer science artifacts such as Pseudocode, design diagrams, and relevant mathematical equations.
\subsection{Procedure}
% What is the computational recipe for a technique?
% What are the data structures and representations used in a technique?
Algorithm~\ref{alg:ma} provides a pseudocode listing of the Memetic Algorithm for minimizing a cost function. The procedure describes a simple or first order Memetic Algorithm that shows the improvement of individual solutions separate from a global search, although does not show the independent evolution of memes. 

\begin{algorithm}[ht]
	\SetLine  

	% data
	\SetKwData{ProblemSize}{ProblemSize}
	\SetKwData{MaxIterations}{$iterations_{max}$}
	\SetKwData{PopulationSize}{$Pop_{size}$}
	\SetKwData{MemeticPopulationSize}{$MemePop_{size}$}
	\SetKwData{Best}{$S_{best}$}
	\SetKwData{Population}{Population}
	\SetKwData{MemeticPopulation}{MemeticPopulation}
	\SetKwData{Candidate}{$S_{i}$}
	\SetKwData{CandidateCost}{$Si_{cost}$}
	
	% functions
	\SetKwFunction{Cost}{Cost}
	\SetKwFunction{InitializePopulation}{InitializePopulation}
	\SetKwFunction{StopCondition}{StopCondition}
	\SetKwFunction{StochasticGlobalSearch}{StochasticGlobalSearch}
	\SetKwFunction{SelectMemeticPopulation}{SelectMemeticPopulation}
	\SetKwFunction{GetBestSolution}{GetBestSolution}
	\SetKwFunction{LocalSearch}{LocalSearch}
	
	% I/O
	\KwIn{\ProblemSize, \PopulationSize, \MemeticPopulationSize}		
	\KwOut{\Best}
  
	% Algorithm
	\Population $\leftarrow$ \InitializePopulation{\ProblemSize, \PopulationSize}\;
	
	% loop
	\While{$\neg$\StopCondition{}} {	
		\ForEach{\Candidate $\in$ \Population} {
			\CandidateCost $\leftarrow$ \Cost{\Candidate}\;
		}
		\Best $\leftarrow$ \GetBestSolution{\Population}\;
		\Population $\leftarrow$ \StochasticGlobalSearch{\Population}\;
		\MemeticPopulation $\leftarrow$ \SelectMemeticPopulation{\Population, \MemeticPopulationSize}\;
		\ForEach{\Candidate $\in$ \MemeticPopulation} {
			\Candidate $\leftarrow$ \LocalSearch{\Candidate}\;
		}
	}
	\Return{\Best}\;
	% end
	\caption{Pseudocode for the Memetic Algorithm.}
	\label{alg:ma}
\end{algorithm}

% Heuristics: Usage guidelines
% The heuristics element describe the commonsense, best practice, and demonstrated rules for applying and configuring a parameterized algorithm. The heuristics relate to the technical details of the techniques procedure and data structures for general classes of application (neither specific implementations not specific problem instances). The heuristics are described textually, such as a series of guidelines in a bullet-point structure.
\subsection{Heuristics}
% What are the suggested configurations for a technique?
% What are the guidelines for the application of a technique to a problem instance?
\begin{itemize}
	\item The global search provides the broad exploration mechanism, whereas the individual solution improvement via local search provides an exploitation mechanism. 
	\item Balance is needed between the local and global mechanisms to ensure the system does not prematurely converge to a local optimum and does not consume unnecessary computational resources.
	\item The local search should be problem and representation specific, where as the global search may be generic and non-specific (black-box).
	\item Memetic Algorithms have been applied to a range of constraint, combinatorial, and continuous function optimization problem domains.
\end{itemize}

% Code Listing
% The code description provides a minimal but functional version of the technique implemented with a programming language. The code description must be able to be typed into an appropriate computer, compiled or interpreted as need be, and provide a working execution of the technique. The technique implementation also includes a minimal problem instance to which it is applied, and both the problem and algorithm implementations are complete enough to demonstrate the techniques procedure. The description is presented as a programming source code listing.
\subsection{Code Listing}
% How is a technique implemented as an executable program?
% How is a technique applied to a concrete problem instance?
Listing~\ref{memetic_algorithm} provides an example of the Memetic Algorithm implemented in the Ruby Programming Language. 
% problem
The demonstration problem is an instance of a continuous function optimization that seeks $\min f(x)$ where $f=\sum_{i=1}^n x_{i}^2$, $-5.0\leq x_i \leq 5.0$ and $n=3$. The optimal solution for this basin function is $(v_0,\ldots,v_{n-1})=0.0$.
% algorithm
The Memetic Algorithm uses a canonical Genetic Algorithm as the global search technique that operates on binary strings, uses tournament selection, point mutations, uniform crossover and a binary coded decimal decoding of bits to real values. A bit climber local search is used that performs probabilistic bit flips (point mutations) and only accepts solutions with the same or improving fitness. 

% the listing
\lstinputlisting[firstline=7,language=ruby,caption=Memetic Algorithm in Ruby, label=memetic_algorithm]{../src/algorithms/physical/memetic_algorithm.rb}

% References: Deeper understanding
% The references element description includes a listing of both primary sources of information about the technique as well as useful introductory sources for novices to gain a deeper understanding of the theory and application of the technique. The description consists of hand-selected reference material including books, peer reviewed conference papers, journal articles, and potentially websites. A bullet-pointed structure is suggested.
\subsection{References}
% What are the primary sources for a technique?
% What are the suggested reference sources for learning more about a technique?

% 
% Primary Sources
% 
\subsubsection{Primary Sources}
% seminal
The concept of a Memetic Algorithm is credited to Moscato \cite{Moscato1989}, who was inspired by the description of meme's in Dawkins' ``The Selfish Gene'' \cite{Dawkins1976}. Moscato proposed Memetic Algorithms as the marriage between population based global search and heuristic local search made by each individual without the constraints of a genetic representation and investigated variations on the Traveling Salesman Problem.
% early

% 
% Learn More
% 
\subsubsection{Learn More}
% reviews
Moscato and Cotta provide a gentle introduction to the field of Memetic Algorithms as a book chapter that covers formal descriptions of the approach, a summary of the fields of application, and the state of the art \cite{Moscato2003}.
A overview and classification of the types of Memetic Algorithms is presented by Ong et al.\ who describe a class of adaptive Memetic Algorithms \cite{Ong2006}. Krasnogor and Smith also provide a taxonomy of Memetic Algorithms, focusing on the properties needed to design `competent' implementations of the approach with examples on a number of combinatorial optimization problems \cite{Krasnogor2005}. Work by Krasnogor and Gustafson investigate what they refer to as `self-generating' Memetic Algorithms that use the memetic principle to co-evolve the local search applied by individual solutions \cite{Krasnogor2004}. 
% books
For a broader overview of the field, see the 2005 book ``Recent Advances in Memetic Algorithms'' that provides an overview and a number of studies \cite{Hart2005}.


\putbib\end{bibunit}



