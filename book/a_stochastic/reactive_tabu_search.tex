% The Clever Algorithms Project: http://www.CleverAlgorithms.com
% (c) Copyright 2010 Jason Brownlee. Some Rights Reserved. 
% This work is licensed under a Creative Commons Attribution-Noncommercial-Share Alike 2.5 Australia License.

% This is an algorithm description, see:
% Jason Brownlee. A Template for Standardized Algorithm Descriptions. Technical Report CA-TR-20100107-1, The Clever Algorithms Project http://www.CleverAlgorithms.com, January 2010.

% Name
% The algorithm name defines the canonical name used to refer to the technique, in addition to common aliases, abbreviations, and acronyms. The name is used in terms of the heading and sub-headings of an algorithm description.
\section{Reactive Tabu Search} 
\label{sec:reactive_tabu_search}

% other names
% What is the canonical name and common aliases for a technique?
% What are the common abbreviations and acronyms for a technique?
\emph{Reactive Tabu Search, RTS, R-TABU, Reactive Taboo Search.}

% Taxonomy: Lineage and locality
% The algorithm taxonomy defines where a techniques fits into the field, both the specific subfields of Computational Intelligence and Biologically Inspired Computation as well as the broader field of Artificial Intelligence. The taxonomy also provides a context for determining the relation- ships between algorithms. The taxonomy may be described in terms of a series of relationship statements or pictorially as a venn diagram or a graph with hierarchical structure.
\subsection{Taxonomy}
% To what fields of study does a technique belong?
Reactive Tabu Search is a Metaheuristic and a Global Optimization algorithm.
% What are the closely related approaches to a technique?
It is an extension of Tabu Search (Section~\ref{sec:tabu_search}) and the basis for a field of reactive techniques called Reactive Local Search and more broadly the field of Reactive Search Optimization.

% Strategy: Problem solving plan
% The strategy is an abstract description of the computational model. The strategy describes the information processing actions a technique shall take in order to achieve an objective. The strategy provides a logical separation between a computational realization (procedure) and a analogous system (metaphor). A given problem solving strategy may be realized as one of a number specific algorithms or problem solving systems. The strategy description is textual using information processing and algorithmic terminology.
\subsection{Strategy}
% What is the information processing objective of a technique?
% What is a techniques plan of action?
The objective of Tabu Search is to avoid cycles while applying a local search technique. The Reactive Tabu Search addresses this objective by explicitly monitoring the search and reacting to the occurrence of cycles and their repetition by adapting the tabu tenure (tabu list size).
% reactive search
The strategy of the broader field of Reactive Search Optimization is to automate the process by which a practitioner configures a search procedure by monitoring its online behavior and to use machine learning techniques to adapt a techniques configuration.

% Procedure: Abstract computation
% The algorithmic procedure summarizes the specifics of realizing a strategy as a systemized and parameterized computation. It outlines how the algorithm is organized in terms of the data structures and representations. The procedure may be described in terms of software engineering and computer science artifacts such as pseudo code, design diagrams, and relevant mathematical equations.
\subsection{Procedure}
% What is the computational recipe for a technique?
% What are the data structures and representations used in a technique?
Algorithm~\ref{alg:reactive_tabu_search} provides a pseudo-code listing of the Reactive Tabu Search algorithm for minimizing a cost function. 
% specifics
The pseudo code is based on a the version of the Reactive Tabu Search described by Battiti and Tecchiolli in \cite{Battiti1995a} with supplements like the \texttt{IsTabu} function from \cite{Battiti1994}. The procedure has been modified for brevity to exude the diversification procedure (escape move). Algorithm~\ref{alg:memory_based_reaction} describes the memory based reaction that manipulates the size of the \texttt{ProhibitionPeriod} in response to identified cycles in the ongoing search. Algorithm~\ref{alg:best_move} describes the selection of the best move from a list of candidate moves in the neighborhood of a given solution. The function permits prohibited moves in the case where a prohibited move is better than the best know solution and the selected admissible move (called aspiration). Algorithm~\ref{alg:tabu} determines whether a given neighborhood move is tabu based on the current \texttt{ProhibitionPeriod}, and is employed by sub-functions of the Algorithm~\ref{alg:best_move} function.

\begin{algorithm}[ht]
	\SetLine
	% data
	\SetKwData{Best}{$S_{best}$}
	\SetKwData{Current}{$S_{curr}$}
	\SetKwData{CurrentFeature}{$Scurr_{feature}$}
	\SetKwData{TabuList}{TabuList}
	\SetKwData{CandidateList}{CandidateList}
	\SetKwData{Increase}{Increase}
	\SetKwData{Decrease}{Decrease}
	\SetKwData{ProhibitionPeriod}{ProhibitionPeriod}
	\SetKwData{MaxIterations}{$Iteration_{max}$}
	\SetKwData{CurrentIteration}{$Iteration_{i}$}
	\SetKwData{ShouldEscape}{ShouldEscape}	
	\SetKwData{ProblemSize}{ProblemSize}
	% functions
	\SetKwFunction{Cost}{Cost}
	\SetKwFunction{ConstructInitialSolution}{ConstructInitialSolution}
	\SetKwFunction{GenerateCandidateNeighborhood}{GenerateCandidateNeighborhood}
	\SetKwFunction{MemoryBasedReaction}{MemoryBasedReaction}
	\SetKwFunction{BestMove}{BestMove}
	% I/O
	\KwIn{\MaxIterations, \Increase, \Decrease, \ProblemSize}
	\KwOut{\Best}
  	% Algorithm
	\Current $\leftarrow$ \ConstructInitialSolution{}\;
	\Best $\leftarrow$ \Current\;
	\TabuList $\leftarrow$ 0\;
	\ProhibitionPeriod $\leftarrow$ 1\;
	
	\ForEach{\CurrentIteration $\in$ \MaxIterations} {
		% memory based reaction
		\MemoryBasedReaction{\Increase, \Decrease, \ProblemSize}\;
		% neighbours
		\CandidateList $\leftarrow$ \GenerateCandidateNeighborhood{\Current}\;
		% best move
		\Current $\leftarrow$ \BestMove{\CandidateList}\;
		% update tabu list
		\TabuList $\leftarrow$ \CurrentFeature\;
		% update best so far
		\If{\Cost{\Current} $\leq$ \Cost{\Best}}{
			\Best $\leftarrow$ \Current\;
		}
	}
	\Return{\Best}\;
	% caption
	\caption{Pseudo Code for the Reactive Tabu Search algorithm.}
	\label{alg:reactive_tabu_search}
\end{algorithm}

\begin{algorithm}[pht]
	\SetLine
	% data
	\SetKwData{Current}{$S_{curr}$}
	\SetKwData{CurrentVisitTime}{$Scurr_{t}$}
	\SetKwData{Increase}{Increase}
	\SetKwData{Decrease}{Decrease}
	\SetKwData{VisitedSolutions}{VisitedSolutions}
	\SetKwData{CurrentIteration}{$Iteration_{i}$}
	\SetKwData{RepetitionInterval}{RepetitionInterval}
	\SetKwData{MostVisitedCandidates}{MostVisitedCandidates}
	\SetKwData{AvgRepetitionInterval}{$RepetitionInterval_{avg}$}
	\SetKwData{ProhibitionPeriod}{ProhibitionPeriod}
	\SetKwData{ProhibitionPeriodTime}{$ProhibitionPeriod_{t}$}
	\SetKwData{ProblemSize}{ProblemSize}
	% functions
	\SetKwFunction{HaveVisitedSolutionBefore}{HaveVisitedSolutionBefore}	
	\SetKwFunction{RetrieveLastTimeVisited}{RetrieveLastTimeVisited}
	\SetKwFunction{Count}{Count}
	\SetKwFunction{Empty}{Empty}
	\SetKwFunction{Max}{Max}
	
	% I/O
	\KwIn{\Increase, \Decrease, \ProblemSize}
	\KwOut{}
  	% Algorithm
	\eIf{\HaveVisitedSolutionBefore{\Current, \VisitedSolutions}}{
		% RepetitionInterval
		\CurrentVisitTime $\leftarrow$ \RetrieveLastTimeVisited{\VisitedSolutions, \Current}\;
		\RepetitionInterval $\leftarrow$ \CurrentIteration $-$ \CurrentVisitTime\;
		\CurrentVisitTime $\leftarrow$ \CurrentIteration\;
		% check for increase
		\If{\RepetitionInterval $<$ $2*\ProblemSize$}{
			\AvgRepetitionInterval $\leftarrow$ $0.1*\RepetitionInterval + 0.9*\AvgRepetitionInterval$\;
			\ProhibitionPeriod $\leftarrow$ $\ProhibitionPeriod * \Increase$\;
			\ProhibitionPeriodTime $\leftarrow$ \CurrentIteration\;
		}
	}{
		% remember solution
		\VisitedSolutions $\leftarrow$ \Current\;
		\CurrentVisitTime $\leftarrow$ \CurrentIteration\;
	}
	
	% check for reduction
	\If{\CurrentIteration-\ProhibitionPeriodTime $>$ \AvgRepetitionInterval}{
		\ProhibitionPeriod $\leftarrow$ \Max{1, $\ProhibitionPeriod * \Decrease$}\;
		\ProhibitionPeriodTime $\leftarrow$ \CurrentIteration\;
	}

	% caption
	\caption{Pseudo Code for the \texttt{MemoryBasedReaction} function in the Reactive Tabu Search algorithm.}
	\label{alg:memory_based_reaction}
\end{algorithm}

\begin{algorithm}[pht]
	\SetLine
	% data	
	\SetKwData{ProblemSize}{ProblemSize}
	\SetKwData{Best}{$S_{best}$}
	\SetKwData{BestTabu}{$Sbest_{tabu}$}
	\SetKwData{Current}{$S_{curr}$}
	\SetKwData{CurrentIteration}{$Iteration_{i}$}
	\SetKwData{ProhibitionPeriod}{ProhibitionPeriod}
	\SetKwData{ProhibitionPeriodTime}{$ProhibitionPeriod_{t}$}
	\SetKwData{CandidateList}{CandidateList}
	\SetKwData{AdmissibleCandidateList}{$CandidateList_{admissible}$}
	\SetKwData{TabuCandidateList}{$CandidateList_{tabu}$}
	% functions
	\SetKwFunction{Size}{Size}
	\SetKwFunction{Cost}{Cost}
	\SetKwFunction{GetAdmissibleMoves}{GetAdmissibleMoves}
	\SetKwFunction{GetBest}{GetBest}
	% I/O
	\KwIn{\ProblemSize}
	\KwOut{\Current}
  	% Algorithm
	\AdmissibleCandidateList $\leftarrow$ \GetAdmissibleMoves{\CandidateList}\;
	\TabuCandidateList $\leftarrow$ \CandidateList $-$ \AdmissibleCandidateList\;
	% modify
	\If{\Size{\AdmissibleCandidateList} $<$ 2} {
		\ProhibitionPeriod $\leftarrow$ $\ProblemSize - 2$\;
		\ProhibitionPeriodTime $\leftarrow$ \CurrentIteration\;
	}
	% aspiration
	\Current $\leftarrow$ \GetBest{\AdmissibleCandidateList}\;
	\BestTabu $\leftarrow$ \GetBest{\TabuCandidateList}\;
	\If{\Cost{\BestTabu} $<$ \Cost{\Best} $\wedge$ \Cost{\BestTabu} $<$ \Cost{\Current} }{
		\Current $\leftarrow$ \BestTabu\;
	}
	\Return{\Current}\;
	% caption
	\caption{Pseudo Code for the \texttt{BestMove} function in the Reactive Tabu Search algorithm.}
	\label{alg:best_move}
\end{algorithm}

\begin{algorithm}[pht]
	\SetLine
	% data	
	\SetKwData{CurrentFeature}{$Scurr_{feature}$}
	\SetKwData{CurrentFeatureLastUsed}{$Scurr_{feature}^{t}$}
	\SetKwData{CurrentIteration}{$Iteration_{curr}$}
	\SetKwData{ProhibitionPeriod}{ProhibitionPeriod}
	\SetKwData{Tabu}{Tabu}
	% functions
	\SetKwFunction{RetrieveTimeFeatureLastUsed}{RetrieveTimeFeatureLastUsed}
	% I/O
	\KwIn{}
	\KwOut{\Tabu}
  	% Algorithm
	\Tabu $\leftarrow$ FALSE\;
	\CurrentFeatureLastUsed $\leftarrow$ \RetrieveTimeFeatureLastUsed{\CurrentFeature}\;
	\If{\CurrentFeatureLastUsed $\geq$ \CurrentIteration$-$\ProhibitionPeriod}{
		\Tabu $\leftarrow$ TRUE\;
	}
	\Return{\Tabu}\;
	% caption
	\caption{Pseudo Code for the \texttt{IsTabu} function in the Reactive Tabu Search algorithm.}
	\label{alg:tabu}
\end{algorithm}

% Heuristics: Usage guidelines
% The heuristics element describe the commonsense, best practice, and demonstrated rules for applying and configuring a parameterized algorithm. The heuristics relate to the technical details of the techniques procedure and data structures for general classes of application (neither specific implementations not specific problem instances). The heuristics are described textually, such as a series of guidelines in a bullet-point structure.
\subsection{Heuristics}
% What are the suggested configurations for a technique?
% What are the guidelines for the application of a technique to a problem instance?
\begin{itemize}
	\item Reactive Tabu Search is an extension of Tabu Search and as such should exploit the best practices used for the parent algorithm.
	\item Reactive Tabu Search was designed for discrete domains such as combinatorial optimization, although has been applied to continuos function optimization.
	\item Reactive Tabu Search was proposed to use efficient memory data structures such as hash tables.
	\item Reactive Tabu Search was proposed to use an long-term memory to diversify the search after a threshold of cycle repetitions has been reached.
	\item The \texttt{increase} parameter should be greater than one (such as 1.1 or 1.3) and the \texttt{decrease} parameter should be less than one (such as 0.9 or 0.8).
\end{itemize}

% The code description provides a minimal but functional version of the technique implemented with a programming language. The code description must be able to be typed into an appropriate computer, compiled or interpreted as need be, and provide a working execution of the technique. The technique implementation also includes a minimal problem instance to which it is applied, and both the problem and algorithm implementations are complete enough to demonstrate the techniques procedure. The description is presented as a programming source code listing.
\subsection{Code Listing}
% How is a technique implemented as an executable program?
% How is a technique applied to a concrete problem instance?
Listing~\ref{reactive_tabu_search} provides an example of the Reactive Tabu Search algorithm implemented in the Ruby Programming Language. 
% problem
The algorithm is applied to the Berlin52 instance of the Traveling Salesman Problem (TSP), taken from the TSPLIB. The problem seeks a permutation of the order to visit cities (called a tour) that minimized the total distance traveled. The optimal tour distance for Berlin52 instance is 7542 units.

% algorithm
The procedure is based on the code listing described by Battiti and Tecchiolli in \cite{Battiti1995a} with supplements like the \texttt{IsTabu} function from \cite{Battiti1994}. The implementation does not use efficient memory data structures such as hash tables. The algorithm is initialized with a stochastic 2-opt local search, and the neighborhood is generated as a fixed candidate list of stochastic 2-opt moves. The edges selected for changing in the 2-opt move are stored as features in the tabu list. The example does not implement the escape procedure for search diversification. 

% the listing
\lstinputlisting[firstline=7,language=ruby,caption=Reactive Tabu Search algorithm in the Ruby Programming Language, label=reactive_tabu_search]{../src/algorithms/stochastic/reactive_tabu_search.rb}

% References: Deeper understanding
% The references element description includes a listing of both primary sources of information about the technique as well as useful introductory sources for novices to gain a deeper understanding of the theory and application of the technique. The description consists of hand-selected reference material including books, peer reviewed conference papers, journal articles, and potentially websites. A bullet-pointed structure is suggested.
\subsection{References}
% What are the primary sources for a technique?
% What are the suggested reference sources for learning more about a technique?

% 
% Primary Sources
% 
\subsection{Primary Sources}
% seminal 
Reactive Tabu Search was proposed by Battiti and Tecchiolli as an extension to Tabu Search that included an adaptive tabu list size in addition to a diversification mechanism \cite{Battiti1994}. The technique also used efficient memory structures that were based on an earlier work by Battiti and Tecchiolli that considered a parallel tabu search \cite{Battiti1992}.
% secondary 
Some early application papers by Battiti and Tecchiolli include a comparison to Simulated Annealing applied to the Quadratic Assignment Problem  \cite{Battiti1994a}, benchmarked on instances of the knapsack problem and N-K models and compared with Repeated Local Minima Search, Simulated Annealing, and Genetic Algorithms \cite{Battiti1995a}, and training neural networks on an array of problem instances \cite{Battiti1995b}.

% 
% Learn More
% 
\subsection{Learn More}
% overviews of RTS
% none?
% reactive local search
Reactive Tabu Search was abstracted to a form called Reactive Local Search that considers adaptive methods that learn suitable parameters for heuristics that manage an embedded local search technique \cite{Battiti1995, Battiti2001}. Under this abstraction, the Reactive Tabu Search algorithm is single example of the Reactive Local Search principle applied to the Tabu Search. 
% reactive search optimization
This framework was further extended to the use of any adaptive machine learning techniques to adapt the parameters of an algorithm by reacting to algorithm outcomes online while solving a problem, called Reactive Search \cite{Battiti1996}. The best reference for this general framework is the book on Reactive Search Optimization by Battiti, Brunato, and Mascia \cite{Battiti2008}. Additionally, the review chapter by Battiti and Brunato provides a contemporary description \cite{Battiti2009}.


