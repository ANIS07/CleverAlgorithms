% Simulated Annealing

% The Clever Algorithms Project: http://www.CleverAlgorithms.com
% (c) Copyright 2010 Jason Brownlee. Some Rights Reserved. 
% This work is licensed under a Creative Commons Attribution-Noncommercial-Share Alike 2.5 Australia License.

\documentclass[a4paper, 11pt]{article}
\usepackage{tabularx}
\usepackage{booktabs}
\usepackage{url}
\usepackage[pdftex,breaklinks=true,colorlinks=true,urlcolor=blue,linkcolor=blue,citecolor=blue,]{hyperref}
\usepackage{geometry}
\usepackage[ruled, linesnumbered]{../algorithm2e}
\usepackage{listings} 
\usepackage{textcomp}
\ifx\pdfoutput\@undefined\usepackage[usenames,dvips]{color}
\else\usepackage[usenames,dvipsnames]{color}
\lstset{basicstyle=\footnotesize\ttfamily,numbers=left,numberstyle=\tiny,frame=single,columns=flexible,upquote=true,showstringspaces=false,tabsize=2,captionpos=b,breaklines=true,breakatwhitespace=true,keywordstyle=\color{blue},stringstyle=\color{ForestGreen}}
\geometry{verbose,a4paper,tmargin=25mm,bmargin=25mm,lmargin=25mm,rmargin=25mm}

% Dear template user: fill these in
\newcommand{\myreporttitle}{Simulated Annealing}
\newcommand{\myreportauthor}{Jason Brownlee}
\newcommand{\myreportemail}{jasonb@CleverAlgorithms.com}
\newcommand{\myreportwebsite}{http://www.CleverAlgorithms.com}
\newcommand{\myreportproject}{The Clever Algorithms Project\\\url{\myreportwebsite}}
\newcommand{\myreportdate}{20101115c}
\newcommand{\myreportversion}{1}
\newcommand{\myreportlicense}{\copyright\ Copyright 2010 Jason Brownlee. Some Rights Reserved. This work is licensed under a Creative Commons Attribution-Noncommercial-Share Alike 2.5 Australia License.}

% leave this alone, it's templated baby!
\title{{\myreporttitle}\footnote{\myreportlicense}}
\author{\myreportauthor\\{\myreportemail}\\\small\myreportproject}
\date{\today\\{\small{Technical Report: CA-TR-{\myreportdate}-\myreportversion}}}
\begin{document}
\maketitle

% write a summary sentence for each major section
\section*{Abstract} 
% project
The Clever Algorithms project aims to describe a large number of Artificial Intelligence algorithms in a complete, consistent, and centralized manner, to improve their general accessibility. 
% template
The project makes use of a standardized algorithm description template that uses well-defined topics that motivate the collection of specific and useful information about each algorithm described.
% report
This report describes the Simulated Annealing algorithm.

\begin{description}
	\item[Keywords:] {\small\texttt{Clever, Algorithms, Description, Optimization, Simulated, Annealing}}
\end{description} 

\section{Introduction} 
\label{sec:intro}
% project
The Clever Algorithms project aims to describe a large number of algorithms from the fields of Computational Intelligence, Biologically Inspired Computation, and Metaheuristics in a complete, consistent and centralized manner \cite{Brownlee2010}.
% description
The project requires all algorithms to be described using a standardized template that includes a fixed number of sections, each of which is motivated by the presentation of specific information about the technique \cite{Brownlee2010a}.
% this report
This report describes the Simulated Annealing algorithm.

% Name
% The algorithm name defines the canonical name used to refer to the technique, in addition to common aliases, abbreviations, and acronyms. The name is used in terms of the heading and sub-headings of an algorithm description.
\section{Name} 
\label{sec:name}
% What is the canonical name and common aliases for a technique?
% What are the common abbreviations and acronyms for a technique?
% The heading and alternate headings for the algorithm description.
Simulated Annealing, SA

% Taxonomy: Lineage and locality
% The algorithm taxonomy defines where a techniques fits into the field, both the specific subfields of Computational Intelligence and Biologically Inspired Computation as well as the broader field of Artificial Intelligence. The taxonomy also provides a context for determining the relation- ships between algorithms. The taxonomy may be described in terms of a series of relationship statements or pictorially as a venn diagram or a graph with hierarchical structure.
\section{Taxonomy}
\label{sec:taxonomy}
% To what fields of study does a technique belong?
Simulated Annealing is a global optimization that belongs to the field of Stochastic Optimization and Metaheuristics.
% What are the closely related approaches to a technique?
Simulated Annealing is an adaptation of the Metropolis-Hastings Monte Carlo algorithm for the use in function optimization. It is a baseline method that like the Genetic Algorithm provides a basis for a large variety of extensions and specialization's of the general method.

% Inspiration: Motivating system
% The inspiration describes the specific system or process that provoked the inception of the algorithm. The inspiring system may non-exclusively be natural, biological, physical, or social. The description of the inspiring system may include relevant domain specific theory, observation, nomenclature, and most important must include those salient attributes of the system that are somehow abstractly or conceptually manifest in the technique. The inspiration is described textually with citations and may include diagrams to highlight features and relationships within the inspiring system.
% Optional
\section{Inspiration}
\label{sec:inspiration}
% What is the system or process that motivated the development of a technique?
Simulated Annealing is inspired by the process of annealing in metallurgy where a material is heated and slowly cooled under controlled conditions to increase the size of the crystals and reduce their defects. This has the effect of improving the strength and durability of the material. The heat increases the energy of the atoms allowing them to move freely, and the slow cooling schedule allows a new low-energy configuration to be discovered and exploited.
% Which features of the motivating system are relevant to a technique?

% Metaphor: Explanation via analogy
% The metaphor is a description of the technique in the context of the inspiring system or a different suitable system. The features of the technique are made apparent through an analogous description of the features of the inspiring system. The explanation through analogy is not expected to be literal scientific truth, rather the method is used as an allegorical communication tool. The inspiring system is not explicitly described, this is the role of the ‘inspiration’ element, which represents a loose dependency for this element. The explanation is textual and uses the nomenclature of the metaphorical system.
% Optional
\section{Metaphor}
\label{sec:metaphor}
% What is the explanation of a technique in the context of the inspiring system?
Each configuration of the problem in the search space represents a different internal energy of the system. Heating the system up results in a broader sampling and a relaxation of the acceptance criteria of the samples taken from the search space. As the system is cooled, the scope of sampling the search space narrows to those areas that have demonstrated some potential of payoff and the acceptance criteria of samples is narrowed to improving movements. Once the system has cooled, the configuration will represent a sample at or close to a global optimum. 
% What are the functionalities inferred for a technique from the analogous inspiring system?


% Strategy: Problem solving plan
% The strategy is an abstract description of the computational model. The strategy describes the information processing actions a technique shall take in order to achieve an objective. The strategy provides a logical separation between a computational realization (procedure) and a analogous system (metaphor). A given problem solving strategy may be realized as one of a number specific algorithms or problem solving systems. The strategy description is textual using information processing and algorithmic terminology.
\section{Strategy}
\label{sec:strategy}
% What is the information processing objective of a technique?
The information processing objective of the technique is to locate the minimum cost configuration in the search space.
% What is a techniques plan of action?
The algorithms plan of action is to probabilistically re-sample the problem space with a step-size that is initially very large and that decreases over the execution time of the algorithm. Similarly, the acceptance of new samples into the currently held sample is managed by a probabilistic function that becomes more discerning of the cost of samples it accepts over the execution time of the algorithm.

% Procedure: Abstract computation
% The algorithmic procedure summarizes the specifics of realizing a strategy as a systemized and parameterized computation. It outlines how the algorithm is organized in terms of the data structures and representations. The procedure may be described in terms of software engineering and computer science artifacts such as pseudo code, design diagrams, and relevant mathematical equations.
\section{Procedure}
\label{sec:procedure}
% What is the computational recipe for a technique?
% What are the data structures and representations used in a technique?
Algorithm~\ref{alg:sa} provides a pseudo-code listing of the main Simulated Annealing algorithm for minimizing a cost function.

\begin{algorithm}[ht]
	\SetLine  

	% data
	\SetKwData{Best}{$P_{best}$}
	\SetKwData{BestCost}{$Pbest_{cost}$}
	\SetKwData{ProblemSize}{ProblemSize}
	\SetKwData{DecayFactor}{$\rho$}
	\SetKwData{HistoryContribution}{$\alpha$}
	\SetKwData{HeuristicContribution}{$\beta$}
	\SetKwData{NumAnts}{$m$}
	\SetKwData{PopulationSize}{$Population_{size}$}
	\SetKwData{Pheromone}{Pheromone}
	\SetKwData{Candidates}{Candidates}
	\SetKwData{Solution}{$S_{i}$}
	\SetKwData{SolutionCost}{$Si_{cost}$}
	\SetKwData{HeuristicSolution}{$S_{h}$}
	\SetKwData{HeuristicSolutionCost}{$Sh_{cost}$}
	
	% functions
	\SetKwFunction{Cost}{Cost}
	\SetKwFunction{StopCondition}{StopCondition}
	\SetKwFunction{InitializePheromone}{InitializePheromone}
	\SetKwFunction{ProbabilisticStepwiseConstruction}{ProbabilisticStepwiseConstruction}
	\SetKwFunction{DecayPheromone}{DecayPheromone}
	\SetKwFunction{UpdatePheromone}{UpdatePheromone}
	\SetKwFunction{CreateHeuristicSolution}{CreateHeuristicSolution}
	
	% I/O
	\KwIn{\ProblemSize}		
	\KwOut{\Best}
  % Algorithm
	
	% loop
	\While{$\neg$\StopCondition{}} {
		
	}
	\Return{\Best}\;
	% end
	\caption{Pseudo Code for the Simulated Annealing algorithm.}
	\label{alg:sa}
\end{algorithm}

% Heuristics: Usage guidelines
% The heuristics element describe the commonsense, best practice, and demonstrated rules for applying and configuring a parameterized algorithm. The heuristics relate to the technical details of the techniques procedure and data structures for general classes of application (neither specific implementations not specific problem instances). The heuristics are described textually, such as a series of guidelines in a bullet-point structure.
\section{Heuristics}
\label{sec:heuristics}
% What are the suggested configurations for a technique?
% What are the guidelines for the application of a technique to a problem instance?
\begin{itemize}
	\item Simulated Annealing was designed for use with combinatorial optimization problems, although it has also been adapted for continuous function optimization problems.
	\item The convergence proof suggests that with a long enough cooling period, that the system will always converge to the global optimum. The downside of this finding is that the number of samples taken for optimum convergence to occur on some problems may be more than a complete enumeration of the search space. 
	\item ...
\end{itemize}

% The code description provides a minimal but functional version of the technique implemented with a programming language. The code description must be able to be typed into an appropriate computer, compiled or interpreted as need be, and provide a working execution of the technique. The technique implementation also includes a minimal problem instance to which it is applied, and both the problem and algorithm implementations are complete enough to demonstrate the techniques procedure. The description is presented as a programming source code listing.
\section{Code Listing}
\label{sec:code}
% How is a technique implemented as an executable program?
% How is a technique applied to a concrete problem instance?
Listing~\ref{simulated_annealing} provides an example of the Simulated Annealing algorithm implemented in the Ruby Programming Language. 
% problem
The algorithm is applied to the Berlin52 instance of the Traveling Salesman Problem (TSP), taken from the TSPLIB. The problem seeks a permutation of the order to visit cities (called a tour) that minimized the total distance traveled. The optimal tour distance for Berlin52 instance is 7542 units.
% algorithm
The algorithm...

% the listing
\lstinputlisting[firstline=7,language=ruby,caption=Simulated Annealing algorithm in the Ruby Programming Language, label=simulated_annealing]{../../src/algorithms/physical/simulated_annealing.rb}


% References: Deeper understanding
% The references element description includes a listing of both primary sources of information about the technique as well as useful introductory sources for novices to gain a deeper understanding of the theory and application of the technique. The description consists of hand-selected reference material including books, peer reviewed conference papers, journal articles, and potentially websites. A bullet-pointed structure is suggested.
\section{References}
\label{sec:references}
% What are the primary sources for a technique?
% What are the suggested reference sources for learning more about a technique?

% 
% Primary Sources
% 
\subsection{Primary Sources}
% seminal
Simulated Annealing is credited to Kirkpatrick, Gelatt, and Vecchi in 1983 \cite{Kirkpatrick1983}. Granville, Krivanek, and Rasson provided the proof for convergence for Simulated Annealing in 1994 \cite{Granville1994}.
% early


% 
% Learn More
% 
\subsection{Learn More}
% reviews
There are many excellent reviews of Simulated Annealing, not limited to...

% books
There are books dedicated to Simulated Annealing, applications and variations. Two examples of good texts include ``Simulated Annealing: Theory and Applications'' by Laarhoven and Aarts \cite{Laarhoven1988} that provides an introduction to the technique and applications, and ``Simulated annealing: parallelization techniques'' by Robert Azencott \cite{Azencott1992} that focuses of the theory and applications of parallel methods for Simulated Annealing.

% 
% Conclusions: What the reader or what thre author learned by completing this this report.
% 
\section{Conclusions}
\label{sec:conclusions}
% report
This report described the Simulated Annealing algorithm using the standardized template.

% 
% Contribute
% 
\section{Contribute}
\label{sec:contribute}
% simple
Found a typo in the content or a bug in the source code? 
% advanced 
Are you an expert in this technique and know some facts that could improve the algorithm description for all?
% incentive
Do you want to get that warm feeling from contributing to an open source project? 
Do you want to see your name as an acknowledgment in print?

%  ideal
Two pillars of this effort are i) that the best domain experts are people outside of the project, and ii) that this work is subjected to continuous improvement. 
% advice
Please help to make this work less wrong by emailing the author `\myreportauthor' at \url{\myreportemail} or visit the project website at \url{\myreportwebsite}.

% bibliography
\bibliographystyle{plain}
\bibliography{../bibtex}

\end{document}
% EOF