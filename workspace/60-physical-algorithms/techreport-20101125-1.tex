% Physical Algorithms

% The Clever Algorithms Project: http://www.CleverAlgorithms.com
% (c) Copyright 2010 Jason Brownlee. Some Rights Reserved. 
% This work is licensed under a Creative Commons Attribution-Noncommercial-Share Alike 2.5 Australia License.

\documentclass[a4paper, 11pt]{article}
\usepackage{tabularx}
\usepackage{booktabs}
\usepackage{url}
\usepackage[pdftex,breaklinks=true,colorlinks=true,urlcolor=blue,linkcolor=blue,citecolor=blue,]{hyperref}
\usepackage{geometry}
\geometry{verbose,a4paper,tmargin=25mm,bmargin=25mm,lmargin=25mm,rmargin=25mm}

% Dear template user: fill these in
\newcommand{\myreporttitle}{Physical Algorithms}
\newcommand{\myreportauthor}{Jason Brownlee}
\newcommand{\myreportemail}{jasonb@CleverAlgorithms.com}
\newcommand{\myreportproject}{The Clever Algorithms Project\\\url{http://www.CleverAlgorithms.com}}
\newcommand{\myreportdate}{20101125}
\newcommand{\myreportfulldate}{November 24, 2010}
\newcommand{\myreportversion}{1}
\newcommand{\myreportlicense}{\copyright\ Copyright 2010 Jason Brownlee. Some Rights Reserved. This work is licensed under a Creative Commons Attribution-Noncommercial-Share Alike 2.5 Australia License.}

% leave this alone, it's templated baby!
\title{{\myreporttitle}\footnote{\myreportlicense}}
\author{\myreportauthor\\{\myreportemail}\\\small\myreportproject}
\date{\myreportfulldate\\{\small{Technical Report: CA-TR-{\myreportdate}-\myreportversion}}}
\begin{document}
\maketitle

% write a summary sentence for each major section
\section*{Abstract} 
% project
The Clever Algorithms project aims to describe a large number of Artificial Intelligence algorithms in a complete, consistent, and centralized manner, to improve their general accessibility. 
% template
The project makes use of a standardized algorithm description template that uses well-defined topics that motivate the collection of specific and useful information about each algorithm described.
% type
A collection of algorithms for the project have been described, all of which are classified as Physical Algorithms under the adopted algorithm taxonomy.
% best practices
This report provides a point of reflection on the preparation of these algorithms.

\begin{description}
	\item[Keywords:] {\small\texttt{Clever, Algorithms, Project, Physical, Findings}}
\end{description} 

% summarise the document breakdown with cross references
\section{Introduction}
\label{sec:introduction}
% project
The Clever Algorithms project aims to describe a large number of algorithms from the fields of Computational Intelligence, Biologically Inspired Computation, and Metaheuristics in a complete, consistent and centralized manner \cite{Brownlee2010}.
% description
The project requires all algorithms to be described using a standardized template that includes a fixed number of sections, each of which is motivated by the presentation of specific information about the technique \cite{Brownlee2010a}.
% this report
This report provides an overview of the Physical Algorithms in the Clever Algorithms project. 
Section~\ref{sec:algorithms} provides background information and reviews common themes for the general class of algorithm and summarizes those physical algorithms that have been described for the Clever Algorithms Project.

% 
% Described Algorithms
% 
\section{Physical Algorithms}
\label{sec:algorithms}

% 
% Background
% 
\subsection{Background}
% broadly
The algorithms to be described in the Clever Algorithms project are drawn from a diverse set of subfields of Artificial Intelligence, such as Computational Intelligence, Biologically Inspired Computation, and Metaheuristics. The majority of the algorithms selected for description in the project are optimization algorithms \cite{Brownlee2010b}. 
% specific
The recently completed algorithms that have been described for the Clever Algorithms project are referred to as Physical Algorithms. They are differentiated from the remainder of the algorithms described in the project that have been designated a taxonomy including swarm, stochastic, immune, probabilistic, neural, and evolutionary algorithms \cite{Brownlee2010b}. 

% biological
\subsubsection{Physical Properties}
Physical algorithms are those algorithms inspired by a physical process. The described physical algorithm generally belong into the fields of Metaheustics and Computational Intelligence, although do not fit neatly into the existing categories of the biological inspired techniques (such as Swarm, Immune, Neural, and Evolution). In this vein, they could just as easily be referred to as nature inspired algorithms.

The inspiring physical systems range from metallurgy, music, the interplay between culture and evolution, and complex dynamic systems such as avalanches. They are generally stochastic optimization algorithms with a mixtures of local (neighborhood-based) and global search techniques. 

% 
% Described Algorithms
% 
\subsection{Described Algorithms}
\label{subsec:algorithms}
% overview
This section lists the Physical Algorithms currently described for inclusion in the Clever Algorithms project. It is proposed that these algorithms will collectively comprise a chapter on `Physical Algorithms' in the Clever Algorithms book. 

\begin{enumerate}
	\item \textbf{Simulated Annealing}: \cite{Brownlee2010ak}
	\item \textbf{Extremal Optimization}: \cite{Brownlee2010al}
	\item \textbf{Harmony Search}: \cite{Brownlee2010am}
	\item \textbf{Cultural Algorithm}: \cite{Brownlee2010an}
	\item \textbf{Memetic Algorithm}: \cite{Brownlee2010ao}
\end{enumerate}

% 
% Extensions
% 
\section{Extensions}
\label{sec:extensions}
There are other algorithms and classes of algorithm that were not described from the field of Neural Intelligence. Some areas that may be considered for algorithm description in follow up works include:

\begin{itemize}
	\item \textbf{More Annealing}: Extensions to the classical Simulated Annealing algorithm, such as Adaptive Simulated Annealing (formally Very Fast Simulated Re-annealing) \cite{Ingber1989, Ingber1996}, and Quantum Annealing \cite{Apolloni1989, Das2005}.
	\item \textbf{Stochastic tunneling}: based on the physical idea of a particle tunneling through structures \cite{Wenzel1999}.
\end{itemize}

% 
% Conclusions
% 
\section{Conclusions}
\label{sec:conclusions}
% overview
This report provided a point of reflection for the batch of Physical Algorithm descriptions prepared for the Clever Algorithms project. All described algorithms were assigned to the `Physical Algorithms' kingdom in the chosen algorithm taxonomy. This report highlighted the commonality for all described Physical Algorithms and provided a definition suitable for use in the proposed book and website.

% bibliography
\bibliographystyle{plain}
\bibliography{../bibtex}

\end{document}
% EOF