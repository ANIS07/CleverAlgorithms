% 

% The Clever Algorithms Project: http://www.CleverAlgorithms.com
% (c) Copyright 2010 Jason Brownlee. All Rights Reserved. 
% This work is licensed under a Creative Commons Attribution-Noncommercial-Share Alike 2.5 Australia License.

\documentclass[a4paper, 11pt]{article}
\usepackage{tabularx}
\usepackage{booktabs}
\usepackage{url}
\usepackage[pdftex,breaklinks=true,colorlinks=true,urlcolor=blue,linkcolor=blue,citecolor=blue,]{hyperref}
\usepackage{geometry}
\geometry{verbose,a4paper,tmargin=25mm,bmargin=25mm,lmargin=25mm,rmargin=25mm}

% Dear template user: fill these in
\newcommand{\myreporttitle}{Unconventional Optimization Algorithms}
\newcommand{\myreportsubtitle}{An Overview}
\newcommand{\myreportauthor}{Jason Brownlee}
\newcommand{\myreportemail}{jasonb@CleverAlgorithms.com}
\newcommand{\myreportproject}{The Clever Algorithms Project\\\url{http://www.CleverAlgorithms.com}}
\newcommand{\myreportdate}{20100118}
\newcommand{\myreportversion}{1}
\newcommand{\myreportlicense}{\copyright\ Copyright 2010 Jason Brownlee. All Rights Reserved. This work is licensed under a Creative Commons Attribution-Noncommercial-Share Alike 2.5 Australia License.}

% leave this alone, it's templated baby!
\title{{\myreporttitle}: {\myreportsubtitle}\footnote{\myreportlicense}}
\author{\myreportauthor\\{\myreportemail}\\\small\myreportproject}
\date{\today\\{\small{Technical Report: CA-TR-{\myreportdate}-\myreportversion}}}
\begin{document}
\maketitle

% write a summary sentence for each major section
\section*{Abstract} 
todo

\begin{description}
	\item[Keywords:] {\small\texttt{Clever, Algorithms, Unconventional, Optimization}}
\end{description} 

% summarise the document breakdown with cross references
\section{Introduction}
\label{sec:introduction}
% project

% report

% breakdown

What do we need to know about this general class of algorithms: unconventional optimization
nomenclature

% 
% Black-Box Methods
% 
\section{Black-Box Methods}
\label{sec:black_box}
They make little or few assumptions about the problem domain


% 
% Randomness
% 
\section{Randomness}
The are stochastic processes.

stochastic global optimization

% 
% State Space
% 
\section{State Space}
The typically require the problem to be phrased as a search space which is traversed and sampled.
We care about the size of moves, the patters of sampling and re-sampling, the number of samples managed.

% 
% Induction
% 
\section{Induction}
The typically learn by doing (trial and error)
generate, guess, revise


% 
% No Free Lunch
% 
\section{No Free Lunch}
\label{sec:nfl}
all the same across all problems with no prior info


% 
%  Problems
% 
\section{Problems}
\label{sec:problems}
lots of hard problems
a book out there has a summary of the general properties of problems to which these techniques are suited

What types of computational problems are we solving with these algorithms?
Give example classes for each, give canonical instances for each (all covered in this book)

\subsection{Function Optimization}
Generate a set of parameters (continuous) or something like a permutation (combinatorial).

\subsection{Function Approximation}
Generate a representation that produces outputs in the presence of inputs.

% 
% Conclusions: summarise the document message and areas for future consideration
% 
\section{Conclusions}
\label{sec:conclusions}
todo

% bibliography
\bibliographystyle{plain}
\bibliography{../bibtex}

\end{document}
% EOF