% Clever Algorithms: A Taxonomy

% The Clever Algorithms Project: http://www.CleverAlgorithms.com
% (c) Copyright 2010 Jason Brownlee. All Rights Reserved. 
% This work is licensed under a Creative Commons Attribution-Noncommercial-Share Alike 2.5 Australia License.

\documentclass[a4paper, 11pt]{article}
\usepackage{tabularx}
\usepackage{booktabs}
\usepackage{url}
\usepackage[pdftex,breaklinks=true,colorlinks=true,urlcolor=blue,linkcolor=blue,citecolor=blue,]{hyperref}
\usepackage{geometry}
\geometry{verbose,a4paper,tmargin=25mm,bmargin=25mm,lmargin=25mm,rmargin=25mm}

% Dear template user: fill these in
\newcommand{\myreporttitle}{Clever Algorithms}
\newcommand{\myreportsubtitle}{A Taxonomy}
\newcommand{\myreportauthor}{Jason Brownlee}
\newcommand{\myreportemail}{jasonb@CleverAlgorithms.com}
\newcommand{\myreportproject}{The Clever Algorithms Project\\\url{http://www.CleverAlgorithms.com}}
\newcommand{\myreportdate}{20100115}
\newcommand{\myreportversion}{1}
\newcommand{\myreportlicense}{\copyright\ Copyright 2010 Jason Brownlee. All Rights Reserved. This work is licensed under a Creative Commons Attribution-Noncommercial-Share Alike 2.5 Australia License.}

% leave this alone, it's templated baby!
\title{{\myreporttitle}: {\myreportsubtitle}\footnote{\myreportlicense}}
\author{\myreportauthor\\{\myreportemail}\\\small\myreportproject}
\date{\today\\{\small{Technical Report: CA-TR-{\myreportdate}-\myreportversion}}}
\begin{document}
\maketitle

% write a summary sentence for each major section
\section*{Abstract} 
todo

\begin{description}
	\item[Keywords:] {\small\texttt{Clever, Algorithms, Artificial, Intelligence, Computational, Metaheuristics}}
\end{description} 

% summarise the document breakdown with cross references
\section{Introduction}
\label{sec:introduction}
% project
The Clever Algorithms project aims to describe a large number of `unconventional optimization algorithms' in a complete, consistent, and centralized way from the fields of Computational Intelligence and Biologically Inspired Computation \cite{Brownlee2010}.
% report
This report provides background for the Clever Algorithms project by defining a taxonomy of the related sub-fields of Artificial Intelligence from which the algorithms are drawn.
% sections
Section~\ref{sec:artificial_intelligence} introduces the field of Artificial Intelligence in terms of the classical symbolic efforts called \emph{neat AI} and the sub-symbolic and descriptive methods referred to as \emph{scruffy AI}. Section~\ref{sec:natural_computation} presents an orthogonal field to AI called Natural Computing that intersects AI with the field of \emph{Biologically Inspired Computation}. Section~\ref{sec:computationl_intelligence} introduces the field of \emph{Computational Intelligence} as an effort that unifies a number of descriptive AI sub-fields such as evolutionary computation, fuzzy logic, and neural networks. Section~\ref{sec:metaheuristics} considers the heuristic perspective on Computational Intelligence methods and the more recent fields of \emph{Metaheuristics} and \emph{Hyperheuristics}. Section~\ref{sec:machine_learning} looks at the sister field of Machine Learning, and highlights the differences between statistical and heuristic methods. Finally, Section~\ref{sec:clever_algorithms} introduces so-named \emph{clever algorithms} from the Clever Algorithms Project and how they relate to the canonical taxonomy of Artificial Intelligence sub-fields introduced in this report.

% 
% Artificial Intelligence (based on copy from my thesis)
% 
\section{Artificial Intelligence}
\label{sec:artificial_intelligence}
The field of classical \emph{Artificial Intelligence} (AI) coalesced after World War II in the 1950s drawing on an understanding of the brain from neuroscience, the new mathematics of information theory, control theory referred to as cybernetics, and the dawn of the digital computer. AI is a cross-disciplinary field of research generally concerned with developing and investigating systems that operate or act intelligently. It is generally considered a discipline in the field of computer science given the strong focus on computation.

Russell and Norvig provide a perspective that defines Artificial Intelligence in four categories: (1) systems that think like humans, (2) systems that act like humans, (3) systems that think rationally, (4) systems that act rationally \cite{Russell2009}. In their definition, acting like a human suggests that a system can do some specific things humans can do, this includes fields such as the Turing test, natural language processing, automated reasoning, knowledge representation, machine learning, computer vision, and robotics. Thinking like a human suggests systems that model the cognitive information processing properties of humans, for example a general problem solver and systems that build internal models of their world. Thinking rationally suggests laws of rationalism and structured thought, such as syllogisms and formal logic. Finally, acting rationally suggests systems that do rational things such as expected utility maximization and rational agents. 

Luger and Stubblefield suggest that AI is a sub-field of computer science concerned with the automation of intelligence, and like other sub-fields of computer science has both theoretical (\emph{how and why do the systems work?}) and application (\emph{where and when can the systems be used?}) concerns \cite{Luger1993}. They suggest a strong empirical focus to research, because although there may be a strong desire for mathematical analysis, the systems themselves defy analysis due to their complexity. The machines and software themselves are not black boxes, rather analysis proceeds by observing the systems interactions with their environment, followed by an internal assessment of the system to relate their structure back to their behavior.

Artificial Intelligence is therefore concerned with investigating mechanisms that underlie intelligence and intelligence behavior. The traditional approach toward designing and investigating AI (the so-called `good old fashioned' AI) has been to employ a symbolic basis for these mechanisms. A newer approach historically referred to as messy artificial intelligence or or soft computing does not use a symbolic basis, instead patterning these mechanisms after biological or natural processes. This represents a modern paradigm shift in interest from symbolic knowledge representations, to inference strategies for adaptation and learning, and has been referred to as neat versus scruffy approaches to AI. The neat philosophy is concerned with formal symbolic models of intelligence that can explain \emph{why} they work, whereas the scruffy philosophy is concerned with intelligent strategies that explain \emph{how} they work \cite{Sloman1990}.

\subsection{Neat AI}
The traditional stream of AI involves a top down perspective of problem solving, generally involving symbolic representations and logic processes that most importantly can explain why they work. The exemplars successes of this neat prescriptive stream include a multitude of specialist approaches such as rule-based expert systems, automatic theorem provers, and operations research techniques that underly modern planning and scheduling software. Although traditional approaches have resulted in significant success they have their limits, most notably scalability. Increases in problem size result in an unmanageable increase in the complexity of such problems meaning that although traditional techniques can guarantee an optimal, precise, or true solution, the computational execution time or computing memory required can be fantastically unreasonable.

\subsection{Scruffy AI}
There have been a number of thrusts in the field of AI toward less crisp techniques that are able to locate approximate, imprecise, or partially-true solutions to such problems with a reasonable cost of resources. Such approaches are typically \emph{descriptive} rather than \emph{prescriptive}, describing a process for achieving a solution (how), but not explaining why they work (like the neater approaches). 

Scruffy AI approaches are defined as relatively simple procedures that result in complex emergent and self-organizing behavior that can defy traditional reductionist analyses, the effects of which can be exploited for quickly locating approximate solutions to intractable problems. A common characteristic of such techniques is the incorporation of randomness in their processes resulting in robust probabilistic and stochastic decision contrasted to the sometimes more fragile crisp approaches. Another important common quality is the adoption of an inductive rather than deductive approach to problem solving, generalizing solutions or decisions from sets of specific observations made by the system.

% 
% Natural Computation - based on copy from my thesis
% 
\section{Natural Computation}
\label{sec:natural_computation}
An important perspective on scruffy Artificial Intelligence is the motivation and inspiration for the core information processing strategy of a given technique. Computers can only do what they are instructed, therefore a consideration of Computational Intelligence is to distill information principles and strategies from other fields of study, such as the physical world and biology. The study of biologically motivated Computational Intelligence may be called Biologically Inspired Computing \cite{Castro2005a}, and is one of three related fields of Natural Computing \cite{Forbes2000, Forbes2005, Paton1994}. 

Natural Computing is an interdisciplinary field concerned with the relationship of computation and biology, which in addition to Biologically Inspired Computing also comprises of Computationally Motivated Biology and Computing with Biology \cite{Paun2005, Marrow2000}. Therefore, the field of Artificial Intelligence, specifically the scruffy variety of Computational Intelligence motivates this work from the perspective of an intelligent problem-solving strategy in computer science, whereas the field of Natural Computing, specifically the Biological Inspired Computation variety motivates the actual principles and information processing capabilities of the strategy.

\subsection{Biologically Inspired Computation}
Computation inspired by biological metaphor, also referred to as \emph{Biomimicry}, and \emph{Biomemetics} in other engineering disciplines \cite{Castro2005, Benyus1998}. The intent is to devise mathematical or engineering tools to address problem domains. Biologically Inspired Computation fits into this category, as do other non-computational areas of problem solving not discussed. At its simplest, its using solutions (a procedure for finding solutions is considered a solution) found in the biological environment.

\subsection{Computationally Motivated Biology}
Investigating biology with computers. The intent of this area is to use information sciences and simulation to model biological systems in digital computers with the aim to replicate and better understand behaviors in biological systems. The field facilitates the ability to better understand life-as-it-is and investigate life-as-it-could-be. Typically, work in this sub-field is not concerned with the construction of mathematical and engineering tools, rather it is focused on simulating natural phenomena. Common examples include Artificial Life, Fractal Geometry (L-systems, Iterative Function Systems, Particle Systems, Brownian motion), and Cellular Automata. A related field is that of Computational Biology generally concerned with modeling biological systems and the application of statistical methods such as in the sub-field of bioinformatics.

\subsection{Computation with Biology}
The investigation of substrates other than silicon in which to implement computation \cite{Aaronson2005}. Common examples include molecular or DNA Computing and Quantum Computing.

% 
% Computational Intelligence - based on copy from my thesis
% 
\section{Computational Intelligence}
\label{sec:computationl_intelligence}
A modern name for the sub-field of AI concerned with sub-symbolic (messy, scruffy, soft) mechanisms is Computational Intelligence. This name provides a banner which groups four principle approaches: Fuzzy Intelligence, Connectionist Intelligence, Evolutionary Intelligence, and Swarm Intelligence \cite{Engelbrecht2007, Pedrycz1997}. Generally, Computational Intelligence describes techniques that focus on \emph{strategy} and \emph{outcome}. This section provides a summary of the four primary areas of study in Computational Intelligence.

\subsection{Evolutionary Computation} 
A paradigm that is concerned with the investigation of systems inspired by the neo-Darwinian theory of evolution by means of natural selection. Popular evolutionary algorithms include the Genetic Algorithm, Evolution Strategy, Genetic and Evolutionary Programming, and Differential Evolution \cite{Baeck2000, Baeck2000a}. The evolutionary process is considered an adaptive strategy and is typically applied to search and optimization domains \cite{Goldberg1989, Holland1975}.

\subsection{Swarm Intelligence} 
A paradigm that considers collective intelligence emerges through the interaction and cooperation of large numbers of lesser intelligent agents. The paradigm consists of two dominant sub-fields (1) Ant Colony Optimization that investigates probabilistic algorithms inspired by the stigmergy and foraging behavior of ants \cite{Bonabeau1999, Dorigo2004}, and (2) Particle Swarm Optimization that investigates probabilistic algorithms inspired by the flocking and foraging behavior of birds and fish \cite{Shi2001}. Like evolutionary computation, swarm intelligences are considered adaptive strategies and are typically applied to search and optimization domains.

\subsection{Connectionist Intelligence}
An approach that is concerned with the investigation of architectures and learning strategies inspired by the modeling of neurons in the brain called Artificial Neural Networks. Learning strategies are typically divided into supervised and unsupervised which manage environmental feedback in different ways. Neural network learning processes are considered adaptive learning and are typically applied to function approximation and pattern recognition domains.

\subsection{Fuzzy Intelligence}
An approach that is concerned with the investigation of fuzzy logic which is a form of logic that is not constrained to true and false, but rather functions which define degree’s of truth. Fuzzy logic and fuzzy systems are a reasoning strategy and is typically applied to expert system and control system domains.

% 
% Metaheuristics - based on some copy from my thesis
% 
\section{Metaheuristics}
\label{sec:metaheuristics}
Another popular and general name for the strategy-outcome perspective of scruffy AI is \emph{Metaheuristics} that evolved from the neater field of heuristics methods applied in Operations Research. A metaheuristic is defined as a general algorithmic framework which can be applied to different optimization problems with relative few modifications to make them adapted to a specific problem \cite{Blum2003}. 

\subsection{Heuristics}
classical heuristics

\subsection{Meta-Heuristics}
todo

\subsection{Hyper-Heuristics}
todo

% 
% Machine Learning
% 
\section{Machine Learning}
\label{sec:machine_learning}
Another important perspective is that provided by the field of Machine Learning that focuses on the learning properties of Artificial Intelligence. The term is commonly used to describe inductive model building techniques (that generalize from specific observations) that are applied to ‘learn’ relationships in data sets (the application of which is referred to as Data Mining [433]), with or without supervision (corrective behavior) [293].

\subsection{Statistical Machine Learning}

\subsection{Data Mining}

% 
% Clever Algorithms
% 
\section{Clever Algorithms}
\label{sec:clever_algorithms}
This books is concerned with the algorithms, their general strategies, and their inspiration drawn from across these sub-fields of Artificial Intelligence and Computer Science. The term \emph{Clever Algorithms} is intended to unify a collection of interesting and useful computational tools under a consistent and accessible banner: \emph{algorithms drawn from the field of artificial intelligence whose strategies are inspired by a natural or physical systems}. The term is intended for accessibility, not as a new branch of study, a branch that perhaps already has too many names.

Really, the project is currently focused on `unconventional optimization algorithms' from Artificial Intelligence.

so where do clever algorithms fit into all of this?
all and any really




% bibliography
\bibliographystyle{plain}
\bibliography{../bibtex}

\end{document}
% EOF