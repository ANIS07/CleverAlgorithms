% Clever Algorithms: A Taxonomy

% The Clever Algorithms Project: http://www.CleverAlgorithms.com
% (c) Copyright 2010 Jason Brownlee. All Rights Reserved. 
% This work is licensed under a Creative Commons Attribution-Noncommercial-Share Alike 2.5 Australia License.

\documentclass[a4paper, 11pt]{article}
\usepackage{tabularx}
\usepackage{booktabs}
\usepackage{url}
\usepackage[pdftex,breaklinks=true,colorlinks=true,urlcolor=blue,linkcolor=blue,citecolor=blue,]{hyperref}
\usepackage{geometry}
\geometry{verbose,a4paper,tmargin=25mm,bmargin=25mm,lmargin=25mm,rmargin=25mm}

% Dear template user: fill these in
\newcommand{\myreporttitle}{Clever Algorithms}
\newcommand{\myreportsubtitle}{A Taxonomy}
\newcommand{\myreportauthor}{Jason Brownlee}
\newcommand{\myreportemail}{jasonb@CleverAlgorithms.com}
\newcommand{\myreportproject}{The Clever Algorithms Project\\\url{http://www.CleverAlgorithms.com}}
\newcommand{\myreportdate}{20100115}
\newcommand{\myreportversion}{1}
\newcommand{\myreportlicense}{\copyright\ Copyright 2010 Jason Brownlee. All Rights Reserved. This work is licensed under a Creative Commons Attribution-Noncommercial-Share Alike 2.5 Australia License.}

% leave this alone, it's templated baby!
\title{{\myreporttitle}: {\myreportsubtitle}\footnote{\myreportlicense}}
\author{\myreportauthor\\{\myreportemail}\\\small\myreportproject}
\date{\today\\{\small{Technical Report: CA-TR-{\myreportdate}-\myreportversion}}}
\begin{document}
\maketitle

% write a summary sentence for each major section
\section*{Abstract} 
todo

\begin{description}
	\item[Keywords:] {\small\texttt{Keywords, Go, Here}}
\end{description} 

% summarise the document breakdown with cross references
\section{Introduction}
\label{sec:introduction}
% project

% report
all about an overview of ai, and specifically the fields related to the clever algorithms project

% 
% Artificial Intelligence
% 
\section{Artificial Intelligence}
\label{sec:artificial_intelligence}
The field of classical \emph{Artificial Intelligence} (AI) coalesced after World War II in the 1950's drawing on an understanding of the brain from neuroscience, the new mathematics of information theory, control theory referred to as cybernetics, and the dawn of the digital computer. AI is a cross-disciplinary field of research generally concerned with developing and investigating systems that operate or act intelligently. It is generally considered a discipline in the field of computer science given the strong focus on computation.

Some great AI books are \cite{Russell2009} and \cite{Luger1993}

\subsection{Neat AI}
The traditional stream of AI involves a top down perspective of problem solving, generally involving symbolic representations and logic processes that most importantly can explain why they work. The exemplars successes of this neat prescriptive stream include a multitude of specialist approaches such as rule-based expert systems, automatic theorem provers, and operations research techniques that underly modern planning and scheduling software. Although traditional approaches have resulted in significant success they have their limits, most notably scalability. Increases in problem size result in an unmanageable increase in the complexity of such problems meaning that although traditional techniques can guarantee an optimal, precise, or true solution, the computational execution time or computing memory required can be fantastically unreasonable.

\subsection{Messy AI}
As such, there have been a number of thrusts in the field of AI toward less neat techniques that are able to locate approximate, imprecise, or partially-true solutions to such problems with a reasonable cost of resources. Such approaches are typically \emph{descriptive} rather than \emph{prescriptive}, describing a process for achieving a solution (how), but not explaining why they work (like the neat perceptive approaches). 

Messy AI approaches are defined as relatively simple procedures that result in complex emergent and self-organizing behavior that can defy traditional reductionist analyses, the effects of which can be exploited for quickly locating approximate solutions to intractable problems. A common characteristic of such techniques is the incorporation of randomness in their processes resulting in robust probabilistic and stochastic decision contrasted to the sometimes more fragile crisp approaches. Another important common quality is the adoption of an inductive rather than deductive approach to problem solving, generalizing solutions or decisions from sets of specific observations made by the system.


% 
% Natural Computation
% 
\section{Natural Computation}
A common theme among these approaches is that they are patterned after natural or biological systems and given the name \emph{Natural Computing}, \emph{Biologically Inspired Computation}, \emph{Biomimicry}, and \emph{Biomemetics}. Natural systems and processes can provide powerful analogies and metaphors for a wide array of problems faced by humans. A related field that is out of scope of this book include \emph{Computing with Nature} concerned with using non-silicon substrates for computation such as DNA Computing, Photonic Computing, and Quantum Computing. Another interesting natural inspired field of research that is out of scope is \textit{Nature with Computers} that is concerned with computational modeling to better understand natural processes such as Computational Biology, Artificial Life (ALife), and Fractal Geometry.



\subsection{Computational Biology}

\subsection{Biologically Inspired Computation}

\subsection{Computation with Biology}


% 
% Computational Intelligence
% 
\section{Computational Intelligence}
\label{sec:computationl_intelligence}
Messy techniques have been given many different names. \emph{Soft computing} was used to describe these techniques, a name that contrasts the classical hard or crisp approaches, as well as the hardness or difficulty of the problems to which they are suited. More recently such approaches have fallen under the heading of \emph{Computational Intelligence}, as an attempt to unify approaches focused on \emph{strategy} and \emph{outcome}.

\subsection{Evolutionary Computation}

\subsection{Artificial Neural Networks}

\subsection{Fuzzy Logic}

% 
% Metaheuristics
% 
\section{Metaheuristics}
\label{sec:metaheuristics}
Approaches embody the the adaptive principles of natural and biological systems and also fall into the field of \emph{Machine Learning}. This is a sub-field of AI primarily concerned with investigating statistical computational learning as well as learning strategies and algorithms. A final and more recent perspective for such approaches is \emph{Metaheuristics} that proposes such strategies guide simpler problem specific heuristics for problem solving.


\subsection{Heuristics}

\subsection{Meta-Heuristics}

\subsection{Hyper-Heuristics}


% 
% Machine Learning
% 
\section{Machine Learning}
\label{sec:machine_learning}

\subsection{Statistical Machine Learning}

\subsection{Data Mining}

% 
% Clever Algorithms
% 
\section{Clever Algorithms}
\label{sec:clever_algorithms}
This books is concerned with the algorithms, their general strategies, and their inspiration drawn from across these sub-fields of Artificial Intelligence and Computer Science. The term \emph{Clever Algorithms} is intended to unify a collection of interesting and useful computational tools under a consistent and accessible banner: \emph{algorithms drawn from the field of artificial intelligence whose strategies are inspired by a natural or physical systems}. The term is intended for accessibility, not as a new branch of study, a branch that perhaps already has too many names.

Really, the project is currently focused on `unconventional optimization algorithms' from Artificial Intelligence.

so where do clever algorithms fit into all of this?
all and any really


% summarise the document message and areas for future consideration
\section{Conclusions}
\label{sec:conclusions}
Overview



% bibliography
\bibliographystyle{plain}
\bibliography{../bibtex}

\end{document}
% EOF