% Scatter Search

% The Clever Algorithms Project: http://www.CleverAlgorithms.com
% (c) Copyright 2010 Jason Brownlee. Some Rights Reserved. 
% This work is licensed under a Creative Commons Attribution-Noncommercial-Share Alike 2.5 Australia License.

\documentclass[a4paper, 11pt]{article}
\usepackage{tabularx}
\usepackage{booktabs}
\usepackage{url}
\usepackage[pdftex,breaklinks=true,colorlinks=true,urlcolor=blue,linkcolor=blue,citecolor=blue,]{hyperref}
\usepackage{geometry}
\usepackage[ruled, linesnumbered]{../algorithm2e}
\usepackage{listings} 
\usepackage{textcomp}
\ifx\pdfoutput\@undefined\usepackage[usenames,dvips]{color}
\else\usepackage[usenames,dvipsnames]{color}
\lstset{basicstyle=\footnotesize\ttfamily,numbers=left,numberstyle=\tiny,frame=single,columns=flexible,upquote=true,showstringspaces=false,tabsize=2,captionpos=b,breaklines=true,breakatwhitespace=true,keywordstyle=\color{blue},stringstyle=\color{ForestGreen}}
\geometry{verbose,a4paper,tmargin=25mm,bmargin=25mm,lmargin=25mm,rmargin=25mm}

% Dear template user: fill these in
\newcommand{\myreporttitle}{Scatter Search}
\newcommand{\myreportauthor}{Jason Brownlee}
\newcommand{\myreportemail}{jasonb@CleverAlgorithms.com}
\newcommand{\myreportwebsite}{http://www.CleverAlgorithms.com}
\newcommand{\myreportproject}{The Clever Algorithms Project\\\url{\myreportwebsite}}
\newcommand{\myreportdate}{20100216}
\newcommand{\myreportversion}{1}
\newcommand{\myreportlicense}{\copyright\ Copyright 2010 Jason Brownlee. Some Rights Reserved. This work is licensed under a Creative Commons Attribution-Noncommercial-Share Alike 2.5 Australia License.}

% leave this alone, it's templated baby!
\title{{\myreporttitle}\footnote{\myreportlicense}}
\author{\myreportauthor\\{\myreportemail}\\\small\myreportproject}
\date{\today\\{\small{Technical Report: CA-TR-{\myreportdate}-\myreportversion}}}
\begin{document}
\maketitle

% write a summary sentence for each major section
\section*{Abstract} 
% project
The Clever Algorithms project aims to describe a large number of Artificial Intelligence algorithms in a complete, consistent, and centralized manner, to improve their general accessibility. 
% template
The project makes use of a standardized algorithm description template that uses well-defined topics that motivate the collection of specific and useful information about each algorithm described.
% report
This report describes the Scatter Search algorithm using the standardized template.

\begin{description}
	\item[Keywords:] {\small\texttt{Clever, Algorithms, Description, Optimization, Scatter, Search}}
\end{description} 

\section{Introduction} 
\label{sec:intro}
% project
The Clever Algorithms project aims to describe a large number of algorithms from the fields of Computational Intelligence, Biologically Inspired Computation, and Metaheuristics in a complete, consistent and centralized manner \cite{Brownlee2010}.
% description
The project requires all algorithms to be described using a standardized template that includes a fixed number of sections, each of which is motivated by the presentation of specific information about the technique \cite{Brownlee2010a}.
% this report
This report describes the Scatter Search algorithm using the standardized template.

% Name
% The algorithm name defines the canonical name used to refer to the technique, in addition to common aliases, abbreviations, and acronyms. The name is used in terms of the heading and sub-headings of an algorithm description.
\section{Name} 
\label{sec:name}
% What is the canonical name and common aliases for a technique?
% What are the common abbreviations and acronyms for a technique?
% The heading and alternate headings for the algorithm description.
Scatter Search, SS

% Taxonomy: Lineage and locality
% The algorithm taxonomy defines where a techniques fits into the field, both the specific subfields of Computational Intelligence and Biologically Inspired Computation as well as the broader field of Artificial Intelligence. The taxonomy also provides a context for determining the relation- ships between algorithms. The taxonomy may be described in terms of a series of relationship statements or pictorially as a venn diagram or a graph with hierarchical structure.
\section{Taxonomy}
\label{sec:taxonomy}
% To what fields of study does a technique belong?
Scatter search is a Metaheuristic and a Global Optimization algorithm. It is also sometimes associated with the field of Evolutionary Computation given the use of a population and recombination in the structure of the technique.
% What are the closely related approaches to a technique?
Scatter Search is a sibling of Tabu Search, developed by the same author and based on similar origins.

% Strategy: Problem solving plan
% The strategy is an abstract description of the computational model. The strategy describes the information processing actions a technique shall take in order to achieve an objective. The strategy provides a logical separation between a computational realization (procedure) and a analogous system (metaphor). A given problem solving strategy may be realized as one of a number specific algorithms or problem solving systems. The strategy description is textual using information processing and algorithmic terminology.
\section{Strategy}
\label{sec:strategy}
% What is the information processing objective of a technique?
The objective of Scatter Search is to maintain a set of diverse and high-quality candidate solutions. The principle of the approach is that useful information about the global optima is stored in a diverse and elite set of solutions (the reference set) and that recombining samples from the set can exploit this information.
% What is a techniques plan of action?
The strategy involves an iterative process, where a population of diverse and high-quality candidate solutions that are partitioned into subsets and linearly recombined to create weighted centroids of sample-based neighborhoods. The results of recombination are refined using an embedded heuristic and assessed in the context of the reference set as to whether or not they are retained. 

% Procedure: Abstract computation
% The algorithmic procedure summarizes the specifics of realizing a strategy as a systemized and parameterized computation. It outlines how the algorithm is organized in terms of the data structures and representations. The procedure may be described in terms of software engineering and computer science artifacts such as pseudo code, design diagrams, and relevant mathematical equations.
\section{Procedure}
\label{sec:procedure}
% What is the computational recipe for a technique?
% What are the data structures and representations used in a technique?
Algorithm~\ref{alg:scatter_search} provides a pseudo-code listing of the Scatter Search algorithm for minimizing a cost function. The procedure is based on the abstract form presented by Glover as a template for the general class of technique \cite{Glover1998a}, with influences from an application of the technique to function optimization by Glover \cite{Glover1998a}.

\begin{algorithm}[ht]
	\SetLine
	% data
	% \SetKwData{Candidate}{$S_{candidate}$}
	\SetKwData{ithCandidate}{$S_{i}$}
	\SetKwData{ithCandidateSet}{$Subset_{i}$}
	\SetKwData{Subsets}{Subsets}
	\SetKwData{ReferenceSetSize}{$ReferenceSet_{size}$}
	\SetKwData{DiverseSetSize}{$DiverseSet_{size}$}
	\SetKwData{ReferenceSet}{ReferenceSet}
	\SetKwData{InitialSet}{InitialSet}
	\SetKwData{RefinedSet}{RefinedSet}
	\SetKwData{RecombinedSet}{RecombinedSet}
	\SetKwData{CandidateSet}{CandidateSet}
	\SetKwData{RecombinedCandidates}{RecombinedCandidates}
	% functions
	\SetKwFunction{Cost}{Cost}
	\SetKwFunction{SelectInitialReferenceSet}{SelectInitialReferenceSet}
	\SetKwFunction{StopCondition}{StopCondition}
	\SetKwFunction{ConstructInitialSolution}{ConstructInitialSolution}
  	\SetKwFunction{LocalSearch}{LocalSearch}
	\SetKwFunction{SelectSubset}{SelectSubset}
	\SetKwFunction{RecombineMembers}{RecombineMembers}
	\SetKwFunction{Select}{Select}
	% I/O
	\KwIn{\DiverseSetSize, \ReferenceSetSize}
	\KwOut{\ReferenceSet}
  	% Algorithm
	\InitialSet $\leftarrow$ \ConstructInitialSolution{\DiverseSetSize}\;
	\RefinedSet $\leftarrow 0$\;
	\For{\ithCandidate $\in$ \InitialSet}{
		\RefinedSet $\leftarrow$ \LocalSearch{\ithCandidate}\;
	} 
	\ReferenceSet $\leftarrow$ \SelectInitialReferenceSet{\ReferenceSetSize}\;
	% loop
	\While{$\neg$ \StopCondition{}} {
		% select subsets
		\Subsets $\leftarrow$ \SelectSubset{\ReferenceSet}\;
		% recombine subsets
		\CandidateSet $\leftarrow 0$\;
		\For{\ithCandidateSet $\in$ \Subsets}{
			\RecombinedCandidates $\leftarrow$ \RecombineMembers{\ithCandidateSet}\;
			% improve
			\For{\ithCandidate $\in$ \RecombinedCandidates}{
				\CandidateSet $\leftarrow$ \LocalSearch{\ithCandidate}\;
			}
		}
		% update reference set
		\ReferenceSet $\leftarrow$ \Select{\ReferenceSet, \CandidateSet, \ReferenceSetSize}\;
	}
	\Return{\ReferenceSet}\;
	% caption
	\caption{Pseudo Code for the Scatter Search algorithm.}
	\label{alg:scatter_search}
\end{algorithm}

% Heuristics: Usage guidelines
% The heuristics element describe the commonsense, best practice, and demonstrated rules for applying and configuring a parameterized algorithm. The heuristics relate to the technical details of the techniques procedure and data structures for general classes of application (neither specific implementations not specific problem instances). The heuristics are described textually, such as a series of guidelines in a bullet-point structure.
\section{Heuristics}
\label{sec:heuristics}
% What are the suggested configurations for a technique?
% What are the guidelines for the application of a technique to a problem instance?
\begin{itemize}
	\item Scatter search is suitable for both discrete domains such as combinatorial optimization as well as continuous domains such as non-linear programming (continuous function optimization).
	\item Small set sizes are preferred for the \texttt{ReferenceSet}, such as 10 or 20 members.
	\item Subset sizes can be 2,3,4 or more members that are all recombined to produce viable candidate solutions within the neighborhood of the members of the subset.
	\item Each subset should comprise at least one member added to the set in the previous algorithm iteration.
	\item The Local Search procedure should be a problem-specific improvement heuristic. 
	\item The selection of members for the \texttt{ReferenceSet} at the end of each iteration favors solutions with higher quality and may also promote diversity.
	\item The \texttt{ReferenceSet} may be updated at the end of an iteration, or dynamically as candidates are created (a so-called steady-state population in some evolutionary computation literature).
	\item A lack of changes to the \texttt{ReferenceSet} may be used as a signal to stop the current search, and potentially restart the search with a newly initialized \texttt{ReferenceSet}.	
\end{itemize}

% The code description provides a minimal but functional version of the technique implemented with a programming language. The code description must be able to be typed into an appropriate computer, compiled or interpreted as need be, and provide a working execution of the technique. The technique implementation also includes a minimal problem instance to which it is applied, and both the problem and algorithm implementations are complete enough to demonstrate the techniques procedure. The description is presented as a programming source code listing.
\section{Code Listing}
\label{sec:code}
% How is a technique implemented as an executable program?
% How is a technique applied to a concrete problem instance?
Listing~\ref{scatter_search} provides an example of the Scatter Search algorithm implemented in the Ruby Programming Language. 
% problem
The example problem is an instance of a continuous function optimization that seeks $min f(x)$ where $f=\sum_{i=1}^n x_{i}^2$, $-5.0\leq x_i \leq 5.0$ and $n=3$. The optimal solution for this basin function is $(v_1,\ldots,v_{n})=0.0$.

% algorithm
The algorithm is an implementation of Scatter Search as described in an application of the technique to unconstrained non-linear optimization by Glover \cite{Glover2003b}. The seeds for initial solutions are generated as random vectors, as opposed to stratified samples. The example was further simplified by not including a restart strategy, and the exclusion of diversity maintenance in the \texttt{ReferenceSet}. A stochastic local search algorithm is used as the embedded heuristic that uses a stochastic step size in the range of half a percent of the search space.

% the listing
\lstinputlisting[firstline=7,language=ruby,caption=Scatter Search algorithm in the Ruby Programming Language, label=scatter_search]{../../src/algorithms/stochastic/scatter_search.rb}


% References: Deeper understanding
% The references element description includes a listing of both primary sources of information about the technique as well as useful introductory sources for novices to gain a deeper understanding of the theory and application of the technique. The description consists of hand-selected reference material including books, peer reviewed conference papers, journal articles, and potentially websites. A bullet-pointed structure is suggested.
\section{References}
\label{sec:references}
% What are the primary sources for a technique?
% What are the suggested reference sources for learning more about a technique?

% 
% Primary Sources
% 
\subsection{Primary Sources}
% seminal
A form of the Scatter Search algorithm was proposed by Glover for integer programming \cite{Glover1977}, based on Glover's earlier work on surrogate constraints.
% update
The approach remained idle until it was revisited by Glover and combined with Tabu Search \cite{Glover1994a}.
% modern
The modern canonical reference of the approach was proposed by Glover who provides a abstract template of the procedure that may be specialized for a given application domain \cite{Glover1998a}. 

% 
% Learn More
% 
\subsection{Learn More}
% historical reviews
The primary reference for the approach is the book by Laguna and Mart\'i that reviews the principles of the approach in detail and presents tutorials on applications of the approach on standard problems using the C programming language \cite{Laguna2003}.
% review
There are many review articles and chapters on Scatter Search that may be used to supplement an understanding of the approach, such as a detailed review chapter by Glover \cite{Glover1999}, a review of the fundamentals of the approach and its relationship to an abstraction called `path linking' by Glover, Laguna, and Mart\'i \cite{Glover2000}, and a modern overview of the technique by Mart\'i, Lagunab, and Glover \cite{Martia2006}.


% 
% Conclusions: What the reader or what thre author learned by completing this this report.
% 
\section{Conclusions}
\label{sec:conclusions}
% report
This report described the Scatter Search algorithm as a procedure that maintains a diverse and high-quality set of candidate solutions that are recombined to create new candidate solutions.

% bad algo
This preparation of this report was particularly difficult compared to recent reports. The reason for this (as proposed by the author) was because of the vagueness of the description of the algorithm in the literature and the over use of poor pseudo code listings in review articles and chapters. It is hoped that experts in the field may be consulted and/or improved descriptions of the approach can be identified.

% 
% Contribute
% 
\section{Contribute}
\label{sec:contribute}
% simple
Found a typo in the content or a bug in the source code? 
% advanced 
Are you an expert in this technique and know some facts that could improve the algorithm description for all?
% incentive
Do you want to get that warm feeling from contributing to an open source project? 
Do you want to see your name as an acknowledgment in print?

%  ideal
Two pillars of this effort are i) that the best domain experts are people outside of the project, and ii) that this work is wrong by default. 
% advice
Please help to make this work less wrong by emailing the author `\myreportauthor' at \url{\myreportemail} or visit the project website at \url{\myreportwebsite}.

% bibliography
\bibliographystyle{plain}
\bibliography{../bibtex}

\end{document}
% EOF