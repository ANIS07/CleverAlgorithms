% Stochastic Climbing Algorithm

% The Clever Algorithms Project: http://www.CleverAlgorithms.com
% (c) Copyright 2010 Jason Brownlee. Some Rights Reserved. 
% This work is licensed under a Creative Commons Attribution-Noncommercial-Share Alike 2.5 Australia License.

\documentclass[a4paper, 11pt]{article}
\usepackage{tabularx}
\usepackage{booktabs}
\usepackage{url}
\usepackage[pdftex,breaklinks=true,colorlinks=true,urlcolor=blue,linkcolor=blue,citecolor=blue,]{hyperref}
\usepackage{geometry}
\usepackage[ruled, linesnumbered]{../algorithm2e}
\usepackage{listings} 
\usepackage{textcomp}
\ifx\pdfoutput\@undefined\usepackage[usenames,dvips]{color}
\else\usepackage[usenames,dvipsnames]{color}
\lstset{basicstyle=\footnotesize\ttfamily,numbers=left,numberstyle=\tiny,frame=single,columns=flexible,upquote=true,showstringspaces=false,tabsize=2,captionpos=b,breaklines=true,breakatwhitespace=true,keywordstyle=\color{blue},stringstyle=\color{ForestGreen}}
\geometry{verbose,a4paper,tmargin=25mm,bmargin=25mm,lmargin=25mm,rmargin=25mm}

% Dear template user: fill these in
\newcommand{\myreporttitle}{Stochastic Hill Climbing}
\newcommand{\myreportauthor}{Jason Brownlee}
\newcommand{\myreportemail}{jasonb@CleverAlgorithms.com}
\newcommand{\myreportwebsite}{http://www.CleverAlgorithms.com}
\newcommand{\myreportproject}{The Clever Algorithms Project\\\url{\myreportwebsite}}
\newcommand{\myreportdate}{2010MMDD}
\newcommand{\myreportversion}{1}
\newcommand{\myreportlicense}{\copyright\ Copyright 2010 Jason Brownlee. Some Rights Reserved. This work is licensed under a Creative Commons Attribution-Noncommercial-Share Alike 2.5 Australia License.}

% leave this alone, it's templated baby!
\title{{\myreporttitle}\footnote{\myreportlicense}}
\author{\myreportauthor\\{\myreportemail}\\\small\myreportproject}
\date{\today\\{\small{Technical Report: CA-TR-{\myreportdate}-\myreportversion}}}
\begin{document}
\maketitle

% write a summary sentence for each major section
\section*{Abstract} 
% project
The Clever Algorithms project aims to describe a large number of Artificial Intelligence algorithms in a complete, consistent, and centralized manner, to improve their general accessibility. 
% template
The project makes use of a standardized algorithm description template that uses well-defined topics that motivate the collection of specific and useful information about each algorithm described.
% report
todo

\begin{description}
	\item[Keywords:] {\small\texttt{Clever, Algorithms, Description, Optimization}}
\end{description} 

\section{Introduction} 
\label{sec:intro}
% project
The Clever Algorithms project aims to describe a large number of algorithms from the fields of Computational Intelligence, Biologically Inspired Computation, and Metaheuristics in a complete, consistent and centralized manner \cite{Brownlee2010}.
% description
The project requires all algorithms to be described using a standardized template that includes a fixed number of sections, each of which is motivated by the presentation of specific information about the technique \cite{Brownlee2010a}.
% this report
This report describes\ldots todo

% Name
% The algorithm name defines the canonical name used to refer to the technique, in addition to common aliases, abbreviations, and acronyms. The name is used in terms of the heading and sub-headings of an algorithm description.
\section{Name} 
\label{sec:name}
% What is the canonical name and common aliases for a technique?
% What are the common abbreviations and acronyms for a technique?
% The heading and alternate headings for the algorithm description.
Hill Climbing Algorithm, HCA

% Taxonomy: Lineage and locality
% The algorithm taxonomy defines where a techniques fits into the field, both the specific subfields of Computational Intelligence and Biologically Inspired Computation as well as the broader field of Artificial Intelligence. The taxonomy also provides a context for determining the relation- ships between algorithms. The taxonomy may be described in terms of a series of relationship statements or pictorially as a venn diagram or a graph with hierarchical structure.
\section{Taxonomy}
\label{sec:taxonomy}
todo

% Inspiration: Motivating system
% The inspiration describes the specific system or process that provoked the inception of the algorithm. The inspiring system may non-exclusively be natural, biological, physical, or social. The description of the inspiring system may include relevant domain specific theory, observation, nomenclature, and most important must include those salient attributes of the system that are somehow abstractly or conceptually manifest in the technique. The inspiration is described textually with citations and may include diagrams to highlight features and relationships within the inspiring system.
% Optional
\section{Inspiration}
\label{sec:inspiration}
% What is the system or process that motivated the development of a technique?
% Which features of the motivating system are relevant to a technique?
todo

% Metaphor: Explanation via analogy
% The metaphor is a description of the technique in the context of the inspiring system or a different suitable system. The features of the technique are made apparent through an analogous description of the features of the inspiring system. The explanation through analogy is not expected to be literal scientific truth, rather the method is used as an allegorical communication tool. The inspiring system is not explicitly described, this is the role of the ‘inspiration’ element, which represents a loose dependency for this element. The explanation is textual and uses the nomenclature of the metaphorical system.
% Optional
\section{Metaphor}
\label{sec:metaphor}
% What is the explanation of a technique in the context of the inspiring system?
% What are the functionalities inferred for a technique from the analogous inspiring system?
todo

% Strategy: Problem solving plan
% The strategy is an abstract description of the computational model. The strategy describes the information processing actions a technique shall take in order to achieve an objective. The strategy provides a logical separation between a computational realization (procedure) and a analogous system (metaphor). A given problem solving strategy may be realized as one of a number specific algorithms or problem solving systems. The strategy description is textual using information processing and algorithmic terminology.
\section{Strategy}
\label{sec:strategy}
% What is the information processing objective of a technique?
% What is a techniques plan of action?
todo

% Procedure: Abstract computation
% The algorithmic procedure summarizes the specifics of realizing a strategy as a systemized and parameterized computation. It outlines how the algorithm is organized in terms of the data structures and representations. The procedure may be described in terms of software engineering and computer science artifacts such as pseudo code, design diagrams, and relevant mathematical equations.
\section{Procedure}
\label{sec:procedure}
% What is the computational recipe for a technique?
% What are the data structures and representations used in a technique?
todo

% Heuristics: Usage guidelines
% The heuristics element describe the commonsense, best practice, and demonstrated rules for applying and configuring a parameterized algorithm. The heuristics relate to the technical details of the techniques procedure and data structures for general classes of application (neither specific implementations not specific problem instances). The heuristics are described textually, such as a series of guidelines in a bullet-point structure.
\section{Heuristics}
\label{sec:heuristics}
% What are the suggested configurations for a technique?
% What are the guidelines for the application of a technique to a problem instance?
\begin{itemize}
	\item they can get stuck in local optima
	\item restart the process
	\item make them parallel, like doing restart
	\item for discrete domains, same principle can be used in continuous domains using step sizes (localized random search, etc)
\end{itemize}

% The code description provides a minimal but functional version of the technique implemented with a programming language. The code description must be able to be typed into an appropriate computer, compiled or interpreted as need be, and provide a working execution of the technique. The technique implementation also includes a minimal problem instance to which it is applied, and both the problem and algorithm implementations are complete enough to demonstrate the techniques procedure. The description is presented as a programming source code listing.
\section{Code Listing}
\label{sec:code}
% How is a technique implemented as an executable program?
% How is a technique applied to a concrete problem instance?
Listing~\ref{stochastic_hill_climbing} provides an example of the Stochastic Hill Climbing Algorithm implemented in the Ruby Programming Language.
% more?
todo

% the listing
\lstinputlisting[firstline=7,language=ruby,caption=Stochastic Hill Climbing Algorithm in the Ruby Programming Language, label=stochastic_hill_climbing]{../../src/algorithms/stochastic/stochastic_hill_climbing.rb}


% References: Deeper understanding
% The references element description includes a listing of both primary sources of information about the technique as well as useful introductory sources for novices to gain a deeper understanding of the theory and application of the technique. The description consists of hand-selected reference material including books, peer reviewed conference papers, journal articles, and potentially websites. A bullet-pointed structure is suggested.
\section{References}
\label{sec:references}
% What are the primary sources for a technique?
% What are the suggested reference sources for learning more about a technique?

variants:
\begin{itemize}
	\item simple hill climbing: accept first improved move
	\item steepest ascent hill climbing: test all neighbors, take best. similar in concept to best-first search (greedy search?)
	\item Stochastic hill climbing: test random neighbors, take best
	\item Random-restart hill climbing (Shotgun hill climbing): restart the search when no improvements can be made
	\item RMHC: Random Mutation Hill Climbing - same as stochastic, used for bit climbing
	\item dynamic hill climbing?
	\item search for (aliases): random hill climbing, mutation hill climbing, also `random search is simulated annealing at zero temp'
\end{itemize}

classic approach gets stuck in local optima, stochastic addresses this, so does random-restart, the others are classical techniques and out of scope.

it is for combinatorial search domains (discrete!!!), like bits or tsp, if you want continuous then go for stochastic gradient descent (approximate gradient descent) or localized random search (equiv of stochastic hill climbing)

also look at `local beam search', basically a parallel stochastic hill climber

\begin{itemize}
	\item (maybe links to seminal HL algos): When Will a Genetic Algorithm Outperform Hill Climbing?
	\item (see for seminal refs, compare classical HC algos against ga, devise RMHC): Relative building-block fitness and the building-block hypothesis
	\item related to parallel hill climbing (perhaps a heuristic or variant for discussion)
	\item (theory???) On the convergence of generalized hill climbing algorithms, and: A class of convergent generalized hill climbing algorithms
	\item Probabilistic Hill-Climbing Algorithms ?
	\item look at AI:AMA page 124!
	\item stochastic hill-climbing as a baseline method for evaluating genetic algorithms (tech report, 1994)
	\item (might have some good refs?) A study on hill climbing algorithms for neural network training
	\item (might have refs) Planning by Guided Hill-Climbing
	\item (maybe useful) how easy is local search? (1988)
	\item (studies random hill climber) an empirical study of bit vector function optimization 
	\item holland - classical works mention the need for mutation operator for hill climbing.
	\item (potentially seminal for RMHC): Prototype and feature selection by sampling and random mutation hill climbing algorithms
	\item (classical forerunner to RMHC) Bit-climbing, representational bias, and test suite design 
	\item (forerunner to RMHC?) Evolution in time and space
	\item (1+1 ES) - continuos domain, discuss on ES description
\end{itemize}

% 
% Primary Sources
% 
\subsection{Primary Sources}
todo

% 
% Learn More
% 
\subsection{Learn More}
% historical reviews
todo


% 
% Conclusions: What the reader or what thre author learned by completing this this report.
% 
\section{Conclusions}
\label{sec:conclusions}
% report
todo

% making use of algorithms
a practitioner will want to apply a technique to a problem - how do they do that?
write an advanced section on this, or explain how in the intro material
search for general class of problem and technique, or determine it. is it discrete or continuous, what problems is it like - model it like those


% 
% Contribute
% 
\section{Contribute}
\label{sec:contribute}
% simple
Found a typo in the content or a bug in the source code? 
% advanced 
Are you an expert in this technique and know some facts that could improve the algorithm description for all?
% incentive
Do you want to get that warm feeling from contributing to an open source project? 
Do you want to see your name as an acknowledgment in print?

%  ideal
Two pillars of this effort are i) that the best domain experts are people outside of the project, and ii) that this work is wrong by default. 
% advice
Please help to make this work less wrong by emailing the author `\myreportauthor' at \url{\myreportemail} or visit the project website at \url{\myreportwebsite}.

% bibliography
\bibliographystyle{plain}
\bibliography{../bibtex}

\end{document}
% EOF