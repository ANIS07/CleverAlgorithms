% Stochastic Hill Climbing

% The Clever Algorithms Project: http://www.CleverAlgorithms.com
% (c) Copyright 2010 Jason Brownlee. Some Rights Reserved. 
% This work is licensed under a Creative Commons Attribution-Noncommercial-Share Alike 2.5 Australia License.

\documentclass[a4paper, 11pt]{article}
\usepackage{tabularx}
\usepackage{booktabs}
\usepackage{url}
\usepackage[pdftex,breaklinks=true,colorlinks=true,urlcolor=blue,linkcolor=blue,citecolor=blue,]{hyperref}
\usepackage{geometry}
\usepackage[ruled, linesnumbered]{../algorithm2e}
\usepackage{listings} 
\usepackage{textcomp}
\ifx\pdfoutput\@undefined\usepackage[usenames,dvips]{color}
\else\usepackage[usenames,dvipsnames]{color}
\lstset{basicstyle=\footnotesize\ttfamily,numbers=left,numberstyle=\tiny,frame=single,columns=flexible,upquote=true,showstringspaces=false,tabsize=2,captionpos=b,breaklines=true,breakatwhitespace=true,keywordstyle=\color{blue},stringstyle=\color{ForestGreen}}
\geometry{verbose,a4paper,tmargin=25mm,bmargin=25mm,lmargin=25mm,rmargin=25mm}

% Dear template user: fill these in
\newcommand{\myreporttitle}{Stochastic Hill Climbing}
\newcommand{\myreportauthor}{Jason Brownlee}
\newcommand{\myreportemail}{jasonb@CleverAlgorithms.com}
\newcommand{\myreportwebsite}{http://www.CleverAlgorithms.com}
\newcommand{\myreportproject}{The Clever Algorithms Project\\\url{\myreportwebsite}}
\newcommand{\myreportdate}{20100202}
\newcommand{\myreportversion}{1}
\newcommand{\myreportlicense}{\copyright\ Copyright 2010 Jason Brownlee. Some Rights Reserved. This work is licensed under a Creative Commons Attribution-Noncommercial-Share Alike 2.5 Australia License.}

% leave this alone, it's templated baby!
\title{{\myreporttitle}\footnote{\myreportlicense}}
\author{\myreportauthor\\{\myreportemail}\\\small\myreportproject}
\date{\today\\{\small{Technical Report: CA-TR-{\myreportdate}-\myreportversion}}}
\begin{document}
\maketitle

% write a summary sentence for each major section
\section*{Abstract} 
% project
The Clever Algorithms project aims to describe a large number of Artificial Intelligence algorithms in a complete, consistent, and centralized manner, to improve their general accessibility. 
% template
The project makes use of a standardized algorithm description template that uses well-defined topics that motivate the collection of specific and useful information about each algorithm described.
% report
This report describes the Stochastic Hill Climbing algorithm using the standardized template.

\begin{description}
	\item[Keywords:] {\small\texttt{Clever, Algorithms, Description, Optimization, Stochastic, Hill, Climbing}}
\end{description} 

\section{Introduction} 
\label{sec:intro}
% project
The Clever Algorithms project aims to describe a large number of algorithms from the fields of Computational Intelligence, Biologically Inspired Computation, and Metaheuristics in a complete, consistent and centralized manner \cite{Brownlee2010}.
% description
The project requires all algorithms to be described using a standardized template that includes a fixed number of sections, each of which is motivated by the presentation of specific information about the technique \cite{Brownlee2010a}.
% this report
This report describes the Stochastic Hill Climbing algorithm using the standardized template.
% conclusions
Section~\ref{sec:conclusions} highlights some additional and related techniques discovered during the research of this work, and considers the problem of a practitioner requiring guidance to apply a given technique to a problem different to the problem instances presented in the Clever Algorithm project code examples.

% Name
% The algorithm name defines the canonical name used to refer to the technique, in addition to common aliases, abbreviations, and acronyms. The name is used in terms of the heading and sub-headings of an algorithm description.
\section{Name} 
\label{sec:name}
% What is the canonical name and common aliases for a technique?
% What are the common abbreviations and acronyms for a technique?
% The heading and alternate headings for the algorithm description.
Stochastic Hill Climbing, SHC, Random Hill Climbing, RHC, Random Mutation Hill Climbing, RMHC 

% Taxonomy: Lineage and locality
% The algorithm taxonomy defines where a techniques fits into the field, both the specific subfields of Computational Intelligence and Biologically Inspired Computation as well as the broader field of Artificial Intelligence. The taxonomy also provides a context for determining the relation- ships between algorithms. The taxonomy may be described in terms of a series of relationship statements or pictorially as a venn diagram or a graph with hierarchical structure.
\section{Taxonomy}
\label{sec:taxonomy}
% general
The Stochastic Hill Climbing algorithm is a Stochastic Optimization algorithm and is a Local Optimization algorithm (contrasted to Global Optimization). It is a direct search technique, as it does not require derivatives of the search space.
% related
Stochastic Hill Climbing is an extension of deterministic hill climbing algorithms such as Simple Hill Climbing (first-best neighbor), Steepest-Ascent Hill Climbing (best neighbor), and a parent of approaches such as Parallel Hill Climbing and Random-Restart Hill Climbing.

% Strategy: Problem solving plan
% The strategy is an abstract description of the computational model. The strategy describes the information processing actions a technique shall take in order to achieve an objective. The strategy provides a logical separation between a computational realization (procedure) and a analogous system (metaphor). A given problem solving strategy may be realized as one of a number specific algorithms or problem solving systems. The strategy description is textual using information processing and algorithmic terminology.
\section{Strategy}
\label{sec:strategy}
% What is the information processing objective of a technique?
% What is a techniques plan of action?
The strategy of the Stochastic Hill Climbing algorithm is iterate the process of randomly selecting a neighbor for a candidate solution and only accept it if it results in an improvement.
% why
The strategy was proposed to address the limitations of deterministic hill climbing techniques that were likely to get stuck in local optima due to their greedy acceptance of neighboring moves. 

% Procedure: Abstract computation
% The algorithmic procedure summarizes the specifics of realizing a strategy as a systemized and parameterized computation. It outlines how the algorithm is organized in terms of the data structures and representations. The procedure may be described in terms of software engineering and computer science artifacts such as pseudo code, design diagrams, and relevant mathematical equations.
\section{Procedure}
\label{sec:procedure}
% What is the computational recipe for a technique?
% What are the data structures and representations used in a technique?
Algorithm~\ref{alg:stochastic_hill_climbing} provides a pseudo-code listing of the Stochastic Hill Climbing algorithm for minimizing a cost function, specifically the Random Mutation Hill Climbing algorithm described by Forrest and Mitchell applied to a maximization optimization problem \cite{Forrest1993}.

\begin{algorithm}[htp]
	\SetLine
	% data
	\SetKwData{NumIterations}{$Iter_{max}$}
	\SetKwData{ProblemSize}{ProblemSize}
	\SetKwData{Current}{Current}
	\SetKwData{Candidate}{Candidate}
	% functions
	\SetKwFunction{Cost}{Cost}
	\SetKwFunction{RandomSolution}{RandomSolution}
	\SetKwFunction{RandomNeighbor}{RandomNeighbor}
  	% I/O
	\KwIn{\NumIterations, \ProblemSize}
	\KwOut{\Current}
  	% Algorithm
	% init
	\Current $\leftarrow$ \RandomSolution{\ProblemSize}\;
	% main loop
	\ForEach{$iter_i \in$ \NumIterations} {
		% small step
		$\Candidate \leftarrow$ \RandomNeighbor{\Current}\;		
		\If{\Cost{\Candidate} $\geq$ \Cost{\Current}} {
			\Current $\leftarrow$ \Candidate\;
		}
	}
	\Return{\Current}\;
	% caption
	\caption{Pseudo Code Listing for the Stochastic Hill Climbing algorithm.}
	\label{alg:stochastic_hill_climbing}
\end{algorithm}

% Heuristics: Usage guidelines
% The heuristics element describe the commonsense, best practice, and demonstrated rules for applying and configuring a parameterized algorithm. The heuristics relate to the technical details of the techniques procedure and data structures for general classes of application (neither specific implementations not specific problem instances). The heuristics are described textually, such as a series of guidelines in a bullet-point structure.
\section{Heuristics}
\label{sec:heuristics}
% What are the suggested configurations for a technique?
% What are the guidelines for the application of a technique to a problem instance?
\begin{itemize}
	\item Stochastic Hill Climbing was designed to be used in discrete domains with explicit neighbors such as combinatorial optimization (compared to continuous function optimization). 
	\item The algorithm's strategy may be applied to continuous domains by making use of a step-size to define candidate-solution neighbors (such as Localized Random Search and Fixed Step-Size Random Search).
	\item Stochastic Hill Climbing is a local search technique (compared to global search) and may be used to refine a result after the execution of a global search algorithm.
	\item Even though the technique uses a stochastic process, it can still get stuck in local optima.
	\item Neighbors with better or equal cost should be accepted, allowing the technique to navigate across plateaus in the response surface.
	\item The algorithm can be restarted and repeated a number of times after it converges to provide an improved result (called Multiple Restart Hill Climbing).
	\item The procedure can be applied to multiple candidate solutions concurrently, allowing multiple algorithm runs to be performed at the same time (called Parallel Hill Climbing).
\end{itemize}

% The code description provides a minimal but functional version of the technique implemented with a programming language. The code description must be able to be typed into an appropriate computer, compiled or interpreted as need be, and provide a working execution of the technique. The technique implementation also includes a minimal problem instance to which it is applied, and both the problem and algorithm implementations are complete enough to demonstrate the techniques procedure. The description is presented as a programming source code listing.
\section{Code Listing}
\label{sec:code}
% How is a technique implemented as an executable program?
% How is a technique applied to a concrete problem instance?
Listing~\ref{stochastic_hill_climbing} provides an example of the Stochastic Hill Climbing algorithm implemented in the Ruby Programming Language, specifically the Random Mutation Hill Climbing algorithm described by Forrest and Mitchell \cite{Forrest1993}.
% problem
The algorithm is executed for a fixed number of iterations and is applied to a binary string optimization problem called `One Max'. The objective of this maximization problem is to prepare a string of all `1' bits, where the cost function only reports the number of bits in a given string.
% the listing
\lstinputlisting[firstline=7,language=ruby,caption=Stochastic Hill Climbing algorithm in the Ruby Programming Language, label=stochastic_hill_climbing]{../../src/algorithms/stochastic/stochastic_hill_climbing.rb}


% References: Deeper understanding
% The references element description includes a listing of both primary sources of information about the technique as well as useful introductory sources for novices to gain a deeper understanding of the theory and application of the technique. The description consists of hand-selected reference material including books, peer reviewed conference papers, journal articles, and potentially websites. A bullet-pointed structure is suggested.
\section{References}
\label{sec:references}
% What are the primary sources for a technique?
% What are the suggested reference sources for learning more about a technique?

% 
% Primary Sources
% 
\subsection{Primary Sources}
% RMHC
Perhaps the most popular implementation of the Stochastic Hill Climbing algorithm is by Forrest and Mitchell, who proposed the Random Mutation Hill Climbing (RMHC) algorithm (with communication from Richard Palmer) in a study that investigated the behavior of the genetic algorithm on a deceptive class of (discrete) bit-string optimization problem called `royal road' functions \cite{Forrest1993}. The RMHC was compared to two other hill climbing algorithms in addition to the genetic algorithm, specifically: the Steepest-Ascent Hill Climber, and the Next-Ascent Hill Climber. This study was then followed up by Mitchell and Holland \cite{Mitchell1993}

% early approach
Jules and Wattenberg were also early to consider stochastic hill climbing as an approach to compare to the genetic algorithm \cite{Juels1994}.
% prototype vectors
Skalak applied the RMHC algorithm to a single long bit-string that represented a number of prototype vectors for use in classification \cite{Skalak1994}.

% 
% Learn More
% 
\subsection{Learn More}
% historical reviews
% bit climbers
The Stochastic Hill Climbing algorithm is related to the genetic algorithm without crossover. Simplified version's of the approach are investigated for bit-string based optimization problems with the population size of the genetic algorithm reduced to one. The general technique has been investigated under the names Iterated Hillclimbing \cite{Muhlenbein1991}, ES(1+1,m,hc) \cite{Muhlenbein1992}, Random Bit Climber \cite{Davis1991}, and (1+1)-Genetic Algorithm \cite{Back1993}. This main difference between RMHC and ES(1+1) is that the latter uses a fixed probability of a mutation for each discrete element of a solution (meaning the neighborhood size is probabilistic), whereas RMHC will only stochastically modify one element.

% 
% Conclusions: What the reader or what thre author learned by completing this this report.
% 
\section{Conclusions}
\label{sec:conclusions}
% report
This report described the Stochastic Hill Climbing algorithm as a strategy for addressing some of the limitations of deterministic hill climbing methods.
% related techniques
Some additional technique names discovered during the investigation of this work include: \emph{Dynamic Hill Climbing} \cite{Yuret1993}, the investigation of a theoretical abstraction of hill climbing algorithms called \emph{Generalized Hill Climbing} \cite{Johnson2002, Sullivan2001}, \emph{Probabilistic Hill Climbing} \cite{Romeo1984} and a second probabilistic hill climbing algorithm called Probably Approximate Locally Optimal (PALO) \cite{Cohen1994}.

% making use of algorithms
A concern highlighted in the research for this report was the guidance needed for a practitioner to apply a technique to a specific problem that differences from the class of problem presented. Guidance may be in the form of a process where articles and papers on, or related to the selected problem are identified, or problem transformations used to map the instance to something that can be addressed by a the chosen technique. This may be included as an advanced topic in the Clever Algorithms project or included in the introductory material.

% 
% Contribute
% 
\section{Contribute}
\label{sec:contribute}
% simple
Found a typo in the content or a bug in the source code? 
% advanced 
Are you an expert in this technique and know some facts that could improve the algorithm description for all?
% incentive
Do you want to get that warm feeling from contributing to an open source project? 
Do you want to see your name as an acknowledgment in print?

%  ideal
Two pillars of this effort are i) that the best domain experts are people outside of the project, and ii) that this work is wrong by default. 
% advice
Please help to make this work less wrong by emailing the author `\myreportauthor' at \url{\myreportemail} or visit the project website at \url{\myreportwebsite}.

% bibliography
\bibliographystyle{plain}
\bibliography{../bibtex}

\end{document}
% EOF