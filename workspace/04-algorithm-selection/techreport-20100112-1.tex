% The Clever Algorithms Project: Algorithm Selection

% The Clever Algorithms Project: http://www.CleverAlgorithms.com
% (c) Copyright 2010 Jason Brownlee. All Rights Reserved. 
% This work is licensed under a Creative Commons Attribution-Noncommercial-Share Alike 2.5 Australia License.

\documentclass[a4paper, 11pt]{article}
\usepackage{tabularx}
\usepackage{booktabs}
\usepackage{url}
\usepackage[pdftex,breaklinks=true,colorlinks=true,urlcolor=blue,linkcolor=blue,citecolor=blue,]{hyperref}
\usepackage{geometry}
\geometry{verbose,a4paper,tmargin=25mm,bmargin=25mm,lmargin=25mm,rmargin=25mm}

% Dear template user: fill these in
\newcommand{\myreporttitle}{The Clever Algorithms Project}
\newcommand{\myreportsubtitle}{Algorithm Selection}
\newcommand{\myreportauthor}{Jason Brownlee}
\newcommand{\myreportemail}{jasonb@CleverAlgorithms.com}
\newcommand{\myreportproject}{The Clever Algorithms Project\\\url{http://www.CleverAlgorithms.com}}
\newcommand{\myreportdate}{20100112}
\newcommand{\myreportversion}{1}
\newcommand{\myreportlicense}{\copyright\ Copyright 2010 Jason Brownlee. All Rights Reserved. This work is licensed under a Creative Commons Attribution-Noncommercial-Share Alike 2.5 Australia License.}

% leave this alone, it's templated baby!
\title{{\myreporttitle}: {\myreportsubtitle}\footnote{\myreportlicense}}
\author{\myreportauthor\\{\myreportemail}\\\small\myreportproject}
\date{\today\\{\small{Technical Report: CA-TR-{\myreportdate}-\myreportversion}}}
\begin{document}
\maketitle

% write a summary sentence for each major section
\section*{Abstract} 
This is the abstract. Consider writing a one sentence summary of each major section in the report.

\begin{description}
	\item[Keywords:] {\small\texttt{Clever, Algorithms, Algorithm, Selection, Methodology}}
\end{description} 

% summarise the document breakdown with cross references
\section{Introduction}
\label{sec:introduction}
% project
The Clever Algorithms project aims to describe a large number of algorithms from the field of Artificial Intelligence in a complete, consistent, and centralized way to improve the accessibility of the methods \cite{Brownlee2010}. 
% report
This report\ldots
% breakdown
Section~\ref{sec:methodology}\ldots
Section~\ref{sec:results}\ldots
Section~\ref{sec:analysis}\ldots
Section~\ref{sec:selection}\ldots
Section~\ref{sec:conclusions}\ldots

\section{Methodology}
\label{sec:methodology}
This section describes the methodology for the selection of algorithms for inclusion in the Clever Algorithms project.

This methodology and its results are based on a blog post by Jason Brownlee on August 2nd 2009 entitled ``What is a good optimization algorithm? A data-driven method for algorithm selection'' located at: \url{http://www.neverreadpassively.com/2009/08/what-is-good-optimization-algorithm.html}.

\subsection{Algorithm List}
prepare an algorithm list. include name, parent field
draw from lots of fields, such as computational intelligence algorithms, metaheuristics, biologically inspired algorithms, natural algorithms, biomimetic algorithms
draw from lots of sources: books, websites, papers, articles, friends

\subsection{Rank Algorithms}


\subsection{Select Algorithms}
order by rank, filter for interestingness
organize by field or sub-field or some commonality


\section{Results}
\label{sec:results}
This section summarizes the results of assessing algorithms for inclusion in the Clever Algorithms project.

Top overall?
Top for each field?
list all results? good for completeness 

\section{Analysis}
\label{sec:analysis}
This section analyses the results of the assessed algorithms for inclusion in the Clever Algorithms project.

common base names boosting some algorithms, like the ga


\section{Selected Algorithms}
\label{sec:selection}
This section provides a listing of 50 algorithms selected for description in the Clever Algorithms Project. The presentation of this list of algorithms is partitioned into sub-domains that are expected to represent Chapters in the final work. Each algorithm is listed with at least one primary or seminal source to verify the existence of the approach.

not final, just a first best guess based on available information
composition must be large, diverse, interesting. 
selection based on a filtered version of the ranked listings

\subsection{Stochastic Algorithms}

\subsection{Evolutionary Algorithms}

\subsection{Swarm Algorithms}

\subsection{Immune Algorithms}

\subsection{Probabilistic Algorithms}

\subsection{Physical Algorithms}

\subsection{Neural Algorithms}
do we want this in the book?

% summarise the document message and areas for future consideration
\section{Conclusions}
\label{sec:conclusions}
This is the conclusion. Consider summarizing the message of the document once again, and highlighting areas for future consideration.

The project now has preliminary algorithm chapters and sections.
this does not have to be maintained, it is just a first attempt at outlining the scope of the project


% bibliography
\bibliographystyle{plain}
\bibliography{../bibtex}

\end{document}
% EOF