% Programming Language Selection for Introducing Unconventional Optimization Algorithms

% The Clever Algorithms Project: http://www.CleverAlgorithms.com
% (c) Copyright 2010 Jason Brownlee. Some Rights Reserved. 
% This work is licensed under a Creative Commons Attribution-Noncommercial-Share Alike 2.5 Australia License.

\documentclass[a4paper, 11pt]{article}
\usepackage{tabularx}
\usepackage{booktabs}
\usepackage{url}
\usepackage[pdftex,breaklinks=true,colorlinks=true,urlcolor=blue,linkcolor=blue,citecolor=blue,]{hyperref}
\usepackage{geometry}
\geometry{verbose,a4paper,tmargin=25mm,bmargin=25mm,lmargin=25mm,rmargin=25mm}

% Dear template user: fill these in
\newcommand{\myreporttitle}{Programming Language Selection for Optimization Algorithms}
% \newcommand{\myreportsubtitle}{}
\newcommand{\myreportauthor}{Jason Brownlee}
\newcommand{\myreportemail}{jasonb@CleverAlgorithms.com}
\newcommand{\myreportproject}{The Clever Algorithms Project\\\url{http://www.CleverAlgorithms.com}}
\newcommand{\myreportdate}{20100122}
\newcommand{\myreportfulldate}{January 22, 2010}
\newcommand{\myreportversion}{1}
\newcommand{\myreportlicense}{\copyright\ Copyright 2010 Jason Brownlee. Some Rights Reserved. This work is licensed under a Creative Commons Attribution-Noncommercial-Share Alike 2.5 Australia License.}

% leave this alone, it's templated baby!
% \title{{\myreporttitle}: {\myreportsubtitle}\footnote{\myreportlicense}}
\title{{\myreporttitle}\footnote{\myreportlicense}}
\author{\myreportauthor\\{\myreportemail}\\\small\myreportproject}
\date{\myreportfulldate\\{\small{Technical Report: CA-TR-{\myreportdate}-\myreportversion}}}
\begin{document}
\maketitle

% write a summary sentence for each major section
\section*{Abstract} 
todo

\begin{description}
	\item[Keywords:] {\small\texttt{Clever, Algorithms, Language, Selection, Optimization}}
\end{description} 

% summarise the document breakdown with cross references
\section{Introduction}
\label{sec:introduction}
% project
The Clever Algorithms project aims to describe a large number of optimization algorithms from the fields of Computational Intelligence, Natural Computation, and Metaheuristics in a complete, consistent, and centralized manner \cite{Brownlee2010}
% overview 
The standardized description of algorithms in the project requires an example implementation of each technique in a tutorial and potentially a code listing format \cite{Brownlee2010a}.
% report
This report addresses the problem of which language to use for describing algorithm implementation examples, and recommends a specific language to be used.

% breakdown
Section~\ref{sec:language_selection} considers the problem of language selection focusing on the types of languages that could be used and examples, as well as a consideration of attributes that might be important when selecting a language.
Section~\ref{sec:methodology} proposes a methodology for evaluating and comparing languages for use in implementation examples for the standardized algorithm description, specifying an algorithm to use in the comparison, a set of languages to be considered, and the specific properties that will be measured and compared.
Four languages are considered: Javascript in Section~\ref{sec:javascript}, Lua in Section~\ref{sec:lua}, Python in Section~\ref{sec:python} and Ruby in Section~\ref{sec:ruby}.
The results of the comparison are reviewed in Section~\ref{sec:analysis} highlighting areas for improvement in the language selection methodology and in the execution of the comparison. 
Finally, the findings are discussed in Section~\ref{sec:findings} and the Ruby Programming Language is selected for use in the Clever Algorithms project.

% 
% Language Selection
% 
\section{Language Selection}
\label{sec:language_selection}
what languages and why?

% 
% Languages
%
\subsection{Languages}
implement an algorithm in js, lua, python, and ruby and assess. best you can do really.
java? c? cpp? too bloated?
lisp? too hard?


% 
% Languages Attributes
%
\subsection{Languages Attributes}
what properties are important?
random numbers, multi-paradigm, etc


debate on this topic:
\url{http://www.reddit.com/r/gamedev/comments/alik1/ask_gamedev_please_help_me_choose_languages_for/}


% 
% Methodology
%
\section{Methodology}
\label{sec:methodology}

implement one algorithm in a range of scripting languages and compare

% 
%  Algorithm
% 
\subsection{Algorithm}
a genetic algorithm, non trivial, small, easy to compare

% 
% Implementation
% 
\subsection{Implementation}
procedural, 4 dynamic languages
js, lua, python, and ruby


language selection? google search results as a rough popularity measure?
top 4 languages?


% 
% Comparison
% 
\subsection{Comparison}
compare on conciseness, compactness, communication qualities - accessibility, etc. readability

rank or score on a set of properties, subjective, but a starting point


% 
% Javascript
% 
\section{Javascript}
\label{sec:javascript}
todo


% 
% Lua
% 
\section{Lua}
\label{sec:lua}
todo


% 
% Python
% 
\section{Python}
\label{sec:python}
todo


% 
% Ruby
% 
\section{Ruby}
\label{sec:ruby}
todo


% 
% Analysis
% 
\section{Analysis}
\label{sec:analysis}

how could the methodology be improved?
how could the execution be improved?


% 
% Findings
% 
\section{Findings}
\label{sec:findings}
% report

% findings

% future


% bibliography
\bibliographystyle{plain}
\bibliography{../bibtex}

\end{document}
% EOF