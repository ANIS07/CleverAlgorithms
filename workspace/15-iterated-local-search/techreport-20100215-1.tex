% Iterated Local Search

% The Clever Algorithms Project: http://www.CleverAlgorithms.com
% (c) Copyright 2010 Jason Brownlee. Some Rights Reserved. 
% This work is licensed under a Creative Commons Attribution-Noncommercial-Share Alike 2.5 Australia License.

\documentclass[a4paper, 11pt]{article}
\usepackage{tabularx}
\usepackage{booktabs}
\usepackage{url}
\usepackage[pdftex,breaklinks=true,colorlinks=true,urlcolor=blue,linkcolor=blue,citecolor=blue,]{hyperref}
\usepackage{geometry}
\usepackage[ruled, linesnumbered]{../algorithm2e}
\usepackage{listings} 
\usepackage{textcomp}
\ifx\pdfoutput\@undefined\usepackage[usenames,dvips]{color}
\else\usepackage[usenames,dvipsnames]{color}
\lstset{basicstyle=\footnotesize\ttfamily,numbers=left,numberstyle=\tiny,frame=single,columns=flexible,upquote=true,showstringspaces=false,tabsize=2,captionpos=b,breaklines=true,breakatwhitespace=true,keywordstyle=\color{blue},stringstyle=\color{ForestGreen}}
\geometry{verbose,a4paper,tmargin=25mm,bmargin=25mm,lmargin=25mm,rmargin=25mm}

% Dear template user: fill these in
\newcommand{\myreporttitle}{Iterated Local Search}
\newcommand{\myreportauthor}{Jason Brownlee}
\newcommand{\myreportemail}{jasonb@CleverAlgorithms.com}
\newcommand{\myreportwebsite}{http://www.CleverAlgorithms.com}
\newcommand{\myreportproject}{The Clever Algorithms Project\\\url{\myreportwebsite}}
\newcommand{\myreportdate}{20100215}
\newcommand{\myreportversion}{1}
\newcommand{\myreportlicense}{\copyright\ Copyright 2010 Jason Brownlee. Some Rights Reserved. This work is licensed under a Creative Commons Attribution-Noncommercial-Share Alike 2.5 Australia License.}

% leave this alone, it's templated baby!
\title{{\myreporttitle}\footnote{\myreportlicense}}
\author{\myreportauthor\\{\myreportemail}\\\small\myreportproject}
\date{\today\\{\small{Technical Report: CA-TR-{\myreportdate}-\myreportversion}}}
\begin{document}
\maketitle

% write a summary sentence for each major section
\section*{Abstract} 
% project
The Clever Algorithms project aims to describe a large number of Artificial Intelligence algorithms in a complete, consistent, and centralized manner, to improve their general accessibility. 
% template
The project makes use of a standardized algorithm description template that uses well-defined topics that motivate the collection of specific and useful information about each algorithm described.
% report
This report describes the Iterative Local Search algorithm using the standardized template.

\begin{description}
	\item[Keywords:] {\small\texttt{Clever, Algorithms, Description, Optimization, Iterative, Local, Search}}
\end{description} 

\section{Introduction} 
\label{sec:intro}
% project
The Clever Algorithms project aims to describe a large number of algorithms from the fields of Computational Intelligence, Biologically Inspired Computation, and Metaheuristics in a complete, consistent and centralized manner \cite{Brownlee2010}.
% description
The project requires all algorithms to be described using a standardized template that includes a fixed number of sections, each of which is motivated by the presentation of specific information about the technique \cite{Brownlee2010a}.
% this report
This report describes the Iterative Local Search algorithm using the standardized template.

% Name
% The algorithm name defines the canonical name used to refer to the technique, in addition to common aliases, abbreviations, and acronyms. The name is used in terms of the heading and sub-headings of an algorithm description.
\section{Name} 
\label{sec:name}
% What is the canonical name and common aliases for a technique?
% What are the common abbreviations and acronyms for a technique?
% The heading and alternate headings for the algorithm description.
Iterative Local Search, ILS

% Taxonomy: Lineage and locality
% The algorithm taxonomy defines where a techniques fits into the field, both the specific subfields of Computational Intelligence and Biologically Inspired Computation as well as the broader field of Artificial Intelligence. The taxonomy also provides a context for determining the relation- ships between algorithms. The taxonomy may be described in terms of a series of relationship statements or pictorially as a venn diagram or a graph with hierarchical structure.
\section{Taxonomy}
\label{sec:taxonomy}
% To what fields of study does a technique belong?
Iterative Local Search is a Metaheuristic and a Global Optimization technique.
% What are the closely related approaches to a technique?
It is an extension of Mutli-Restart Search and may be considered a parent of many two-phase search approaches such as Greedy Randomized Adaptive Search Procedure and Variable Neighborhood Search.

% Strategy: Problem solving plan
% The strategy is an abstract description of the computational model. The strategy describes the information processing actions a technique shall take in order to achieve an objective. The strategy provides a logical separation between a computational realization (procedure) and a analogous system (metaphor). A given problem solving strategy may be realized as one of a number specific algorithms or problem solving systems. The strategy description is textual using information processing and algorithmic terminology.
\section{Strategy}
\label{sec:strategy}
% What is the information processing objective of a technique?
The objective of Iterative Local Search is to improve upon stochastic Mutli-Restart Search by sampling in the broader neighborhood of candidate solutions and using a Local Search technique to refine solutions to their local optima.
% What is a techniques plan of action?
Iterative Local Search explores a sequence of solutions created as perturbations of the current best solution, the result of which is refined using an embedded heuristic, most commonly a Local Search algorithm.

% Procedure: Abstract computation
% The algorithmic procedure summarizes the specifics of realizing a strategy as a systemized and parameterized computation. It outlines how the algorithm is organized in terms of the data structures and representations. The procedure may be described in terms of software engineering and computer science artifacts such as pseudo code, design diagrams, and relevant mathematical equations.
\section{Procedure}
\label{sec:procedure}
% What is the computational recipe for a technique?
% What are the data structures and representations used in a technique?
Algorithm~\ref{alg:iterative_local_search} provides a pseudo-code listing of the Iterative Local Search algorithm for minimizing a cost function.

\begin{algorithm}[ht]
	\SetLine
	% data
	\SetKwData{Best}{$S_{best}$}
	\SetKwData{Candidate}{$S_{candidate}$}
	\SetKwData{SearchHistory}{SearchHistory}
	% functions
	\SetKwFunction{Cost}{Cost}
	\SetKwFunction{Perturbation}{Perturbation}
	\SetKwFunction{StopCondition}{StopCondition}
	\SetKwFunction{ConstructInitialSolution}{ConstructInitialSolution}
  	\SetKwFunction{LocalSearch}{LocalSearch}
	\SetKwFunction{AcceptanceCriterion}{AcceptanceCriterion}
	% I/O
	\KwIn{}
	\KwOut{\Best}
  	% Algorithm
	\Best $\leftarrow$ \ConstructInitialSolution{}\;
	\Best $\leftarrow$ \LocalSearch{}\;
	\SearchHistory $\leftarrow$ \Best\;
	\While{$\neg$ \StopCondition{}} {
		% greedy randomized solution
		\Candidate $\leftarrow$ \Perturbation{\Best, \SearchHistory}\;
		% local search
		\Candidate $\leftarrow$ \LocalSearch{\Candidate}\;
		\SearchHistory $\leftarrow$ \Candidate\;
		% keep best
		\If{\AcceptanceCriterion{\Best, \Candidate, \SearchHistory}} {
			\Best $\leftarrow$ \Candidate\;
		}
	}
	\Return{\Best}\;
	% caption
	\caption{Pseudo Code for the Iterative Local Search algorithm.}
	\label{alg:iterative_local_search}
\end{algorithm}

% Heuristics: Usage guidelines
% The heuristics element describe the commonsense, best practice, and demonstrated rules for applying and configuring a parameterized algorithm. The heuristics relate to the technical details of the techniques procedure and data structures for general classes of application (neither specific implementations not specific problem instances). The heuristics are described textually, such as a series of guidelines in a bullet-point structure.
\section{Heuristics}
\label{sec:heuristics}
% What are the suggested configurations for a technique?
% What are the guidelines for the application of a technique to a problem instance?

\begin{itemize}
	\item the perturbation should not easily be undone
	\item the local search technique should be problem specific
	\item can use random initial solution, or a heuristically created solution, like NN
	\item perturbation - random move in a local neighborhood bigger than the one considered by the local search
	\item perturbations can be deterministic, stochastic, or probabilistic (adaptive), prefer the latter two
	\item can store as much or little history as possible. no history is the most common, more history may mean more improvement in the search (more information known about problem space). no history is a random walk in a larger neighborhood.
	\item simplest acceptance criteria is improvement in quality of the result, could use more advanced things like probabilistic acceptance, etc.
	\item perturbations that are too small makes the algorithm too greedy, too large and the search becomes more random - control of exploration/exploitation
	\item there is a balance between perturbation and acceptance criteria as to how much exploration/exploitation the algorithm performs
\end{itemize}

% The code description provides a minimal but functional version of the technique implemented with a programming language. The code description must be able to be typed into an appropriate computer, compiled or interpreted as need be, and provide a working execution of the technique. The technique implementation also includes a minimal problem instance to which it is applied, and both the problem and algorithm implementations are complete enough to demonstrate the techniques procedure. The description is presented as a programming source code listing.
\section{Code Listing}
\label{sec:code}
% How is a technique implemented as an executable program?
% How is a technique applied to a concrete problem instance?
Listing~\ref{iterated_local_search} provides an example of the Iterated Local Search algorithm implemented in the Ruby Programming Language. 
% problem
The algorithm is applied to the Berlin52 instance of the Traveling Salesman Problem (TSP), taken from the TSPLIB. The problem seeks a permutation of the order to visit cities (called a tour) that minimized the total distance traveled. The optimal tour distance for Berlin52 instance is 7542 units.

% algorithm

double bridge move is discussed as the perturbation, and 2-opt for the local search for TSP in \cite{Ramalhinho-Lourenco2003}
simple first-pass approach

double-bridge move (4-opt), break into 4 parts (a,b,c,d), the put it back together (a,d,c,b)

% the listing
\lstinputlisting[firstline=7,language=ruby,caption=Iterated Local Search algorithm in the Ruby Programming Language, label=iterated_local_search]{../../src/algorithms/stochastic/iterated_local_search.rb}


% References: Deeper understanding
% The references element description includes a listing of both primary sources of information about the technique as well as useful introductory sources for novices to gain a deeper understanding of the theory and application of the technique. The description consists of hand-selected reference material including books, peer reviewed conference papers, journal articles, and potentially websites. A bullet-pointed structure is suggested.
\section{References}
\label{sec:references}
% What are the primary sources for a technique?
% What are the suggested reference sources for learning more about a technique?

NOTES
\begin{itemize}
	\item proposed to be a the iterative application of an embedded heuristic to create a sequence of solutions, better than if one were to use repeated random trials with the same heuristic
	\item more intelligent starting points for local search than stochastically selected points
	\item \cite{Ramalhinho-Lourenco2003} suggest that the general approach has been used by many authors and in many specialized cases: iterated descent, large-step Markov chains, iterated Lin-Kernighan, chained local optimization
	\item \cite{Ramalhinho-Lourenco2003} propose constraints on what is an iterative local search: i) a single chain of solutions, the search for improved solutions occurs within a reduced space provided by a black-box heuristic.
	
\end{itemize}


% 
% Primary Sources
% 
\subsection{Primary Sources}
todo

\begin{itemize}
	\item early technical report \cite{Stuetzle1999} for QAP
	\item another early tech report \cite{Stutzle1998a} permutation flow shop problem
	\item (seminal?) introductory conference paper \cite{Lourenco2001} - review of related techniques, outlines the framework
	\item defined in his thesis \cite{Stutzle1998}
	\item early assessment of TSP for ILS \cite{Stutzle1999}
\end{itemize}

% 
% Learn More
% 
\subsection{Learn More}
% historical reviews
todo

\begin{itemize}
	\item great review article \cite{Ramalhinho-Lourenco2003} - framework for the general approach, considers examples of generating initial solutions, perturbation techniques, local search techniques, and acceptance criterion. examples with TSP, QAP, 
\end{itemize}

% 
% Conclusions: What the reader or what thre author learned by completing this this report.
% 
\section{Conclusions}
\label{sec:conclusions}
% report
todo

% additional 
need a discussion of exploration vs exploitation  aka diversification vs intensification

% also
random restart of a local search is a basic approach that many of these algorithms improve upon. should be in the book 

% 
% Contribute
% 
\section{Contribute}
\label{sec:contribute}
% simple
Found a typo in the content or a bug in the source code? 
% advanced 
Are you an expert in this technique and know some facts that could improve the algorithm description for all?
% incentive
Do you want to get that warm feeling from contributing to an open source project? 
Do you want to see your name as an acknowledgment in print?

%  ideal
Two pillars of this effort are i) that the best domain experts are people outside of the project, and ii) that this work is wrong by default. 
% advice
Please help to make this work less wrong by emailing the author `\myreportauthor' at \url{\myreportemail} or visit the project website at \url{\myreportwebsite}.

% bibliography
\bibliographystyle{plain}
\bibliography{../bibtex}

\end{document}
% EOF