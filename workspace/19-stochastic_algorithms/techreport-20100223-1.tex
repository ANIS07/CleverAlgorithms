% Stochastic Algorithms

% The Clever Algorithms Project: http://www.CleverAlgorithms.com
% (c) Copyright 2010 Jason Brownlee. Some Rights Reserved. 
% This work is licensed under a Creative Commons Attribution-Noncommercial-Share Alike 2.5 Australia License.

\documentclass[a4paper, 11pt]{article}
\usepackage{tabularx}
\usepackage{booktabs}
\usepackage{url}
\usepackage[pdftex,breaklinks=true,colorlinks=true,urlcolor=blue,linkcolor=blue,citecolor=blue,]{hyperref}
\usepackage{geometry}
\geometry{verbose,a4paper,tmargin=25mm,bmargin=25mm,lmargin=25mm,rmargin=25mm}

% Dear template user: fill these in
\newcommand{\myreporttitle}{Stochastic Algorithms}
\newcommand{\myreportauthor}{Jason Brownlee}
\newcommand{\myreportemail}{jasonb@CleverAlgorithms.com}
\newcommand{\myreportproject}{The Clever Algorithms Project\\\url{http://www.CleverAlgorithms.com}}
\newcommand{\myreportdate}{20100223}
\newcommand{\myreportfulldate}{February 23, 2010}
\newcommand{\myreportversion}{1}
\newcommand{\myreportlicense}{\copyright\ Copyright 2010 Jason Brownlee. Some Rights Reserved. This work is licensed under a Creative Commons Attribution-Noncommercial-Share Alike 2.5 Australia License.}

% leave this alone, it's templated baby!
\title{{\myreporttitle}\footnote{\myreportlicense}}
\author{\myreportauthor\\{\myreportemail}\\\small\myreportproject}
\date{\myreportfulldate\\{\small{Technical Report: CA-TR-{\myreportdate}-\myreportversion}}}
\begin{document}
\maketitle

% write a summary sentence for each major section
\section*{Abstract} 
% project
The Clever Algorithms project aims to describe a large number of Artificial Intelligence algorithms in a complete, consistent, and centralized manner, to improve their general accessibility. 
% template
The project makes use of a standardized algorithm description template that uses well-defined topics that motivate the collection of specific and useful information about each algorithm described.
% report
todo - provides a point of reflection on the preparation of the first 10 algorithms and the lessons learned and best practices advocated...


\begin{description}
	\item[Keywords:] {\small\texttt{Clever, Algorithms, Project, Stochastic, Optimization}}
\end{description} 

% summarise the document breakdown with cross references
\section{Introduction}
\label{sec:introduction}
% project
The Clever Algorithms project aims to describe a large number of algorithms from the fields of Computational Intelligence, Biologically Inspired Computation, and Metaheuristics in a complete, consistent and centralized manner \cite{Brownlee2010}.
% description
The project requires all algorithms to be described using a standardized template that includes a fixed number of sections, each of which is motivated by the presentation of specific information about the technique \cite{Brownlee2010a}.
% this report
This report provides an overview of the Stochastic Algorithms in the Clever Algorithms project. 
Section~\ref{sec:algorithms} provides background information and reviews common themes for general class of algorithm and summarizes those stochastic algorithms that have been described for the Clever Algorithms Project.
Section~\ref{sec:outcomes} summarizes the findings from preparing the stochastic algorithm descriptions, and lists a number of recommendations for both describing algorithms in the future and the modification of the existing algorithm descriptions for inclusion in the Clever Algorithms book and web page.

% 
% Described Stochastic Algorithms
% 
\section{Stochastic Algorithms}
\label{sec:algorithms}

% 
% Background
% 
\subsection{Background}
background topics / themes

what is common to all the reviewed algorithms?

is any of the content from this report relevant? should it be re-used? \cite{Brownlee2010n}

local search? global search? metaheuristic?
stochastic local search or a stochastic metaheuristic to manage an embedded local search
no inspiration or metaphor 


% 
% Described Algorithms
% 
\subsection{Described Algorithms}
% overview
This section summarizes the stochastic algorithms currently described for inclusion in the Clever Algorithms project. It is proposed that these algorithms will collectively comprise a chapter in the Clever Algorithms book. 

\begin{itemize}
	\item \textbf{Random Search}: \cite{Brownlee2010g}
	\item \textbf{Adaptive Random Search}: \cite{Brownlee2010h}
	\item \textbf{Stochastic Hill Climbing}: \cite{Brownlee2010i}
	\item \textbf{Guided Local Search}: \cite{Brownlee2010j}
	\item \textbf{Variable Neighborhood Search}: \cite{Brownlee2010e}
	\item \textbf{Greedy Randomized Adaptive Search Procedure}: \cite{Brownlee2010d}
	\item \textbf{Iterated Local Search}: \cite{Brownlee2010k}
	\item \textbf{Tabu Search}: \cite{Brownlee2010f}
	\item \textbf{Scatter Search}: \cite{Brownlee2010l}
	\item \textbf{Reactive Tabu Search}: \cite{Brownlee2010m}
\end{itemize}

% 
% Outcomes
% 
\section{Outcomes}
\label{sec:outcomes}

% 
% Findings
% 
\subsection{Findings}
% overview
This section summarizes the interesting and relevant observations made while preparing the first ten algorithm descriptions for the Clever Algorithms project. Some of these observations also make specific suggestions for the project. 

\begin{itemize}
	\item \textbf{Change to Description Template}: The standardized algorithm description template \cite{Brownlee2010a} was varied with the description of the first algorithm in the project (Random Search \cite{Brownlee2010g}). The \emph{Tutorial} element was replaced with a \emph{Code Listing} that provides an implementation in the Ruby Programming Language and a terse summary of the implementation. 	
	\item \textbf{Practical Focus}: The Clever Algorithms project takes a practical focus to algorithms, specifically focusing on realization and implementation \cite{Brownlee2010g}. This chosen approach explicitly excludes formal and mathematical descriptions of algorithms, a fact that should be emphasized in the introductory material for the project. 
	\item \textbf{Bounded Claims}: The project aims to describe algorithms in a complete, consistent and centralized manner. These claims of completeness and consistency are bounded: the algorithm descriptions are \emph{complete enough} to cover the proposed standardized algorithm template and consistent enough in that all descriptions in the project conform to the proposed template \cite{Brownlee2010g}. The algorithm descriptions in the clever algorithms project are neither absolutely complete nor absolutely consistent, and the specifically chosen bounds on these aims should be emphasized in the introductory material for the project.
	\item \textbf{Local versus Global Search}: There is a common taxonomy that separates algorithms into local (or neighborhood searching) techniques and global techniques. A related subject is the abstract description of the structure of the response surface of search spaces, including topics such as hills, valleys, and modality. These subjects are required prior knowledge and context for understanding the stochastic algorithms described as well as the remaining algorithms to be described in the Clever Algorithms project \cite{Brownlee2010h}. It is suggested that these subjects be described in the introductory material for the project.
	\item \textbf{Practical Guidance}: The standardized algorithm descriptions provide a practical implementation of each algorithm applied to a specific problem instance, although the project does not provide information as to how a given algorithm description (abstract or concrete) may be adapted to a practitioners specific problem \cite{Brownlee2010i}. A guide and a set of heuristics (for example, see \cite{Gendreau2003}) for generally researching a technique for a given problem as well as adapting a given algorithm description in the Clever Algorithms project to a practitioners problem should be described in the introductory or advanced topics material for the project.
	\item \textbf{Intensification versus Diversification}: There is a common taxonomy for search procedures that considers the exploration (diversification) of a given technique and the exploitation (intensification) of a given technique. This taxonomy is required prior knowledge and context for understanding the stochastic algorithms described as well as the remaining algorithms to be described in the Clever Algorithms project \cite{Brownlee2010k}. It is suggested that these subjects be described in the introductory material for the project.
	\item \textbf{Value of Template Sections}: The standardized algorithm template contains hand-chosen elements for a range of target audiences \cite{Brownlee2010a}. The value of each element to each or all target audiences should be made clear in the introductory material for the Clever Algorithms project \cite{Brownlee2010k}. For example, the primary sources can be used directly to cite a technique in a paper, and the code listing may be directly adapted for a given problem instance.
	\item \textbf{Change of Selected Algorithms}: A collection of 10 stochastic algorithms were previously identified and chosen to be described in the Clever Algorithms project \cite{Brownlee2010b}. One algorithms on that list were removed (Reactive search optimization) and replaced with an alternative (Iterated Local Search). Also, the `Hill Climbing Search' algorithm was renamed to `Stochastic Hill Climbing' to align the technique with the theme of stochastic optimization algorithms.
	\item \textbf{Additional Algorithm Name}: Many algorithm names were discovered in the preparation of the technical reports. The total number of stochastic algorithms listed in the original algorithm name assessment was 23. This number has ballooned to 52 at the time of writing.
\end{itemize}


% 
% Recommendations
% 
\subsection{Recommendations}
% overview
This section lists recommendations for algorithm descriptions to be writing for the Clever Algorithms project in the future and related concerns. 

\begin{itemize}
	\item \textbf{Function Name Consistency}: The function names used in the \emph{Pseudo Code} listing and the \emph{Code Listing} parts of the description should match or be as close to matching as possible \cite{Brownlee2010g}. This will aid in the transferability of understanding from the abstract presentation of a technique to its concrete realization.
	\item \textbf{Algorithm Research Due Diligence}: A minimum number of sources must be consulted to ensure that sufficient coverage of a given algorithm is achieved. This involves searching for a given algorithm name across a specific set of web based sources. A set of seven sources were identified and suggested as the minimum due diligence while researching a given algorithm name for description in the clever algorithms project \cite{Brownlee2010h}. This list includes: Google Web, Google Books, Google SCholar, IEEE Explore, Springer Link, ACM Digital Library, and Scirus.
	\item \textbf{Consult Experts for Difficult Algorithms}: Some algorithms are more difficult to understand that others, requiring extended effort locating alternate algorithm descriptions and deciphering seminal descriptions.  This difficulty can be caused by excessive (or required) detail or complexity in the presentation of a technique and in some cases the poor choice of algorithm description style in seminal papers. Two algorithms that required additional effort to describe were the Scatter Search \cite{Brownlee2010l} and the Reactive Tabu Search \cite{Brownlee2010m}. These difficulties and associated risks may be mitigated by consulting algorithm experts and practitioners to verify understandings and implementations. Experts may be located by locating and contacting authors of papers and/or authors of libraries that implement the technique. 
	\item \textbf{Pseudo Code Nomenclature, Idioms, and Terms}: The nomenclature and idioms used in pseudo code across all algorithm descriptions should be listed in the introductory material for the project, and the specific terms used for a given algorithm description should be defined in that description \cite{Brownlee2010m}.		
	\item \textbf{Consistent Problem Instances}: All ten of the current algorithm descriptions use the same combinatorial optimization problem instance (TSP Berlin52) or continuous function optimization problem instance (basin function) in the code listings. If this trend continues, the small set of specific problem instances should be described in the introductory material for the project and referred to across each algorithm description.
	\item \textbf{Code Reviews}: The code listings for each algorithm description were written from scratch in the Ruby Programming Language \cite{Brownlee2010o}. As such, the listings are expected to contain bugs and inconsistently conform to the adopted coding standard. To mitigate this risk, the code should be subjected to code reviews by language experts to illicit comments and suggestions for improvement. The listings may also be subjected to code reviews from technique experts, if available.
\end{itemize}

% 
% Conclusions
% 
\section{Conclusions}
\label{sec:conclusions}
% overview
This report provided a point of reflection for the first 10 algorithm descriptions prepared for the Clever Algorithms project. All described algorithms were assigned to the `Stochastic Algorithms' kingdom in the Clever Algorithms project. This report highlighted the commonality for all described stochastic algorithms and provided a definition suitable for use in the proposed book and website.
%  forward
The report provided a summary of the interesting findings identified in the preparation of the algorithm descriptions, and provided a set of recommendations for going forward both with future algorithm descriptions and with regard to migrating the existing descriptions into the Clever Algorithms book and website mediums.

% bibliography
\bibliographystyle{plain}
\bibliography{../bibtex}

\end{document}
% EOF