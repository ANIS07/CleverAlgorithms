% The Clever Algorithms Project: Inspiring Works

% The Clever Algorithms Project: http://www.CleverAlgorithms.com
% (c) Copyright 2010 Jason Brownlee. All Rights Reserved. 
% This work is licensed under a Creative Commons Attribution-Noncommercial-Share Alike 2.5 Australia License.

\documentclass[a4paper, 11pt]{article}
\usepackage{tabularx}
\usepackage{booktabs}
\usepackage{url}
\usepackage[pdftex,breaklinks=true,colorlinks=true,urlcolor=blue,linkcolor=blue,citecolor=blue,]{hyperref}
\usepackage{geometry}
\geometry{verbose,a4paper,tmargin=25mm,bmargin=25mm,lmargin=25mm,rmargin=25mm}

% Dear template user: fill these in
\newcommand{\myreporttitle}{The Clever Algorithms Project}
\newcommand{\myreportsubtitle}{Inspiring Works}
\newcommand{\myreportauthor}{Jason Brownlee}
\newcommand{\myreportemail}{jasonb@CleverAlgorithms.com}
\newcommand{\myreportproject}{The Clever Algorithms Project\\\url{http://www.CleverAlgorithms.com}}
\newcommand{\myreportdate}{20100110}
\newcommand{\myreportversion}{1}
\newcommand{\myreportlicense}{\copyright\ Copyright 2010 Jason Brownlee. All Rights Reserved. This work is licensed under a Creative Commons Attribution-Noncommercial-Share Alike 2.5 Australia License.}

% leave this alone, it's templated baby!
\title{{\myreporttitle}: {\myreportsubtitle}\footnote{\myreportlicense}}
\author{\myreportauthor\\{\myreportemail}\\\small\myreportproject}
\date{\today\\{\small{Technical Report: CA-TR-{\myreportdate}-\myreportversion}}}
\begin{document}
\maketitle

% write a summary sentence for each major section
\section*{Abstract} 
todo

\begin{description}
	\item[Keywords:] {\small\texttt{Clever, Algorithms, Inspiration, Works, Motivation}}
\end{description} 

% summarise the document breakdown with cross references
\section{Introduction}
\label{sec:introduction}
% message
The Clever Algorithms project seeks to describe a large number of algorithms from the fields of Computational Intelligence and Biologically Inspired Computation in a complete, consistent, and centralized manner \cite{Brownlee2010}. The project was not devised in isolation, it was inspired and influenced by a diverse collection of books (Section~\ref{sec:books}), software (Section~\ref{sec:software}), and websites (Section~\ref{sec:websites}). This technical report provides an overview of the works that influenced the inception and development of the Clever Algorithms project and distills the specific features from all the inspirations into a set of principles to help guide the project (Section~\ref{sec:findings}).

\section{Books}
\label{sec:books}
This section summarizes specific books that influenced the Clever Algorithms project.

\subsection{Society of Mind}
published by Marvin Minsky \cite{Minsky1988}

a set of discrete and semi-independent one-page essays that collectively describe a society model of the human conscious and thought.

discrete publication of parts of the work over an extended period of time, then collected together to form a whole, a complete theory


\subsection{Evolutionary Computation}

EC books 1 \cite{Baeck2000} and 2 \cite{Baeck2000a}

description of many ec techniques by many different authors, a compendium of techniques. a pre-cursor to these books was the handbook of ec \cite{Baeck1997} same format.


\subsection{Handbook of Exact String Matching Algorithms}
Christian Charras and Thierry Lecroq in 2004 \cite{Charras2004}, first book of a pair, second book is on the general field of string matching i believe.

can access a pdf and ps version of the book for free online \url{http://www-igm.univ-mlv.fr/~lecroq/livres.html}
author has many books on string matching algorithms, it's his area 

a model for what i want to achieve although in a different domain


\subsection{Clonal Selection as an Inspiration for Adaptive and Distributed Information Processing}

PhD dissertation by this author \cite{Brownlee2008}

discrete work product released here \url{http://www.it.swin.edu.au/personal/jbrownlee/}, contributing to the complete product 
description of many algorithmic techniques in a standard manner


\subsection{Pro Git}
book written by Scott Chacon and published by Apress \cite{Chacon2009}
has a book website \url{http://progit.org}
has not only the examples released in version control \url{http://github.com/progit/book-examples} but also the content was released (and developed) under version control \url{http://github.com/progit/progit} under a permissive creative commons license. 

putting the content in a publicly accessible version control system, specifically git, allows consumers of the books to not only highlight errata, but also suggest and provide patches to fix errata, contribute to subsequent revisions of the book, and more easily create new versions of the book.



\subsection{Global Optimization Algorithms: Theory and Applications}
another open book as a side project, ongoing development, not yet complete \cite{Weise2007}

uses the release date as the book revision, for example the revision I just looked at was `2009-06-26', and he himself is the publisher

available online \url{http://www.it-weise.de}

has a long preface describing the ongoing project. i don't want to version the book this way i think. it's there as an option though.


\subsection{A Field Programmers Guide to Genetic Programming}
A book, self published on LuLu \cite{Poli2008}
A supporting website \url{http://www.gp-field-guide.org.uk} - blog that documents happenings and errata from the book and has a user group.



\section{Software}
\label{sec:software}
This section summarizes specific software that influenced the Clever Algorithms project.

\subsection{WEKA}
From the website: 

\emph{Weka is a collection of machine learning algorithms for data mining tasks. The algorithms can either be applied directly to a dataset or called from your own Java code. Weka contains tools for data pre-processing, classification, regression, clustering, association rules, and visualization. It is also well-suited for developing new machine learning schemes.}



standardization effort for implementing data mining algorithms \cite{Hall2009}
books, papers, community, many many algorithms

open source, located online \url{http://www.cs.waikato.ac.nz/~ml/weka/}

\subsection{Optimization Algorithm Toolkit}
OAT, created by this author \cite{Brownlee2007}, inspired by WEKA and other projects.

a model for what i want to achieve - implementation for lots of optimization algorithms from the fields of CI and BIC. open source code, open to contributions, online at \url{http://optalgtoolkit.sourceforge.net}

From the website: 
\emph{The Optimization Algorithm Toolkit (OAT) is a workbench and toolkit for developing, evaluating, experimenting, and playing with classical and state-of-the-art optimization algorithms on standard benchmark problem domains. The software includes reference algorithm implementations, graphing, visualizations, and much more.}


\section{Websites}
\label{sec:websites}
This section summarizes specific websites that influenced the Clever Algorithms project.


% summarise the document message and areas for future consideration
\section{Findings}
\label{sec:findings}
% what did we cover?

% what can we use from all this?
The following describe a set of properties distilled from the inspiring works:

\begin{itemize}
	\item \emph{Openness}: let people consume/reproduce the way they want
	\item \emph{Discreteness}: semi-independence of content for information to promote discrete consumption and grazing
	\item \emph{Compactness}: minimal content, enough to be useful to target audience, pointers to more, address information overload 
	\item \emph{more\ldots} blah!
\end{itemize}



% bibliography
\bibliographystyle{plain}
\bibliography{../bibtex}

\end{document}
% EOF