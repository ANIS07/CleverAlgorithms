% Bacterial Foraging Optimization Algorithm

% The Clever Algorithms Project: http://www.CleverAlgorithms.com
% (c) Copyright 2010 Jason Brownlee. Some Rights Reserved. 
% This work is licensed under a Creative Commons Attribution-Noncommercial-Share Alike 2.5 Australia License.

\documentclass[a4paper, 11pt]{article}
\usepackage{amsmath}
\usepackage{tabularx}
\usepackage{booktabs}
\usepackage{url}
\usepackage[pdftex,breaklinks=true,colorlinks=true,urlcolor=blue,linkcolor=blue,citecolor=blue,]{hyperref}
\usepackage{geometry}
\usepackage[ruled, linesnumbered]{../algorithm2e}
\usepackage{listings} 
\usepackage{textcomp}
\ifx\pdfoutput\@undefined\usepackage[usenames,dvips]{color}
\else\usepackage[usenames,dvipsnames]{color}
\lstset{basicstyle=\footnotesize\ttfamily,numbers=left,numberstyle=\tiny,frame=single,columns=flexible,upquote=true,showstringspaces=false,tabsize=2,captionpos=b,breaklines=true,breakatwhitespace=true,keywordstyle=\color{blue},stringstyle=\color{ForestGreen}}
\geometry{verbose,a4paper,tmargin=25mm,bmargin=25mm,lmargin=25mm,rmargin=25mm}

% Dear template user: fill these in
\newcommand{\myreporttitle}{Bacterial Foraging Optimization Algorithm}
\newcommand{\myreportauthor}{Jason Brownlee}
\newcommand{\myreportemail}{jasonb@CleverAlgorithms.com}
\newcommand{\myreportwebsite}{http://www.CleverAlgorithms.com}
\newcommand{\myreportproject}{The Clever Algorithms Project\\\url{\myreportwebsite}}
\newcommand{\myreportdate}{20101121}
\newcommand{\myreportversion}{1}
\newcommand{\myreportlicense}{\copyright\ Copyright 2010 Jason Brownlee. Some Rights Reserved. This work is licensed under a Creative Commons Attribution-Noncommercial-Share Alike 2.5 Australia License.}

% leave this alone, it's templated baby!
\title{{\myreporttitle}\footnote{\myreportlicense}}
\author{\myreportauthor\\{\myreportemail}\\\small\myreportproject}
\date{\today\\{\small{Technical Report: CA-TR-{\myreportdate}-\myreportversion}}}
\begin{document}
\maketitle

% write a summary sentence for each major section
\section*{Abstract} 
% project
The Clever Algorithms project aims to describe a large number of Artificial Intelligence algorithms in a complete, consistent, and centralized manner, to improve their general accessibility. 
% template
The project makes use of a standardized algorithm description template that uses well-defined topics that motivate the collection of specific and useful information about each algorithm described.
% report
This report describes the Bacterial Foraging Optimization Algorithm using the standardized algorithm description template.


\begin{description}
	\item[Keywords:] {\small\texttt{Clever, Algorithms, Description, Optimization, Bacterial, Foraging, \\Optimization, Algorithm}}
\end{description} 

\section{Introduction} 
\label{sec:intro}
% project
The Clever Algorithms project aims to describe a large number of algorithms from the fields of Computational Intelligence, Biologically Inspired Computation, and Metaheuristics in a complete, consistent and centralized manner \cite{Brownlee2010}.
% description
The project requires all algorithms to be described using a standardized template that includes a fixed number of sections, each of which is motivated by the presentation of specific information about the technique \cite{Brownlee2010a}.
% this report
This report describes the Bacterial Foraging Optimization Algorithm using the standardized algorithm description template.

% Name
% The algorithm name defines the canonical name used to refer to the technique, in addition to common aliases, abbreviations, and acronyms. The name is used in terms of the heading and sub-headings of an algorithm description.
\section{Name} 
\label{sec:name}
% What is the canonical name and common aliases for a technique?
% What are the common abbreviations and acronyms for a technique?
% The heading and alternate headings for the algorithm description.
Bacterial Foraging Optimization Algorithm, BFOA, Bacterial Foraging Optimization, BFO

% Taxonomy: Lineage and locality
% The algorithm taxonomy defines where a techniques fits into the field, both the specific subfields of Computational Intelligence and Biologically Inspired Computation as well as the broader field of Artificial Intelligence. The taxonomy also provides a context for determining the relation- ships between algorithms. The taxonomy may be described in terms of a series of relationship statements or pictorially as a venn diagram or a graph with hierarchical structure.
\section{Taxonomy}
\label{sec:taxonomy}
% To what fields of study does a technique belong?
The Bacterial Foraging Optimization Algorithm belongs to the field of Bacteria Optimization Algorithms and Swarm Optimization, and more broadly to the fields of Computational Intelligence and Metaheuristics.
% What are the closely related approaches to a technique?
It is related to other Bacteria Optimization Algorithms such as the Bacteria Chemotaxis Algorithm \cite{Muller2002}, and other Swarm Intelligence algorithms such as Ant Colony Optimization and Particle Swarm Optimization.
There have been many extensions of the approach that attempt to hybridize the algorithm with other Computational Intelligence algorithms and Metaheuristics such as Particle Swarm Optimization, Genetic Algorithm, and Tabu Search.

% Inspiration: Motivating system
% The inspiration describes the specific system or process that provoked the inception of the algorithm. The inspiring system may non-exclusively be natural, biological, physical, or social. The description of the inspiring system may include relevant domain specific theory, observation, nomenclature, and most important must include those salient attributes of the system that are somehow abstractly or conceptually manifest in the technique. The inspiration is described textually with citations and may include diagrams to highlight features and relationships within the inspiring system.
% Optional
\section{Inspiration}
\label{sec:inspiration}
% What is the system or process that motivated the development of a technique?
The Bacterial Foraging Optimization Algorithm is inspired by the group foraging behavior of bacteria such as E.coli and M.xanthus.
% Which features of the motivating system are relevant to a technique?
Specifically, the BFOA is inspired by the chemotaxis behavior of bacteria such as E.coli that will perceive chemical gradients in the environment (such as nutrients) and will move toward or away from specific signals.

% Metaphor: Explanation via analogy
% The metaphor is a description of the technique in the context of the inspiring system or a different suitable system. The features of the technique are made apparent through an analogous description of the features of the inspiring system. The explanation through analogy is not expected to be literal scientific truth, rather the method is used as an allegorical communication tool. The inspiring system is not explicitly described, this is the role of the ‘inspiration’ element, which represents a loose dependency for this element. The explanation is textual and uses the nomenclature of the metaphorical system.
% Optional
\section{Metaphor}
\label{sec:metaphor}
% What is the explanation of a technique in the context of the inspiring system?
% What are the functionalities inferred for a technique from the analogous inspiring system?
todo

% Strategy: Problem solving plan
% The strategy is an abstract description of the computational model. The strategy describes the information processing actions a technique shall take in order to achieve an objective. The strategy provides a logical separation between a computational realization (procedure) and a analogous system (metaphor). A given problem solving strategy may be realized as one of a number specific algorithms or problem solving systems. The strategy description is textual using information processing and algorithmic terminology.
\section{Strategy}
\label{sec:strategy}
% What is the information processing objective of a technique?
The information processing strategy of the algorithm is to allow cells to stochastically walk toward optima.
% What is a techniques plan of action?
This is achieved through a series of three processes on a population of simulated cells: 1) `Chemotaxis' where the cost of cells is derated by the proximity to other cells and cells move along the manipulated cost surface one at a time (the majority of the work of the algorithm), 2) `Reproduction' where only those cells that performed well over their lifetime may contribute to the next iteration, and 3) `Elimination-dispersal' where cells are discarded and new random samples are inserted with a low probability.

% Procedure: Abstract computation
% The algorithmic procedure summarizes the specifics of realizing a strategy as a systemized and parameterized computation. It outlines how the algorithm is organized in terms of the data structures and representations. The procedure may be described in terms of software engineering and computer science artifacts such as pseudo code, design diagrams, and relevant mathematical equations.
\section{Procedure}
\label{sec:procedure}
% What is the computational recipe for a technique?
% What are the data structures and representations used in a technique?
Algorithm~\ref{alg:bfoa} provides a pseudo-code listing of the Bacterial Foraging Optimization Algorithm for minimizing a cost function. 
A cells cost is derated by its interaction with other cells. This interaction is calculated as follows: 

\begin{align*}
	interaction(cell_k) = 
	&\sum_{i=1}^S\bigg[-d_{attract}\times exp\bigg(-w_{attract}\times \sum_{m=1}^P (cell_m^k - other_m^i)^2 \bigg) \bigg] + \\
	&\sum_{i=1}^S\bigg[h_{repellant}\times exp\bigg(-w_{repellant}\times \sum_{m=1}^P (cell_m^k - other_m^i)^2 \bigg) \bigg]
\end{align*}

where $cell_k$ is a given cell, $d_{attract}$ and $w_{attract}$ are attraction coefficients, $h_{repellant}$ and $w_{repellant}$ are repulsion coefficients, $S$ is the number of cells in the population, $P$ is the number of dimensions on a given cells position vector.

\begin{algorithm}[ht]
	\SetLine

	% params
	\SetKwData{ProblemSize}{$Problem_{size}$}
	\SetKwData{NumCells}{$Cells_{num}$}
	
	
	% data
	\SetKwData{Population}{Population}
	\SetKwData{Best}{Best}
	
	% functions
	\SetKwFunction{InitializePopulation}{InitializePopulation}  
  
	% I/O
	\KwIn{\ProblemSize, \NumCells}		
	\KwOut{\Best}
  % Algorithm

	\Population $\leftarrow$ \InitializePopulation{\NumCells, \ProblemSize}\;

	
	
	\Return{\Best}\;
	% end
	\caption{Pseudo Code for the Bacterial Foraging Optimization Algorithm.}
	\label{alg:bfoa}
\end{algorithm}

% Heuristics: Usage guidelines
% The heuristics element describe the commonsense, best practice, and demonstrated rules for applying and configuring a parameterized algorithm. The heuristics relate to the technical details of the techniques procedure and data structures for general classes of application (neither specific implementations not specific problem instances). The heuristics are described textually, such as a series of guidelines in a bullet-point structure.
\section{Heuristics}
\label{sec:heuristics}
% What are the suggested configurations for a technique?
% What are the guidelines for the application of a technique to a problem instance?
\begin{itemize}
	\item The algorithm was designed for application o continuous function optimization problem domains.
	\item Given the loops in the algorithm, it can be configured numerous ways to elicit different search behavior. It is common to have a large number of chemotaxis iterations, and small numbers of the other iterations.
	\item The default coefficients for swarming behavior (cell-cell interactions) are as follows $d_{attract}=0.1$, $w_{attract}=0.2$, $h_{repellant}=d_{attract}$, and $w_{repellant}=10$.	
	\item The step size is commonly a fraction of the search space, such as 0.1.
	\item During reproduction, typically half the population with a low health metric are discarded, and two copies of each member from the first (high-health) half of the population are retained.
	\item The probability of elimination and dispersal ($p_{ed}$) is commonly set quite large, such as 0.25.
\end{itemize}

% The code description provides a minimal but functional version of the technique implemented with a programming language. The code description must be able to be typed into an appropriate computer, compiled or interpreted as need be, and provide a working execution of the technique. The technique implementation also includes a minimal problem instance to which it is applied, and both the problem and algorithm implementations are complete enough to demonstrate the techniques procedure. The description is presented as a programming source code listing.
\section{Code Listing}
\label{sec:code}
% How is a technique implemented as an executable program?
% How is a technique applied to a concrete problem instance?
Listing~\ref{bfoa} provides an example of the Bacterial Foraging Optimization Algorithm implemented in the Ruby Programming Language. 
% problem
The demonstration problem is an instance of a continuous function optimization that seeks $min f(x)$ where $f=\sum_{i=1}^n x_{i}^2$, $-5.0\leq x_i \leq 5.0$ and $n=3$. The optimal solution for this basin function is $(v_0,\ldots,v_{n-1})=0.0$.
% algorithm
The algorithm is an implementation based on the description on the seminal work \cite{Passino2002}. The parameters for cell-cell interactions (attraction and repulsion) were taken from the paper, and the various loop parameters were taken from the `Swarming Effects' example.

% the listing
\lstinputlisting[firstline=7,language=ruby,caption=Bacterial Foraging Optimization Algorithm in the Ruby Programming Language, label=bfoa]{../../src/algorithms/swarm/bfoa.rb}

% References: Deeper understanding
% The references element description includes a listing of both primary sources of information about the technique as well as useful introductory sources for novices to gain a deeper understanding of the theory and application of the technique. The description consists of hand-selected reference material including books, peer reviewed conference papers, journal articles, and potentially websites. A bullet-pointed structure is suggested.
\section{References}
\label{sec:references}
% What are the primary sources for a technique?
% What are the suggested reference sources for learning more about a technique?

% 
% Primary Sources
% 
\subsection{Primary Sources}
% seminal
Early work by Liu and Passino considered models of chemotaxis as optimization for both E.coli and M.xanthus which were applied to continuous function optimization \cite{Liu2002}.
This work was consolidated by Passino who presented the Bacterial Foraging Optimization Algorithm that includes a detailed presentation of the algorithm and equations themselves, heuristics for configuration, and demonstration applications and behavior dynamics \cite{Passino2002}.

% 
% Learn More
% 
\subsection{Learn More}
% reviews
A detailed summary of social foraging and the BFOA is provided in the book by Passino \cite{Passino2005}.
Passino provides a follow-up review of the background models of chemotaxis as optimization and describes the equations of the  Bacterial Foraging Optimization Algorithm in detail in a Journal article \cite{Passino2010}.
Das, et al. present the algorithm and its inspiration, and go on to provide an in depth analysis the dynamics of chemotaxis using simplified mathematical models \cite{Das2009}.
% books


% 
% Conclusions: What the reader or what thre author learned by completing this this report.
% 
\section{Conclusions}
\label{sec:conclusions}
% report
This report described the Bacterial Foraging Optimization Algorithm using the standardized algorithm description template.

% 
% Contribute
% 
\section{Contribute}
\label{sec:contribute}
% simple
Found a typo in the content or a bug in the source code? 
% advanced 
Are you an expert in this technique and know some facts that could improve the algorithm description for all?
% incentive
Do you want to get that warm feeling from contributing to an open source project? 
Do you want to see your name as an acknowledgment in print?

%  ideal
Two pillars of this effort are i) that the best domain experts are people outside of the project, and ii) that this work is subjected to continuous improvement. 
% advice
Please help to make this work less wrong by emailing the author `\myreportauthor' at \url{\myreportemail} or visit the project website at \url{\myreportwebsite}.

% bibliography
\bibliographystyle{plain}
\bibliography{../bibtex}

\end{document}
% EOF