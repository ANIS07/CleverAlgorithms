% Differential Evolution

% The Clever Algorithms Project: http://www.CleverAlgorithms.com
% (c) Copyright 2010 Jason Brownlee. Some Rights Reserved. 
% This work is licensed under a Creative Commons Attribution-Noncommercial-Share Alike 2.5 Australia License.

\documentclass[a4paper, 11pt]{article}
\usepackage{tabularx}
\usepackage{booktabs}
\usepackage{url}
\usepackage[pdftex,breaklinks=true,colorlinks=true,urlcolor=blue,linkcolor=blue,citecolor=blue,]{hyperref}
\usepackage{geometry}
\usepackage[ruled, linesnumbered]{../algorithm2e}
\usepackage{listings} 
\usepackage{textcomp}
\ifx\pdfoutput\@undefined\usepackage[usenames,dvips]{color}
\else\usepackage[usenames,dvipsnames]{color}
\lstset{basicstyle=\footnotesize\ttfamily,numbers=left,numberstyle=\tiny,frame=single,columns=flexible,upquote=true,showstringspaces=false,tabsize=2,captionpos=b,breaklines=true,breakatwhitespace=true,keywordstyle=\color{blue},stringstyle=\color{ForestGreen}}
\geometry{verbose,a4paper,tmargin=25mm,bmargin=25mm,lmargin=25mm,rmargin=25mm}

% Dear template user: fill these in
\newcommand{\myreporttitle}{Differential Evolution}
\newcommand{\myreportauthor}{Jason Brownlee}
\newcommand{\myreportemail}{jasonb@CleverAlgorithms.com}
\newcommand{\myreportwebsite}{http://www.CleverAlgorithms.com}
\newcommand{\myreportproject}{The Clever Algorithms Project\\\url{\myreportwebsite}}
\newcommand{\myreportdate}{20100403}
\newcommand{\myreportversion}{1}
\newcommand{\myreportlicense}{\copyright\ Copyright 2010 Jason Brownlee. Some Rights Reserved. This work is licensed under a Creative Commons Attribution-Noncommercial-Share Alike 2.5 Australia License.}

% leave this alone, it's templated baby!
\title{{\myreporttitle}\footnote{\myreportlicense}}
\author{\myreportauthor\\{\myreportemail}\\\small\myreportproject}
\date{\today\\{\small{Technical Report: CA-TR-{\myreportdate}-\myreportversion}}}
\begin{document}
\maketitle

% write a summary sentence for each major section
\section*{Abstract} 
% project
The Clever Algorithms project aims to describe a large number of Artificial Intelligence algorithms in a complete, consistent, and centralized manner, to improve their general accessibility. 
% template
The project makes use of a standardized algorithm description template that uses well-defined topics that motivate the collection of specific and useful information about each algorithm described.
% report
This report describes the Differential Evolution algorithm using the standardized template.

\begin{description}
	\item[Keywords:] {\small\texttt{Clever, Algorithms, Description, Optimization, Differential Evolution}}
\end{description} 

\section{Introduction} 
\label{sec:intro}
% project
The Clever Algorithms project aims to describe a large number of algorithms from the fields of Computational Intelligence, Biologically Inspired Computation, and Metaheuristics in a complete, consistent and centralized manner \cite{Brownlee2010}.
% description
The project requires all algorithms to be described using a standardized template that includes a fixed number of sections, each of which is motivated by the presentation of specific information about the technique \cite{Brownlee2010a}.
% this report
This report describes the Differential Evolution algorithm using the standardized template.

% Name
% The algorithm name defines the canonical name used to refer to the technique, in addition to common aliases, abbreviations, and acronyms. The name is used in terms of the heading and sub-headings of an algorithm description.
\section{Name} 
\label{sec:name}
% What is the canonical name and common aliases for a technique?
% What are the common abbreviations and acronyms for a technique?
% The heading and alternate headings for the algorithm description.
Differential Evolution, DE

% Taxonomy: Lineage and locality
% The algorithm taxonomy defines where a techniques fits into the field, both the specific subfields of Computational Intelligence and Biologically Inspired Computation as well as the broader field of Artificial Intelligence. The taxonomy also provides a context for determining the relation- ships between algorithms. The taxonomy may be described in terms of a series of relationship statements or pictorially as a venn diagram or a graph with hierarchical structure.
\section{Taxonomy}
\label{sec:taxonomy}
% To what fields of study does a technique belong?
Differential Evolution is a Global Optimization algorithm and is an instance of an Evolutionary Algorithm from the field of Evolutionary Computation.
% What are the closely related approaches to a technique?
It is related to sibling Evolutionary Algorithms such as the Genetic Algorithm and Evolutionary Programming, and Evolution Strategies, and shows some similarities to Particle Swarm Optimization.

% Strategy: Problem solving plan
% The strategy is an abstract description of the computational model. The strategy describes the information processing actions a technique shall take in order to achieve an objective. The strategy provides a logical separation between a computational realization (procedure) and a analogous system (metaphor). A given problem solving strategy may be realized as one of a number specific algorithms or problem solving systems. The strategy description is textual using information processing and algorithmic terminology.
\section{Strategy}
\label{sec:strategy}
% What is the information processing objective of a technique?
% What is a techniques plan of action?

The Differential Evolution algorithm involves maintaining a population of candidate solutions that under go generations of recombination, evaluation, and selection. The recombination strategy is quite different to similar evolutionary algorithms involving the creation of new candidate solution components based on the weighted difference between two randomly selected population members added to a third population member (so-called classic DE). This approach perturbs population members relative relative to the spread of the broader population. In conjunction with selection, the perturbation effect self-organizes or bound the resampling of the problem space to known areas of interest.

extracts direction and distance information from the population, indirectly


% Procedure: Abstract computation
% The algorithmic procedure summarizes the specifics of realizing a strategy as a systemized and parameterized computation. It outlines how the algorithm is organized in terms of the data structures and representations. The procedure may be described in terms of software engineering and computer science artifacts such as pseudo code, design diagrams, and relevant mathematical equations.
\section{Procedure}
\label{sec:procedure}
% What is the computational recipe for a technique?

% What are the data structures and representations used in a technique?
The algorithm as a specialized nomenclature for each variation of the technique describing the specifics of the configuration used in the variation. This takes the form of DE/x/y/z, where x represents the solution to be perturbed such a random or best (in the population). The y signifies the number of difference vectors used in the perturbation of x, where a difference vectors is essentially the difference between two randomly selected although distinct members of the population. Finally, z signifies the recombination operator performed such as bin for binomial and exp for exponential. 

The classical DE is noted as DE/rand/1/bin, where as variation of this approach that creates new solutions as perturbations of the best solution in the generation is noted as DE/best/1/bin.


Algorithm~\ref{alg:differential_evolution} provides a pseudo-code listing of the Differential Evolution algorithm for minimizing a cost function. 

\begin{algorithm}[htp]
	\SetLine  

	% data
	\SetKwData{Best}{$S_{best}$}
	\SetKwData{Member}{$P_{i}$}
	\SetKwData{Population}{Population}
	
	\SetKwData{PopulationSize}{G}
	\SetKwData{NumParameters}{NP}
	\SetKwData{WeightingFactor}{F}
	\SetKwData{CrossoverRate}{CR}
	
	% functions
	\SetKwFunction{StopCondition}{StopCondition}
	\SetKwFunction{InitializePopulation}{InitializePopulation}
	\SetKwFunction{EvaluateCost}{EvaluateCost}
	\SetKwFunction{GetBestSolution}{GetBestSolution}
  
	% I/O
	\KwIn{\PopulationSize, \NumParameters, \WeightingFactor, \CrossoverRate}		
	\KwOut{\Best}
  	% Algorithm
	% initialize	
	\Population $\leftarrow$ \InitializePopulation{\PopulationSize, \NumParameters}\;
	% evaluate
	\EvaluateCost{\Population}\;
	% best
	\Best $\leftarrow$ \GetBestSolution{\Population}\;
	% loop
	\While{$\neg$\StopCondition{}} {
		
		
	}
	\Return{\Best}\;
	% end
	\caption{Pseudo Code for the Differential Evolution algorithm.}
	\label{alg:differential_evolution}
\end{algorithm}

% Heuristics: Usage guidelines
% The heuristics element describe the commonsense, best practice, and demonstrated rules for applying and configuring a parameterized algorithm. The heuristics relate to the technical details of the techniques procedure and data structures for general classes of application (neither specific implementations not specific problem instances). The heuristics are described textually, such as a series of guidelines in a bullet-point structure.
\section{Heuristics}
\label{sec:heuristics}
% What are the suggested configurations for a technique?
% What are the guidelines for the application of a technique to a problem instance?
\begin{itemize}
	\item The number of parents (NP) should be 10 times the number of parameters being optimized
	\item Use a weighting factor (F) of 0.8
	\item Use a crossover rate (CR) of 0.9
	\item Initialize new vectors anywhere in the valid search space
	\item Increase to NP should decrease F and vice-versa
	\item Higher crossover for DE/rand/1/bin than for DE/rand/1/exp
	\item CR and F almost always in [0.5, 1.0]
	\item Common approaches: DE/rand/1/ DE/best/2/
	
	\item designed for nonlinear, non-differentiable continuous function optimization
	\item initialize with a random population
	
\end{itemize}

% The code description provides a minimal but functional version of the technique implemented with a programming language. The code description must be able to be typed into an appropriate computer, compiled or interpreted as need be, and provide a working execution of the technique. The technique implementation also includes a minimal problem instance to which it is applied, and both the problem and algorithm implementations are complete enough to demonstrate the techniques procedure. The description is presented as a programming source code listing.
\section{Code Listing}
\label{sec:code}
% How is a technique implemented as an executable program?
% How is a technique applied to a concrete problem instance?
Listing~\ref{differential_evolution} provides an example of the Differential Evolution algorithm implemented in the Ruby Programming Language.
% problem
The demonstration problem is an instance of a continuous function optimization that seeks $min f(x)$ where $f=\sum_{i=1}^n x_{i}^2$, $-5.0\leq x_i \leq 5.0$ and $n=2$. The optimal solution for this basin function is $(v_0,\ldots,v_{n-1})=0.0$.
% algorithm
The algorithm...

% the listing
\lstinputlisting[firstline=7,language=ruby,caption=Differential Evolution algorithm in the Ruby Programming Language, label=differential_evolution]{../../src/algorithms/evolutionary/differential_evolution.rb}


% References: Deeper understanding
% The references element description includes a listing of both primary sources of information about the technique as well as useful introductory sources for novices to gain a deeper understanding of the theory and application of the technique. The description consists of hand-selected reference material including books, peer reviewed conference papers, journal articles, and potentially websites. A bullet-pointed structure is suggested.
\section{References}
\label{sec:references}
% What are the primary sources for a technique?
% What are the suggested reference sources for learning more about a technique?

% 
% Primary Sources
% 
\subsection{Primary Sources}
% seminal
The Differential Evolution algorithm was presented by Storn and Price in a technical report that considered DE1 and DE2 variants of the approach applied to a suite of continuous function optimization problems  \cite{Storn1995}. A revised version of the technical report was later published as a journal article \cite{Storn1997}. 
% early papers, maturation
An early paper by Storn applied the approach to the optimization of an IIR-filter (Infinite Impulse Response) \cite{Storn1996a}. A second early paper applied the approach to a second suite of benchmark problem instances, adopting the contemporary nomenclature for describing the approach, including the DE/rand/1 and DE/best/2 variations \cite{Storn1996b}.


% 
% Learn More
% 
\subsection{Learn More}
% historical reviews
A classical overview of Differential Evolution is presented by Price and Storn \cite{Price1997}, and terse introduction to the approach for function optimization is presented by Storn \cite{Storn1996}. A seminal extended description of the algorithm with sample applications was presented by Storn and Price as a book chapter \cite{Price1999}.
% books
Price, Storn, and Lampinen release a contemporary book dedicated to Differential Evolution including theory, benchmarks, sample code and numerous application demonstrations \cite{Price2005}. Chakraborty also released a book considering extensions to the approach to address complexities such as rotation invariance and stopping criteria  \cite{Chakraborty2008}.

% 
% Conclusions: What the reader or what thre author learned by completing this this report.
% 
\section{Conclusions}
\label{sec:conclusions}
% report
todo


% 
% Contribute
% 
\section{Contribute}
\label{sec:contribute}
% simple
Found a typo in the content or a bug in the source code? 
% advanced 
Are you an expert in this technique and know some facts that could improve the algorithm description for all?
% incentive
Do you want to get that warm feeling from contributing to an open source project? 
Do you want to see your name as an acknowledgment in print?

%  ideal
Two pillars of this effort are i) that the best domain experts are people outside of the project, and ii) that this work is wrong by default. 
% advice
Please help to make this work less wrong by emailing the author `\myreportauthor' at \url{\myreportemail} or visit the project website at \url{\myreportwebsite}.

% bibliography
\bibliographystyle{plain}
\bibliography{../bibtex}

\end{document}
% EOF