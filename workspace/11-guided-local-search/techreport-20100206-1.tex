% Guided Local Search

% The Clever Algorithms Project: http://www.CleverAlgorithms.com
% (c) Copyright 2010 Jason Brownlee. Some Rights Reserved. 
% This work is licensed under a Creative Commons Attribution-Noncommercial-Share Alike 2.5 Australia License.

\documentclass[a4paper, 11pt]{article}
\usepackage{tabularx}
\usepackage{booktabs}
\usepackage{url}
\usepackage[pdftex,breaklinks=true,colorlinks=true,urlcolor=blue,linkcolor=blue,citecolor=blue,]{hyperref}
\usepackage{geometry}
\usepackage[ruled, linesnumbered]{../algorithm2e}
\usepackage{listings} 
\usepackage{textcomp}
\ifx\pdfoutput\@undefined\usepackage[usenames,dvips]{color}
\else\usepackage[usenames,dvipsnames]{color}
\lstset{basicstyle=\footnotesize\ttfamily,numbers=left,numberstyle=\tiny,frame=single,columns=flexible,upquote=true,showstringspaces=false,tabsize=2,captionpos=b,breaklines=true,breakatwhitespace=true,keywordstyle=\color{blue},stringstyle=\color{ForestGreen}}
\geometry{verbose,a4paper,tmargin=25mm,bmargin=25mm,lmargin=25mm,rmargin=25mm}

% Dear template user: fill these in
\newcommand{\myreporttitle}{Guided Local Search}
\newcommand{\myreportauthor}{Jason Brownlee}
\newcommand{\myreportemail}{jasonb@CleverAlgorithms.com}
\newcommand{\myreportwebsite}{http://www.CleverAlgorithms.com}
\newcommand{\myreportproject}{The Clever Algorithms Project\\\url{\myreportwebsite}}
\newcommand{\myreportdate}{20100206}
\newcommand{\myreportversion}{1}
\newcommand{\myreportlicense}{\copyright\ Copyright 2010 Jason Brownlee. Some Rights Reserved. This work is licensed under a Creative Commons Attribution-Noncommercial-Share Alike 2.5 Australia License.}

% leave this alone, it's templated baby!
\title{{\myreporttitle}\footnote{\myreportlicense}}
\author{\myreportauthor\\{\myreportemail}\\\small\myreportproject}
\date{\today\\{\small{Technical Report: CA-TR-{\myreportdate}-\myreportversion}}}
\begin{document}
\maketitle

% write a summary sentence for each major section
\section*{Abstract} 
% project
The Clever Algorithms project aims to describe a large number of Artificial Intelligence algorithms in a complete, consistent, and centralized manner, to improve their general accessibility. 
% template
The project makes use of a standardized algorithm description template that uses well-defined topics that motivate the collection of specific and useful information about each algorithm described.
% report
todo

\begin{description}
	\item[Keywords:] {\small\texttt{Clever, Algorithms, Description, Optimization}}
\end{description} 

\section{Introduction} 
\label{sec:intro}
% project
The Clever Algorithms project aims to describe a large number of algorithms from the fields of Computational Intelligence, Biologically Inspired Computation, and Metaheuristics in a complete, consistent and centralized manner \cite{Brownlee2010}.
% description
The project requires all algorithms to be described using a standardized template that includes a fixed number of sections, each of which is motivated by the presentation of specific information about the technique \cite{Brownlee2010a}.
% this report
This report describes\ldots todo

% Name
% The algorithm name defines the canonical name used to refer to the technique, in addition to common aliases, abbreviations, and acronyms. The name is used in terms of the heading and sub-headings of an algorithm description.
\section{Name} 
\label{sec:name}
% What is the canonical name and common aliases for a technique?
% What are the common abbreviations and acronyms for a technique?
% The heading and alternate headings for the algorithm description.
Guided Local Search, GLS

% Taxonomy: Lineage and locality
% The algorithm taxonomy defines where a techniques fits into the field, both the specific subfields of Computational Intelligence and Biologically Inspired Computation as well as the broader field of Artificial Intelligence. The taxonomy also provides a context for determining the relation- ships between algorithms. The taxonomy may be described in terms of a series of relationship statements or pictorially as a venn diagram or a graph with hierarchical structure.
\section{Taxonomy}
\label{sec:taxonomy}
metaheuristic for local search
an extension to hill climbing
related to tabu search (penalties)


% Strategy: Problem solving plan
% The strategy is an abstract description of the computational model. The strategy describes the information processing actions a technique shall take in order to achieve an objective. The strategy provides a logical separation between a computational realization (procedure) and a analogous system (metaphor). A given problem solving strategy may be realized as one of a number specific algorithms or problem solving systems. The strategy description is textual using information processing and algorithmic terminology.
\section{Strategy}
\label{sec:strategy}
% What is the information processing objective of a technique?
% What is a techniques plan of action?
metaheuristic to guide local search
penalize specific features in candidate solutions, modifies the response surface for the local search technique
designed to sit on top of hill climbing algorithms

run local search until it gets stuck in a local optima, record the position, update the cost function to penalize the search for heading toward the local optima, repeat


% Procedure: Abstract computation
% The algorithmic procedure summarizes the specifics of realizing a strategy as a systemized and parameterized computation. It outlines how the algorithm is organized in terms of the data structures and representations. The procedure may be described in terms of software engineering and computer science artifacts such as pseudo code, design diagrams, and relevant mathematical equations.
\section{Procedure}
\label{sec:procedure}
% What is the computational recipe for a technique?
% What are the data structures and representations used in a technique?
Algorithm~\ref{alg:guided_local_search} provides a pseudo-code listing of the Guided Local Search algorithm for \ldots todo

\begin{algorithm}[htp]
	\SetLine
	% data
	\SetKwData{NumIterations}{$Iter_{max}$}
	\SetKwData{ProblemSize}{ProblemSize}
	\SetKwData{Current}{Current}
	\SetKwData{Candidate}{Candidate}
	% functions
	\SetKwFunction{Cost}{Cost}
	\SetKwFunction{RandomSolution}{RandomSolution}
	\SetKwFunction{RandomNeighbor}{RandomNeighbor}
  	% I/O
	\KwIn{\NumIterations, \ProblemSize}
	\KwOut{\Current}
  	% Algorithm
	% init
	\Current $\leftarrow$ \RandomSolution{\ProblemSize}\;
	% main loop
	\ForEach{$iter_i \in$ \NumIterations} {
		% small step
		$\Candidate \leftarrow$ \RandomNeighbor{\Current}\;		
		\If{\Cost{\Candidate} $\geq$ \Cost{\Current}} {
			\Current $\leftarrow$ \Candidate\;
		}
	}
	\Return{\Current}\;
	% caption
	\caption{Pseudo Code Listing for the Guided Local Search algorithm.}
	\label{alg:guided_local_search}
\end{algorithm}

% Heuristics: Usage guidelines
% The heuristics element describe the commonsense, best practice, and demonstrated rules for applying and configuring a parameterized algorithm. The heuristics relate to the technical details of the techniques procedure and data structures for general classes of application (neither specific implementations not specific problem instances). The heuristics are described textually, such as a series of guidelines in a bullet-point structure.
\section{Heuristics}
\label{sec:heuristics}
% What are the suggested configurations for a technique?
% What are the guidelines for the application of a technique to a problem instance?
\begin{itemize}
	\item todo
\end{itemize}

% The code description provides a minimal but functional version of the technique implemented with a programming language. The code description must be able to be typed into an appropriate computer, compiled or interpreted as need be, and provide a working execution of the technique. The technique implementation also includes a minimal problem instance to which it is applied, and both the problem and algorithm implementations are complete enough to demonstrate the techniques procedure. The description is presented as a programming source code listing.
\section{Code Listing}
\label{sec:code}
% How is a technique implemented as an executable program?
% How is a technique applied to a concrete problem instance?
Listing~\ref{guided_local_search} provides an example of the Guided Local Search algorithm implemented in the Ruby Programming Language.

% problem
todo

% the listing
\lstinputlisting[firstline=7,language=ruby,caption=Guided Local Search algorithm in the Ruby Programming Language, label=guided_local_search]{../../src/algorithms/stochastic/guided_local_search.rb}


% References: Deeper understanding
% The references element description includes a listing of both primary sources of information about the technique as well as useful introductory sources for novices to gain a deeper understanding of the theory and application of the technique. The description consists of hand-selected reference material including books, peer reviewed conference papers, journal articles, and potentially websites. A bullet-pointed structure is suggested.
\section{References}
\label{sec:references}
% What are the primary sources for a technique?
% What are the suggested reference sources for learning more about a technique?

came from: GENET
\begin{itemize}
	\item GLS is an extension of GENET
	\item homepage for the algorithm: \url{http://www.bracil.net/CSP/gls.html}
	\item GENET applied to a binary problem: SOLVING CONSTRAINT SATISFACTION PROBLEMS USING NEURAL NETWORKS (1991)
	\item GENET extended beyond binary problems: A Generic Neural Network Approach For Constraint Satisfaction Problems (1992)
	\item Extensions and evaluation of GENET in Constraint satisfaction (1996)
\end{itemize}


extensions: extended guided local search, guided genetic search
\begin{itemize}
	\item extension: Guided Genetic Algorithm (GGA) (ga + GLS): Guided Genetic Algorithm (1999)
	\item extensions: Applying an Extended Guided Local Search on the Quadratic Assignment Problem (2003)
\end{itemize}


makes use of an updated local search - a fast local search


% 
% Primary Sources
% 
\subsection{Primary Sources}
% based on GNET
Guided Local Search emerged from an approach called GENET, which is a connectionist approach to constraint satisfaction.
% seminal papers
Guided Local Search was presented by Voudouris and Tsang in a series of technical reports (that were later published) that described the technique and provided example applications of it to constraint satisfaction \cite{Voudouris1994}, combinatorial optimization \cite{Voudouris1995b, Voudouris1995}, and function optimization \cite{Voudouris1995a}.
% thesis
The seminal work on the technique was Voudouris' PhD dissertation \cite{Voudouris1997}.


% 
% Learn More
% 
\subsection{Learn More}
% review
Voudouris and Tsang provide a high-level introduction to the technique \cite{Voudouris1998}, and a contemporary summary of the approach in Glover and Kochenberger's `Handbook of metaheuristics' \cite{Glover2003a} that includes a review of the technique, application areas, and demonstration applications on a diverse set of problem instances.
% Mills' extensions
Mills, et al. elaborated on the approach, devising an `Extended Guided Local Search' (EGLS) technique that add `aspiration criteria' and random moves to the procedure \cite{Mills2003}, work which culminated in Mills' PhD dissertation \cite{Mills2002}.  
% Lau's extensions
Lau and Tsang further extended the approach by integrating it with a Genetic Algorithm, called the `Guided Genetic Algorithm' (GGA) \cite{Lau1998}, that also culminated in a PhD dissertation \cite{Lau1999}.

% 
% Conclusions: What the reader or what thre author learned by completing this this report.
% 
\section{Conclusions}
\label{sec:conclusions}
% report
todo

% 
% Contribute
% 
\section{Contribute}
\label{sec:contribute}
% simple
Found a typo in the content or a bug in the source code? 
% advanced 
Are you an expert in this technique and know some facts that could improve the algorithm description for all?
% incentive
Do you want to get that warm feeling from contributing to an open source project? 
Do you want to see your name as an acknowledgment in print?

%  ideal
Two pillars of this effort are i) that the best domain experts are people outside of the project, and ii) that this work is wrong by default. 
% advice
Please help to make this work less wrong by emailing the author `\myreportauthor' at \url{\myreportemail} or visit the project website at \url{\myreportwebsite}.

% bibliography
\bibliographystyle{plain}
\bibliography{../bibtex}

\end{document}
% EOF