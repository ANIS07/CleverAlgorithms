% Evolutionary Algorithms

% The Clever Algorithms Project: http://www.CleverAlgorithms.com
% (c) Copyright 2010 Jason Brownlee. Some Rights Reserved. 
% This work is licensed under a Creative Commons Attribution-Noncommercial-Share Alike 2.5 Australia License.

\documentclass[a4paper, 11pt]{article}
\usepackage{tabularx}
\usepackage{booktabs}
\usepackage{url}
\usepackage[pdftex,breaklinks=true,colorlinks=true,urlcolor=blue,linkcolor=blue,citecolor=blue,]{hyperref}
\usepackage{geometry}
\geometry{verbose,a4paper,tmargin=25mm,bmargin=25mm,lmargin=25mm,rmargin=25mm}

% Dear template user: fill these in
\newcommand{\myreporttitle}{Evolutionary Algorithms}
\newcommand{\myreportauthor}{Jason Brownlee}
\newcommand{\myreportemail}{jasonb@CleverAlgorithms.com}
\newcommand{\myreportproject}{The Clever Algorithms Project\\\url{http://www.CleverAlgorithms.com}}
\newcommand{\myreportdate}{20100509}
\newcommand{\myreportversion}{1}
\newcommand{\myreportlicense}{\copyright\ Copyright 2010 Jason Brownlee. Some Rights Reserved. This work is licensed under a Creative Commons Attribution-Noncommercial-Share Alike 2.5 Australia License.}

% leave this alone, it's templated baby!
\title{{\myreporttitle}\footnote{\myreportlicense}}
\author{\myreportauthor\\{\myreportemail}\\\small\myreportproject}
\date{\today\\{\small{Technical Report: CA-TR-{\myreportdate}-\myreportversion}}}
\begin{document}
\maketitle

% write a summary sentence for each major section
\section*{Abstract} 
% project
The Clever Algorithms project aims to describe a large number of Artificial Intelligence algorithms in a complete, consistent, and centralized manner, to improve their general accessibility. 
% template
The project makes use of a standardized algorithm description template that uses well-defined topics that motivate the collection of specific and useful information about each algorithm described.
% stochastic
The second batch of ten algorithms for the project have been described, all of which are classified as Evolutionary Algorithms under the adopted algorithm taxonomy.
% best practices
This report provides a point of reflection on the preparation of the second ten algorithms to be described for the Clever Algorithms project and summarizes the findings, lessons learned, and recommends some best practices for going forward with the project.

\begin{description}
	\item[Keywords:] {\small\texttt{Clever, Algorithms, Project, Evolutionary, Optimization, Findings}}
\end{description} 

% summarise the document breakdown with cross references
\section{Introduction}
\label{sec:introduction}
% project
The Clever Algorithms project aims to describe a large number of algorithms from the fields of Computational Intelligence, Biologically Inspired Computation, and Metaheuristics in a complete, consistent and centralized manner \cite{Brownlee2010}.
% description
The project requires all algorithms to be described using a standardized template that includes a fixed number of sections, each of which is motivated by the presentation of specific information about the technique \cite{Brownlee2010a}.
% this report
This report provides an overview of the Evolutionary Algorithms in the Clever Algorithms project. 
Section~\ref{sec:algorithms} provides background information and reviews common themes for the general class of algorithm and summarizes those evolutionary algorithms that have been described for the Clever Algorithms Project.
Section~\ref{sec:outcomes} summarizes the findings from preparing the evolutionary algorithm descriptions, and lists a number of recommendations for both describing algorithms in the future and the modification of the existing algorithm descriptions for inclusion in the Clever Algorithms book and web page.

% 
% Described Evolutionary Algorithms
% 
\section{Evolutionary Algorithms}
\label{sec:algorithms}

% 
% Background
% 
\subsection{Background}
% broadly
The algorithms to be described in the Clever Algorithms project are drawn from a diverse set of subfields of Artificial Intelligence, such as Computational Intelligence, Biologically Inspired Computation, and Metaheuristics. The majority of the algorithms selected for description in the project are optimization algorithms \cite{Brownlee2010b}. 
% unconventional
This collection of selected algorithms have been referred to as Unconventional Optimization algorithms in order to differentiate them from the more traditional deterministic approaches from operations research and mathematical programming such as Nelder-Mead method \cite{Brownlee2010n}. So-called unconventional optimization algorithms are characterized by the incorporation of stochastic and probabilistic processes and the heuristic or approximate (sub-optimal) nature of the solutions that are designed to achieve. As such, these algorithms can be understood by topics such as black box methods, stochastic optimization, and inductive learning (refer to Brownlee \cite{Brownlee2010n}). 

% specific
The ten algorithms that have been described for the Clever Algorithms project are referred to as Evolutionary Algorithms. They are differentiated from the remainder of the algorithms described in the project that have been designated a taxonomy including stochastic, swarm, probabilistic, physical, and immune algorithms \cite{Brownlee2010b}. 

% differences
\subsubsection{Evolution}
Evolutionary Algorithms belong to the Evolutionary Computation field of study concerned with methods inspired by the process and mechanisms of biological evolution. The process of evolution by means of natural selection (descent with modification) was proposed by Darwin to account for the variety of life and its suitability (adaptive fit) for its environment. The mechanisms of evolution describe how evolution actually takes place through the modification and propagation of genetic material (proteins). Evolutionary algorithms are concerned with investigating computations that resemble simplified version of the processes and mechanisms of evolution toward achieving the effects of these processes and mechanisms, namely the development of adaptive systems.
% related
Additional subject areas that fall within the realm of Evolutionary Computation are algorithms that seek to exploit the properties from the related fields of Population Genetics, Population Ecology, Coevolutionary Biology, and Developmental Biology. 

% features
\subsubsection{References}
Section~\ref{subsec:algorithms} lists the evolutionary algorithms algorithms that have been described. These algorithms share properties of adaptation through an iterative process of trial and error that accumulates and amplifies beneficial variation. Candidate solutions represent members of a virtual population striving to survive in an environment defined by a problem specific objective function. In each case, the evolutionary process refines the adaptive fit of the population of candidate solutions to the environment, typically using surrogates for the mechanisms of evolution such as genetic recombination and genetic mutation.

% references
There are many excellent texts on the theory of evolution, although Darwin's the original source can be an interesting and surprisingly enjoyable read \cite{Darwin1859}. Huxley's book defined the modern synthesis in evolutionary biology that combined Darwin's natural selection with Mendel's genetic mechanisms \cite{Huxley1942}, although any good textbook on evolution would suffice such as Futuyma's `Evolution' \cite{Futuyma2009}. Popular science books on evolution are an easy place to start such as Dawkins' `The Selfish Gene' that presents on a gene-centric perspective on evolution \cite{Dawkins1976}, and Dennett's `Darwin's Dangerous Idea' that considers on the algorithmic properties of the processes \cite{Dennett1995}.

% classical
Goldberg's classic text is still a valuable resource for the Genetic Algorithm \cite{Goldberg1989}, and Holland's text is interesting for those looking to learn about the research into adaptive systems that became the Genetic Algorithm \cite{Holland1975}. Additionally, the seminal work by Koza should be considered for those interested in Genetic Programming \cite{Koza1992}, and Schwefel's seminal work should be considered for those with an interest in Evolution Strategies \cite{Schwefel1981}.
% modern
For a rounded overview of the field of Evolutionary Computation B\"ack, Fogel, and Michalewicz's two volumes of `Evolutionary Computation' are an excellent resource covering the major techniques, theory, and application specific concerns \cite{Baeck2000, Baeck2000a}.
% other books
For some additional modern books on the unified field of Evolutionary Computation and Evolutionary Algorithms, see De Jong \cite{Jong2006}, a recent edition of Fogel \cite{Fogel1995}, and Eiben and Smith \cite{Eiben2003}. 

% 
% Described Algorithms
% 
\subsection{Described Algorithms}
\label{subsec:algorithms}
% overview
This section summarizes the stochastic algorithms currently described for inclusion in the Clever Algorithms project. It is proposed that these algorithms will collectively comprise a chapter on `Evolutionary Algorithms' in the Clever Algorithms book. 

\begin{enumerate}
	\item \textbf{Genetic Algorithm}: \cite{Brownlee2010p}
	\item \textbf{Genetic Programming}: \cite{Brownlee2010q}
	\item \textbf{Evolutionary Programming}: \cite{Brownlee2010r}
	\item \textbf{Evolution Strategies}: \cite{Brownlee2010s}
	\item \textbf{Differential Evolution}: \cite{Brownlee2010t}
	\item \textbf{Grammatical Evolution}: \cite{Brownlee2010u}
	\item \textbf{Gene Expression Programming}: \cite{Brownlee2010v}
	\item \textbf{Learning Classifier System}: \cite{Brownlee2010w}
	\item \textbf{Non-dominated Sorting Genetic Algorithm}: \cite{Brownlee2010x}
	\item \textbf{Strength Pareto Evolutionary Algorithm}: \cite{Brownlee2010y}
\end{enumerate}

% 
% Outcomes
% 
\section{Outcomes}
\label{sec:outcomes}

% 
% Findings
% 
\subsection{Findings}
% overview
This section summarizes the interesting and relevant observations made while preparing the second ten algorithm descriptions for the Clever Algorithms project. Some of these observations also make specific suggestions for the project. 

\begin{itemize}
	\item \textbf{words}: words.
\end{itemize}


% 
% Recommendations
% 
\subsection{Recommendations}
% overview
This section lists recommendations for algorithm descriptions to be written for the Clever Algorithms project in the future. 

\begin{itemize}
	\item \textbf{words}: words.
\end{itemize}

% 
% Conclusions
% 
\section{Conclusions}
\label{sec:conclusions}
% overview
This report provided a point of reflection for the second batch of ten algorithm descriptions prepared for the Clever Algorithms project. All described algorithms were assigned to the `Evolutionary Algorithms' kingdom in the chosen algorithm taxonomy. This report highlighted the commonality for all described Evolutionary Algorithms and provided a definition suitable for use in the proposed book and website.
%  forward
The report provided a summary of the interesting findings identified in the preparation of the algorithm descriptions, and provided a set of recommendations for going forward both with future algorithm descriptions and with regard to migrating the existing descriptions into the Clever Algorithms book and website mediums.

% bibliography
\bibliographystyle{plain}
\bibliography{../bibtex}

\end{document}
% EOF