% Clever Algorithms: Algorithm Visualization

% The Clever Algorithms Project: http://www.CleverAlgorithms.com
% (c) Copyright 2010 Jason Brownlee. Some Rights Reserved. 
% This work is licensed under a Creative Commons Attribution-Noncommercial-Share Alike 2.5 Australia License.

\documentclass[a4paper, 11pt]{article}
\usepackage{tabularx}
\usepackage{booktabs}
\usepackage{url}
\usepackage[pdftex,breaklinks=true,colorlinks=true,urlcolor=blue,linkcolor=blue,citecolor=blue,]{hyperref}
\usepackage{geometry}
\geometry{verbose,a4paper,tmargin=25mm,bmargin=25mm,lmargin=25mm,rmargin=25mm}

% Dear template user: fill these in
\newcommand{\myreporttitle}{Clever Algorithms}
\newcommand{\myreportsubtitle}{Algorithm Visualization}
\newcommand{\myreportauthor}{Jason Brownlee}
\newcommand{\myreportemail}{jasonb@CleverAlgorithms.com}
\newcommand{\myreportproject}{The Clever Algorithms Project\\\url{http://www.CleverAlgorithms.com}}
\newcommand{\myreportdate}{20101220}
\newcommand{\myreportfulldate}{December 20, 2010}
\newcommand{\myreportversion}{1}
\newcommand{\myreportlicense}{\copyright\ Copyright 2010 Jason Brownlee. Some Rights Reserved. This work is licensed under a Creative Commons Attribution-Noncommercial-Share Alike 2.5 Australia License.}

% leave this alone, it's templated baby!
\title{{\myreporttitle}: {\myreportsubtitle}\footnote{\myreportlicense}}
\author{\myreportauthor\\{\myreportemail}\\\small\myreportproject}
\date{\myreportfulldate\\{\small{Technical Report: CA-TR-{\myreportdate}-\myreportversion}}}
\begin{document}
\maketitle

% write a summary sentence for each major section
\section*{Abstract} 
% project
The Clever Algorithms project aims to describe a large number of Artificial Intelligence algorithms in a complete, consistent, and centralized manner, to improve their general accessibility. 
% template
The project makes use of a standardized algorithm description template that uses well-defined topics that motivate the collection of specific and useful information about each algorithm described.
% report
This report...

\begin{description}
	\item[Keywords:] {\small\texttt{Clever, Algorithms, Visualization}}
\end{description} 

% summarise the document breakdown with cross references
\section{Introduction}
\label{sec:introduction}
The Clever Algorithms project aims to describe a large number of algorithms from the fields of Computational Intelligence, Biologically Inspired Computation, and Metaheuristics in a complete, consistent and centralized manner \cite{Brownlee2010}.
% description
The project requires all algorithms to be described using a standardized template that includes a fixed number of sections, each of which is motivated by the presentation of specific information about the technique \cite{Brownlee2010a}.

% this report
This report...

the idea is to provide the tools and examples for visualizing algorithms used in book

all examples provided in gnu plot. it's focused, simple ad accessible.... also free

provide script examples as well as images


%
% Problems
%
\section{Problems}
should be the first thing
consider at reduced dimensions
consider at different scales, focused regions

problem specific

\subsection{Continuous Function Optimization}

plot the surface


\subsection{Traveling Salesman Problem}

plot the domain

plot optimal solutions


%
% Performance
%
\section{Performance}
easy, first thing

\subsection{Single Algorithm Run}
line graphs

TODO: give example with a GA

\subsection{Multiple Algorithm Run}
same algorithm, different random seed

compare using boxplots generally

TODO: give example with a GA, ES, EP, ...


%
% Candidate Solutions
%
\section{Candidate Solutions}
asdasd

problem specific

\subsection{Continuous Function Optimization}
todo
on the plot

\subsection{Traveling Salesman Problem}
todo

%
% Tools
%
\section{Visualization Methods}
asdasd

\subsection{Tools}
first step should be to use a specailized tool if available

\subsection{GNU Plot}
focused on plotting

great for performance graphics and surfaces

\subsection{R Project}
stats methods and visualization


\subsection{Libraries}
write code if you must, but use a lib

may, for example, some used in java, ruby etc

open source physics, jfreegraph, jung, etc


\subsection{DIY}
last resort, better to use a lib


%
% Conclusions
%
\section{Conclusions}
\label{sec:conclusions}
This repor...



% bibliography
\bibliographystyle{plain}
\bibliography{../bibtex}

\end{document}
% EOF
