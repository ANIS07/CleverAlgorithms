% Algorithm Description: Random Search

% The Clever Algorithms Project: http://www.CleverAlgorithms.com
% (c) Copyright 2010 Jason Brownlee. Some Rights Reserved. 
% This work is licensed under a Creative Commons Attribution-Noncommercial-Share Alike 2.5 Australia License.

\documentclass[a4paper, 11pt]{article}
\usepackage{tabularx}
\usepackage{booktabs}
\usepackage{url}
\usepackage[pdftex,breaklinks=true,colorlinks=true,urlcolor=blue,linkcolor=blue,citecolor=blue,]{hyperref}
\usepackage{geometry}
\usepackage[ruled, linesnumbered]{../algorithm2e}
\usepackage{listings} 
\usepackage{textcomp}
\ifx\pdfoutput\@undefined\usepackage[usenames,dvips]{color}
\else\usepackage[usenames,dvipsnames]{color}
\lstset{basicstyle=\footnotesize\ttfamily,numbers=left,numberstyle=\tiny,frame=single,columns=flexible,upquote=true,showstringspaces=false,tabsize=2,captionpos=b,breaklines=true,breakatwhitespace=true,keywordstyle=\color{blue},stringstyle=\color{ForestGreen}}
\geometry{verbose,a4paper,tmargin=25mm,bmargin=25mm,lmargin=25mm,rmargin=25mm}

% Dear template user: fill these in
\newcommand{\myreporttitle}{Algorithm Description}
\newcommand{\myreportsubtitle}{Random Search}
\newcommand{\myreportauthor}{Jason Brownlee}
\newcommand{\myreportemail}{jasonb@CleverAlgorithms.com}
\newcommand{\myreportproject}{The Clever Algorithms Project\\\url{http://www.CleverAlgorithms.com}}
\newcommand{\myreportdate}{20100128}
\newcommand{\myreportversion}{1}
\newcommand{\myreportlicense}{\copyright\ Copyright 2010 Jason Brownlee. Some Rights Reserved. This work is licensed under a Creative Commons Attribution-Noncommercial-Share Alike 2.5 Australia License.}

% leave this alone, it's templated baby!
\title{{\myreporttitle}: {\myreportsubtitle}\footnote{\myreportlicense}}
\author{\myreportauthor\\{\myreportemail}\\\small\myreportproject}
\date{\today\\{\small{Technical Report: CA-TR-{\myreportdate}-\myreportversion}}}
\begin{document}
\maketitle

% write a summary sentence for each major section
\section*{Abstract} 
% project
The Clever Algorithms project aims to describe a large number of Artificial Intelligence algorithms in a complete, consistent and centralized manner.
% algorithm
This report describes the Random Search algorithm using a standardized algorithm description template.
% outcomes
As the first algorithm description in the project, a few modifications are suggested for the standardized description template and some bounds are suggested for the broader project ideals.

\begin{description}
	\item[Keywords:] {\small\texttt{Algorithm, Description, Random, Search, Optimization}}
\end{description} 

\section{Introduction} 
\label{sec:intro}
% project
The Clever Algorithms project aims to describe a large number of algorithms from the fields of Computational Intelligence, Biologically Inspired Computation, and Metaheuristics in a complete, consistent and centralized manner \cite{Brownlee2010}.
% description
The project requires all algorithms to be described using a standardized template that includes a fixed number of sections, each of which is motivated by the collection of specific information about the technique \cite{Brownlee2010a}.
% this report
This report describes the `Random Search' algorithm using the standardized algorithm description template. 
% conclusions 
Section~\ref{sec:conclusions} summarizes some changes to the standardized description template, comments on the limits the projects ideals, and highlights some related algorithms that may be considered for inclusion in the project.

% Name
% The algorithm name defines the canonical name used to refer to the technique, in addition to common aliases, abbreviations, and acronyms. The name is used in terms of the heading and sub-headings of an algorithm description.
\section{Name} 
\label{sec:name}
% What is the canonical name and common aliases for a technique?
% What are the common abbreviations and acronyms for a technique?
% The heading and alternate headings for the algorithm description.
Random Search, RS, Blind Random Search, Blind Search, Pure Random Search, PRS.

% Taxonomy: Lineage and locality
% The algorithm taxonomy defines where a techniques fits into the field, both the specific subfields of Computational Intelligence and Biologically Inspired Computation as well as the broader field of Artificial Intelligence. The taxonomy also provides a context for determining the relation- ships between algorithms. The taxonomy may be described in terms of a series of relationship statements or pictorially as a venn diagram or a graph with hierarchical structure.
\section{Taxonomy}
\label{sec:taxonomy}
% To what fields of study does a technique belong?
Random search belongs to the fields of Stochastic Optimization and Global Optimization.
% type
Random search is also called a direct search method as it does not require derivatives to search a continuous domain.
% What are the closely related approaches to a technique?
This base approach is related to techniques that provide small improvements such as Directed Random Search, and Adaptive Random Search. 

% Inspiration: Motivating system
% The inspiration describes the specific system or process that provoked the inception of the algorithm. The inspiring system may non-exclusively be natural, biological, physical, or social. The description of the inspiring system may include relevant domain specific theory, observation, nomenclature, and most important must include those salient attributes of the system that are somehow abstractly or conceptually manifest in the technique. The inspiration is described textually with citations and may include diagrams to highlight features and relationships within the inspiring system.
% Optional
\section{Inspiration}
\label{sec:inspiration}
% What is the system or process that motivated the development of a technique?
% Which features of the motivating system are relevant to a technique?
N/A

% Metaphor: Explanation via analogy
% The metaphor is a description of the technique in the context of the inspiring system or a different suitable system. The features of the technique are made apparent through an analogous description of the features of the inspiring system. The explanation through analogy is not expected to be literal scientific truth, rather the method is used as an allegorical communication tool. The inspiring system is not explicitly described, this is the role of the ‘inspiration’ element, which represents a loose dependency for this element. The explanation is textual and uses the nomenclature of the metaphorical system.
% Optional
\section{Metaphor}
\label{sec:metaphor}
% What is the explanation of a technique in the context of the inspiring system?
% What are the functionalities inferred for a technique from the analogous inspiring system?
N/A

% Strategy: Problem solving plan
% The strategy is an abstract description of the computational model. The strategy describes the information processing actions a technique shall take in order to achieve an objective. The strategy provides a logical separation between a computational realization (procedure) and a analogous system (metaphor). A given problem solving strategy may be realized as one of a number specific algorithms or problem solving systems. The strategy description is textual using information processing and algorithmic terminology.
\section{Strategy}
\label{sec:strategy}
% What is the information processing objective of a technique?
% What is a techniques plan of action?
The strategy of Random Search is to sample solutions using a uniform probability distribution from across the entire search space. Each future sample is independent of the samples that come before it.

% Procedure: Abstract computation
% The algorithmic procedure summarizes the specifics of realizing a strategy as a systemized and parameterized computation. It outlines how the algorithm is organized in terms of the data structures and representations. The procedure may be described in terms of software engineering and computer science artifacts such as pseudo code, design diagrams, and relevant mathematical equations.
\section{Procedure}
\label{sec:procedure}
% What is the computational recipe for a technique?
% What are the data structures and representations used in a technique?
Algorithm~\ref{alg:random_search} provides a pseudo-code listing of the Random Search Algorithm for minimizing a cost function.

\begin{algorithm}[htp]
	\SetLine
	% data
	\SetKwData{NumIterations}{NumIterations}
	\SetKwData{ProblemSize}{ProblemSize}
	\SetKwData{Best}{Best}
	\SetKwData{Null}{Null}
	% functions
	\SetKwFunction{Cost}{Cost}
	\SetKwFunction{RandomSolution}{RandomSolution}
  	% I/O
	\KwIn{\NumIterations, \ProblemSize}
	\KwOut{\Best}
  	% Algorithm
	\Best $\leftarrow 0$\;
	\ForEach{$iter_i \in$ \NumIterations} {
		$candidate_i$ = \RandomSolution{\ProblemSize}\;
		\If{\Cost{$candidate_i$} $<$ \Cost{\Best}} {
			\Best $\leftarrow$ $candidate_i$\;
		}
	}
	\Return{\Best}\;
	% caption
	\caption{Pseudo Code Listing for the Random Search Algorithm.}
	\label{alg:random_search}
\end{algorithm}

% Heuristics: Usage guidelines
% The heuristics element describe the commonsense, best practice, and demonstrated rules for applying and configuring a parameterized algorithm. The heuristics relate to the technical details of the techniques procedure and data structures for general classes of application (neither specific implementations not specific problem instances). The heuristics are described textually, such as a series of guidelines in a bullet-point structure.
\section{Heuristics}
\label{sec:heuristics}
% What are the suggested configurations for a technique?
% What are the guidelines for the application of a technique to a problem instance?
\begin{itemize}
	\item Random search is minimal in that it only requires a candidate solution construction routine and a candidate solution evaluation routine.
	\item Random Search is a default technique to try before any other heuristic method to calibrate solution construction and evaluation.
	\item The worst case performance for Random Search is worse than an Enumeration of the search domain ($O(n)$), given that Random Search has no memory and can blindly resample.
	\item Random Search can return a reasonable approximation of the optimal solution within a reasonable time under low problem dimensionality, although the approach does not scale well. (references???) - the number of samples to take of the domain may be estimated to provide assurances about information coverage in the domain (probabilistic model???)
	\item Care must be taken with some problem domains to ensure that random candidate solution construction is in fact uniformly random (unbiased)
	\item The Random Search algorithm is the parent of all stochastic optimization methods that may be seen to introduce models that bias the sampling of the domain away from a uniform probability, and potentially introduce sample rejection criteria.
	\item The results of a Random Search can be used to seed another search technique, like a local search technique (such as a hill climbing algorithm) that can be used to locate the best solution in the neighborhood of the candidate solution.
\end{itemize}

% The code description provides a minimal but functional version of the technique implemented with a programming language. The code description must be able to be typed into an appropriate computer, compiled or interpreted as need be, and provide a working execution of the technique. The technique implementation also includes a minimal problem instance to which it is applied, and both the problem and algorithm implementations are complete enough to demonstrate the techniques procedure. The description is presented as a programming source code listing.
\section{Code Listing}
\label{sec:code}
% How is a technique implemented as an executable program?
% How is a technique applied to a concrete problem instance?
Listing~\ref{random_search} provides an example of the Random Search Algorithm implemented in the Ruby Programming Language. 
% about the algorithm
In the example, the algorithm runs for a fixed number of iterations and returns the best candidate solution discovered. Random candidate solution construction involves creating a vector in the problem space one dimension at a time.
% about the problem
The example problem is an instance of a continuous function optimization that seeks $min f(x)$ where $f=\sum_{i=1}^n x_{i}^2$, $-5.0\leq x_i \leq 5.0$ and $n=2$. The optimal solution for this basin function is $(v_0,\ldots v_{n-1})=0.0$.

% the listing
\lstinputlisting[firstline=7,language=ruby,caption=Random Search Algorithm in the Ruby Programming Language, label=random_search]{../../src/algorithms/stochastic/random_search.rb}

% References: Deeper understanding
% The references element description includes a listing of both primary sources of information about the technique as well as useful introductory sources for novices to gain a deeper understanding of the theory and application of the technique. The description consists of hand-selected reference material including books, peer reviewed conference papers, journal articles, and potentially websites. A bullet-pointed structure is suggested.
\section{References}
\label{sec:references}
% What are the primary sources for a technique?
% What are the suggested reference sources for learning more about a technique?

% 
% Primary Sources
% 
\subsection{Primary Sources}
There is no seminal specification of the Random Search algorithm, rather there are discussions of the general approach and related random search methods from the 1950's 60's and 70's around the time that pattern and direct search methods were actively researched.
% seminal
Brooks is credited with the so-called `pure random search' \cite{Brooks1958}. Some seminal reviewing `random search methods' of the time, include: Kul'chitskii \cite{Kul'chitskii1976} and Karnopp \cite{Karnopp1963}.


% 
% Learn More
% 
\subsection{Learn More}
% random search methods
For overviews of into Random Search Methods see Zhigljavsky \cite{Zhigljavsky1991}, Solis and Wets Solis \cite{Solis1981}, and White \cite{White1971} who provides an excellent review article.
% stochastic
Spall provides a detailed overview of the field of Stochastic Optimization, including the Random Search method \cite{Spall2003} (for example, see Chapter 2). For a shorter introduction by Spall, see \cite{Spall2004} (specifically Section 6.2). Also see Zabinsky for another detailed review of the broader field \cite{Zabinsky2003}.


% 
% Conclusions: What the reader or what thre author learned by completing this this report.
% 
\section{Conclusions}
\label{sec:conclusions}
% technique
This report introduced the Random Search technique as a baseline procedure suitable a default technique for a difficult problem, and a minimal method suitable for calibrating solution construction and evaluation procedures.
% template
The description of this technique did not cover inspiration and metaphor as these topics were not relevant to the algorithm. The description deviated from the standardized template in that a \emph{code listing} was provided with a brief description, instead of a \emph{tutorial} \cite{Brownlee2010a}. It is believed that a minimal code listing and small context-setting description are more suitable for the Clever Algorithms project, and this change is suggested for all algorithm descriptions to be prepared for the Clever Algorithms project. 

% focus
It is apparent that the Clever Algorithms project takes a very practical approach to algorithm description, not directly considering mathematical descriptions of algorithms or the expected behavior.  
% bounded
This further highlights that the ideals of the Clever Algorithms project are in fact bounded by default \cite{Brownlee2010}. Specifically, the project will describe algorithms in a complete, consistent and centralized manner, although the descriptions will be `complete enough' for a practitioner to understand and use them, they will be `consistent enough' to ensure continuity between algorithm descriptions, and the descriptions will be `centralized enough' to be consumed in a book and website format. These clear bounds on the project ideals must be specified in the introductory information provided with the descriptions, such as the introduction chapter of the book.

% related
Random Search is closely related to \emph{Localized Random Search} (use of a step size), \emph{Adaptive Random Search} (use of a adaptive step size), and \emph{Directed Random Search} (sometimes a synonym for Localized and/or Adaptive Random Search) techniques, all of which appear in the data-driven algorithm selection results, and some of which have been selected for inclusion in the Clever Algorithms project \cite{Brownlee2010b}.
Some additional techniques uncovered during research include: \emph{Controlled Random Search} \cite{Price1977}, \emph{Creeping Random Search} \cite{Rastrigin1963}.

% bibliography
\bibliographystyle{plain}
\bibliography{../bibtex}

\end{document}
% EOF