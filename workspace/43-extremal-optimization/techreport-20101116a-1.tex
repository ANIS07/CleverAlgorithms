% Extremal Optimization

% The Clever Algorithms Project: http://www.CleverAlgorithms.com
% (c) Copyright 2010 Jason Brownlee. Some Rights Reserved. 
% This work is licensed under a Creative Commons Attribution-Noncommercial-Share Alike 2.5 Australia License.

\documentclass[a4paper, 11pt]{article}
\usepackage{tabularx}
\usepackage{booktabs}
\usepackage{url}
\usepackage[pdftex,breaklinks=true,colorlinks=true,urlcolor=blue,linkcolor=blue,citecolor=blue,]{hyperref}
\usepackage{geometry}
\usepackage[ruled, linesnumbered]{../algorithm2e}
\usepackage{listings} 
\usepackage{textcomp}
\ifx\pdfoutput\@undefined\usepackage[usenames,dvips]{color}
\else\usepackage[usenames,dvipsnames]{color}
\lstset{basicstyle=\footnotesize\ttfamily,numbers=left,numberstyle=\tiny,frame=single,columns=flexible,upquote=true,showstringspaces=false,tabsize=2,captionpos=b,breaklines=true,breakatwhitespace=true,keywordstyle=\color{blue},stringstyle=\color{ForestGreen}}
\geometry{verbose,a4paper,tmargin=25mm,bmargin=25mm,lmargin=25mm,rmargin=25mm}

% Dear template user: fill these in
\newcommand{\myreporttitle}{Extremal Optimization}
\newcommand{\myreportauthor}{Jason Brownlee}
\newcommand{\myreportemail}{jasonb@CleverAlgorithms.com}
\newcommand{\myreportwebsite}{http://www.CleverAlgorithms.com}
\newcommand{\myreportproject}{The Clever Algorithms Project\\\url{\myreportwebsite}}
\newcommand{\myreportdate}{20101116a}
\newcommand{\myreportversion}{1}
\newcommand{\myreportlicense}{\copyright\ Copyright 2010 Jason Brownlee. Some Rights Reserved. This work is licensed under a Creative Commons Attribution-Noncommercial-Share Alike 2.5 Australia License.}

% leave this alone, it's templated baby!
\title{{\myreporttitle}\footnote{\myreportlicense}}
\author{\myreportauthor\\{\myreportemail}\\\small\myreportproject}
\date{\today\\{\small{Technical Report: CA-TR-{\myreportdate}-\myreportversion}}}
\begin{document}
\maketitle

% write a summary sentence for each major section
\section*{Abstract}
% project
The Clever Algorithms project aims to describe a large number of Artificial Intelligence algorithms in a complete, consistent, and centralized manner, to improve their general accessibility. 
% template
The project makes use of a standardized algorithm description template that uses well-defined topics that motivate the collection of specific and useful information about each algorithm described.
% report
This report describes the Extremal Optimization algorithm using the standardized algorithm template.

\begin{description}
	\item[Keywords:] {\small\texttt{Clever, Algorithms, Description, Optimization, Extremal}}
\end{description} 

\section{Introduction} 
\label{sec:intro}
% project
The Clever Algorithms project aims to describe a large number of algorithms from the fields of Computational Intelligence, Biologically Inspired Computation, and Metaheuristics in a complete, consistent and centralized manner \cite{Brownlee2010}.
% description
The project requires all algorithms to be described using a standardized template that includes a fixed number of sections, each of which is motivated by the presentation of specific information about the technique \cite{Brownlee2010a}.
% this report
This report describes the Extremal Optimization algorithm using the standardized algorithm template.

% Name
% The algorithm name defines the canonical name used to refer to the technique, in addition to common aliases, abbreviations, and acronyms. The name is used in terms of the heading and sub-headings of an algorithm description.
\section{Name} 
\label{sec:name}
% What is the canonical name and common aliases for a technique?
% What are the common abbreviations and acronyms for a technique?
% The heading and alternate headings for the algorithm description.
Extremal Optimization, EO

% Taxonomy: Lineage and locality
% The algorithm taxonomy defines where a techniques fits into the field, both the specific subfields of Computational Intelligence and Biologically Inspired Computation as well as the broader field of Artificial Intelligence. The taxonomy also provides a context for determining the relation- ships between algorithms. The taxonomy may be described in terms of a series of relationship statements or pictorially as a venn diagram or a graph with hierarchical structure.
\section{Taxonomy}
\label{sec:taxonomy}
% To what fields of study does a technique belong?
Extremal Optimization is a stochastic search technique that has properties of being a local and global search method.
% What are the closely related approaches to a technique?
It is generally related to hill climbing algorithms and provides the basis for extensions such as Generalized Extremal Optimization.

% Inspiration: Motivating system
% The inspiration describes the specific system or process that provoked the inception of the algorithm. The inspiring system may non-exclusively be natural, biological, physical, or social. The description of the inspiring system may include relevant domain specific theory, observation, nomenclature, and most important must include those salient attributes of the system that are somehow abstractly or conceptually manifest in the technique. The inspiration is described textually with citations and may include diagrams to highlight features and relationships within the inspiring system.
% Optional
\section{Inspiration}
\label{sec:inspiration}
% What is the system or process that motivated the development of a technique?
Extremal Optimization is inspired by the Bak-Sneppen self-organized criticality model of co-evolution from the field of statistical physics. 
% Which features of the motivating system are relevant to a technique?
The self-organized criticality model suggests that some dynamical systems have a critical point as an attractor, whereby the systems exhibit periods of slow movement or accumulation followed by short periods of avalanche or instability. Examples of such systems include land formation, earthquakes, and the dynamics of sand piles. The Bak-Sneppen considers such dynamics in co-evolutionary systems and the description of punctuated equilibrium (long periods of status followed by short short periods of extension and large evolutionary change).

% Metaphor: Explanation via analogy
% The metaphor is a description of the technique in the context of the inspiring system or a different suitable system. The features of the technique are made apparent through an analogous description of the features of the inspiring system. The explanation through analogy is not expected to be literal scientific truth, rather the method is used as an allegorical communication tool. The inspiring system is not explicitly described, this is the role of the ‘inspiration’ element, which represents a loose dependency for this element. The explanation is textual and uses the nomenclature of the metaphorical system.
% Optional
\section{Metaphor}
\label{sec:metaphor}
% What is the explanation of a technique in the context of the inspiring system?
% What are the functionalities inferred for a technique from the analogous inspiring system?
todo

% Strategy: Problem solving plan
% The strategy is an abstract description of the computational model. The strategy describes the information processing actions a technique shall take in order to achieve an objective. The strategy provides a logical separation between a computational realization (procedure) and a analogous system (metaphor). A given problem solving strategy may be realized as one of a number specific algorithms or problem solving systems. The strategy description is textual using information processing and algorithmic terminology.
\section{Strategy}
\label{sec:strategy}
% What is the information processing objective of a technique?
THe objective of the information processing strategy is to iteratively identify the worst performing components of a given solution and replace or swap them with other components.
% What is a techniques plan of action?
Through the extended application of this simple heuristic, broader dynamics arise that result in the steady improvement of a candidate solution with sudden and large crashes in the quality of the candidate solution. These dynamics allow the heuristic to explore the search space, exploit higher quality solutions, then escape possible local optima.

% Procedure: Abstract computation
% The algorithmic procedure summarizes the specifics of realizing a strategy as a systemized and parameterized computation. It outlines how the algorithm is organized in terms of the data structures and representations. The procedure may be described in terms of software engineering and computer science artifacts such as pseudo code, design diagrams, and relevant mathematical equations.
\section{Procedure}
\label{sec:procedure}
% What is the computational recipe for a technique?
% What are the data structures and representations used in a technique?
Algorithm~\ref{alg:eo} provides a pseudo-code listing of the main Extremal Optimization algorithm for minimizing a cost function.

\begin{algorithm}[ht]
	\SetLine  

	% data
	\SetKwData{ProblemSize}{ProblemSize}
	\SetKwData{MaxIterations}{$iterations_{max}$}
	\SetKwData{InitialTemperature}{$temp_{max}$}
	\SetKwData{Temp}{$temp_{curr}$}
	\SetKwData{ProblemSize}{ProblemSize}
	\SetKwData{Best}{$S_{best}$}
	\SetKwData{Current}{$S_{current}$}
	\SetKwData{Candidate}{$S_{i}$}
	
	% functions
	\SetKwFunction{Cost}{Cost}
	\SetKwFunction{Rand}{Rand}
	\SetKwFunction{StopCondition}{StopCondition}
	\SetKwFunction{CreateInitialSolution}{CreateInitialSolution}
	\SetKwFunction{CreateNeighborSolution}{CreateNeighborSolution}
	\SetKwFunction{CalculateTemperature}{CalculateTemperature}
	
	% I/O
	\KwIn{\ProblemSize, \MaxIterations, \InitialTemperature}		
	\KwOut{\Best}
  
	% Algorithm
	\Current $\leftarrow$ \CreateInitialSolution{\ProblemSize}\;
	\Best $\leftarrow$ \Current\;
	% loop
	\For{$i=1$ \KwTo \MaxIterations} {
		\Candidate $\leftarrow$ \CreateNeighborSolution{\Current}\;
		\Temp $\leftarrow$ \CalculateTemperature{$i$, \InitialTemperature}\;
		\uIf{\Cost{\Candidate} $\leq$ \Cost{\Current}} {
			\Current $\leftarrow$ \Candidate\;
			\If{\Cost{\Candidate} $\leq$ \Cost{\Best}} {
				\Best $\leftarrow$ \Candidate\;
			}
		}
		\ElseIf{$\exp(\frac{\Cost{\Current}-\Cost{\Candidate}}{\Temp})$ $>$ \Rand{}} {
			\Current $\leftarrow$ \Candidate\;
		}
	}
	\Return{\Best}\;
	% end
	\caption{Pseudo Code for the Extremal Optimization algorithm.}
	\label{alg:eo}
\end{algorithm}

% Heuristics: Usage guidelines
% The heuristics element describe the commonsense, best practice, and demonstrated rules for applying and configuring a parameterized algorithm. The heuristics relate to the technical details of the techniques procedure and data structures for general classes of application (neither specific implementations not specific problem instances). The heuristics are described textually, such as a series of guidelines in a bullet-point structure.
\section{Heuristics}
\label{sec:heuristics}
% What are the suggested configurations for a technique?
% What are the guidelines for the application of a technique to a problem instance?
\begin{itemize}
	\item Extremal Optimization was designed for combinatorial optimization problems, although variations have been applied to continuous function optimization.
	\item ...
\end{itemize}

% The code description provides a minimal but functional version of the technique implemented with a programming language. The code description must be able to be typed into an appropriate computer, compiled or interpreted as need be, and provide a working execution of the technique. The technique implementation also includes a minimal problem instance to which it is applied, and both the problem and algorithm implementations are complete enough to demonstrate the techniques procedure. The description is presented as a programming source code listing.
\section{Code Listing}
\label{sec:code}
% How is a technique implemented as an executable program?
% How is a technique applied to a concrete problem instance?
Listing~\ref{extremal_optimization} provides an example of the Extremal Optimization algorithm implemented in the Ruby Programming Language. 
% problem
The algorithm is applied to the Berlin52 instance of the Traveling Salesman Problem (TSP), taken from the TSPLIB. The problem seeks a permutation of the order to visit cities (called a tour) that minimized the total distance traveled. The optimal tour distance for Berlin52 instance is 7542 units.

% algorithm
The algorithm implementation...

% the listing
\lstinputlisting[firstline=7,language=ruby,caption=Extremal Optimization algorithm in the Ruby Programming Language, label=extremal_optimization]{../../src/algorithms/physical/extremal_optimization.rb}


% References: Deeper understanding
% The references element description includes a listing of both primary sources of information about the technique as well as useful introductory sources for novices to gain a deeper understanding of the theory and application of the technique. The description consists of hand-selected reference material including books, peer reviewed conference papers, journal articles, and potentially websites. A bullet-pointed structure is suggested.
\section{References}
\label{sec:references}
% What are the primary sources for a technique?
% What are the suggested reference sources for learning more about a technique?

% 
% Primary Sources
% 
\subsection{Primary Sources}
% seminal
Extremal Optimization was proposed as an optimization heuristic by Boettcher and Percus applied to graph partitioning and the Traveling Salesman Problem \cite{Boettcher1999}. The approach was inspired by the Bak-Sneppen self-organized criticality model of co-evolution \cite{Bak1987, Bak1993}.

% 
% Learn More
% 
\subsection{Learn More}
% reviews
A number of detailed reviews of Extremal Optimization have been presented, including a review and studies by Boettcher and Percus \cite{Boettcher2000}, an accessible review by Boettcher \cite{Boettcher2000a}, and a focused study on the Spin Glass problem by Boettcher and Percus \cite{Boettcher2001}.
% books


% 
% Conclusions: What the reader or what thre author learned by completing this this report.
% 
\section{Conclusions}
\label{sec:conclusions}
% report
This report described the Extremal Optimization algorithm using the standardized algorithm template.

% 
% Contribute
% 
\section{Contribute}
\label{sec:contribute}
% simple
Found a typo in the content or a bug in the source code? 
% advanced 
Are you an expert in this technique and know some facts that could improve the algorithm description for all?
% incentive
Do you want to get that warm feeling from contributing to an open source project? 
Do you want to see your name as an acknowledgment in print?

%  ideal
Two pillars of this effort are i) that the best domain experts are people outside of the project, and ii) that this work is subjected to continuous improvement. 
% advice
Please help to make this work less wrong by emailing the author `\myreportauthor' at \url{\myreportemail} or visit the project website at \url{\myreportwebsite}.

% bibliography
\bibliographystyle{plain}
\bibliography{../bibtex}

\end{document}
% EOF