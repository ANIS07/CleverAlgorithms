% project overview
% Copyright (C) 2010 Jason Brownlee
% This work is licensed under a Creative Commons Attribution-Noncommercial-Share Alike 2.5 Australia License.

\documentclass[a4paper, 11pt]{article}
\usepackage{url}
\usepackage[pdftex,breaklinks=true,colorlinks=true,urlcolor=blue,linkcolor=blue,citecolor=blue,]{hyperref}

% Fill these in
\newcommand{\myreporttitle}{Clever Algorithms}
\newcommand{\myreportsubtitle}{Project Overview}
\newcommand{\myreportname}{Jason Brownlee}
\newcommand{\myreportemail}{jasonb@CleverAlgorithms.com}
\newcommand{\myreportproject}{\url{http://www.CleverAlgorithms.com}}
\newcommand{\myreportdate}{20090105}

% todo add 'Clever Algorithms'
% todo setup clever algorithms email
% todo setup header and footer - copyright, license, etc

\title{{\myreporttitle}: {\myreportsubtitle}\\{\normalsize\myreportproject}}
\author{\myreportname\\\myreportemail}
\date{Technical Report: CA-TR-\myreportdate}
\begin{document}
\maketitle

\section*{Abstract} 
\label{sec:abstract}
\emph{This is where the abstract goes.}
\\
\textbf{Keywords}: \texttt{Clever, Algorithms, Project, Overview, Motivation}

\section{Introduction}
\label{sec:introduction}

this report provides an overview of the clever algorithms project

\section{Motivation}
This section motivates the Clever Algorithms Project.

\subsection{Problem}
Artificial Intelligence is a very large field of study and the description and investigation of algorithmic techniques is a central pursuit in subfields such as Machine Learning, Biologically Inspired Computation, and Computational Intelligence. There is a critical open problem in the field of Artificial Intelligence: communication. The fields of Biologically Inspired Computation and Computational Intelligence are of particular concern because the techniques they are concerned with are typically heuristic procedures based in metaphor making them susceptible to description my analogy and metaphorical nomenclature.

\textbf{Open Problem}: \emph{The communication of algorithmic techniques in the fields of Biologically Inspired Computation and Computational Intelligence is a difficult open problem.}

\begin{itemize}
	\item The description of techniques are typically \emph{incomplete}. Many techniques are ambiguously described, partially described, or not described at all.
	\item The description of techniques are typically \emph{inconsistent}. A given technique may be described using a variety of formal and semi-formal methods that also vary across different techniques limiting the transferability of the skills an audience used to realize a technique (such as mathematics, pseudo code, program code, and narratives). An inconsistent representation for techniques mean that the skills used to realize one technique may not be transferable to realizing other techniques or even updated versions of the same technique.
	\item The description of techniques are typically \emph{distributed}. The description of data structures, operations, and parameterization of a given technique may span an array of papers, articles, books, and source code published over a number of years, the access to some of which may be restricted or difficult to obtain.
\end{itemize}

\subsection{Affect}
For the individual, an ill described algorithm may be a frustration where the gaps are filled with intuition and `best guess' or they may an example of bad science and the failure of the scientific method where the inability to understand and implement a technique may prevent the replication of results or the investigation and extension of a technique. 

\textbf{General Affect}: \emph{Poorly described algorithmic techniques in the fields of Biologically Inspired Computation and Computational Intelligence damage those fields.}

\begin{itemize}
	\item Poorly described techniques result in \emph{inconsistent interpretation}. A field of study is generally concerned with building a corpus of knowledge the momentum of which is dependent on a common shared understanding. Diversity of the understanding provided through the effort required for the continued reinterpretation of an approach may promote a deeper understanding, although forward progress may become difficult. Without a consistent understanding of a technique it cannot be accurately compared, broadly and concurrently investigated, or ultimately make meaningful contributions to the broader field. 
	\item Poorly described techniques result in \emph{undue attrition}. Techniques that can only be realistically implemented by the original author may as well not exist, regardless of related claims of efficiency or efficacy. Practitioners must sift through volumes of papers, articles, books, and sample source code to formulate a viable interpretation of a given technique. This investment of effort is required for each practitioner and each technique resulting in the vast majority of published approaches disappearing into obscurity. If a practitioner cannot realize a described the technique it will likely be lost to continuous and ruthless attrition as the practitioners and move onto those techniques that can realize.
	\item Poorly described techniques result in \emph{bad science}. In a scientific setting, an algorithmic procedure is investigated both theoretically as an abstraction and empirically as an implementation. As such, investigation of an algorithmic technique using the scientific method requires an unambiguous description of the technique. Those algorithms that cannot be clearly communicated, cannot be subjected to broader study, the results cannot be reproduced, and the approach cannot be applied, investigated, or extended. 
\end{itemize}

\section{Clever Algorithms Project}
This section describes the objectives of the Clever Algorithms Project.

\subsection{The Solution}
consistent clear, unambiguous presentation of lots of algorithms together
in a book, random access, wide access, web access, free access

\subsection{Objectives}

same rep is not for all
diversity is good for progress

what exactly does this entail?

\begin{itemize}
	\item \textbf{Completeness}
	\item \textbf{Consistency}
	\item \textbf{Centralization}
\end{itemize}

\begin{itemize}
	\item \textbf{Complete Description}
	\item \textbf{Unambiguous Representation}
	\item \textbf{Centralized Corpus}
	\item \textbf{Compact}
	\item \textbf{Directly Usable}
\end{itemize}

\section{Audience}
\label{sec:audience}
who is the book for?

who is the book for 

\begin{itemize}
	\item \textbf{Research Scientists}: asdf
	\item \textbf{Developers}: asdf
	\item \textbf{Students}: asdf
	\item \textbf{Interested Amateurs}: asdf
\end{itemize}


\section{Project Home}
the where?

\section{Methodology}
\label{sec:methodology}
how is this book going to get done?


\section{Conclusions}
\label{sec:conclusions}
what was this all about, what is next, what big questions and problems does this raise?


follow ups
- need to define fields of interest
- need to support open problem
- need to show technique development process in the feild


\end{document}
