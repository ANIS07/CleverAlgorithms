% Book Pitch
% Copyright (C) 2010 Jason Brownlee
% This work is licensed under a Creative Commons Attribution-Noncommercial-Share Alike 2.5 Australia License.

\documentclass[a4paper, 11pt]{article}
\usepackage{url}
\usepackage[pdftex,breaklinks=true,colorlinks=true,urlcolor=blue,linkcolor=blue,citecolor=blue,]{hyperref}

% Fill these in
\newcommand{\myreporttitle}{Clever Algorithms}
\newcommand{\myreportsubtitle}{Project Overview}
\newcommand{\myreportname}{Jason Brownlee}
\newcommand{\myreportemail}{jasonb@CleverAlgorithms.com}
\newcommand{\myreportproject}{\url{http://www.CleverAlgorithms.com}}

% todo add 'tech report' also add 'Clever Algorithms'
% todo setup clever algorithms email
% todo setup tech report numbering system
% todo setup header and footer - copyright, license, etc

\title{{\myreporttitle}: {\myreportsubtitle}\\{\normalsize\myreportproject}}
\author{\myreportname\\\myreportemail}
\begin{document}
\maketitle

\section*{Abstract} 
\label{sec:abstract}
\emph{This is where the abstract goes.}
\\
\textbf{Keywords}: \texttt{Clever, Algorithms, Project, Overview, Motivation}

\section{Introduction}
\label{sec:introduction}

this report provides an overview of the clever algorithms project

\section{Motivation}
This section describes the motivation for the Clever Algorithms Project.

\subsection{The Problem}
There is a critical open problem in the field of Artificial Intelligence: communication. Generally, techniques are poorly described, the descriptions a distributed, spanning multiple documents, and the descriptions of a given technique and the descriptions across techniques are not consistent. 

Artificial Intelligence is a very large field of study, and the description and investigation of algorithmic techniques is a central pursuit in subfields such as Operations Research, Machine Learning, Biologically Inspired Computation, and Computational Intelligence. The latter two fields are exemplars for the open problem of algorithm communication, where an average conference proceeding will show algorithms described mathematically, using ad hoc pseudo code, real programming language code, and in some cases algorithms are only partially described or not described at all. 

\subsection{The Affect}
For the individual, an ill described algorithm may be a frustration where the gaps are filled with intuition and `best guess' or they may an example of bad science and the failure of the scientific method where the inability to understand and implement a technique may prevent the replication of results or the investigation and extension of a technique. 

In practice, developers and research scientists typically must sift through volumes of papers, articles, books, and sample source code before a viable interpretation of a given technique. Ideally, the presentation of an algorithm must be clear, unambiguous, and devoid of interpretation. This is generally difficult given the variability of skills in the author of the technique and in the audience interested in the technique. For example, an unambiguous mathematical description may not always be appropriate or possible.

It is true that once a technique is published it exists in the permanent record of scientific progress, although those techniques that cannot be deciphered due to poor communication may as well not have been published. If a practitioner or research scientist cannot realize a described the technique, the results cannot be reproduced, the approach cannot be exploited, investigated, or extended, and as such is lost to continuous and ruthless attrition as the practitioners and the scientists move onto those techniques that can realize.

\subsection{Summary}
The problem may be summarized as follows:
\begin{itemize}
	\item \textbf{Problem}: Poor communication of algorithmic techniques, specifically those from the fields of Biologically Inspired Computation and Computational Intelligence.
	\begin{itemize}
		\item \textbf{Incomplete Descriptions}
		\item \textbf{Inconsistent Descriptions}
		\item \textbf{Decentralized Descriptions}
	\end{itemize}
	\item \textbf{Affect}: 
	\begin{itemize}
		\item \textbf{Technique Interpretation}
		\item \textbf{Bad Science}		
		\item \textbf{Technique Attrition}: 
	\end{itemize}
\end{itemize}

\section{Clever Algorithms Project}
This section describes the objectives of the Clever Algorithms Project.

\subsection{The Solution}
consistent clear, unambiguous presentation of lots of algorithms together
in a book, random access, wide access, web access, free access

\subsection{Objectives}

what exactly does this entail?

\begin{itemize}
	\item \textbf{Complete Description}
	\item \textbf{Unambiguous Representation}
	\item \textbf{Centralized Corpus}
	\item \textbf{Compact}
	\item \textbf{Directly Usable}
\end{itemize}

\section{Audience}
\label{sec:audience}
who is the book for?

who is the book for 

\begin{itemize}
	\item \textbf{Research Scientists}: asdf
	\item \textbf{Developers}: asdf
	\item \textbf{Students}: asdf
	\item \textbf{Interested Amateurs}: asdf
\end{itemize}


\section{Project Home}
the where?

\section{Methodology}
\label{sec:methodology}
how is this book going to get done?


\section{Conclusions}
\label{sec:conclusions}
what was this all about, what is next, what big questions and problems does this raise?


\end{document}
