% project overview
% Copyright (C) 2010 Jason Brownlee
% This work is licensed under a Creative Commons Attribution-Noncommercial-Share Alike 2.5 Australia License.

\documentclass[a4paper, 11pt]{article}
\usepackage{url}
\usepackage[pdftex,breaklinks=true,colorlinks=true,urlcolor=blue,linkcolor=blue,citecolor=blue,]{hyperref}

% Fill these in
\newcommand{\myreporttitle}{Clever Algorithms}
\newcommand{\myreportsubtitle}{Project Overview}
\newcommand{\myreportname}{Jason Brownlee}
\newcommand{\myreportemail}{jasonb@CleverAlgorithms.com}
\newcommand{\myreportproject}{\url{http://www.CleverAlgorithms.com}}
\newcommand{\myreportdate}{20090105}

% todo add 'Clever Algorithms'
% todo setup clever algorithms email
% todo setup header and footer - copyright, license, etc

\title{{\myreporttitle}: {\myreportsubtitle}\\{\normalsize\myreportproject}}
\author{\myreportname\\\myreportemail}
\date{Technical Report: CA-TR-\myreportdate}
\begin{document}
\maketitle

\section*{Abstract} 
\label{sec:abstract}
\emph{This is where the abstract goes.}
\\
\textbf{Keywords}: \texttt{Clever, Algorithms, Project, Overview, Motivation}

\section{Introduction}
\label{sec:introduction}

this report provides an overview of the clever algorithms project

\section{Motivation}

\subsection{Problem}
Artificial Intelligence is a very large field of study and the description and investigation of algorithmic techniques is a central pursuit in subfields such as Machine Learning, Biologically Inspired Computation, and Computational Intelligence. There is a critical open problem in the field of Artificial Intelligence: communication. The fields of Biologically Inspired Computation and Computational Intelligence are of particular concern because the techniques they are concerned with are typically heuristic procedures based in metaphor making them susceptible to description my analogy and metaphorical nomenclature.

\textbf{Open Problem}: \emph{The communication of algorithmic techniques in the fields of Biologically Inspired Computation and Computational Intelligence is a difficult open problem.}

\begin{itemize}
	\item The description of techniques are typically \emph{incomplete}. Many techniques are ambiguously described, partially described, or not described at all.
	\item The description of techniques are typically \emph{inconsistent}. A given technique may be described using a variety of formal and semi-formal methods that also vary across different techniques limiting the transferability of the skills an audience used to realize a technique (such as mathematics, pseudo code, program code, and narratives). An inconsistent representation for techniques mean that the skills used to realize one technique may not be transferable to realizing other techniques or even updated versions of the same technique.
	\item The description of techniques are typically \emph{distributed}. The description of data structures, operations, and parameterization of a given technique may span an array of papers, articles, books, and source code published over a number of years, the access to some of which may be restricted or difficult to obtain.
\end{itemize}

\subsection{Affect}
For the individual, an ill described algorithm may be a frustration where the gaps are filled with intuition and `best guess' or they may an example of bad science and the failure of the scientific method where the inability to understand and implement a technique may prevent the replication of results or the investigation and extension of a technique. 

\textbf{General Affect}: \emph{Poorly described algorithmic techniques in the fields of Biologically Inspired Computation and Computational Intelligence damage those fields.}

\begin{itemize}
	\item Poorly described techniques result in \emph{inconsistent interpretation}. A field of study is generally concerned with building a corpus of knowledge the momentum of which is dependent on a common shared understanding. Diversity of the understanding provided through the effort required for the continued reinterpretation of an approach may promote a deeper understanding, although forward progress may become difficult. Without a consistent understanding of a technique it cannot be accurately compared, broadly and concurrently investigated, or ultimately make meaningful contributions to the broader field. 
	\item Poorly described techniques result in \emph{undue attrition}. Techniques that can only be realistically implemented by the original author may as well not exist, regardless of related claims of efficiency or efficacy. Practitioners must sift through volumes of papers, articles, books, and sample source code to formulate a viable interpretation of a given technique. This investment of effort is required for each practitioner and each technique resulting in the vast majority of published approaches disappearing into obscurity. If a practitioner cannot realize a described the technique it will likely be lost to continuous and ruthless attrition as the practitioners and move onto those techniques that can realize.
	\item Poorly described techniques result in \emph{bad science}. In a scientific setting, an algorithmic procedure is investigated both theoretically as an abstraction and empirically as an implementation. As such, investigation of an algorithmic technique using the scientific method requires an unambiguous description of the technique. Those algorithms that cannot be clearly communicated, cannot be subjected to broader study, the results cannot be reproduced, and the approach cannot be applied, investigated, or extended. 
\end{itemize}

\section{Clever Algorithms Project}

\subsection{Solution}
Algorithmic techniques that emerge from the fields of Computational Intelligence and Biologically Inspired Computation are interesting, surprising, and will potentially contribute alternative paradigms of computation to address the limits of more tradition approaches. 

\textbf{Observed Solution}: \emph{A strategy to address the open problem of poor technique communication is to present complete algorithms in a consistent manner in a centralized location.} 

Much effort has been put into this solution with the production of many books and libraries of program code. Algorithms from the fields of Computational Intelligence and Biologically Inspired Computation are more nuanced than algorithms from Operations Research for example. They are typically rooted in metaphor and explained via analogy. Although this may contribute to the technical ambiguity of their description, this is also a strength, provoking interest, imagination, creativity, and motivation in practitioners. 

The audience defines the completeness of an algorithm's description. Different audiences seek out and different descriptions with varied levels of rigor, from a theoretician that requires an unambiguous mathematical description, to an engineer who requires an implementable algorithmic procedure, to the algorithm designer interested in abstracting the strategy from the inspiring system. 

\subsection{Objectives}

The Clever Algorithms Project proposes a compendium algorithm descriptions. The primary objectives of the project are \emph{completeness}, \emph{consistence}, and \emph{centralization}:

\begin{itemize}
	\item \textbf{Completeness}: A well-defined template will be defined for describing an algorithm to a selected audience, each section of which will have a clear intention. Algorithms conforming to the template will be considered complete, and all algorithm descriptions listed in the compendium will conform to the proposed template.
	\item \textbf{Consistency}: The conformation of a collection of algorithm descriptions will consider the descriptions consistent, and all algorithm descriptions listed in the compendium will conform to the proposed template.
	\item \textbf{Centralization}: Algorithm descriptions may be accessible via multiple means, although will be managed at a central point of dissemination. The audience of the compendium will consume its content from a centralized point such as a book or a website.
\end{itemize}

The secondary objectives of the project are \emph{accessibility}, \emph{usability}, \emph{understandability}:

\begin{itemize}
	\item \textbf{Accessibility}: The algorithm descriptions will be widely accessible by the intended target audience both physically (such as electronically online and tangibly printed) and practically (such as loosely coupled descriptions amenable to electronic search and hoc access).
	\item \textbf{Usability}: The algorithm descriptions must be directly usable by the intended audience. The use of an algorithm description will be defined by a specific use case of a specific target audience, each section in the proposed template will address at least one specific use case of a specific target audience.
	\item \textbf{Understandability}: The algorithm descriptions will be in a structure and nomenclature suitable to be understood by the target audience and where appropriate written in the English language.
\end{itemize}

\subsection{Audience}
The audience for the The Clever Algorithms Project are practitioners somehow concerned or interested with the fields of Computational Intelligence and Biologically Inspired Computation:

\textbf{Scientists} - Research scientists concerned with theoretically or empirically investigating algorithms, addressing the questions:
\begin{itemize}
	\item What is the motivating system and strategy of a given technique?
	\item What are some algorithm that may be used in a comparison within a given subfield or across subfields?
\end{itemize}

\textbf{Engineers} - Programmers and developers concerned with implementing, applying, or maintaining algorithms, addressing the questions:
\begin{itemize}
	\item What is the algorithm procedure for a given technique?	
	\item What are the best practice heuristics for applying a given technique?
\end{itemize}

\textbf{Students} - Graduate and Undergraduate students interested in learning algorithms, addressing the questions:

\begin{itemize}
	\item  
	\item  algorithms.
	\item 
	\item \textbf{Amateurs}: Practitioners interested in knowing more about algorithms.
\end{itemize}


\section{Location}
the where?

\section{Methodology}
\label{sec:methodology}
how is this book going to get done?


\section{Conclusions}
\label{sec:conclusions}
what was this all about, what is next, what big questions and problems does this raise?


follow ups - extensions
- need to define fields of interest
- need to support open problem
- need to show technique development process in the field


\end{document}
