% The Clever Algorithms Project: Algorithm Descriptions

% The Clever Algorithms Project: http://www.CleverAlgorithms.com
% (c) Copyright 2010 Jason Brownlee. All Rights Reserved. 
% This work is licensed under a Creative Commons Attribution-Noncommercial-Share Alike 2.5 Australia License.

\documentclass[a4paper, 11pt]{article}
\usepackage{tabularx}
\usepackage{booktabs}
\usepackage{url}
\usepackage[pdftex,breaklinks=true,colorlinks=true,urlcolor=blue,linkcolor=blue,citecolor=blue,]{hyperref}
\usepackage{geometry}
\geometry{verbose,a4paper,tmargin=25mm,bmargin=25mm,lmargin=25mm,rmargin=25mm}

% Dear template user: fill these in
\newcommand{\myreporttitle}{A Template for Standardized Algorithm Descriptions}
\newcommand{\myreportauthor}{Jason Brownlee}
\newcommand{\myreportemail}{jasonb@CleverAlgorithms.com}
\newcommand{\myreportproject}{The Clever Algorithms Project\\\url{http://www.CleverAlgorithms.com}}
\newcommand{\myreportdate}{20100110}
\newcommand{\myreportversion}{1}
\newcommand{\myreportlicense}{\copyright\ Copyright 2010 Jason Brownlee. All Rights Reserved. This work is licensed under a Creative Commons Attribution-Noncommercial-Share Alike 2.5 Australia License.}

% leave this alone, it's templated baby!
\title{{\myreporttitle}\footnote{\myreportlicense}}
\author{\myreportauthor\\{\myreportemail}\\\small\myreportproject}
\date{\today\\{\small{Technical Report: CA-TR-{\myreportdate}-\myreportversion}}}
\begin{document}
\maketitle

% write a summary sentence for each major section
\section*{Abstract} 
todo

\begin{description}
	\item[Keywords:] {\small\texttt{Clever, Algorithms, Standard, Description, Procedure, Template, Algorithm}}
\end{description} 

% summarise the document breakdown with cross references
\section{Introduction}
\label{sec:introduction}
%  project
The Clever Algorithm Project is concerned with the complete, consistent, and centralized description of algorithms from the fields of Computational Intelligence and Biologically Inspired Computation to ensure that they are accessible, usable, and understandable \cite{Brownlee2010}.
% this report
This report provides an exploration and definition of the standardized structure for algorithm descriptions in the project.

% breakdown
Section~\ref{sec:elements} provides a summary of algorithm description elements that may be used in a standardized algorithm description template. Section~\ref{sec:template} proposes a template with specific description elements and expectations as to how the template should be adopted and refined.

\section{Descriptive Elements}
\label{sec:elements}
This section provides a summary of descriptive elements that may form apart of a standardized algorithm description. Each element is 1) defined, 2) specifies the intended nature of an instantiated description (such as size, scope and form), and 3) specifies two questions that motivate the content for the element.

\subsection{Name}
The algorithm name defines the canonical name used to refer to the technique, in addition to common aliases, abbreviations, and acronyms. The name is described in terms of the heading and sub-headings of an algorithm.

\begin{itemize}
	% names
	\item \emph{What is the canonical name and common aliases for a technique?}
	% short names
	\item \emph{What are the common abbreviations and acronyms for a technique?}
\end{itemize}

\subsection{Taxonomy: Inter-relationships with fields and other techniques}
The algorithm taxonomy defines where a techniques fits into the field, both the specific subfields of Computational Intelligence and Biologically Inspired Computation as well as the broader field of Artificial Intelligence. The taxonomy also provides a context for determining the relationships between algorithms. The taxonomy may be described in terms of a series of relationship statements or pictorially as a venn diagram or a graph with hierarchical structure.

\begin{itemize}
	% membership
	\item \emph{To what fields of study does a technique belong?}
	% inter-relationship
	\item \emph{What are closely related techniques to a technique?}
\end{itemize}

\subsection{Inspiration: The motivating system}
The inspiration describes the specific system or process that provoked the inception of the algorithm. The inspiring system may be natural, biological, physical, or social. The description of the inspiring system may include relevant domain specific theory, observation, nomenclature, and most important must include those salient attributes of the system that are somehow abstractly or conceptually manifest in the technique. The inspiration is described textually with source and may include diagrams to highlight features and relationships within the inspiring system.

\begin{itemize}
	% description
	\item \emph{What is the system that motivated the development of a technique?}
	% scope
	\item \emph{Which features of the motivating system are relevant to a technique?}
\end{itemize}

\subsection{Metaphor: Explanation via analogy}
The metaphor is a description of the technique in the context of the inspiring system or a different suitable system. The features of the technique are made apparent through an analogous description of the features of the inspiring system. The explanation through analogy is not expected to be literal scientific truth, rather the method is used as an allegorical communication tool. The inspiring system is not explicitly described, this is the role of the `inspiration' element, which represents a loose dependency for this element. The explanation is textual and uses the nomenclature of the metaphorical system. 

\begin{itemize}
	% metaphor (how does it work)
	\item \emph{What is the explanation of the technique in the context of the inspiring system?}
	% analogy (what can it do)
	\item \emph{What are the functionalities inferred from the analogous inspiring system?}
\end{itemize}

\subsection{Strategy: Problem solving information processing}
the abstracted computation in terms of information processing. abstraction of inspired metaphor with a problem, solution, and suggestion an computational system / algorithm

The algorithm strategy is an abstraction of the inspiration toward a computational model. The strategy highlights a canonical interpretation of the salient attributes in the inspiring system and how they map onto a computational approach with the goal of problem solving. The strategy provides a logical separation between a computational realization (algorithm) and a metaphoric system (inspiration). A given problem solving strategy may be realized as one of a number specific algorithms or problem solving systems.

\begin{itemize}
	\item \emph{What is the techniques plan of action?}
	\item \emph{What are the information processing properties of the technique?}
\end{itemize}

\subsection{Procedure}
The algorithm procedure summarizes the specifics of realizing the strategy as a systemized and parameterized computation. It outlines how the algorithm is organized in terms of the data structures and adopted representations. The procedure may be described in terms of software engineering and computer science artifacts such as pseudo code, design diagrams, and relevant mathematical equations.

\begin{itemize}
	% recipe
	\item \emph{What is the computational recipe for the technique?}
	% data
	\item \emph{What are the data structures and representations used in the technique?}
\end{itemize}

\subsection{Heuristics}
best practices and rules of thumb for configuring a parameterized algorithm. guidelines for using executing and using the strategy

\begin{itemize}
	% parameters
	\item \emph{What are the suggested configurations for the technique?}
	% application
	\item \emph{What are the guidelines for the application of the technique to a problem instance?}
\end{itemize}

\subsection{Applications}
the general and/or specific domains the technique has been applied or is suited

\begin{itemize}
	\item \emph{What are the suggested classes of problems for which the technique is suited?}
	\item \emph{What are some exemplar problem instances to which the technique has been applied?}
\end{itemize}

\subsection{History}
the dates and people behind the development and progression of the technique

\begin{itemize}
	\item \emph{Under what conditions was the original technique proposed?}
	\item \emph{What are the significant milestones in the development of the technique?}
\end{itemize}

\subsection{Further Reading}
sources of information to seek if more detail is required

The presented algorithm inspiration, strategy, and procedure were written for conciseness. As such, each algorithm provides a further reading section for those readers interested in a deeper understanding of the theory and application of the technique. The listing consists of hand-selected reference material including books, peer reviewed conference papers and journal articles.

\begin{itemize}
	\item \emph{What are the primary sources for a technique?}
	\item \emph{What are the suggested reference sources for learning more about a technique?}
\end{itemize}

\subsection{Source Code}
complete source code listing
a technique is not understood until is implemented and seen running, observing and manipulating the effects of these things is where the real work is 

\begin{itemize}
	\item \emph{How is a technique implemented as an executable program?}
	\item \emph{How is a technique applied to an actual problem instance?}
\end{itemize}

\subsection{Tutorial}
source code listing presented as a narrative. example of realizing the approach on a problem - demonstrate strategy, realize the procedure, manifest the heuristics

Tutorials are complete in that they result in a functional program, although they are demonstrative, implementing only just enough features to demonstrate a specific technique on a specific problem instance. They provide an exemplar realization of the technique suitable for demonstrating the fine details of the technique, and a template for specializing the technique for practical problem solving. 

\begin{itemize}
	\item \emph{What is the rationale when implementing a technique as an executable program?}
	\item \emph{What is the rationale when applying a technique to an actual problem instance?}
\end{itemize}

\section{Examples} 
\label{sec:examples}
This section elaborates the elements of a standardized algorithm description presented in Section~\ref{sec:elements} by briefly summarizing how three popular algorithms from the fields of Computational Intelligence and Biologically Inspired Computation may be described (in the simplest possible form).

\subsection{Genetic Algorithm}
todo

\subsection{Simulated Annealing}
todo

\subsection{Particle Swarm Optimization}
todo

\section{Standardized Description} 
\label{sec:template}
This section proposes a standardized algorithm template that includes some of the elements drawn from Section~\ref{sec:elements}.

\subsection{Template Composition}
need a mixture of abstraction and concreteness, need a mixture of narrative, programatic and potentially even diagrammatic 

\subsection{Algorithm Description Template}
Table~\ref{tab:template} contains a standardized algorithm description template to be completed by all algorithms described in the Clever Algorithms project.

\begin{table}[ht]
	\centering
		\begin{tabularx}{\textwidth}{lX}
		\toprule
		\textbf{Element} & \textbf{Description} \\ 
		\toprule
		\emph{Name} & The heading and alternate headings for the algorithm description.  \\ 
		\midrule
		\emph{Further Reading} & An annotated reference list of followup resources. \\
		\bottomrule
		\end{tabularx}	
	\caption{Standardized algorithm description template.}
	\label{tab:template}
\end{table}

\subsection{Template Usage}
how should the template be used? are all sections required? little of the information is provided in an appropriate format, most likely the hard work will be formulating the elements of the description for each algorithm
not really separable all the time

\subsection{A Living Standard}
The proposed template should be considered a first draft of a standard that is expected to be refined through adoption and use throughout the execution of the Clever Algorithms project. The final version of this standard that the project converges to will be used for all algorithms described in the projects compendium to ensure the objective of consistency is met. This constraint is suggested, but not required for the algorithms described workspace during content development.

% summarise the document message and areas for future consideration
\section{Conclusions}
\label{sec:conclusions}
todo

% bibliography
\bibliographystyle{plain}
\bibliography{../bibtex}

\end{document}
% EOF