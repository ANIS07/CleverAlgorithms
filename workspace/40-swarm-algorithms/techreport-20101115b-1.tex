% Swarm Algorithms

% The Clever Algorithms Project: http://www.CleverAlgorithms.com
% (c) Copyright 2010 Jason Brownlee. Some Rights Reserved. 
% This work is licensed under a Creative Commons Attribution-Noncommercial-Share Alike 2.5 Australia License.

\documentclass[a4paper, 11pt]{article}
\usepackage{tabularx}
\usepackage{booktabs}
\usepackage{url}
\usepackage[pdftex,breaklinks=true,colorlinks=true,urlcolor=blue,linkcolor=blue,citecolor=blue,]{hyperref}
\usepackage{geometry}
\geometry{verbose,a4paper,tmargin=25mm,bmargin=25mm,lmargin=25mm,rmargin=25mm}

% Dear template user: fill these in
\newcommand{\myreporttitle}{Swarm Algorithms}
\newcommand{\myreportauthor}{Jason Brownlee}
\newcommand{\myreportemail}{jasonb@CleverAlgorithms.com}
\newcommand{\myreportproject}{The Clever Algorithms Project\\\url{http://www.CleverAlgorithms.com}}
\newcommand{\myreportdate}{20101015b}
\newcommand{\myreportfulldate}{November 15, 2010}
\newcommand{\myreportversion}{1}
\newcommand{\myreportlicense}{\copyright\ Copyright 2010 Jason Brownlee. Some Rights Reserved. This work is licensed under a Creative Commons Attribution-Noncommercial-Share Alike 2.5 Australia License.}

% leave this alone, it's templated baby!
\title{{\myreporttitle}\footnote{\myreportlicense}}
\author{\myreportauthor\\{\myreportemail}\\\small\myreportproject}
\date{\myreportfulldate\\{\small{Technical Report: CA-TR-{\myreportdate}-\myreportversion}}}
\begin{document}
\maketitle

% write a summary sentence for each major section
\section*{Abstract} 
% project
The Clever Algorithms project aims to describe a large number of Artificial Intelligence algorithms in a complete, consistent, and centralized manner, to improve their general accessibility. 
% template
The project makes use of a standardized algorithm description template that uses well-defined topics that motivate the collection of specific and useful information about each algorithm described.
% type
A batch of algorithms for the project have been described, all of which are classified as Swarm Algorithms under the adopted algorithm taxonomy.
% best practices
This report provides a point of reflection on the preparation of these algorithms.

\begin{description}
	\item[Keywords:] {\small\texttt{Clever, Algorithms, Project, Swarm, Optimization, Findings}}
\end{description} 

% summarise the document breakdown with cross references
\section{Introduction}
\label{sec:introduction}
% project
The Clever Algorithms project aims to describe a large number of algorithms from the fields of Computational Intelligence, Biologically Inspired Computation, and Metaheuristics in a complete, consistent and centralized manner \cite{Brownlee2010}.
% description
The project requires all algorithms to be described using a standardized template that includes a fixed number of sections, each of which is motivated by the presentation of specific information about the technique \cite{Brownlee2010a}.
% this report
This report provides an overview of the Swarm Algorithms in the Clever Algorithms project. 
Section~\ref{sec:algorithms} provides background information and reviews common themes for the general class of algorithm and summarizes those immune algorithms that have been described for the Clever Algorithms Project.

% 
% Described Algorithms
% 
\section{Swarm Algorithms}
\label{sec:algorithms}

% 
% Background
% 
\subsection{Background}
% broadly
The algorithms to be described in the Clever Algorithms project are drawn from a diverse set of subfields of Artificial Intelligence, such as Computational Intelligence, Biologically Inspired Computation, and Metaheuristics. The majority of the algorithms selected for description in the project are optimization algorithms \cite{Brownlee2010b}. 
% specific
The recently completed algorithms that have been described for the Clever Algorithms project are referred to as Swarm Algorithms. They are differentiated from the remainder of the algorithms described in the project that have been designated a taxonomy including stochastic, immune, probabilistic, physical, and evolutionary algorithms \cite{Brownlee2010b}. 

% Immune
\subsubsection{Swarm Intelligence}
Swarm intelligence is the study of computational systems inspired by the collective intelligence that emerges through the cooperation of large numbers of homogeneous agents in the environment. Examples include schools of fish, flocks of birds, and colonies of ants. Such intelligence is decentralized, self-organizing and distributed through out an environment. In nature such systems are commonly used to solve problems such as effective foraging for food, prey evading, or colony re-location. The information is typically stored throughout the participating homogeneous agents, or is stored or communicated in the environment itself such as through the use of pheromones in ants, dancing in bees, and proximity in fish and birds.

The paradigm consists of two dominant sub-fields (1) \emph{Ant Colony Optimization} that investigates probabilistic algorithms inspired by the stigmergy and foraging behavior of ants, and (2) \emph{Particle Swarm Optimization} that investigates probabilistic algorithms inspired by the flocking, schooling and herding. Like evolutionary computation, swarm intelligences are considered adaptive strategies and are typically applied to search and optimization domains.

% References
\subsubsection{References}
% classical
The seminal books on the field of Swarm Intelligence include ``Swarm Intelligence'' by Kennedy, Eberhart and Shi \cite{Kennedy2001}, and ``Swarm Intelligence: From Natural to Artificial Systems'' by Bonabeau, Dorigo, and Theraulaz \cite{Bonabeau1999}. Another excellent text book on the area is ``Fundamentals of Computational Swarm Intelligence'' by Engelbrecht \cite{Engelbrecht2006}. The seminal book reference for the field of Ant Colony Optimization is ``Ant colony optimization'' by Dorigo and St\"utzle \cite{Dorigo2004}.

% 
% Described Algorithms
% 
\subsection{Described Algorithms}
\label{subsec:algorithms}
% overview
This section lists the swarm algorithms currently described for inclusion in the Clever Algorithms project. It is proposed that these algorithms will collectively comprise a chapter on `Swarm Algorithms' in the Clever Algorithms book. 

\begin{enumerate}
	\item \textbf{Particle Swarm Optimization}: \cite{Brownlee2010ac}
	\item \textbf{Ant System}: \cite{Brownlee2010ad}
	\item \textbf{Ant Colony System}: \cite{Brownlee2010ae}
\end{enumerate}

% 
% Extensions
% 
\section{Extensions}
\label{sec:extensions}
There are other algorithms and classes of algorithm that were not described from the field of Swarm Intelligence. Some areas that may be considered for algorithm description in follow up works include:

\begin{itemize}
	\item \textbf{Social Insects}: There are algorithms inspired by other social insects besides ants, such as: bees, wasps, fireflies, and termites.
	\item \textbf{Bacteria}: Another related area is that of algorithms inspired by the social-like properties of some bacteria which may be considered a swarm intelligence.
	\item \textbf{Cellular Systems}: There are algorithms inspired by social-like mechanisms in cellular systems like the immune system and cancers.
\end{itemize}

% 
% Conclusions
% 
\section{Conclusions}
\label{sec:conclusions}
% overview
This report provided a point of reflection for the batch of swarm algorithm descriptions prepared for the Clever Algorithms project. All described algorithms were assigned to the `Swarm Algorithms' kingdom in the chosen algorithm taxonomy. This report highlighted the commonality for all described Swarm Algorithms and provided a definition suitable for use in the proposed book and website.

% bibliography
\bibliographystyle{plain}
\bibliography{../bibtex}

\end{document}
% EOF